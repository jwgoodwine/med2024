\section{USING THE INTEGER TRAINED NETWORK ON FRACTIONAL ORDER STEP RESPONSES}
\label{sec:generalize}

Now we give a fractional order step response to the trained neural network to
see if it can generalize the training on first and second order transfer
functions to fractional order step responses. As indicated above, we use the
\texttt{numfracpy} python library to numerically compute the fractional order
step response for time from 0 to 10. The solution is numerically computed with a
time step of $\Delta t = 0.01$, but the input to the nextwork is 101 nodes, so
only every 10th element of the numerical solution is used.

We tested the network on 1000 fractional order step responses to transfer
functions of the form
\[
  X(s) = \left( \frac{k}{s^\alpha + k} \right) \left( \frac{1}{s} \right),
\]
or in the time domain
\begin{equation}
\frac{\d^\alpha x}{\d t^\alpha}(t) + k x(t) = k
\label{eq:fracstep}
\end{equation}
with zero initial conditions. 

The 1000 responses were each generated and tested as follows: 
\begin{itemize}
\item Randomly select an order between 1 and 2 from a uniform distribution.
\item Randomly select a $k$ between 5 and 9 from a uniform distribution. This
  range of values was selected to produce responses with a period of oscillation
  similar to the natural frequencies in the training set.
\item Numerically compute the step response to Equation~\ref{eq:fracstep} using
the \texttt{FODE()} function from the \texttt{numfracpy} library. 
\end{itemize}

\begin{figure}
\centering
%% Creator: Matplotlib, PGF backend
%%
%% To include the figure in your LaTeX document, write
%%   \input{<filename>.pgf}
%%
%% Make sure the required packages are loaded in your preamble
%%   \usepackage{pgf}
%%
%% Also ensure that all the required font packages are loaded; for instance,
%% the lmodern package is sometimes necessary when using math font.
%%   \usepackage{lmodern}
%%
%% Figures using additional raster images can only be included by \input if
%% they are in the same directory as the main LaTeX file. For loading figures
%% from other directories you can use the `import` package
%%   \usepackage{import}
%%
%% and then include the figures with
%%   \import{<path to file>}{<filename>.pgf}
%%
%% Matplotlib used the following preamble
%%   \def\mathdefault#1{#1}
%%   \everymath=\expandafter{\the\everymath\displaystyle}
%%   
%%   \usepackage{fontspec}
%%   \setmainfont{DejaVuSerif.ttf}[Path=\detokenize{/Users/billgoodwine/research/step/steps/lib/python3.11/site-packages/matplotlib/mpl-data/fonts/ttf/}]
%%   \setsansfont{DejaVuSans.ttf}[Path=\detokenize{/Users/billgoodwine/research/step/steps/lib/python3.11/site-packages/matplotlib/mpl-data/fonts/ttf/}]
%%   \setmonofont{DejaVuSansMono.ttf}[Path=\detokenize{/Users/billgoodwine/research/step/steps/lib/python3.11/site-packages/matplotlib/mpl-data/fonts/ttf/}]
%%   \makeatletter\@ifpackageloaded{underscore}{}{\usepackage[strings]{underscore}}\makeatother
%%
\begingroup%
\makeatletter%
\begin{pgfpicture}%
\pgfpathrectangle{\pgfpointorigin}{\pgfqpoint{3.500000in}{2.379431in}}%
\pgfusepath{use as bounding box, clip}%
\begin{pgfscope}%
\pgfsetbuttcap%
\pgfsetmiterjoin%
\definecolor{currentfill}{rgb}{1.000000,1.000000,1.000000}%
\pgfsetfillcolor{currentfill}%
\pgfsetlinewidth{0.000000pt}%
\definecolor{currentstroke}{rgb}{1.000000,1.000000,1.000000}%
\pgfsetstrokecolor{currentstroke}%
\pgfsetdash{}{0pt}%
\pgfpathmoveto{\pgfqpoint{0.000000in}{0.000000in}}%
\pgfpathlineto{\pgfqpoint{3.500000in}{0.000000in}}%
\pgfpathlineto{\pgfqpoint{3.500000in}{2.379431in}}%
\pgfpathlineto{\pgfqpoint{0.000000in}{2.379431in}}%
\pgfpathlineto{\pgfqpoint{0.000000in}{0.000000in}}%
\pgfpathclose%
\pgfusepath{fill}%
\end{pgfscope}%
\begin{pgfscope}%
\pgfsetbuttcap%
\pgfsetmiterjoin%
\definecolor{currentfill}{rgb}{1.000000,1.000000,1.000000}%
\pgfsetfillcolor{currentfill}%
\pgfsetlinewidth{0.000000pt}%
\definecolor{currentstroke}{rgb}{0.000000,0.000000,0.000000}%
\pgfsetstrokecolor{currentstroke}%
\pgfsetstrokeopacity{0.000000}%
\pgfsetdash{}{0pt}%
\pgfpathmoveto{\pgfqpoint{0.619136in}{0.571603in}}%
\pgfpathlineto{\pgfqpoint{3.350000in}{0.571603in}}%
\pgfpathlineto{\pgfqpoint{3.350000in}{2.229431in}}%
\pgfpathlineto{\pgfqpoint{0.619136in}{2.229431in}}%
\pgfpathlineto{\pgfqpoint{0.619136in}{0.571603in}}%
\pgfpathclose%
\pgfusepath{fill}%
\end{pgfscope}%
\begin{pgfscope}%
\pgfpathrectangle{\pgfqpoint{0.619136in}{0.571603in}}{\pgfqpoint{2.730864in}{1.657828in}}%
\pgfusepath{clip}%
\pgfsetrectcap%
\pgfsetroundjoin%
\pgfsetlinewidth{0.803000pt}%
\definecolor{currentstroke}{rgb}{0.690196,0.690196,0.690196}%
\pgfsetstrokecolor{currentstroke}%
\pgfsetdash{}{0pt}%
\pgfpathmoveto{\pgfqpoint{0.743267in}{0.571603in}}%
\pgfpathlineto{\pgfqpoint{0.743267in}{2.229431in}}%
\pgfusepath{stroke}%
\end{pgfscope}%
\begin{pgfscope}%
\pgfsetbuttcap%
\pgfsetroundjoin%
\definecolor{currentfill}{rgb}{0.000000,0.000000,0.000000}%
\pgfsetfillcolor{currentfill}%
\pgfsetlinewidth{0.803000pt}%
\definecolor{currentstroke}{rgb}{0.000000,0.000000,0.000000}%
\pgfsetstrokecolor{currentstroke}%
\pgfsetdash{}{0pt}%
\pgfsys@defobject{currentmarker}{\pgfqpoint{0.000000in}{-0.048611in}}{\pgfqpoint{0.000000in}{0.000000in}}{%
\pgfpathmoveto{\pgfqpoint{0.000000in}{0.000000in}}%
\pgfpathlineto{\pgfqpoint{0.000000in}{-0.048611in}}%
\pgfusepath{stroke,fill}%
}%
\begin{pgfscope}%
\pgfsys@transformshift{0.743267in}{0.571603in}%
\pgfsys@useobject{currentmarker}{}%
\end{pgfscope}%
\end{pgfscope}%
\begin{pgfscope}%
\definecolor{textcolor}{rgb}{0.000000,0.000000,0.000000}%
\pgfsetstrokecolor{textcolor}%
\pgfsetfillcolor{textcolor}%
\pgftext[x=0.743267in,y=0.474381in,,top]{\color{textcolor}{\rmfamily\fontsize{10.000000}{12.000000}\selectfont\catcode`\^=\active\def^{\ifmmode\sp\else\^{}\fi}\catcode`\%=\active\def%{\%}$\mathdefault{0}$}}%
\end{pgfscope}%
\begin{pgfscope}%
\pgfpathrectangle{\pgfqpoint{0.619136in}{0.571603in}}{\pgfqpoint{2.730864in}{1.657828in}}%
\pgfusepath{clip}%
\pgfsetrectcap%
\pgfsetroundjoin%
\pgfsetlinewidth{0.803000pt}%
\definecolor{currentstroke}{rgb}{0.690196,0.690196,0.690196}%
\pgfsetstrokecolor{currentstroke}%
\pgfsetdash{}{0pt}%
\pgfpathmoveto{\pgfqpoint{1.239787in}{0.571603in}}%
\pgfpathlineto{\pgfqpoint{1.239787in}{2.229431in}}%
\pgfusepath{stroke}%
\end{pgfscope}%
\begin{pgfscope}%
\pgfsetbuttcap%
\pgfsetroundjoin%
\definecolor{currentfill}{rgb}{0.000000,0.000000,0.000000}%
\pgfsetfillcolor{currentfill}%
\pgfsetlinewidth{0.803000pt}%
\definecolor{currentstroke}{rgb}{0.000000,0.000000,0.000000}%
\pgfsetstrokecolor{currentstroke}%
\pgfsetdash{}{0pt}%
\pgfsys@defobject{currentmarker}{\pgfqpoint{0.000000in}{-0.048611in}}{\pgfqpoint{0.000000in}{0.000000in}}{%
\pgfpathmoveto{\pgfqpoint{0.000000in}{0.000000in}}%
\pgfpathlineto{\pgfqpoint{0.000000in}{-0.048611in}}%
\pgfusepath{stroke,fill}%
}%
\begin{pgfscope}%
\pgfsys@transformshift{1.239787in}{0.571603in}%
\pgfsys@useobject{currentmarker}{}%
\end{pgfscope}%
\end{pgfscope}%
\begin{pgfscope}%
\definecolor{textcolor}{rgb}{0.000000,0.000000,0.000000}%
\pgfsetstrokecolor{textcolor}%
\pgfsetfillcolor{textcolor}%
\pgftext[x=1.239787in,y=0.474381in,,top]{\color{textcolor}{\rmfamily\fontsize{10.000000}{12.000000}\selectfont\catcode`\^=\active\def^{\ifmmode\sp\else\^{}\fi}\catcode`\%=\active\def%{\%}$\mathdefault{2}$}}%
\end{pgfscope}%
\begin{pgfscope}%
\pgfpathrectangle{\pgfqpoint{0.619136in}{0.571603in}}{\pgfqpoint{2.730864in}{1.657828in}}%
\pgfusepath{clip}%
\pgfsetrectcap%
\pgfsetroundjoin%
\pgfsetlinewidth{0.803000pt}%
\definecolor{currentstroke}{rgb}{0.690196,0.690196,0.690196}%
\pgfsetstrokecolor{currentstroke}%
\pgfsetdash{}{0pt}%
\pgfpathmoveto{\pgfqpoint{1.736308in}{0.571603in}}%
\pgfpathlineto{\pgfqpoint{1.736308in}{2.229431in}}%
\pgfusepath{stroke}%
\end{pgfscope}%
\begin{pgfscope}%
\pgfsetbuttcap%
\pgfsetroundjoin%
\definecolor{currentfill}{rgb}{0.000000,0.000000,0.000000}%
\pgfsetfillcolor{currentfill}%
\pgfsetlinewidth{0.803000pt}%
\definecolor{currentstroke}{rgb}{0.000000,0.000000,0.000000}%
\pgfsetstrokecolor{currentstroke}%
\pgfsetdash{}{0pt}%
\pgfsys@defobject{currentmarker}{\pgfqpoint{0.000000in}{-0.048611in}}{\pgfqpoint{0.000000in}{0.000000in}}{%
\pgfpathmoveto{\pgfqpoint{0.000000in}{0.000000in}}%
\pgfpathlineto{\pgfqpoint{0.000000in}{-0.048611in}}%
\pgfusepath{stroke,fill}%
}%
\begin{pgfscope}%
\pgfsys@transformshift{1.736308in}{0.571603in}%
\pgfsys@useobject{currentmarker}{}%
\end{pgfscope}%
\end{pgfscope}%
\begin{pgfscope}%
\definecolor{textcolor}{rgb}{0.000000,0.000000,0.000000}%
\pgfsetstrokecolor{textcolor}%
\pgfsetfillcolor{textcolor}%
\pgftext[x=1.736308in,y=0.474381in,,top]{\color{textcolor}{\rmfamily\fontsize{10.000000}{12.000000}\selectfont\catcode`\^=\active\def^{\ifmmode\sp\else\^{}\fi}\catcode`\%=\active\def%{\%}$\mathdefault{4}$}}%
\end{pgfscope}%
\begin{pgfscope}%
\pgfpathrectangle{\pgfqpoint{0.619136in}{0.571603in}}{\pgfqpoint{2.730864in}{1.657828in}}%
\pgfusepath{clip}%
\pgfsetrectcap%
\pgfsetroundjoin%
\pgfsetlinewidth{0.803000pt}%
\definecolor{currentstroke}{rgb}{0.690196,0.690196,0.690196}%
\pgfsetstrokecolor{currentstroke}%
\pgfsetdash{}{0pt}%
\pgfpathmoveto{\pgfqpoint{2.232829in}{0.571603in}}%
\pgfpathlineto{\pgfqpoint{2.232829in}{2.229431in}}%
\pgfusepath{stroke}%
\end{pgfscope}%
\begin{pgfscope}%
\pgfsetbuttcap%
\pgfsetroundjoin%
\definecolor{currentfill}{rgb}{0.000000,0.000000,0.000000}%
\pgfsetfillcolor{currentfill}%
\pgfsetlinewidth{0.803000pt}%
\definecolor{currentstroke}{rgb}{0.000000,0.000000,0.000000}%
\pgfsetstrokecolor{currentstroke}%
\pgfsetdash{}{0pt}%
\pgfsys@defobject{currentmarker}{\pgfqpoint{0.000000in}{-0.048611in}}{\pgfqpoint{0.000000in}{0.000000in}}{%
\pgfpathmoveto{\pgfqpoint{0.000000in}{0.000000in}}%
\pgfpathlineto{\pgfqpoint{0.000000in}{-0.048611in}}%
\pgfusepath{stroke,fill}%
}%
\begin{pgfscope}%
\pgfsys@transformshift{2.232829in}{0.571603in}%
\pgfsys@useobject{currentmarker}{}%
\end{pgfscope}%
\end{pgfscope}%
\begin{pgfscope}%
\definecolor{textcolor}{rgb}{0.000000,0.000000,0.000000}%
\pgfsetstrokecolor{textcolor}%
\pgfsetfillcolor{textcolor}%
\pgftext[x=2.232829in,y=0.474381in,,top]{\color{textcolor}{\rmfamily\fontsize{10.000000}{12.000000}\selectfont\catcode`\^=\active\def^{\ifmmode\sp\else\^{}\fi}\catcode`\%=\active\def%{\%}$\mathdefault{6}$}}%
\end{pgfscope}%
\begin{pgfscope}%
\pgfpathrectangle{\pgfqpoint{0.619136in}{0.571603in}}{\pgfqpoint{2.730864in}{1.657828in}}%
\pgfusepath{clip}%
\pgfsetrectcap%
\pgfsetroundjoin%
\pgfsetlinewidth{0.803000pt}%
\definecolor{currentstroke}{rgb}{0.690196,0.690196,0.690196}%
\pgfsetstrokecolor{currentstroke}%
\pgfsetdash{}{0pt}%
\pgfpathmoveto{\pgfqpoint{2.729349in}{0.571603in}}%
\pgfpathlineto{\pgfqpoint{2.729349in}{2.229431in}}%
\pgfusepath{stroke}%
\end{pgfscope}%
\begin{pgfscope}%
\pgfsetbuttcap%
\pgfsetroundjoin%
\definecolor{currentfill}{rgb}{0.000000,0.000000,0.000000}%
\pgfsetfillcolor{currentfill}%
\pgfsetlinewidth{0.803000pt}%
\definecolor{currentstroke}{rgb}{0.000000,0.000000,0.000000}%
\pgfsetstrokecolor{currentstroke}%
\pgfsetdash{}{0pt}%
\pgfsys@defobject{currentmarker}{\pgfqpoint{0.000000in}{-0.048611in}}{\pgfqpoint{0.000000in}{0.000000in}}{%
\pgfpathmoveto{\pgfqpoint{0.000000in}{0.000000in}}%
\pgfpathlineto{\pgfqpoint{0.000000in}{-0.048611in}}%
\pgfusepath{stroke,fill}%
}%
\begin{pgfscope}%
\pgfsys@transformshift{2.729349in}{0.571603in}%
\pgfsys@useobject{currentmarker}{}%
\end{pgfscope}%
\end{pgfscope}%
\begin{pgfscope}%
\definecolor{textcolor}{rgb}{0.000000,0.000000,0.000000}%
\pgfsetstrokecolor{textcolor}%
\pgfsetfillcolor{textcolor}%
\pgftext[x=2.729349in,y=0.474381in,,top]{\color{textcolor}{\rmfamily\fontsize{10.000000}{12.000000}\selectfont\catcode`\^=\active\def^{\ifmmode\sp\else\^{}\fi}\catcode`\%=\active\def%{\%}$\mathdefault{8}$}}%
\end{pgfscope}%
\begin{pgfscope}%
\pgfpathrectangle{\pgfqpoint{0.619136in}{0.571603in}}{\pgfqpoint{2.730864in}{1.657828in}}%
\pgfusepath{clip}%
\pgfsetrectcap%
\pgfsetroundjoin%
\pgfsetlinewidth{0.803000pt}%
\definecolor{currentstroke}{rgb}{0.690196,0.690196,0.690196}%
\pgfsetstrokecolor{currentstroke}%
\pgfsetdash{}{0pt}%
\pgfpathmoveto{\pgfqpoint{3.225870in}{0.571603in}}%
\pgfpathlineto{\pgfqpoint{3.225870in}{2.229431in}}%
\pgfusepath{stroke}%
\end{pgfscope}%
\begin{pgfscope}%
\pgfsetbuttcap%
\pgfsetroundjoin%
\definecolor{currentfill}{rgb}{0.000000,0.000000,0.000000}%
\pgfsetfillcolor{currentfill}%
\pgfsetlinewidth{0.803000pt}%
\definecolor{currentstroke}{rgb}{0.000000,0.000000,0.000000}%
\pgfsetstrokecolor{currentstroke}%
\pgfsetdash{}{0pt}%
\pgfsys@defobject{currentmarker}{\pgfqpoint{0.000000in}{-0.048611in}}{\pgfqpoint{0.000000in}{0.000000in}}{%
\pgfpathmoveto{\pgfqpoint{0.000000in}{0.000000in}}%
\pgfpathlineto{\pgfqpoint{0.000000in}{-0.048611in}}%
\pgfusepath{stroke,fill}%
}%
\begin{pgfscope}%
\pgfsys@transformshift{3.225870in}{0.571603in}%
\pgfsys@useobject{currentmarker}{}%
\end{pgfscope}%
\end{pgfscope}%
\begin{pgfscope}%
\definecolor{textcolor}{rgb}{0.000000,0.000000,0.000000}%
\pgfsetstrokecolor{textcolor}%
\pgfsetfillcolor{textcolor}%
\pgftext[x=3.225870in,y=0.474381in,,top]{\color{textcolor}{\rmfamily\fontsize{10.000000}{12.000000}\selectfont\catcode`\^=\active\def^{\ifmmode\sp\else\^{}\fi}\catcode`\%=\active\def%{\%}$\mathdefault{10}$}}%
\end{pgfscope}%
\begin{pgfscope}%
\definecolor{textcolor}{rgb}{0.000000,0.000000,0.000000}%
\pgfsetstrokecolor{textcolor}%
\pgfsetfillcolor{textcolor}%
\pgftext[x=1.984568in,y=0.284413in,,top]{\color{textcolor}{\rmfamily\fontsize{10.000000}{12.000000}\selectfont\catcode`\^=\active\def^{\ifmmode\sp\else\^{}\fi}\catcode`\%=\active\def%{\%}$t$}}%
\end{pgfscope}%
\begin{pgfscope}%
\pgfpathrectangle{\pgfqpoint{0.619136in}{0.571603in}}{\pgfqpoint{2.730864in}{1.657828in}}%
\pgfusepath{clip}%
\pgfsetrectcap%
\pgfsetroundjoin%
\pgfsetlinewidth{0.803000pt}%
\definecolor{currentstroke}{rgb}{0.690196,0.690196,0.690196}%
\pgfsetstrokecolor{currentstroke}%
\pgfsetdash{}{0pt}%
\pgfpathmoveto{\pgfqpoint{0.619136in}{0.646959in}}%
\pgfpathlineto{\pgfqpoint{3.350000in}{0.646959in}}%
\pgfusepath{stroke}%
\end{pgfscope}%
\begin{pgfscope}%
\pgfsetbuttcap%
\pgfsetroundjoin%
\definecolor{currentfill}{rgb}{0.000000,0.000000,0.000000}%
\pgfsetfillcolor{currentfill}%
\pgfsetlinewidth{0.803000pt}%
\definecolor{currentstroke}{rgb}{0.000000,0.000000,0.000000}%
\pgfsetstrokecolor{currentstroke}%
\pgfsetdash{}{0pt}%
\pgfsys@defobject{currentmarker}{\pgfqpoint{-0.048611in}{0.000000in}}{\pgfqpoint{-0.000000in}{0.000000in}}{%
\pgfpathmoveto{\pgfqpoint{-0.000000in}{0.000000in}}%
\pgfpathlineto{\pgfqpoint{-0.048611in}{0.000000in}}%
\pgfusepath{stroke,fill}%
}%
\begin{pgfscope}%
\pgfsys@transformshift{0.619136in}{0.646959in}%
\pgfsys@useobject{currentmarker}{}%
\end{pgfscope}%
\end{pgfscope}%
\begin{pgfscope}%
\definecolor{textcolor}{rgb}{0.000000,0.000000,0.000000}%
\pgfsetstrokecolor{textcolor}%
\pgfsetfillcolor{textcolor}%
\pgftext[x=0.344444in, y=0.594198in, left, base]{\color{textcolor}{\rmfamily\fontsize{10.000000}{12.000000}\selectfont\catcode`\^=\active\def^{\ifmmode\sp\else\^{}\fi}\catcode`\%=\active\def%{\%}$\mathdefault{0.0}$}}%
\end{pgfscope}%
\begin{pgfscope}%
\pgfpathrectangle{\pgfqpoint{0.619136in}{0.571603in}}{\pgfqpoint{2.730864in}{1.657828in}}%
\pgfusepath{clip}%
\pgfsetrectcap%
\pgfsetroundjoin%
\pgfsetlinewidth{0.803000pt}%
\definecolor{currentstroke}{rgb}{0.690196,0.690196,0.690196}%
\pgfsetstrokecolor{currentstroke}%
\pgfsetdash{}{0pt}%
\pgfpathmoveto{\pgfqpoint{0.619136in}{1.025622in}}%
\pgfpathlineto{\pgfqpoint{3.350000in}{1.025622in}}%
\pgfusepath{stroke}%
\end{pgfscope}%
\begin{pgfscope}%
\pgfsetbuttcap%
\pgfsetroundjoin%
\definecolor{currentfill}{rgb}{0.000000,0.000000,0.000000}%
\pgfsetfillcolor{currentfill}%
\pgfsetlinewidth{0.803000pt}%
\definecolor{currentstroke}{rgb}{0.000000,0.000000,0.000000}%
\pgfsetstrokecolor{currentstroke}%
\pgfsetdash{}{0pt}%
\pgfsys@defobject{currentmarker}{\pgfqpoint{-0.048611in}{0.000000in}}{\pgfqpoint{-0.000000in}{0.000000in}}{%
\pgfpathmoveto{\pgfqpoint{-0.000000in}{0.000000in}}%
\pgfpathlineto{\pgfqpoint{-0.048611in}{0.000000in}}%
\pgfusepath{stroke,fill}%
}%
\begin{pgfscope}%
\pgfsys@transformshift{0.619136in}{1.025622in}%
\pgfsys@useobject{currentmarker}{}%
\end{pgfscope}%
\end{pgfscope}%
\begin{pgfscope}%
\definecolor{textcolor}{rgb}{0.000000,0.000000,0.000000}%
\pgfsetstrokecolor{textcolor}%
\pgfsetfillcolor{textcolor}%
\pgftext[x=0.344444in, y=0.972861in, left, base]{\color{textcolor}{\rmfamily\fontsize{10.000000}{12.000000}\selectfont\catcode`\^=\active\def^{\ifmmode\sp\else\^{}\fi}\catcode`\%=\active\def%{\%}$\mathdefault{0.5}$}}%
\end{pgfscope}%
\begin{pgfscope}%
\pgfpathrectangle{\pgfqpoint{0.619136in}{0.571603in}}{\pgfqpoint{2.730864in}{1.657828in}}%
\pgfusepath{clip}%
\pgfsetrectcap%
\pgfsetroundjoin%
\pgfsetlinewidth{0.803000pt}%
\definecolor{currentstroke}{rgb}{0.690196,0.690196,0.690196}%
\pgfsetstrokecolor{currentstroke}%
\pgfsetdash{}{0pt}%
\pgfpathmoveto{\pgfqpoint{0.619136in}{1.404285in}}%
\pgfpathlineto{\pgfqpoint{3.350000in}{1.404285in}}%
\pgfusepath{stroke}%
\end{pgfscope}%
\begin{pgfscope}%
\pgfsetbuttcap%
\pgfsetroundjoin%
\definecolor{currentfill}{rgb}{0.000000,0.000000,0.000000}%
\pgfsetfillcolor{currentfill}%
\pgfsetlinewidth{0.803000pt}%
\definecolor{currentstroke}{rgb}{0.000000,0.000000,0.000000}%
\pgfsetstrokecolor{currentstroke}%
\pgfsetdash{}{0pt}%
\pgfsys@defobject{currentmarker}{\pgfqpoint{-0.048611in}{0.000000in}}{\pgfqpoint{-0.000000in}{0.000000in}}{%
\pgfpathmoveto{\pgfqpoint{-0.000000in}{0.000000in}}%
\pgfpathlineto{\pgfqpoint{-0.048611in}{0.000000in}}%
\pgfusepath{stroke,fill}%
}%
\begin{pgfscope}%
\pgfsys@transformshift{0.619136in}{1.404285in}%
\pgfsys@useobject{currentmarker}{}%
\end{pgfscope}%
\end{pgfscope}%
\begin{pgfscope}%
\definecolor{textcolor}{rgb}{0.000000,0.000000,0.000000}%
\pgfsetstrokecolor{textcolor}%
\pgfsetfillcolor{textcolor}%
\pgftext[x=0.344444in, y=1.351524in, left, base]{\color{textcolor}{\rmfamily\fontsize{10.000000}{12.000000}\selectfont\catcode`\^=\active\def^{\ifmmode\sp\else\^{}\fi}\catcode`\%=\active\def%{\%}$\mathdefault{1.0}$}}%
\end{pgfscope}%
\begin{pgfscope}%
\pgfpathrectangle{\pgfqpoint{0.619136in}{0.571603in}}{\pgfqpoint{2.730864in}{1.657828in}}%
\pgfusepath{clip}%
\pgfsetrectcap%
\pgfsetroundjoin%
\pgfsetlinewidth{0.803000pt}%
\definecolor{currentstroke}{rgb}{0.690196,0.690196,0.690196}%
\pgfsetstrokecolor{currentstroke}%
\pgfsetdash{}{0pt}%
\pgfpathmoveto{\pgfqpoint{0.619136in}{1.782948in}}%
\pgfpathlineto{\pgfqpoint{3.350000in}{1.782948in}}%
\pgfusepath{stroke}%
\end{pgfscope}%
\begin{pgfscope}%
\pgfsetbuttcap%
\pgfsetroundjoin%
\definecolor{currentfill}{rgb}{0.000000,0.000000,0.000000}%
\pgfsetfillcolor{currentfill}%
\pgfsetlinewidth{0.803000pt}%
\definecolor{currentstroke}{rgb}{0.000000,0.000000,0.000000}%
\pgfsetstrokecolor{currentstroke}%
\pgfsetdash{}{0pt}%
\pgfsys@defobject{currentmarker}{\pgfqpoint{-0.048611in}{0.000000in}}{\pgfqpoint{-0.000000in}{0.000000in}}{%
\pgfpathmoveto{\pgfqpoint{-0.000000in}{0.000000in}}%
\pgfpathlineto{\pgfqpoint{-0.048611in}{0.000000in}}%
\pgfusepath{stroke,fill}%
}%
\begin{pgfscope}%
\pgfsys@transformshift{0.619136in}{1.782948in}%
\pgfsys@useobject{currentmarker}{}%
\end{pgfscope}%
\end{pgfscope}%
\begin{pgfscope}%
\definecolor{textcolor}{rgb}{0.000000,0.000000,0.000000}%
\pgfsetstrokecolor{textcolor}%
\pgfsetfillcolor{textcolor}%
\pgftext[x=0.344444in, y=1.730187in, left, base]{\color{textcolor}{\rmfamily\fontsize{10.000000}{12.000000}\selectfont\catcode`\^=\active\def^{\ifmmode\sp\else\^{}\fi}\catcode`\%=\active\def%{\%}$\mathdefault{1.5}$}}%
\end{pgfscope}%
\begin{pgfscope}%
\pgfpathrectangle{\pgfqpoint{0.619136in}{0.571603in}}{\pgfqpoint{2.730864in}{1.657828in}}%
\pgfusepath{clip}%
\pgfsetrectcap%
\pgfsetroundjoin%
\pgfsetlinewidth{0.803000pt}%
\definecolor{currentstroke}{rgb}{0.690196,0.690196,0.690196}%
\pgfsetstrokecolor{currentstroke}%
\pgfsetdash{}{0pt}%
\pgfpathmoveto{\pgfqpoint{0.619136in}{2.161611in}}%
\pgfpathlineto{\pgfqpoint{3.350000in}{2.161611in}}%
\pgfusepath{stroke}%
\end{pgfscope}%
\begin{pgfscope}%
\pgfsetbuttcap%
\pgfsetroundjoin%
\definecolor{currentfill}{rgb}{0.000000,0.000000,0.000000}%
\pgfsetfillcolor{currentfill}%
\pgfsetlinewidth{0.803000pt}%
\definecolor{currentstroke}{rgb}{0.000000,0.000000,0.000000}%
\pgfsetstrokecolor{currentstroke}%
\pgfsetdash{}{0pt}%
\pgfsys@defobject{currentmarker}{\pgfqpoint{-0.048611in}{0.000000in}}{\pgfqpoint{-0.000000in}{0.000000in}}{%
\pgfpathmoveto{\pgfqpoint{-0.000000in}{0.000000in}}%
\pgfpathlineto{\pgfqpoint{-0.048611in}{0.000000in}}%
\pgfusepath{stroke,fill}%
}%
\begin{pgfscope}%
\pgfsys@transformshift{0.619136in}{2.161611in}%
\pgfsys@useobject{currentmarker}{}%
\end{pgfscope}%
\end{pgfscope}%
\begin{pgfscope}%
\definecolor{textcolor}{rgb}{0.000000,0.000000,0.000000}%
\pgfsetstrokecolor{textcolor}%
\pgfsetfillcolor{textcolor}%
\pgftext[x=0.344444in, y=2.108849in, left, base]{\color{textcolor}{\rmfamily\fontsize{10.000000}{12.000000}\selectfont\catcode`\^=\active\def^{\ifmmode\sp\else\^{}\fi}\catcode`\%=\active\def%{\%}$\mathdefault{2.0}$}}%
\end{pgfscope}%
\begin{pgfscope}%
\definecolor{textcolor}{rgb}{0.000000,0.000000,0.000000}%
\pgfsetstrokecolor{textcolor}%
\pgfsetfillcolor{textcolor}%
\pgftext[x=0.288889in,y=1.400517in,,bottom,rotate=90.000000]{\color{textcolor}{\rmfamily\fontsize{10.000000}{12.000000}\selectfont\catcode`\^=\active\def^{\ifmmode\sp\else\^{}\fi}\catcode`\%=\active\def%{\%}$x(t)$}}%
\end{pgfscope}%
\begin{pgfscope}%
\pgfpathrectangle{\pgfqpoint{0.619136in}{0.571603in}}{\pgfqpoint{2.730864in}{1.657828in}}%
\pgfusepath{clip}%
\pgfsetrectcap%
\pgfsetroundjoin%
\pgfsetlinewidth{1.505625pt}%
\definecolor{currentstroke}{rgb}{0.121569,0.466667,0.705882}%
\pgfsetstrokecolor{currentstroke}%
\pgfsetdash{}{0pt}%
\pgfpathmoveto{\pgfqpoint{0.743267in}{0.646959in}}%
\pgfpathlineto{\pgfqpoint{0.748232in}{0.686817in}}%
\pgfpathlineto{\pgfqpoint{0.768093in}{0.892119in}}%
\pgfpathlineto{\pgfqpoint{0.782988in}{1.034080in}}%
\pgfpathlineto{\pgfqpoint{0.795401in}{1.134904in}}%
\pgfpathlineto{\pgfqpoint{0.807814in}{1.218824in}}%
\pgfpathlineto{\pgfqpoint{0.820227in}{1.286680in}}%
\pgfpathlineto{\pgfqpoint{0.830158in}{1.330469in}}%
\pgfpathlineto{\pgfqpoint{0.840088in}{1.365986in}}%
\pgfpathlineto{\pgfqpoint{0.850018in}{1.394248in}}%
\pgfpathlineto{\pgfqpoint{0.859949in}{1.416251in}}%
\pgfpathlineto{\pgfqpoint{0.869879in}{1.432935in}}%
\pgfpathlineto{\pgfqpoint{0.879810in}{1.445165in}}%
\pgfpathlineto{\pgfqpoint{0.889740in}{1.453718in}}%
\pgfpathlineto{\pgfqpoint{0.899671in}{1.459281in}}%
\pgfpathlineto{\pgfqpoint{0.909601in}{1.462453in}}%
\pgfpathlineto{\pgfqpoint{0.922014in}{1.463831in}}%
\pgfpathlineto{\pgfqpoint{0.936910in}{1.462767in}}%
\pgfpathlineto{\pgfqpoint{0.954288in}{1.459162in}}%
\pgfpathlineto{\pgfqpoint{0.984079in}{1.450362in}}%
\pgfpathlineto{\pgfqpoint{1.033731in}{1.435911in}}%
\pgfpathlineto{\pgfqpoint{1.068488in}{1.428374in}}%
\pgfpathlineto{\pgfqpoint{1.105727in}{1.422759in}}%
\pgfpathlineto{\pgfqpoint{1.150413in}{1.418510in}}%
\pgfpathlineto{\pgfqpoint{1.209996in}{1.415324in}}%
\pgfpathlineto{\pgfqpoint{1.306817in}{1.412673in}}%
\pgfpathlineto{\pgfqpoint{1.497978in}{1.409996in}}%
\pgfpathlineto{\pgfqpoint{1.823199in}{1.407869in}}%
\pgfpathlineto{\pgfqpoint{2.458745in}{1.406283in}}%
\pgfpathlineto{\pgfqpoint{3.225870in}{1.405549in}}%
\pgfpathlineto{\pgfqpoint{3.225870in}{1.405549in}}%
\pgfusepath{stroke}%
\end{pgfscope}%
\begin{pgfscope}%
\pgfpathrectangle{\pgfqpoint{0.619136in}{0.571603in}}{\pgfqpoint{2.730864in}{1.657828in}}%
\pgfusepath{clip}%
\pgfsetrectcap%
\pgfsetroundjoin%
\pgfsetlinewidth{1.505625pt}%
\definecolor{currentstroke}{rgb}{1.000000,0.498039,0.054902}%
\pgfsetstrokecolor{currentstroke}%
\pgfsetdash{}{0pt}%
\pgfpathmoveto{\pgfqpoint{0.743267in}{0.646959in}}%
\pgfpathlineto{\pgfqpoint{0.745749in}{0.647858in}}%
\pgfpathlineto{\pgfqpoint{0.750714in}{0.652997in}}%
\pgfpathlineto{\pgfqpoint{0.755680in}{0.661572in}}%
\pgfpathlineto{\pgfqpoint{0.763127in}{0.679811in}}%
\pgfpathlineto{\pgfqpoint{0.773058in}{0.712684in}}%
\pgfpathlineto{\pgfqpoint{0.782988in}{0.753840in}}%
\pgfpathlineto{\pgfqpoint{0.795401in}{0.814905in}}%
\pgfpathlineto{\pgfqpoint{0.810297in}{0.899275in}}%
\pgfpathlineto{\pgfqpoint{0.832640in}{1.041009in}}%
\pgfpathlineto{\pgfqpoint{0.897188in}{1.462922in}}%
\pgfpathlineto{\pgfqpoint{0.917049in}{1.573408in}}%
\pgfpathlineto{\pgfqpoint{0.931944in}{1.645393in}}%
\pgfpathlineto{\pgfqpoint{0.946840in}{1.706708in}}%
\pgfpathlineto{\pgfqpoint{0.959253in}{1.749056in}}%
\pgfpathlineto{\pgfqpoint{0.971666in}{1.783144in}}%
\pgfpathlineto{\pgfqpoint{0.981596in}{1.804385in}}%
\pgfpathlineto{\pgfqpoint{0.991527in}{1.820276in}}%
\pgfpathlineto{\pgfqpoint{0.998975in}{1.828727in}}%
\pgfpathlineto{\pgfqpoint{1.006423in}{1.834263in}}%
\pgfpathlineto{\pgfqpoint{1.013870in}{1.836953in}}%
\pgfpathlineto{\pgfqpoint{1.021318in}{1.836884in}}%
\pgfpathlineto{\pgfqpoint{1.028766in}{1.834155in}}%
\pgfpathlineto{\pgfqpoint{1.036214in}{1.828881in}}%
\pgfpathlineto{\pgfqpoint{1.043662in}{1.821186in}}%
\pgfpathlineto{\pgfqpoint{1.053592in}{1.807401in}}%
\pgfpathlineto{\pgfqpoint{1.063522in}{1.789908in}}%
\pgfpathlineto{\pgfqpoint{1.075935in}{1.763403in}}%
\pgfpathlineto{\pgfqpoint{1.090831in}{1.725818in}}%
\pgfpathlineto{\pgfqpoint{1.108209in}{1.675769in}}%
\pgfpathlineto{\pgfqpoint{1.133035in}{1.596952in}}%
\pgfpathlineto{\pgfqpoint{1.187653in}{1.421496in}}%
\pgfpathlineto{\pgfqpoint{1.207513in}{1.365506in}}%
\pgfpathlineto{\pgfqpoint{1.224892in}{1.322601in}}%
\pgfpathlineto{\pgfqpoint{1.239787in}{1.291119in}}%
\pgfpathlineto{\pgfqpoint{1.254683in}{1.264972in}}%
\pgfpathlineto{\pgfqpoint{1.267096in}{1.247445in}}%
\pgfpathlineto{\pgfqpoint{1.279509in}{1.233864in}}%
\pgfpathlineto{\pgfqpoint{1.289439in}{1.225836in}}%
\pgfpathlineto{\pgfqpoint{1.299370in}{1.220291in}}%
\pgfpathlineto{\pgfqpoint{1.309300in}{1.217167in}}%
\pgfpathlineto{\pgfqpoint{1.319231in}{1.216376in}}%
\pgfpathlineto{\pgfqpoint{1.329161in}{1.217809in}}%
\pgfpathlineto{\pgfqpoint{1.339091in}{1.221338in}}%
\pgfpathlineto{\pgfqpoint{1.351504in}{1.228472in}}%
\pgfpathlineto{\pgfqpoint{1.363917in}{1.238336in}}%
\pgfpathlineto{\pgfqpoint{1.378813in}{1.253288in}}%
\pgfpathlineto{\pgfqpoint{1.396191in}{1.274226in}}%
\pgfpathlineto{\pgfqpoint{1.418535in}{1.305068in}}%
\pgfpathlineto{\pgfqpoint{1.502943in}{1.426215in}}%
\pgfpathlineto{\pgfqpoint{1.522804in}{1.448909in}}%
\pgfpathlineto{\pgfqpoint{1.540182in}{1.465493in}}%
\pgfpathlineto{\pgfqpoint{1.555078in}{1.477028in}}%
\pgfpathlineto{\pgfqpoint{1.569973in}{1.485971in}}%
\pgfpathlineto{\pgfqpoint{1.584869in}{1.492287in}}%
\pgfpathlineto{\pgfqpoint{1.599765in}{1.496009in}}%
\pgfpathlineto{\pgfqpoint{1.614660in}{1.497234in}}%
\pgfpathlineto{\pgfqpoint{1.629556in}{1.496113in}}%
\pgfpathlineto{\pgfqpoint{1.644452in}{1.492845in}}%
\pgfpathlineto{\pgfqpoint{1.661830in}{1.486639in}}%
\pgfpathlineto{\pgfqpoint{1.681691in}{1.476921in}}%
\pgfpathlineto{\pgfqpoint{1.706517in}{1.461860in}}%
\pgfpathlineto{\pgfqpoint{1.743756in}{1.436104in}}%
\pgfpathlineto{\pgfqpoint{1.788443in}{1.405776in}}%
\pgfpathlineto{\pgfqpoint{1.815751in}{1.390088in}}%
\pgfpathlineto{\pgfqpoint{1.838095in}{1.379735in}}%
\pgfpathlineto{\pgfqpoint{1.860438in}{1.371972in}}%
\pgfpathlineto{\pgfqpoint{1.880299in}{1.367367in}}%
\pgfpathlineto{\pgfqpoint{1.900160in}{1.364913in}}%
\pgfpathlineto{\pgfqpoint{1.920021in}{1.364508in}}%
\pgfpathlineto{\pgfqpoint{1.942364in}{1.366269in}}%
\pgfpathlineto{\pgfqpoint{1.967190in}{1.370557in}}%
\pgfpathlineto{\pgfqpoint{1.996981in}{1.378127in}}%
\pgfpathlineto{\pgfqpoint{2.044151in}{1.392914in}}%
\pgfpathlineto{\pgfqpoint{2.098768in}{1.409481in}}%
\pgfpathlineto{\pgfqpoint{2.131042in}{1.417039in}}%
\pgfpathlineto{\pgfqpoint{2.160833in}{1.421868in}}%
\pgfpathlineto{\pgfqpoint{2.190624in}{1.424451in}}%
\pgfpathlineto{\pgfqpoint{2.220416in}{1.424855in}}%
\pgfpathlineto{\pgfqpoint{2.252689in}{1.423148in}}%
\pgfpathlineto{\pgfqpoint{2.292411in}{1.418763in}}%
\pgfpathlineto{\pgfqpoint{2.376820in}{1.406315in}}%
\pgfpathlineto{\pgfqpoint{2.428954in}{1.400079in}}%
\pgfpathlineto{\pgfqpoint{2.473641in}{1.397011in}}%
\pgfpathlineto{\pgfqpoint{2.518328in}{1.396264in}}%
\pgfpathlineto{\pgfqpoint{2.567980in}{1.397741in}}%
\pgfpathlineto{\pgfqpoint{2.644941in}{1.402628in}}%
\pgfpathlineto{\pgfqpoint{2.734314in}{1.407779in}}%
\pgfpathlineto{\pgfqpoint{2.798862in}{1.409168in}}%
\pgfpathlineto{\pgfqpoint{2.870858in}{1.408319in}}%
\pgfpathlineto{\pgfqpoint{3.126566in}{1.402959in}}%
\pgfpathlineto{\pgfqpoint{3.225870in}{1.404042in}}%
\pgfpathlineto{\pgfqpoint{3.225870in}{1.404042in}}%
\pgfusepath{stroke}%
\end{pgfscope}%
\begin{pgfscope}%
\pgfpathrectangle{\pgfqpoint{0.619136in}{0.571603in}}{\pgfqpoint{2.730864in}{1.657828in}}%
\pgfusepath{clip}%
\pgfsetrectcap%
\pgfsetroundjoin%
\pgfsetlinewidth{1.505625pt}%
\definecolor{currentstroke}{rgb}{0.172549,0.627451,0.172549}%
\pgfsetstrokecolor{currentstroke}%
\pgfsetdash{}{0pt}%
\pgfpathmoveto{\pgfqpoint{0.743267in}{0.646959in}}%
\pgfpathlineto{\pgfqpoint{0.745749in}{0.648082in}}%
\pgfpathlineto{\pgfqpoint{0.750714in}{0.654815in}}%
\pgfpathlineto{\pgfqpoint{0.755680in}{0.666320in}}%
\pgfpathlineto{\pgfqpoint{0.763127in}{0.691170in}}%
\pgfpathlineto{\pgfqpoint{0.770575in}{0.723944in}}%
\pgfpathlineto{\pgfqpoint{0.780506in}{0.778159in}}%
\pgfpathlineto{\pgfqpoint{0.792919in}{0.859781in}}%
\pgfpathlineto{\pgfqpoint{0.807814in}{0.972865in}}%
\pgfpathlineto{\pgfqpoint{0.832640in}{1.181493in}}%
\pgfpathlineto{\pgfqpoint{0.864914in}{1.451610in}}%
\pgfpathlineto{\pgfqpoint{0.882292in}{1.580522in}}%
\pgfpathlineto{\pgfqpoint{0.897188in}{1.675768in}}%
\pgfpathlineto{\pgfqpoint{0.909601in}{1.742266in}}%
\pgfpathlineto{\pgfqpoint{0.922014in}{1.795872in}}%
\pgfpathlineto{\pgfqpoint{0.931944in}{1.829013in}}%
\pgfpathlineto{\pgfqpoint{0.939392in}{1.848074in}}%
\pgfpathlineto{\pgfqpoint{0.946840in}{1.862163in}}%
\pgfpathlineto{\pgfqpoint{0.954288in}{1.871328in}}%
\pgfpathlineto{\pgfqpoint{0.959253in}{1.874748in}}%
\pgfpathlineto{\pgfqpoint{0.964218in}{1.876062in}}%
\pgfpathlineto{\pgfqpoint{0.969183in}{1.875316in}}%
\pgfpathlineto{\pgfqpoint{0.974149in}{1.872568in}}%
\pgfpathlineto{\pgfqpoint{0.981596in}{1.864830in}}%
\pgfpathlineto{\pgfqpoint{0.989044in}{1.852973in}}%
\pgfpathlineto{\pgfqpoint{0.996492in}{1.837271in}}%
\pgfpathlineto{\pgfqpoint{1.006423in}{1.810878in}}%
\pgfpathlineto{\pgfqpoint{1.018836in}{1.770224in}}%
\pgfpathlineto{\pgfqpoint{1.033731in}{1.712380in}}%
\pgfpathlineto{\pgfqpoint{1.053592in}{1.624796in}}%
\pgfpathlineto{\pgfqpoint{1.110692in}{1.365928in}}%
\pgfpathlineto{\pgfqpoint{1.128070in}{1.300674in}}%
\pgfpathlineto{\pgfqpoint{1.142966in}{1.253891in}}%
\pgfpathlineto{\pgfqpoint{1.155379in}{1.222253in}}%
\pgfpathlineto{\pgfqpoint{1.165309in}{1.202064in}}%
\pgfpathlineto{\pgfqpoint{1.175240in}{1.186553in}}%
\pgfpathlineto{\pgfqpoint{1.185170in}{1.175742in}}%
\pgfpathlineto{\pgfqpoint{1.192618in}{1.170685in}}%
\pgfpathlineto{\pgfqpoint{1.200066in}{1.168186in}}%
\pgfpathlineto{\pgfqpoint{1.207513in}{1.168165in}}%
\pgfpathlineto{\pgfqpoint{1.214961in}{1.170526in}}%
\pgfpathlineto{\pgfqpoint{1.222409in}{1.175148in}}%
\pgfpathlineto{\pgfqpoint{1.232339in}{1.184593in}}%
\pgfpathlineto{\pgfqpoint{1.242270in}{1.197452in}}%
\pgfpathlineto{\pgfqpoint{1.254683in}{1.217704in}}%
\pgfpathlineto{\pgfqpoint{1.269578in}{1.246997in}}%
\pgfpathlineto{\pgfqpoint{1.289439in}{1.291940in}}%
\pgfpathlineto{\pgfqpoint{1.351504in}{1.437482in}}%
\pgfpathlineto{\pgfqpoint{1.368883in}{1.470442in}}%
\pgfpathlineto{\pgfqpoint{1.383778in}{1.493750in}}%
\pgfpathlineto{\pgfqpoint{1.396191in}{1.509252in}}%
\pgfpathlineto{\pgfqpoint{1.408604in}{1.520983in}}%
\pgfpathlineto{\pgfqpoint{1.418535in}{1.527597in}}%
\pgfpathlineto{\pgfqpoint{1.428465in}{1.531760in}}%
\pgfpathlineto{\pgfqpoint{1.438395in}{1.533525in}}%
\pgfpathlineto{\pgfqpoint{1.448326in}{1.532987in}}%
\pgfpathlineto{\pgfqpoint{1.458256in}{1.530274in}}%
\pgfpathlineto{\pgfqpoint{1.468187in}{1.525547in}}%
\pgfpathlineto{\pgfqpoint{1.480600in}{1.517092in}}%
\pgfpathlineto{\pgfqpoint{1.495495in}{1.503761in}}%
\pgfpathlineto{\pgfqpoint{1.512874in}{1.484816in}}%
\pgfpathlineto{\pgfqpoint{1.537700in}{1.453926in}}%
\pgfpathlineto{\pgfqpoint{1.582386in}{1.397955in}}%
\pgfpathlineto{\pgfqpoint{1.602247in}{1.376942in}}%
\pgfpathlineto{\pgfqpoint{1.619626in}{1.361897in}}%
\pgfpathlineto{\pgfqpoint{1.634521in}{1.351893in}}%
\pgfpathlineto{\pgfqpoint{1.649417in}{1.344751in}}%
\pgfpathlineto{\pgfqpoint{1.661830in}{1.341028in}}%
\pgfpathlineto{\pgfqpoint{1.674243in}{1.339299in}}%
\pgfpathlineto{\pgfqpoint{1.686656in}{1.339481in}}%
\pgfpathlineto{\pgfqpoint{1.701551in}{1.342037in}}%
\pgfpathlineto{\pgfqpoint{1.716447in}{1.346856in}}%
\pgfpathlineto{\pgfqpoint{1.733825in}{1.354853in}}%
\pgfpathlineto{\pgfqpoint{1.756169in}{1.367880in}}%
\pgfpathlineto{\pgfqpoint{1.800856in}{1.397641in}}%
\pgfpathlineto{\pgfqpoint{1.830647in}{1.415848in}}%
\pgfpathlineto{\pgfqpoint{1.852990in}{1.426817in}}%
\pgfpathlineto{\pgfqpoint{1.872851in}{1.434006in}}%
\pgfpathlineto{\pgfqpoint{1.890229in}{1.438111in}}%
\pgfpathlineto{\pgfqpoint{1.907607in}{1.440137in}}%
\pgfpathlineto{\pgfqpoint{1.927468in}{1.440014in}}%
\pgfpathlineto{\pgfqpoint{1.947329in}{1.437556in}}%
\pgfpathlineto{\pgfqpoint{1.969673in}{1.432503in}}%
\pgfpathlineto{\pgfqpoint{1.999464in}{1.423191in}}%
\pgfpathlineto{\pgfqpoint{2.078907in}{1.396770in}}%
\pgfpathlineto{\pgfqpoint{2.106216in}{1.390781in}}%
\pgfpathlineto{\pgfqpoint{2.131042in}{1.387594in}}%
\pgfpathlineto{\pgfqpoint{2.155868in}{1.386622in}}%
\pgfpathlineto{\pgfqpoint{2.183176in}{1.387891in}}%
\pgfpathlineto{\pgfqpoint{2.215450in}{1.391868in}}%
\pgfpathlineto{\pgfqpoint{2.265102in}{1.400695in}}%
\pgfpathlineto{\pgfqpoint{2.319720in}{1.409727in}}%
\pgfpathlineto{\pgfqpoint{2.354476in}{1.413269in}}%
\pgfpathlineto{\pgfqpoint{2.389233in}{1.414555in}}%
\pgfpathlineto{\pgfqpoint{2.426472in}{1.413583in}}%
\pgfpathlineto{\pgfqpoint{2.473641in}{1.409915in}}%
\pgfpathlineto{\pgfqpoint{2.577910in}{1.401001in}}%
\pgfpathlineto{\pgfqpoint{2.625080in}{1.399667in}}%
\pgfpathlineto{\pgfqpoint{2.677215in}{1.400555in}}%
\pgfpathlineto{\pgfqpoint{2.776519in}{1.405263in}}%
\pgfpathlineto{\pgfqpoint{2.846032in}{1.407229in}}%
\pgfpathlineto{\pgfqpoint{2.915544in}{1.406831in}}%
\pgfpathlineto{\pgfqpoint{3.143944in}{1.403354in}}%
\pgfpathlineto{\pgfqpoint{3.225870in}{1.404385in}}%
\pgfpathlineto{\pgfqpoint{3.225870in}{1.404385in}}%
\pgfusepath{stroke}%
\end{pgfscope}%
\begin{pgfscope}%
\pgfpathrectangle{\pgfqpoint{0.619136in}{0.571603in}}{\pgfqpoint{2.730864in}{1.657828in}}%
\pgfusepath{clip}%
\pgfsetrectcap%
\pgfsetroundjoin%
\pgfsetlinewidth{1.505625pt}%
\definecolor{currentstroke}{rgb}{0.839216,0.152941,0.156863}%
\pgfsetstrokecolor{currentstroke}%
\pgfsetdash{}{0pt}%
\pgfpathmoveto{\pgfqpoint{0.743267in}{0.646959in}}%
\pgfpathlineto{\pgfqpoint{0.748232in}{0.696772in}}%
\pgfpathlineto{\pgfqpoint{0.782988in}{1.114002in}}%
\pgfpathlineto{\pgfqpoint{0.795401in}{1.222724in}}%
\pgfpathlineto{\pgfqpoint{0.805332in}{1.291586in}}%
\pgfpathlineto{\pgfqpoint{0.815262in}{1.345877in}}%
\pgfpathlineto{\pgfqpoint{0.825192in}{1.387431in}}%
\pgfpathlineto{\pgfqpoint{0.835123in}{1.418188in}}%
\pgfpathlineto{\pgfqpoint{0.842571in}{1.435303in}}%
\pgfpathlineto{\pgfqpoint{0.850018in}{1.448143in}}%
\pgfpathlineto{\pgfqpoint{0.857466in}{1.457387in}}%
\pgfpathlineto{\pgfqpoint{0.864914in}{1.463646in}}%
\pgfpathlineto{\pgfqpoint{0.872362in}{1.467466in}}%
\pgfpathlineto{\pgfqpoint{0.882292in}{1.469581in}}%
\pgfpathlineto{\pgfqpoint{0.892223in}{1.469159in}}%
\pgfpathlineto{\pgfqpoint{0.904636in}{1.466223in}}%
\pgfpathlineto{\pgfqpoint{0.922014in}{1.459607in}}%
\pgfpathlineto{\pgfqpoint{0.979114in}{1.436116in}}%
\pgfpathlineto{\pgfqpoint{1.003940in}{1.428918in}}%
\pgfpathlineto{\pgfqpoint{1.031249in}{1.423335in}}%
\pgfpathlineto{\pgfqpoint{1.063522in}{1.419103in}}%
\pgfpathlineto{\pgfqpoint{1.105727in}{1.415926in}}%
\pgfpathlineto{\pgfqpoint{1.175240in}{1.413256in}}%
\pgfpathlineto{\pgfqpoint{1.316748in}{1.410480in}}%
\pgfpathlineto{\pgfqpoint{1.552595in}{1.408224in}}%
\pgfpathlineto{\pgfqpoint{1.996981in}{1.406536in}}%
\pgfpathlineto{\pgfqpoint{3.004918in}{1.405361in}}%
\pgfpathlineto{\pgfqpoint{3.225870in}{1.405243in}}%
\pgfpathlineto{\pgfqpoint{3.225870in}{1.405243in}}%
\pgfusepath{stroke}%
\end{pgfscope}%
\begin{pgfscope}%
\pgfpathrectangle{\pgfqpoint{0.619136in}{0.571603in}}{\pgfqpoint{2.730864in}{1.657828in}}%
\pgfusepath{clip}%
\pgfsetrectcap%
\pgfsetroundjoin%
\pgfsetlinewidth{1.505625pt}%
\definecolor{currentstroke}{rgb}{0.580392,0.403922,0.741176}%
\pgfsetstrokecolor{currentstroke}%
\pgfsetdash{}{0pt}%
\pgfpathmoveto{\pgfqpoint{0.743267in}{0.646959in}}%
\pgfpathlineto{\pgfqpoint{0.745749in}{0.654297in}}%
\pgfpathlineto{\pgfqpoint{0.753197in}{0.692004in}}%
\pgfpathlineto{\pgfqpoint{0.765610in}{0.773198in}}%
\pgfpathlineto{\pgfqpoint{0.817745in}{1.133852in}}%
\pgfpathlineto{\pgfqpoint{0.835123in}{1.231656in}}%
\pgfpathlineto{\pgfqpoint{0.850018in}{1.302708in}}%
\pgfpathlineto{\pgfqpoint{0.864914in}{1.361917in}}%
\pgfpathlineto{\pgfqpoint{0.877327in}{1.402586in}}%
\pgfpathlineto{\pgfqpoint{0.889740in}{1.435899in}}%
\pgfpathlineto{\pgfqpoint{0.902153in}{1.462464in}}%
\pgfpathlineto{\pgfqpoint{0.914566in}{1.482955in}}%
\pgfpathlineto{\pgfqpoint{0.924497in}{1.495443in}}%
\pgfpathlineto{\pgfqpoint{0.934427in}{1.504851in}}%
\pgfpathlineto{\pgfqpoint{0.944357in}{1.511528in}}%
\pgfpathlineto{\pgfqpoint{0.954288in}{1.515814in}}%
\pgfpathlineto{\pgfqpoint{0.964218in}{1.518031in}}%
\pgfpathlineto{\pgfqpoint{0.976631in}{1.518351in}}%
\pgfpathlineto{\pgfqpoint{0.989044in}{1.516452in}}%
\pgfpathlineto{\pgfqpoint{1.003940in}{1.511928in}}%
\pgfpathlineto{\pgfqpoint{1.023801in}{1.503262in}}%
\pgfpathlineto{\pgfqpoint{1.053592in}{1.487373in}}%
\pgfpathlineto{\pgfqpoint{1.105727in}{1.459486in}}%
\pgfpathlineto{\pgfqpoint{1.135518in}{1.446292in}}%
\pgfpathlineto{\pgfqpoint{1.162827in}{1.436575in}}%
\pgfpathlineto{\pgfqpoint{1.190135in}{1.429091in}}%
\pgfpathlineto{\pgfqpoint{1.219926in}{1.423178in}}%
\pgfpathlineto{\pgfqpoint{1.254683in}{1.418664in}}%
\pgfpathlineto{\pgfqpoint{1.296887in}{1.415624in}}%
\pgfpathlineto{\pgfqpoint{1.353987in}{1.413931in}}%
\pgfpathlineto{\pgfqpoint{1.497978in}{1.412816in}}%
\pgfpathlineto{\pgfqpoint{2.302341in}{1.407562in}}%
\pgfpathlineto{\pgfqpoint{3.225870in}{1.406033in}}%
\pgfpathlineto{\pgfqpoint{3.225870in}{1.406033in}}%
\pgfusepath{stroke}%
\end{pgfscope}%
\begin{pgfscope}%
\pgfpathrectangle{\pgfqpoint{0.619136in}{0.571603in}}{\pgfqpoint{2.730864in}{1.657828in}}%
\pgfusepath{clip}%
\pgfsetrectcap%
\pgfsetroundjoin%
\pgfsetlinewidth{1.505625pt}%
\definecolor{currentstroke}{rgb}{0.549020,0.337255,0.294118}%
\pgfsetstrokecolor{currentstroke}%
\pgfsetdash{}{0pt}%
\pgfpathmoveto{\pgfqpoint{0.743267in}{0.646959in}}%
\pgfpathlineto{\pgfqpoint{0.745749in}{0.653585in}}%
\pgfpathlineto{\pgfqpoint{0.750714in}{0.678950in}}%
\pgfpathlineto{\pgfqpoint{0.758162in}{0.731942in}}%
\pgfpathlineto{\pgfqpoint{0.770575in}{0.840620in}}%
\pgfpathlineto{\pgfqpoint{0.812779in}{1.230226in}}%
\pgfpathlineto{\pgfqpoint{0.827675in}{1.341651in}}%
\pgfpathlineto{\pgfqpoint{0.840088in}{1.418370in}}%
\pgfpathlineto{\pgfqpoint{0.852501in}{1.479867in}}%
\pgfpathlineto{\pgfqpoint{0.862432in}{1.518351in}}%
\pgfpathlineto{\pgfqpoint{0.872362in}{1.547841in}}%
\pgfpathlineto{\pgfqpoint{0.879810in}{1.564466in}}%
\pgfpathlineto{\pgfqpoint{0.887258in}{1.576757in}}%
\pgfpathlineto{\pgfqpoint{0.894705in}{1.585085in}}%
\pgfpathlineto{\pgfqpoint{0.902153in}{1.589834in}}%
\pgfpathlineto{\pgfqpoint{0.909601in}{1.591400in}}%
\pgfpathlineto{\pgfqpoint{0.917049in}{1.590177in}}%
\pgfpathlineto{\pgfqpoint{0.924497in}{1.586552in}}%
\pgfpathlineto{\pgfqpoint{0.934427in}{1.578627in}}%
\pgfpathlineto{\pgfqpoint{0.946840in}{1.564919in}}%
\pgfpathlineto{\pgfqpoint{0.961736in}{1.544824in}}%
\pgfpathlineto{\pgfqpoint{1.021318in}{1.460418in}}%
\pgfpathlineto{\pgfqpoint{1.038696in}{1.441473in}}%
\pgfpathlineto{\pgfqpoint{1.053592in}{1.428332in}}%
\pgfpathlineto{\pgfqpoint{1.068488in}{1.418053in}}%
\pgfpathlineto{\pgfqpoint{1.083383in}{1.410460in}}%
\pgfpathlineto{\pgfqpoint{1.098279in}{1.405265in}}%
\pgfpathlineto{\pgfqpoint{1.115657in}{1.401761in}}%
\pgfpathlineto{\pgfqpoint{1.135518in}{1.400383in}}%
\pgfpathlineto{\pgfqpoint{1.160344in}{1.401318in}}%
\pgfpathlineto{\pgfqpoint{1.200066in}{1.405614in}}%
\pgfpathlineto{\pgfqpoint{1.254683in}{1.411186in}}%
\pgfpathlineto{\pgfqpoint{1.296887in}{1.413119in}}%
\pgfpathlineto{\pgfqpoint{1.346539in}{1.413036in}}%
\pgfpathlineto{\pgfqpoint{1.664312in}{1.407955in}}%
\pgfpathlineto{\pgfqpoint{2.404128in}{1.405865in}}%
\pgfpathlineto{\pgfqpoint{3.225870in}{1.405168in}}%
\pgfpathlineto{\pgfqpoint{3.225870in}{1.405168in}}%
\pgfusepath{stroke}%
\end{pgfscope}%
\begin{pgfscope}%
\pgfpathrectangle{\pgfqpoint{0.619136in}{0.571603in}}{\pgfqpoint{2.730864in}{1.657828in}}%
\pgfusepath{clip}%
\pgfsetrectcap%
\pgfsetroundjoin%
\pgfsetlinewidth{1.505625pt}%
\definecolor{currentstroke}{rgb}{0.890196,0.466667,0.760784}%
\pgfsetstrokecolor{currentstroke}%
\pgfsetdash{}{0pt}%
\pgfpathmoveto{\pgfqpoint{0.743267in}{0.646959in}}%
\pgfpathlineto{\pgfqpoint{0.745749in}{0.653137in}}%
\pgfpathlineto{\pgfqpoint{0.750714in}{0.674086in}}%
\pgfpathlineto{\pgfqpoint{0.760645in}{0.730231in}}%
\pgfpathlineto{\pgfqpoint{0.778023in}{0.847416in}}%
\pgfpathlineto{\pgfqpoint{0.815262in}{1.102548in}}%
\pgfpathlineto{\pgfqpoint{0.832640in}{1.205998in}}%
\pgfpathlineto{\pgfqpoint{0.847536in}{1.283086in}}%
\pgfpathlineto{\pgfqpoint{0.862432in}{1.348806in}}%
\pgfpathlineto{\pgfqpoint{0.874845in}{1.394913in}}%
\pgfpathlineto{\pgfqpoint{0.887258in}{1.433437in}}%
\pgfpathlineto{\pgfqpoint{0.899671in}{1.464820in}}%
\pgfpathlineto{\pgfqpoint{0.912084in}{1.489613in}}%
\pgfpathlineto{\pgfqpoint{0.922014in}{1.505118in}}%
\pgfpathlineto{\pgfqpoint{0.931944in}{1.517135in}}%
\pgfpathlineto{\pgfqpoint{0.941875in}{1.526005in}}%
\pgfpathlineto{\pgfqpoint{0.951805in}{1.532068in}}%
\pgfpathlineto{\pgfqpoint{0.961736in}{1.535656in}}%
\pgfpathlineto{\pgfqpoint{0.971666in}{1.537088in}}%
\pgfpathlineto{\pgfqpoint{0.984079in}{1.536311in}}%
\pgfpathlineto{\pgfqpoint{0.996492in}{1.533196in}}%
\pgfpathlineto{\pgfqpoint{1.011388in}{1.527080in}}%
\pgfpathlineto{\pgfqpoint{1.031249in}{1.516135in}}%
\pgfpathlineto{\pgfqpoint{1.061040in}{1.496705in}}%
\pgfpathlineto{\pgfqpoint{1.108209in}{1.466077in}}%
\pgfpathlineto{\pgfqpoint{1.135518in}{1.451057in}}%
\pgfpathlineto{\pgfqpoint{1.160344in}{1.439805in}}%
\pgfpathlineto{\pgfqpoint{1.185170in}{1.430898in}}%
\pgfpathlineto{\pgfqpoint{1.212479in}{1.423608in}}%
\pgfpathlineto{\pgfqpoint{1.242270in}{1.418210in}}%
\pgfpathlineto{\pgfqpoint{1.274544in}{1.414706in}}%
\pgfpathlineto{\pgfqpoint{1.314265in}{1.412688in}}%
\pgfpathlineto{\pgfqpoint{1.373848in}{1.412155in}}%
\pgfpathlineto{\pgfqpoint{1.607212in}{1.411906in}}%
\pgfpathlineto{\pgfqpoint{2.021807in}{1.408528in}}%
\pgfpathlineto{\pgfqpoint{2.669767in}{1.406694in}}%
\pgfpathlineto{\pgfqpoint{3.225870in}{1.405985in}}%
\pgfpathlineto{\pgfqpoint{3.225870in}{1.405985in}}%
\pgfusepath{stroke}%
\end{pgfscope}%
\begin{pgfscope}%
\pgfpathrectangle{\pgfqpoint{0.619136in}{0.571603in}}{\pgfqpoint{2.730864in}{1.657828in}}%
\pgfusepath{clip}%
\pgfsetrectcap%
\pgfsetroundjoin%
\pgfsetlinewidth{1.505625pt}%
\definecolor{currentstroke}{rgb}{0.498039,0.498039,0.498039}%
\pgfsetstrokecolor{currentstroke}%
\pgfsetdash{}{0pt}%
\pgfpathmoveto{\pgfqpoint{0.743267in}{0.646959in}}%
\pgfpathlineto{\pgfqpoint{0.748232in}{0.695703in}}%
\pgfpathlineto{\pgfqpoint{0.780506in}{1.068104in}}%
\pgfpathlineto{\pgfqpoint{0.792919in}{1.177401in}}%
\pgfpathlineto{\pgfqpoint{0.802849in}{1.248356in}}%
\pgfpathlineto{\pgfqpoint{0.812779in}{1.305822in}}%
\pgfpathlineto{\pgfqpoint{0.822710in}{1.351302in}}%
\pgfpathlineto{\pgfqpoint{0.832640in}{1.386428in}}%
\pgfpathlineto{\pgfqpoint{0.842571in}{1.412814in}}%
\pgfpathlineto{\pgfqpoint{0.852501in}{1.431974in}}%
\pgfpathlineto{\pgfqpoint{0.862432in}{1.445273in}}%
\pgfpathlineto{\pgfqpoint{0.872362in}{1.453910in}}%
\pgfpathlineto{\pgfqpoint{0.882292in}{1.458914in}}%
\pgfpathlineto{\pgfqpoint{0.892223in}{1.461153in}}%
\pgfpathlineto{\pgfqpoint{0.904636in}{1.461139in}}%
\pgfpathlineto{\pgfqpoint{0.919531in}{1.458423in}}%
\pgfpathlineto{\pgfqpoint{0.941875in}{1.451577in}}%
\pgfpathlineto{\pgfqpoint{0.996492in}{1.433961in}}%
\pgfpathlineto{\pgfqpoint{1.026283in}{1.427068in}}%
\pgfpathlineto{\pgfqpoint{1.058557in}{1.421895in}}%
\pgfpathlineto{\pgfqpoint{1.098279in}{1.417898in}}%
\pgfpathlineto{\pgfqpoint{1.152896in}{1.414829in}}%
\pgfpathlineto{\pgfqpoint{1.244752in}{1.412189in}}%
\pgfpathlineto{\pgfqpoint{1.421017in}{1.409603in}}%
\pgfpathlineto{\pgfqpoint{1.726377in}{1.407576in}}%
\pgfpathlineto{\pgfqpoint{2.342063in}{1.406075in}}%
\pgfpathlineto{\pgfqpoint{3.225870in}{1.405327in}}%
\pgfpathlineto{\pgfqpoint{3.225870in}{1.405327in}}%
\pgfusepath{stroke}%
\end{pgfscope}%
\begin{pgfscope}%
\pgfpathrectangle{\pgfqpoint{0.619136in}{0.571603in}}{\pgfqpoint{2.730864in}{1.657828in}}%
\pgfusepath{clip}%
\pgfsetrectcap%
\pgfsetroundjoin%
\pgfsetlinewidth{1.505625pt}%
\definecolor{currentstroke}{rgb}{0.737255,0.741176,0.133333}%
\pgfsetstrokecolor{currentstroke}%
\pgfsetdash{}{0pt}%
\pgfpathmoveto{\pgfqpoint{0.743267in}{0.646959in}}%
\pgfpathlineto{\pgfqpoint{0.748232in}{0.649380in}}%
\pgfpathlineto{\pgfqpoint{0.753197in}{0.655850in}}%
\pgfpathlineto{\pgfqpoint{0.758162in}{0.665959in}}%
\pgfpathlineto{\pgfqpoint{0.765610in}{0.687439in}}%
\pgfpathlineto{\pgfqpoint{0.773058in}{0.715950in}}%
\pgfpathlineto{\pgfqpoint{0.782988in}{0.763912in}}%
\pgfpathlineto{\pgfqpoint{0.795401in}{0.837962in}}%
\pgfpathlineto{\pgfqpoint{0.810297in}{0.943896in}}%
\pgfpathlineto{\pgfqpoint{0.827675in}{1.084759in}}%
\pgfpathlineto{\pgfqpoint{0.862432in}{1.390763in}}%
\pgfpathlineto{\pgfqpoint{0.887258in}{1.600260in}}%
\pgfpathlineto{\pgfqpoint{0.904636in}{1.729477in}}%
\pgfpathlineto{\pgfqpoint{0.919531in}{1.823636in}}%
\pgfpathlineto{\pgfqpoint{0.931944in}{1.888157in}}%
\pgfpathlineto{\pgfqpoint{0.941875in}{1.929718in}}%
\pgfpathlineto{\pgfqpoint{0.951805in}{1.961835in}}%
\pgfpathlineto{\pgfqpoint{0.959253in}{1.979544in}}%
\pgfpathlineto{\pgfqpoint{0.966701in}{1.991713in}}%
\pgfpathlineto{\pgfqpoint{0.971666in}{1.996742in}}%
\pgfpathlineto{\pgfqpoint{0.976631in}{1.999313in}}%
\pgfpathlineto{\pgfqpoint{0.981596in}{1.999448in}}%
\pgfpathlineto{\pgfqpoint{0.986562in}{1.997176in}}%
\pgfpathlineto{\pgfqpoint{0.991527in}{1.992536in}}%
\pgfpathlineto{\pgfqpoint{0.998975in}{1.981245in}}%
\pgfpathlineto{\pgfqpoint{1.006423in}{1.964940in}}%
\pgfpathlineto{\pgfqpoint{1.016353in}{1.935825in}}%
\pgfpathlineto{\pgfqpoint{1.026283in}{1.898917in}}%
\pgfpathlineto{\pgfqpoint{1.038696in}{1.843039in}}%
\pgfpathlineto{\pgfqpoint{1.053592in}{1.764036in}}%
\pgfpathlineto{\pgfqpoint{1.073453in}{1.643982in}}%
\pgfpathlineto{\pgfqpoint{1.140483in}{1.222295in}}%
\pgfpathlineto{\pgfqpoint{1.155379in}{1.146251in}}%
\pgfpathlineto{\pgfqpoint{1.167792in}{1.092236in}}%
\pgfpathlineto{\pgfqpoint{1.180205in}{1.047900in}}%
\pgfpathlineto{\pgfqpoint{1.190135in}{1.019953in}}%
\pgfpathlineto{\pgfqpoint{1.200066in}{0.999006in}}%
\pgfpathlineto{\pgfqpoint{1.207513in}{0.987988in}}%
\pgfpathlineto{\pgfqpoint{1.214961in}{0.981023in}}%
\pgfpathlineto{\pgfqpoint{1.219926in}{0.978625in}}%
\pgfpathlineto{\pgfqpoint{1.224892in}{0.978008in}}%
\pgfpathlineto{\pgfqpoint{1.229857in}{0.979153in}}%
\pgfpathlineto{\pgfqpoint{1.234822in}{0.982031in}}%
\pgfpathlineto{\pgfqpoint{1.242270in}{0.989522in}}%
\pgfpathlineto{\pgfqpoint{1.249718in}{1.000693in}}%
\pgfpathlineto{\pgfqpoint{1.259648in}{1.021007in}}%
\pgfpathlineto{\pgfqpoint{1.269578in}{1.047059in}}%
\pgfpathlineto{\pgfqpoint{1.281991in}{1.086819in}}%
\pgfpathlineto{\pgfqpoint{1.296887in}{1.143394in}}%
\pgfpathlineto{\pgfqpoint{1.314265in}{1.218529in}}%
\pgfpathlineto{\pgfqpoint{1.346539in}{1.370452in}}%
\pgfpathlineto{\pgfqpoint{1.373848in}{1.494981in}}%
\pgfpathlineto{\pgfqpoint{1.391226in}{1.565038in}}%
\pgfpathlineto{\pgfqpoint{1.406122in}{1.616487in}}%
\pgfpathlineto{\pgfqpoint{1.418535in}{1.652082in}}%
\pgfpathlineto{\pgfqpoint{1.428465in}{1.675282in}}%
\pgfpathlineto{\pgfqpoint{1.438395in}{1.693509in}}%
\pgfpathlineto{\pgfqpoint{1.448326in}{1.706594in}}%
\pgfpathlineto{\pgfqpoint{1.455774in}{1.712980in}}%
\pgfpathlineto{\pgfqpoint{1.463222in}{1.716424in}}%
\pgfpathlineto{\pgfqpoint{1.470669in}{1.716950in}}%
\pgfpathlineto{\pgfqpoint{1.478117in}{1.714610in}}%
\pgfpathlineto{\pgfqpoint{1.485565in}{1.709479in}}%
\pgfpathlineto{\pgfqpoint{1.493013in}{1.701658in}}%
\pgfpathlineto{\pgfqpoint{1.502943in}{1.687263in}}%
\pgfpathlineto{\pgfqpoint{1.512874in}{1.668657in}}%
\pgfpathlineto{\pgfqpoint{1.525287in}{1.640107in}}%
\pgfpathlineto{\pgfqpoint{1.540182in}{1.599302in}}%
\pgfpathlineto{\pgfqpoint{1.557560in}{1.544902in}}%
\pgfpathlineto{\pgfqpoint{1.589834in}{1.434406in}}%
\pgfpathlineto{\pgfqpoint{1.617143in}{1.343383in}}%
\pgfpathlineto{\pgfqpoint{1.634521in}{1.291958in}}%
\pgfpathlineto{\pgfqpoint{1.649417in}{1.254040in}}%
\pgfpathlineto{\pgfqpoint{1.661830in}{1.227686in}}%
\pgfpathlineto{\pgfqpoint{1.674243in}{1.206654in}}%
\pgfpathlineto{\pgfqpoint{1.684173in}{1.193908in}}%
\pgfpathlineto{\pgfqpoint{1.694104in}{1.184916in}}%
\pgfpathlineto{\pgfqpoint{1.701551in}{1.180668in}}%
\pgfpathlineto{\pgfqpoint{1.708999in}{1.178559in}}%
\pgfpathlineto{\pgfqpoint{1.716447in}{1.178567in}}%
\pgfpathlineto{\pgfqpoint{1.723895in}{1.180650in}}%
\pgfpathlineto{\pgfqpoint{1.731343in}{1.184750in}}%
\pgfpathlineto{\pgfqpoint{1.741273in}{1.193220in}}%
\pgfpathlineto{\pgfqpoint{1.751203in}{1.204912in}}%
\pgfpathlineto{\pgfqpoint{1.763617in}{1.223638in}}%
\pgfpathlineto{\pgfqpoint{1.776030in}{1.246350in}}%
\pgfpathlineto{\pgfqpoint{1.793408in}{1.283494in}}%
\pgfpathlineto{\pgfqpoint{1.815751in}{1.337284in}}%
\pgfpathlineto{\pgfqpoint{1.865403in}{1.458624in}}%
\pgfpathlineto{\pgfqpoint{1.882781in}{1.494830in}}%
\pgfpathlineto{\pgfqpoint{1.897677in}{1.521118in}}%
\pgfpathlineto{\pgfqpoint{1.910090in}{1.539067in}}%
\pgfpathlineto{\pgfqpoint{1.922503in}{1.553049in}}%
\pgfpathlineto{\pgfqpoint{1.932434in}{1.561224in}}%
\pgfpathlineto{\pgfqpoint{1.942364in}{1.566649in}}%
\pgfpathlineto{\pgfqpoint{1.952294in}{1.569306in}}%
\pgfpathlineto{\pgfqpoint{1.962225in}{1.569222in}}%
\pgfpathlineto{\pgfqpoint{1.972155in}{1.566470in}}%
\pgfpathlineto{\pgfqpoint{1.982086in}{1.561162in}}%
\pgfpathlineto{\pgfqpoint{1.992016in}{1.553448in}}%
\pgfpathlineto{\pgfqpoint{2.004429in}{1.540712in}}%
\pgfpathlineto{\pgfqpoint{2.019325in}{1.521470in}}%
\pgfpathlineto{\pgfqpoint{2.036703in}{1.494664in}}%
\pgfpathlineto{\pgfqpoint{2.059046in}{1.455677in}}%
\pgfpathlineto{\pgfqpoint{2.111181in}{1.363158in}}%
\pgfpathlineto{\pgfqpoint{2.128559in}{1.337136in}}%
\pgfpathlineto{\pgfqpoint{2.143455in}{1.318352in}}%
\pgfpathlineto{\pgfqpoint{2.155868in}{1.305616in}}%
\pgfpathlineto{\pgfqpoint{2.168281in}{1.295787in}}%
\pgfpathlineto{\pgfqpoint{2.180694in}{1.289022in}}%
\pgfpathlineto{\pgfqpoint{2.190624in}{1.285868in}}%
\pgfpathlineto{\pgfqpoint{2.200555in}{1.284723in}}%
\pgfpathlineto{\pgfqpoint{2.210485in}{1.285555in}}%
\pgfpathlineto{\pgfqpoint{2.220416in}{1.288298in}}%
\pgfpathlineto{\pgfqpoint{2.232829in}{1.294272in}}%
\pgfpathlineto{\pgfqpoint{2.245242in}{1.302843in}}%
\pgfpathlineto{\pgfqpoint{2.260137in}{1.316134in}}%
\pgfpathlineto{\pgfqpoint{2.277515in}{1.335005in}}%
\pgfpathlineto{\pgfqpoint{2.299859in}{1.362898in}}%
\pgfpathlineto{\pgfqpoint{2.359441in}{1.439286in}}%
\pgfpathlineto{\pgfqpoint{2.376820in}{1.457527in}}%
\pgfpathlineto{\pgfqpoint{2.391715in}{1.470473in}}%
\pgfpathlineto{\pgfqpoint{2.406611in}{1.480536in}}%
\pgfpathlineto{\pgfqpoint{2.419024in}{1.486533in}}%
\pgfpathlineto{\pgfqpoint{2.431437in}{1.490266in}}%
\pgfpathlineto{\pgfqpoint{2.443850in}{1.491714in}}%
\pgfpathlineto{\pgfqpoint{2.456263in}{1.490912in}}%
\pgfpathlineto{\pgfqpoint{2.468676in}{1.487953in}}%
\pgfpathlineto{\pgfqpoint{2.483571in}{1.481759in}}%
\pgfpathlineto{\pgfqpoint{2.498467in}{1.473002in}}%
\pgfpathlineto{\pgfqpoint{2.515845in}{1.460102in}}%
\pgfpathlineto{\pgfqpoint{2.538189in}{1.440461in}}%
\pgfpathlineto{\pgfqpoint{2.612667in}{1.372155in}}%
\pgfpathlineto{\pgfqpoint{2.630045in}{1.360091in}}%
\pgfpathlineto{\pgfqpoint{2.647423in}{1.350742in}}%
\pgfpathlineto{\pgfqpoint{2.662319in}{1.345151in}}%
\pgfpathlineto{\pgfqpoint{2.677215in}{1.341923in}}%
\pgfpathlineto{\pgfqpoint{2.692110in}{1.341085in}}%
\pgfpathlineto{\pgfqpoint{2.707006in}{1.342577in}}%
\pgfpathlineto{\pgfqpoint{2.721901in}{1.346256in}}%
\pgfpathlineto{\pgfqpoint{2.739280in}{1.353018in}}%
\pgfpathlineto{\pgfqpoint{2.759140in}{1.363444in}}%
\pgfpathlineto{\pgfqpoint{2.783966in}{1.379315in}}%
\pgfpathlineto{\pgfqpoint{2.855962in}{1.427407in}}%
\pgfpathlineto{\pgfqpoint{2.875823in}{1.437396in}}%
\pgfpathlineto{\pgfqpoint{2.893201in}{1.444011in}}%
\pgfpathlineto{\pgfqpoint{2.910579in}{1.448390in}}%
\pgfpathlineto{\pgfqpoint{2.927957in}{1.450416in}}%
\pgfpathlineto{\pgfqpoint{2.945336in}{1.450094in}}%
\pgfpathlineto{\pgfqpoint{2.962714in}{1.447544in}}%
\pgfpathlineto{\pgfqpoint{2.982575in}{1.442193in}}%
\pgfpathlineto{\pgfqpoint{3.004918in}{1.433641in}}%
\pgfpathlineto{\pgfqpoint{3.034709in}{1.419498in}}%
\pgfpathlineto{\pgfqpoint{3.096774in}{1.389178in}}%
\pgfpathlineto{\pgfqpoint{3.121600in}{1.379993in}}%
\pgfpathlineto{\pgfqpoint{3.143944in}{1.374268in}}%
\pgfpathlineto{\pgfqpoint{3.163805in}{1.371483in}}%
\pgfpathlineto{\pgfqpoint{3.183666in}{1.370941in}}%
\pgfpathlineto{\pgfqpoint{3.203526in}{1.372565in}}%
\pgfpathlineto{\pgfqpoint{3.225870in}{1.376709in}}%
\pgfpathlineto{\pgfqpoint{3.225870in}{1.376709in}}%
\pgfusepath{stroke}%
\end{pgfscope}%
\begin{pgfscope}%
\pgfpathrectangle{\pgfqpoint{0.619136in}{0.571603in}}{\pgfqpoint{2.730864in}{1.657828in}}%
\pgfusepath{clip}%
\pgfsetrectcap%
\pgfsetroundjoin%
\pgfsetlinewidth{1.505625pt}%
\definecolor{currentstroke}{rgb}{0.090196,0.745098,0.811765}%
\pgfsetstrokecolor{currentstroke}%
\pgfsetdash{}{0pt}%
\pgfpathmoveto{\pgfqpoint{0.743267in}{0.646959in}}%
\pgfpathlineto{\pgfqpoint{0.748232in}{0.676074in}}%
\pgfpathlineto{\pgfqpoint{0.760645in}{0.774340in}}%
\pgfpathlineto{\pgfqpoint{0.787953in}{0.994337in}}%
\pgfpathlineto{\pgfqpoint{0.802849in}{1.097955in}}%
\pgfpathlineto{\pgfqpoint{0.817745in}{1.186421in}}%
\pgfpathlineto{\pgfqpoint{0.830158in}{1.248533in}}%
\pgfpathlineto{\pgfqpoint{0.842571in}{1.300658in}}%
\pgfpathlineto{\pgfqpoint{0.854984in}{1.343612in}}%
\pgfpathlineto{\pgfqpoint{0.867397in}{1.378341in}}%
\pgfpathlineto{\pgfqpoint{0.879810in}{1.405839in}}%
\pgfpathlineto{\pgfqpoint{0.892223in}{1.427084in}}%
\pgfpathlineto{\pgfqpoint{0.902153in}{1.440208in}}%
\pgfpathlineto{\pgfqpoint{0.912084in}{1.450374in}}%
\pgfpathlineto{\pgfqpoint{0.922014in}{1.457995in}}%
\pgfpathlineto{\pgfqpoint{0.934427in}{1.464513in}}%
\pgfpathlineto{\pgfqpoint{0.946840in}{1.468298in}}%
\pgfpathlineto{\pgfqpoint{0.959253in}{1.469923in}}%
\pgfpathlineto{\pgfqpoint{0.974149in}{1.469703in}}%
\pgfpathlineto{\pgfqpoint{0.994009in}{1.466851in}}%
\pgfpathlineto{\pgfqpoint{1.021318in}{1.460235in}}%
\pgfpathlineto{\pgfqpoint{1.115657in}{1.435399in}}%
\pgfpathlineto{\pgfqpoint{1.155379in}{1.428232in}}%
\pgfpathlineto{\pgfqpoint{1.197583in}{1.422911in}}%
\pgfpathlineto{\pgfqpoint{1.249718in}{1.418734in}}%
\pgfpathlineto{\pgfqpoint{1.319231in}{1.415608in}}%
\pgfpathlineto{\pgfqpoint{1.433430in}{1.413004in}}%
\pgfpathlineto{\pgfqpoint{1.666795in}{1.410245in}}%
\pgfpathlineto{\pgfqpoint{2.054081in}{1.408049in}}%
\pgfpathlineto{\pgfqpoint{2.796379in}{1.406408in}}%
\pgfpathlineto{\pgfqpoint{3.225870in}{1.405957in}}%
\pgfpathlineto{\pgfqpoint{3.225870in}{1.405957in}}%
\pgfusepath{stroke}%
\end{pgfscope}%
\begin{pgfscope}%
\pgfpathrectangle{\pgfqpoint{0.619136in}{0.571603in}}{\pgfqpoint{2.730864in}{1.657828in}}%
\pgfusepath{clip}%
\pgfsetrectcap%
\pgfsetroundjoin%
\pgfsetlinewidth{1.505625pt}%
\definecolor{currentstroke}{rgb}{0.121569,0.466667,0.705882}%
\pgfsetstrokecolor{currentstroke}%
\pgfsetdash{}{0pt}%
\pgfpathmoveto{\pgfqpoint{0.743267in}{0.646959in}}%
\pgfpathlineto{\pgfqpoint{0.745749in}{0.656126in}}%
\pgfpathlineto{\pgfqpoint{0.750714in}{0.687645in}}%
\pgfpathlineto{\pgfqpoint{0.760645in}{0.771633in}}%
\pgfpathlineto{\pgfqpoint{0.810297in}{1.228120in}}%
\pgfpathlineto{\pgfqpoint{0.822710in}{1.315012in}}%
\pgfpathlineto{\pgfqpoint{0.835123in}{1.386461in}}%
\pgfpathlineto{\pgfqpoint{0.845053in}{1.432680in}}%
\pgfpathlineto{\pgfqpoint{0.854984in}{1.469702in}}%
\pgfpathlineto{\pgfqpoint{0.864914in}{1.498239in}}%
\pgfpathlineto{\pgfqpoint{0.874845in}{1.519129in}}%
\pgfpathlineto{\pgfqpoint{0.882292in}{1.530325in}}%
\pgfpathlineto{\pgfqpoint{0.889740in}{1.538126in}}%
\pgfpathlineto{\pgfqpoint{0.897188in}{1.542923in}}%
\pgfpathlineto{\pgfqpoint{0.904636in}{1.545099in}}%
\pgfpathlineto{\pgfqpoint{0.912084in}{1.545024in}}%
\pgfpathlineto{\pgfqpoint{0.919531in}{1.543045in}}%
\pgfpathlineto{\pgfqpoint{0.929462in}{1.538006in}}%
\pgfpathlineto{\pgfqpoint{0.941875in}{1.528824in}}%
\pgfpathlineto{\pgfqpoint{0.959253in}{1.512672in}}%
\pgfpathlineto{\pgfqpoint{1.011388in}{1.461685in}}%
\pgfpathlineto{\pgfqpoint{1.031249in}{1.446241in}}%
\pgfpathlineto{\pgfqpoint{1.048627in}{1.435384in}}%
\pgfpathlineto{\pgfqpoint{1.066005in}{1.426966in}}%
\pgfpathlineto{\pgfqpoint{1.085866in}{1.420040in}}%
\pgfpathlineto{\pgfqpoint{1.105727in}{1.415516in}}%
\pgfpathlineto{\pgfqpoint{1.128070in}{1.412624in}}%
\pgfpathlineto{\pgfqpoint{1.157861in}{1.411180in}}%
\pgfpathlineto{\pgfqpoint{1.205031in}{1.411526in}}%
\pgfpathlineto{\pgfqpoint{1.309300in}{1.412535in}}%
\pgfpathlineto{\pgfqpoint{1.445843in}{1.410519in}}%
\pgfpathlineto{\pgfqpoint{1.649417in}{1.408498in}}%
\pgfpathlineto{\pgfqpoint{2.116146in}{1.406652in}}%
\pgfpathlineto{\pgfqpoint{3.101740in}{1.405404in}}%
\pgfpathlineto{\pgfqpoint{3.225870in}{1.405328in}}%
\pgfpathlineto{\pgfqpoint{3.225870in}{1.405328in}}%
\pgfusepath{stroke}%
\end{pgfscope}%
\begin{pgfscope}%
\pgfpathrectangle{\pgfqpoint{0.619136in}{0.571603in}}{\pgfqpoint{2.730864in}{1.657828in}}%
\pgfusepath{clip}%
\pgfsetrectcap%
\pgfsetroundjoin%
\pgfsetlinewidth{1.505625pt}%
\definecolor{currentstroke}{rgb}{1.000000,0.498039,0.054902}%
\pgfsetstrokecolor{currentstroke}%
\pgfsetdash{}{0pt}%
\pgfpathmoveto{\pgfqpoint{0.743267in}{0.646959in}}%
\pgfpathlineto{\pgfqpoint{0.748232in}{0.649012in}}%
\pgfpathlineto{\pgfqpoint{0.753197in}{0.654340in}}%
\pgfpathlineto{\pgfqpoint{0.758162in}{0.662542in}}%
\pgfpathlineto{\pgfqpoint{0.765610in}{0.679778in}}%
\pgfpathlineto{\pgfqpoint{0.773058in}{0.702473in}}%
\pgfpathlineto{\pgfqpoint{0.782988in}{0.740460in}}%
\pgfpathlineto{\pgfqpoint{0.795401in}{0.798991in}}%
\pgfpathlineto{\pgfqpoint{0.810297in}{0.882960in}}%
\pgfpathlineto{\pgfqpoint{0.827675in}{0.995712in}}%
\pgfpathlineto{\pgfqpoint{0.852501in}{1.174345in}}%
\pgfpathlineto{\pgfqpoint{0.904636in}{1.554424in}}%
\pgfpathlineto{\pgfqpoint{0.924497in}{1.680339in}}%
\pgfpathlineto{\pgfqpoint{0.939392in}{1.762393in}}%
\pgfpathlineto{\pgfqpoint{0.954288in}{1.831803in}}%
\pgfpathlineto{\pgfqpoint{0.966701in}{1.878983in}}%
\pgfpathlineto{\pgfqpoint{0.976631in}{1.909328in}}%
\pgfpathlineto{\pgfqpoint{0.986562in}{1.932886in}}%
\pgfpathlineto{\pgfqpoint{0.994009in}{1.946036in}}%
\pgfpathlineto{\pgfqpoint{1.001457in}{1.955299in}}%
\pgfpathlineto{\pgfqpoint{1.008905in}{1.960690in}}%
\pgfpathlineto{\pgfqpoint{1.013870in}{1.962154in}}%
\pgfpathlineto{\pgfqpoint{1.018836in}{1.961934in}}%
\pgfpathlineto{\pgfqpoint{1.023801in}{1.960054in}}%
\pgfpathlineto{\pgfqpoint{1.031249in}{1.954186in}}%
\pgfpathlineto{\pgfqpoint{1.038696in}{1.944763in}}%
\pgfpathlineto{\pgfqpoint{1.046144in}{1.931922in}}%
\pgfpathlineto{\pgfqpoint{1.056075in}{1.909759in}}%
\pgfpathlineto{\pgfqpoint{1.066005in}{1.882227in}}%
\pgfpathlineto{\pgfqpoint{1.078418in}{1.840992in}}%
\pgfpathlineto{\pgfqpoint{1.093314in}{1.782882in}}%
\pgfpathlineto{\pgfqpoint{1.110692in}{1.705665in}}%
\pgfpathlineto{\pgfqpoint{1.135518in}{1.584044in}}%
\pgfpathlineto{\pgfqpoint{1.187653in}{1.325512in}}%
\pgfpathlineto{\pgfqpoint{1.207513in}{1.239470in}}%
\pgfpathlineto{\pgfqpoint{1.224892in}{1.174532in}}%
\pgfpathlineto{\pgfqpoint{1.239787in}{1.128082in}}%
\pgfpathlineto{\pgfqpoint{1.252200in}{1.096564in}}%
\pgfpathlineto{\pgfqpoint{1.262131in}{1.076324in}}%
\pgfpathlineto{\pgfqpoint{1.272061in}{1.060636in}}%
\pgfpathlineto{\pgfqpoint{1.281991in}{1.049562in}}%
\pgfpathlineto{\pgfqpoint{1.289439in}{1.044287in}}%
\pgfpathlineto{\pgfqpoint{1.296887in}{1.041590in}}%
\pgfpathlineto{\pgfqpoint{1.304335in}{1.041435in}}%
\pgfpathlineto{\pgfqpoint{1.311783in}{1.043769in}}%
\pgfpathlineto{\pgfqpoint{1.319231in}{1.048522in}}%
\pgfpathlineto{\pgfqpoint{1.326678in}{1.055612in}}%
\pgfpathlineto{\pgfqpoint{1.336609in}{1.068528in}}%
\pgfpathlineto{\pgfqpoint{1.346539in}{1.085154in}}%
\pgfpathlineto{\pgfqpoint{1.358952in}{1.110686in}}%
\pgfpathlineto{\pgfqpoint{1.373848in}{1.147404in}}%
\pgfpathlineto{\pgfqpoint{1.391226in}{1.197009in}}%
\pgfpathlineto{\pgfqpoint{1.413569in}{1.268142in}}%
\pgfpathlineto{\pgfqpoint{1.478117in}{1.478808in}}%
\pgfpathlineto{\pgfqpoint{1.497978in}{1.533563in}}%
\pgfpathlineto{\pgfqpoint{1.512874in}{1.568789in}}%
\pgfpathlineto{\pgfqpoint{1.527769in}{1.598177in}}%
\pgfpathlineto{\pgfqpoint{1.540182in}{1.617806in}}%
\pgfpathlineto{\pgfqpoint{1.550113in}{1.630169in}}%
\pgfpathlineto{\pgfqpoint{1.560043in}{1.639487in}}%
\pgfpathlineto{\pgfqpoint{1.569973in}{1.645734in}}%
\pgfpathlineto{\pgfqpoint{1.579904in}{1.648919in}}%
\pgfpathlineto{\pgfqpoint{1.589834in}{1.649092in}}%
\pgfpathlineto{\pgfqpoint{1.599765in}{1.646332in}}%
\pgfpathlineto{\pgfqpoint{1.609695in}{1.640755in}}%
\pgfpathlineto{\pgfqpoint{1.619626in}{1.632507in}}%
\pgfpathlineto{\pgfqpoint{1.632039in}{1.618708in}}%
\pgfpathlineto{\pgfqpoint{1.644452in}{1.601402in}}%
\pgfpathlineto{\pgfqpoint{1.659347in}{1.576636in}}%
\pgfpathlineto{\pgfqpoint{1.676725in}{1.543309in}}%
\pgfpathlineto{\pgfqpoint{1.701551in}{1.490180in}}%
\pgfpathlineto{\pgfqpoint{1.761134in}{1.360250in}}%
\pgfpathlineto{\pgfqpoint{1.780995in}{1.323285in}}%
\pgfpathlineto{\pgfqpoint{1.798373in}{1.295757in}}%
\pgfpathlineto{\pgfqpoint{1.813269in}{1.276372in}}%
\pgfpathlineto{\pgfqpoint{1.825682in}{1.263468in}}%
\pgfpathlineto{\pgfqpoint{1.838095in}{1.253669in}}%
\pgfpathlineto{\pgfqpoint{1.850508in}{1.247048in}}%
\pgfpathlineto{\pgfqpoint{1.860438in}{1.244051in}}%
\pgfpathlineto{\pgfqpoint{1.870368in}{1.243077in}}%
\pgfpathlineto{\pgfqpoint{1.880299in}{1.244081in}}%
\pgfpathlineto{\pgfqpoint{1.890229in}{1.246994in}}%
\pgfpathlineto{\pgfqpoint{1.902642in}{1.253184in}}%
\pgfpathlineto{\pgfqpoint{1.915055in}{1.262003in}}%
\pgfpathlineto{\pgfqpoint{1.929951in}{1.275691in}}%
\pgfpathlineto{\pgfqpoint{1.947329in}{1.295290in}}%
\pgfpathlineto{\pgfqpoint{1.967190in}{1.321360in}}%
\pgfpathlineto{\pgfqpoint{1.996981in}{1.364722in}}%
\pgfpathlineto{\pgfqpoint{2.041668in}{1.429534in}}%
\pgfpathlineto{\pgfqpoint{2.064011in}{1.457703in}}%
\pgfpathlineto{\pgfqpoint{2.081390in}{1.476297in}}%
\pgfpathlineto{\pgfqpoint{2.096285in}{1.489466in}}%
\pgfpathlineto{\pgfqpoint{2.111181in}{1.499818in}}%
\pgfpathlineto{\pgfqpoint{2.126077in}{1.507190in}}%
\pgfpathlineto{\pgfqpoint{2.138490in}{1.511003in}}%
\pgfpathlineto{\pgfqpoint{2.150903in}{1.512696in}}%
\pgfpathlineto{\pgfqpoint{2.163316in}{1.512312in}}%
\pgfpathlineto{\pgfqpoint{2.175729in}{1.509931in}}%
\pgfpathlineto{\pgfqpoint{2.190624in}{1.504599in}}%
\pgfpathlineto{\pgfqpoint{2.205520in}{1.496813in}}%
\pgfpathlineto{\pgfqpoint{2.222898in}{1.485045in}}%
\pgfpathlineto{\pgfqpoint{2.242759in}{1.468757in}}%
\pgfpathlineto{\pgfqpoint{2.270068in}{1.443095in}}%
\pgfpathlineto{\pgfqpoint{2.332133in}{1.383306in}}%
\pgfpathlineto{\pgfqpoint{2.354476in}{1.365294in}}%
\pgfpathlineto{\pgfqpoint{2.374337in}{1.352173in}}%
\pgfpathlineto{\pgfqpoint{2.391715in}{1.343314in}}%
\pgfpathlineto{\pgfqpoint{2.409093in}{1.337120in}}%
\pgfpathlineto{\pgfqpoint{2.423989in}{1.334007in}}%
\pgfpathlineto{\pgfqpoint{2.438885in}{1.332922in}}%
\pgfpathlineto{\pgfqpoint{2.453780in}{1.333810in}}%
\pgfpathlineto{\pgfqpoint{2.471158in}{1.337194in}}%
\pgfpathlineto{\pgfqpoint{2.488537in}{1.342870in}}%
\pgfpathlineto{\pgfqpoint{2.508397in}{1.351753in}}%
\pgfpathlineto{\pgfqpoint{2.533224in}{1.365616in}}%
\pgfpathlineto{\pgfqpoint{2.570463in}{1.389589in}}%
\pgfpathlineto{\pgfqpoint{2.615149in}{1.418001in}}%
\pgfpathlineto{\pgfqpoint{2.639975in}{1.431335in}}%
\pgfpathlineto{\pgfqpoint{2.662319in}{1.440901in}}%
\pgfpathlineto{\pgfqpoint{2.682180in}{1.447102in}}%
\pgfpathlineto{\pgfqpoint{2.702041in}{1.450951in}}%
\pgfpathlineto{\pgfqpoint{2.721901in}{1.452397in}}%
\pgfpathlineto{\pgfqpoint{2.741762in}{1.451502in}}%
\pgfpathlineto{\pgfqpoint{2.761623in}{1.448440in}}%
\pgfpathlineto{\pgfqpoint{2.783966in}{1.442734in}}%
\pgfpathlineto{\pgfqpoint{2.811275in}{1.433210in}}%
\pgfpathlineto{\pgfqpoint{2.848514in}{1.417532in}}%
\pgfpathlineto{\pgfqpoint{2.905614in}{1.393465in}}%
\pgfpathlineto{\pgfqpoint{2.935405in}{1.383514in}}%
\pgfpathlineto{\pgfqpoint{2.960231in}{1.377497in}}%
\pgfpathlineto{\pgfqpoint{2.985057in}{1.373873in}}%
\pgfpathlineto{\pgfqpoint{3.007401in}{1.372746in}}%
\pgfpathlineto{\pgfqpoint{3.032227in}{1.373791in}}%
\pgfpathlineto{\pgfqpoint{3.057053in}{1.377017in}}%
\pgfpathlineto{\pgfqpoint{3.086844in}{1.383213in}}%
\pgfpathlineto{\pgfqpoint{3.126566in}{1.393983in}}%
\pgfpathlineto{\pgfqpoint{3.201044in}{1.414499in}}%
\pgfpathlineto{\pgfqpoint{3.225870in}{1.419589in}}%
\pgfpathlineto{\pgfqpoint{3.225870in}{1.419589in}}%
\pgfusepath{stroke}%
\end{pgfscope}%
\begin{pgfscope}%
\pgfpathrectangle{\pgfqpoint{0.619136in}{0.571603in}}{\pgfqpoint{2.730864in}{1.657828in}}%
\pgfusepath{clip}%
\pgfsetrectcap%
\pgfsetroundjoin%
\pgfsetlinewidth{1.505625pt}%
\definecolor{currentstroke}{rgb}{0.172549,0.627451,0.172549}%
\pgfsetstrokecolor{currentstroke}%
\pgfsetdash{}{0pt}%
\pgfpathmoveto{\pgfqpoint{0.743267in}{0.646959in}}%
\pgfpathlineto{\pgfqpoint{0.755680in}{0.846437in}}%
\pgfpathlineto{\pgfqpoint{0.765610in}{0.971182in}}%
\pgfpathlineto{\pgfqpoint{0.775540in}{1.069468in}}%
\pgfpathlineto{\pgfqpoint{0.785471in}{1.146448in}}%
\pgfpathlineto{\pgfqpoint{0.795401in}{1.206502in}}%
\pgfpathlineto{\pgfqpoint{0.805332in}{1.253204in}}%
\pgfpathlineto{\pgfqpoint{0.815262in}{1.289421in}}%
\pgfpathlineto{\pgfqpoint{0.825192in}{1.317434in}}%
\pgfpathlineto{\pgfqpoint{0.835123in}{1.339045in}}%
\pgfpathlineto{\pgfqpoint{0.845053in}{1.355673in}}%
\pgfpathlineto{\pgfqpoint{0.854984in}{1.368430in}}%
\pgfpathlineto{\pgfqpoint{0.864914in}{1.378187in}}%
\pgfpathlineto{\pgfqpoint{0.877327in}{1.387186in}}%
\pgfpathlineto{\pgfqpoint{0.889740in}{1.393555in}}%
\pgfpathlineto{\pgfqpoint{0.904636in}{1.398747in}}%
\pgfpathlineto{\pgfqpoint{0.922014in}{1.402524in}}%
\pgfpathlineto{\pgfqpoint{0.946840in}{1.405355in}}%
\pgfpathlineto{\pgfqpoint{0.981596in}{1.406778in}}%
\pgfpathlineto{\pgfqpoint{1.051109in}{1.406829in}}%
\pgfpathlineto{\pgfqpoint{1.435913in}{1.405238in}}%
\pgfpathlineto{\pgfqpoint{2.510880in}{1.404622in}}%
\pgfpathlineto{\pgfqpoint{3.225870in}{1.404520in}}%
\pgfpathlineto{\pgfqpoint{3.225870in}{1.404520in}}%
\pgfusepath{stroke}%
\end{pgfscope}%
\begin{pgfscope}%
\pgfpathrectangle{\pgfqpoint{0.619136in}{0.571603in}}{\pgfqpoint{2.730864in}{1.657828in}}%
\pgfusepath{clip}%
\pgfsetrectcap%
\pgfsetroundjoin%
\pgfsetlinewidth{1.505625pt}%
\definecolor{currentstroke}{rgb}{0.839216,0.152941,0.156863}%
\pgfsetstrokecolor{currentstroke}%
\pgfsetdash{}{0pt}%
\pgfpathmoveto{\pgfqpoint{0.743267in}{0.646959in}}%
\pgfpathlineto{\pgfqpoint{0.750714in}{0.722944in}}%
\pgfpathlineto{\pgfqpoint{0.773058in}{0.963875in}}%
\pgfpathlineto{\pgfqpoint{0.785471in}{1.074523in}}%
\pgfpathlineto{\pgfqpoint{0.797884in}{1.165833in}}%
\pgfpathlineto{\pgfqpoint{0.810297in}{1.239203in}}%
\pgfpathlineto{\pgfqpoint{0.820227in}{1.286453in}}%
\pgfpathlineto{\pgfqpoint{0.830158in}{1.324868in}}%
\pgfpathlineto{\pgfqpoint{0.840088in}{1.355663in}}%
\pgfpathlineto{\pgfqpoint{0.850018in}{1.379979in}}%
\pgfpathlineto{\pgfqpoint{0.859949in}{1.398857in}}%
\pgfpathlineto{\pgfqpoint{0.869879in}{1.413223in}}%
\pgfpathlineto{\pgfqpoint{0.879810in}{1.423887in}}%
\pgfpathlineto{\pgfqpoint{0.889740in}{1.431549in}}%
\pgfpathlineto{\pgfqpoint{0.899671in}{1.436807in}}%
\pgfpathlineto{\pgfqpoint{0.912084in}{1.440756in}}%
\pgfpathlineto{\pgfqpoint{0.926979in}{1.442682in}}%
\pgfpathlineto{\pgfqpoint{0.944357in}{1.442362in}}%
\pgfpathlineto{\pgfqpoint{0.969183in}{1.439283in}}%
\pgfpathlineto{\pgfqpoint{1.085866in}{1.421544in}}%
\pgfpathlineto{\pgfqpoint{1.138000in}{1.417323in}}%
\pgfpathlineto{\pgfqpoint{1.207513in}{1.414082in}}%
\pgfpathlineto{\pgfqpoint{1.316748in}{1.411451in}}%
\pgfpathlineto{\pgfqpoint{1.517839in}{1.409107in}}%
\pgfpathlineto{\pgfqpoint{1.895194in}{1.407216in}}%
\pgfpathlineto{\pgfqpoint{2.692110in}{1.405837in}}%
\pgfpathlineto{\pgfqpoint{3.225870in}{1.405449in}}%
\pgfpathlineto{\pgfqpoint{3.225870in}{1.405449in}}%
\pgfusepath{stroke}%
\end{pgfscope}%
\begin{pgfscope}%
\pgfpathrectangle{\pgfqpoint{0.619136in}{0.571603in}}{\pgfqpoint{2.730864in}{1.657828in}}%
\pgfusepath{clip}%
\pgfsetrectcap%
\pgfsetroundjoin%
\pgfsetlinewidth{1.505625pt}%
\definecolor{currentstroke}{rgb}{0.580392,0.403922,0.741176}%
\pgfsetstrokecolor{currentstroke}%
\pgfsetdash{}{0pt}%
\pgfpathmoveto{\pgfqpoint{0.743267in}{0.646959in}}%
\pgfpathlineto{\pgfqpoint{0.748232in}{0.648907in}}%
\pgfpathlineto{\pgfqpoint{0.753197in}{0.654034in}}%
\pgfpathlineto{\pgfqpoint{0.758162in}{0.661981in}}%
\pgfpathlineto{\pgfqpoint{0.765610in}{0.678782in}}%
\pgfpathlineto{\pgfqpoint{0.773058in}{0.701017in}}%
\pgfpathlineto{\pgfqpoint{0.782988in}{0.738404in}}%
\pgfpathlineto{\pgfqpoint{0.795401in}{0.796284in}}%
\pgfpathlineto{\pgfqpoint{0.807814in}{0.864876in}}%
\pgfpathlineto{\pgfqpoint{0.825192in}{0.975395in}}%
\pgfpathlineto{\pgfqpoint{0.847536in}{1.134681in}}%
\pgfpathlineto{\pgfqpoint{0.912084in}{1.607178in}}%
\pgfpathlineto{\pgfqpoint{0.929462in}{1.715172in}}%
\pgfpathlineto{\pgfqpoint{0.944357in}{1.795472in}}%
\pgfpathlineto{\pgfqpoint{0.956770in}{1.852361in}}%
\pgfpathlineto{\pgfqpoint{0.969183in}{1.899275in}}%
\pgfpathlineto{\pgfqpoint{0.979114in}{1.929225in}}%
\pgfpathlineto{\pgfqpoint{0.989044in}{1.952216in}}%
\pgfpathlineto{\pgfqpoint{0.996492in}{1.964825in}}%
\pgfpathlineto{\pgfqpoint{1.003940in}{1.973443in}}%
\pgfpathlineto{\pgfqpoint{1.011388in}{1.978086in}}%
\pgfpathlineto{\pgfqpoint{1.016353in}{1.978993in}}%
\pgfpathlineto{\pgfqpoint{1.021318in}{1.978169in}}%
\pgfpathlineto{\pgfqpoint{1.026283in}{1.975640in}}%
\pgfpathlineto{\pgfqpoint{1.033731in}{1.968711in}}%
\pgfpathlineto{\pgfqpoint{1.041179in}{1.958126in}}%
\pgfpathlineto{\pgfqpoint{1.048627in}{1.944025in}}%
\pgfpathlineto{\pgfqpoint{1.058557in}{1.920035in}}%
\pgfpathlineto{\pgfqpoint{1.068488in}{1.890519in}}%
\pgfpathlineto{\pgfqpoint{1.080901in}{1.846606in}}%
\pgfpathlineto{\pgfqpoint{1.095796in}{1.785036in}}%
\pgfpathlineto{\pgfqpoint{1.113174in}{1.703534in}}%
\pgfpathlineto{\pgfqpoint{1.138000in}{1.575577in}}%
\pgfpathlineto{\pgfqpoint{1.187653in}{1.316717in}}%
\pgfpathlineto{\pgfqpoint{1.207513in}{1.225554in}}%
\pgfpathlineto{\pgfqpoint{1.224892in}{1.156509in}}%
\pgfpathlineto{\pgfqpoint{1.239787in}{1.106965in}}%
\pgfpathlineto{\pgfqpoint{1.252200in}{1.073251in}}%
\pgfpathlineto{\pgfqpoint{1.262131in}{1.051543in}}%
\pgfpathlineto{\pgfqpoint{1.272061in}{1.034672in}}%
\pgfpathlineto{\pgfqpoint{1.281991in}{1.022716in}}%
\pgfpathlineto{\pgfqpoint{1.289439in}{1.016988in}}%
\pgfpathlineto{\pgfqpoint{1.296887in}{1.014024in}}%
\pgfpathlineto{\pgfqpoint{1.304335in}{1.013789in}}%
\pgfpathlineto{\pgfqpoint{1.311783in}{1.016233in}}%
\pgfpathlineto{\pgfqpoint{1.319231in}{1.021288in}}%
\pgfpathlineto{\pgfqpoint{1.326678in}{1.028869in}}%
\pgfpathlineto{\pgfqpoint{1.336609in}{1.042733in}}%
\pgfpathlineto{\pgfqpoint{1.346539in}{1.060631in}}%
\pgfpathlineto{\pgfqpoint{1.358952in}{1.088190in}}%
\pgfpathlineto{\pgfqpoint{1.373848in}{1.127930in}}%
\pgfpathlineto{\pgfqpoint{1.391226in}{1.181769in}}%
\pgfpathlineto{\pgfqpoint{1.413569in}{1.259221in}}%
\pgfpathlineto{\pgfqpoint{1.480600in}{1.498013in}}%
\pgfpathlineto{\pgfqpoint{1.497978in}{1.550330in}}%
\pgfpathlineto{\pgfqpoint{1.512874in}{1.589209in}}%
\pgfpathlineto{\pgfqpoint{1.527769in}{1.621688in}}%
\pgfpathlineto{\pgfqpoint{1.540182in}{1.643403in}}%
\pgfpathlineto{\pgfqpoint{1.550113in}{1.657081in}}%
\pgfpathlineto{\pgfqpoint{1.560043in}{1.667385in}}%
\pgfpathlineto{\pgfqpoint{1.569973in}{1.674275in}}%
\pgfpathlineto{\pgfqpoint{1.577421in}{1.677202in}}%
\pgfpathlineto{\pgfqpoint{1.584869in}{1.678227in}}%
\pgfpathlineto{\pgfqpoint{1.592317in}{1.677378in}}%
\pgfpathlineto{\pgfqpoint{1.599765in}{1.674696in}}%
\pgfpathlineto{\pgfqpoint{1.609695in}{1.668363in}}%
\pgfpathlineto{\pgfqpoint{1.619626in}{1.659024in}}%
\pgfpathlineto{\pgfqpoint{1.629556in}{1.646867in}}%
\pgfpathlineto{\pgfqpoint{1.641969in}{1.628042in}}%
\pgfpathlineto{\pgfqpoint{1.656865in}{1.600771in}}%
\pgfpathlineto{\pgfqpoint{1.674243in}{1.563683in}}%
\pgfpathlineto{\pgfqpoint{1.696586in}{1.510140in}}%
\pgfpathlineto{\pgfqpoint{1.766099in}{1.338505in}}%
\pgfpathlineto{\pgfqpoint{1.783477in}{1.302537in}}%
\pgfpathlineto{\pgfqpoint{1.798373in}{1.275908in}}%
\pgfpathlineto{\pgfqpoint{1.813269in}{1.253760in}}%
\pgfpathlineto{\pgfqpoint{1.825682in}{1.239040in}}%
\pgfpathlineto{\pgfqpoint{1.838095in}{1.227900in}}%
\pgfpathlineto{\pgfqpoint{1.848025in}{1.221632in}}%
\pgfpathlineto{\pgfqpoint{1.857955in}{1.217733in}}%
\pgfpathlineto{\pgfqpoint{1.867886in}{1.216190in}}%
\pgfpathlineto{\pgfqpoint{1.877816in}{1.216963in}}%
\pgfpathlineto{\pgfqpoint{1.887747in}{1.219983in}}%
\pgfpathlineto{\pgfqpoint{1.897677in}{1.225156in}}%
\pgfpathlineto{\pgfqpoint{1.910090in}{1.234471in}}%
\pgfpathlineto{\pgfqpoint{1.922503in}{1.246692in}}%
\pgfpathlineto{\pgfqpoint{1.937399in}{1.264738in}}%
\pgfpathlineto{\pgfqpoint{1.954777in}{1.289653in}}%
\pgfpathlineto{\pgfqpoint{1.977120in}{1.326102in}}%
\pgfpathlineto{\pgfqpoint{2.056564in}{1.460555in}}%
\pgfpathlineto{\pgfqpoint{2.073942in}{1.484196in}}%
\pgfpathlineto{\pgfqpoint{2.088838in}{1.501355in}}%
\pgfpathlineto{\pgfqpoint{2.103733in}{1.515271in}}%
\pgfpathlineto{\pgfqpoint{2.116146in}{1.524205in}}%
\pgfpathlineto{\pgfqpoint{2.128559in}{1.530619in}}%
\pgfpathlineto{\pgfqpoint{2.140972in}{1.534472in}}%
\pgfpathlineto{\pgfqpoint{2.153385in}{1.535772in}}%
\pgfpathlineto{\pgfqpoint{2.165798in}{1.534575in}}%
\pgfpathlineto{\pgfqpoint{2.178211in}{1.530982in}}%
\pgfpathlineto{\pgfqpoint{2.190624in}{1.525136in}}%
\pgfpathlineto{\pgfqpoint{2.205520in}{1.515403in}}%
\pgfpathlineto{\pgfqpoint{2.220416in}{1.503060in}}%
\pgfpathlineto{\pgfqpoint{2.237794in}{1.485940in}}%
\pgfpathlineto{\pgfqpoint{2.260137in}{1.460793in}}%
\pgfpathlineto{\pgfqpoint{2.344546in}{1.362344in}}%
\pgfpathlineto{\pgfqpoint{2.364406in}{1.344456in}}%
\pgfpathlineto{\pgfqpoint{2.381785in}{1.331864in}}%
\pgfpathlineto{\pgfqpoint{2.396680in}{1.323615in}}%
\pgfpathlineto{\pgfqpoint{2.411576in}{1.317859in}}%
\pgfpathlineto{\pgfqpoint{2.426472in}{1.314655in}}%
\pgfpathlineto{\pgfqpoint{2.441367in}{1.313994in}}%
\pgfpathlineto{\pgfqpoint{2.456263in}{1.315798in}}%
\pgfpathlineto{\pgfqpoint{2.471158in}{1.319925in}}%
\pgfpathlineto{\pgfqpoint{2.488537in}{1.327407in}}%
\pgfpathlineto{\pgfqpoint{2.505915in}{1.337379in}}%
\pgfpathlineto{\pgfqpoint{2.528258in}{1.353089in}}%
\pgfpathlineto{\pgfqpoint{2.558050in}{1.377255in}}%
\pgfpathlineto{\pgfqpoint{2.615149in}{1.424263in}}%
\pgfpathlineto{\pgfqpoint{2.637493in}{1.439741in}}%
\pgfpathlineto{\pgfqpoint{2.657354in}{1.451027in}}%
\pgfpathlineto{\pgfqpoint{2.674732in}{1.458644in}}%
\pgfpathlineto{\pgfqpoint{2.692110in}{1.463956in}}%
\pgfpathlineto{\pgfqpoint{2.709488in}{1.466869in}}%
\pgfpathlineto{\pgfqpoint{2.726867in}{1.467383in}}%
\pgfpathlineto{\pgfqpoint{2.744245in}{1.465583in}}%
\pgfpathlineto{\pgfqpoint{2.761623in}{1.461637in}}%
\pgfpathlineto{\pgfqpoint{2.781484in}{1.454807in}}%
\pgfpathlineto{\pgfqpoint{2.803827in}{1.444708in}}%
\pgfpathlineto{\pgfqpoint{2.833619in}{1.428528in}}%
\pgfpathlineto{\pgfqpoint{2.910579in}{1.385283in}}%
\pgfpathlineto{\pgfqpoint{2.935405in}{1.374571in}}%
\pgfpathlineto{\pgfqpoint{2.957749in}{1.367368in}}%
\pgfpathlineto{\pgfqpoint{2.977610in}{1.363163in}}%
\pgfpathlineto{\pgfqpoint{2.997470in}{1.361131in}}%
\pgfpathlineto{\pgfqpoint{3.017331in}{1.361258in}}%
\pgfpathlineto{\pgfqpoint{3.039675in}{1.363837in}}%
\pgfpathlineto{\pgfqpoint{3.062018in}{1.368703in}}%
\pgfpathlineto{\pgfqpoint{3.089327in}{1.377137in}}%
\pgfpathlineto{\pgfqpoint{3.126566in}{1.391416in}}%
\pgfpathlineto{\pgfqpoint{3.191113in}{1.416599in}}%
\pgfpathlineto{\pgfqpoint{3.220905in}{1.425589in}}%
\pgfpathlineto{\pgfqpoint{3.225870in}{1.426823in}}%
\pgfpathlineto{\pgfqpoint{3.225870in}{1.426823in}}%
\pgfusepath{stroke}%
\end{pgfscope}%
\begin{pgfscope}%
\pgfpathrectangle{\pgfqpoint{0.619136in}{0.571603in}}{\pgfqpoint{2.730864in}{1.657828in}}%
\pgfusepath{clip}%
\pgfsetrectcap%
\pgfsetroundjoin%
\pgfsetlinewidth{1.505625pt}%
\definecolor{currentstroke}{rgb}{0.549020,0.337255,0.294118}%
\pgfsetstrokecolor{currentstroke}%
\pgfsetdash{}{0pt}%
\pgfpathmoveto{\pgfqpoint{0.743267in}{0.646959in}}%
\pgfpathlineto{\pgfqpoint{0.758162in}{0.843342in}}%
\pgfpathlineto{\pgfqpoint{0.770575in}{0.973926in}}%
\pgfpathlineto{\pgfqpoint{0.780506in}{1.058110in}}%
\pgfpathlineto{\pgfqpoint{0.790436in}{1.127112in}}%
\pgfpathlineto{\pgfqpoint{0.800366in}{1.183395in}}%
\pgfpathlineto{\pgfqpoint{0.810297in}{1.229124in}}%
\pgfpathlineto{\pgfqpoint{0.820227in}{1.266150in}}%
\pgfpathlineto{\pgfqpoint{0.830158in}{1.296034in}}%
\pgfpathlineto{\pgfqpoint{0.840088in}{1.320078in}}%
\pgfpathlineto{\pgfqpoint{0.850018in}{1.339363in}}%
\pgfpathlineto{\pgfqpoint{0.862432in}{1.358120in}}%
\pgfpathlineto{\pgfqpoint{0.874845in}{1.372223in}}%
\pgfpathlineto{\pgfqpoint{0.887258in}{1.382765in}}%
\pgfpathlineto{\pgfqpoint{0.899671in}{1.390589in}}%
\pgfpathlineto{\pgfqpoint{0.914566in}{1.397301in}}%
\pgfpathlineto{\pgfqpoint{0.931944in}{1.402485in}}%
\pgfpathlineto{\pgfqpoint{0.951805in}{1.406064in}}%
\pgfpathlineto{\pgfqpoint{0.979114in}{1.408507in}}%
\pgfpathlineto{\pgfqpoint{1.018836in}{1.409499in}}%
\pgfpathlineto{\pgfqpoint{1.103244in}{1.408706in}}%
\pgfpathlineto{\pgfqpoint{1.331644in}{1.406596in}}%
\pgfpathlineto{\pgfqpoint{1.776030in}{1.405458in}}%
\pgfpathlineto{\pgfqpoint{3.186148in}{1.404739in}}%
\pgfpathlineto{\pgfqpoint{3.225870in}{1.404731in}}%
\pgfpathlineto{\pgfqpoint{3.225870in}{1.404731in}}%
\pgfusepath{stroke}%
\end{pgfscope}%
\begin{pgfscope}%
\pgfpathrectangle{\pgfqpoint{0.619136in}{0.571603in}}{\pgfqpoint{2.730864in}{1.657828in}}%
\pgfusepath{clip}%
\pgfsetrectcap%
\pgfsetroundjoin%
\pgfsetlinewidth{1.505625pt}%
\definecolor{currentstroke}{rgb}{0.890196,0.466667,0.760784}%
\pgfsetstrokecolor{currentstroke}%
\pgfsetdash{}{0pt}%
\pgfpathmoveto{\pgfqpoint{0.743267in}{0.646959in}}%
\pgfpathlineto{\pgfqpoint{0.745749in}{0.652394in}}%
\pgfpathlineto{\pgfqpoint{0.750714in}{0.674554in}}%
\pgfpathlineto{\pgfqpoint{0.758162in}{0.722791in}}%
\pgfpathlineto{\pgfqpoint{0.768093in}{0.803463in}}%
\pgfpathlineto{\pgfqpoint{0.787953in}{0.989180in}}%
\pgfpathlineto{\pgfqpoint{0.812779in}{1.217959in}}%
\pgfpathlineto{\pgfqpoint{0.827675in}{1.336608in}}%
\pgfpathlineto{\pgfqpoint{0.840088in}{1.420343in}}%
\pgfpathlineto{\pgfqpoint{0.852501in}{1.489010in}}%
\pgfpathlineto{\pgfqpoint{0.862432in}{1.532936in}}%
\pgfpathlineto{\pgfqpoint{0.872362in}{1.567334in}}%
\pgfpathlineto{\pgfqpoint{0.882292in}{1.592703in}}%
\pgfpathlineto{\pgfqpoint{0.889740in}{1.606211in}}%
\pgfpathlineto{\pgfqpoint{0.897188in}{1.615362in}}%
\pgfpathlineto{\pgfqpoint{0.904636in}{1.620527in}}%
\pgfpathlineto{\pgfqpoint{0.912084in}{1.622095in}}%
\pgfpathlineto{\pgfqpoint{0.919531in}{1.620469in}}%
\pgfpathlineto{\pgfqpoint{0.926979in}{1.616053in}}%
\pgfpathlineto{\pgfqpoint{0.934427in}{1.609246in}}%
\pgfpathlineto{\pgfqpoint{0.944357in}{1.597121in}}%
\pgfpathlineto{\pgfqpoint{0.956770in}{1.578292in}}%
\pgfpathlineto{\pgfqpoint{0.976631in}{1.543279in}}%
\pgfpathlineto{\pgfqpoint{1.013870in}{1.476852in}}%
\pgfpathlineto{\pgfqpoint{1.031249in}{1.450537in}}%
\pgfpathlineto{\pgfqpoint{1.046144in}{1.431662in}}%
\pgfpathlineto{\pgfqpoint{1.061040in}{1.416469in}}%
\pgfpathlineto{\pgfqpoint{1.073453in}{1.406616in}}%
\pgfpathlineto{\pgfqpoint{1.085866in}{1.399187in}}%
\pgfpathlineto{\pgfqpoint{1.098279in}{1.393984in}}%
\pgfpathlineto{\pgfqpoint{1.113174in}{1.390322in}}%
\pgfpathlineto{\pgfqpoint{1.128070in}{1.389010in}}%
\pgfpathlineto{\pgfqpoint{1.145448in}{1.389769in}}%
\pgfpathlineto{\pgfqpoint{1.167792in}{1.393172in}}%
\pgfpathlineto{\pgfqpoint{1.214961in}{1.403557in}}%
\pgfpathlineto{\pgfqpoint{1.249718in}{1.410024in}}%
\pgfpathlineto{\pgfqpoint{1.281991in}{1.413669in}}%
\pgfpathlineto{\pgfqpoint{1.314265in}{1.415037in}}%
\pgfpathlineto{\pgfqpoint{1.356470in}{1.414358in}}%
\pgfpathlineto{\pgfqpoint{1.555078in}{1.407778in}}%
\pgfpathlineto{\pgfqpoint{2.051598in}{1.406302in}}%
\pgfpathlineto{\pgfqpoint{3.106705in}{1.405124in}}%
\pgfpathlineto{\pgfqpoint{3.225870in}{1.405065in}}%
\pgfpathlineto{\pgfqpoint{3.225870in}{1.405065in}}%
\pgfusepath{stroke}%
\end{pgfscope}%
\begin{pgfscope}%
\pgfpathrectangle{\pgfqpoint{0.619136in}{0.571603in}}{\pgfqpoint{2.730864in}{1.657828in}}%
\pgfusepath{clip}%
\pgfsetrectcap%
\pgfsetroundjoin%
\pgfsetlinewidth{1.505625pt}%
\definecolor{currentstroke}{rgb}{0.498039,0.498039,0.498039}%
\pgfsetstrokecolor{currentstroke}%
\pgfsetdash{}{0pt}%
\pgfpathmoveto{\pgfqpoint{0.743267in}{0.646959in}}%
\pgfpathlineto{\pgfqpoint{0.748232in}{0.649862in}}%
\pgfpathlineto{\pgfqpoint{0.753197in}{0.656705in}}%
\pgfpathlineto{\pgfqpoint{0.760645in}{0.672792in}}%
\pgfpathlineto{\pgfqpoint{0.768093in}{0.694869in}}%
\pgfpathlineto{\pgfqpoint{0.778023in}{0.732313in}}%
\pgfpathlineto{\pgfqpoint{0.790436in}{0.789996in}}%
\pgfpathlineto{\pgfqpoint{0.805332in}{0.871997in}}%
\pgfpathlineto{\pgfqpoint{0.825192in}{0.996797in}}%
\pgfpathlineto{\pgfqpoint{0.859949in}{1.234637in}}%
\pgfpathlineto{\pgfqpoint{0.889740in}{1.433020in}}%
\pgfpathlineto{\pgfqpoint{0.909601in}{1.551676in}}%
\pgfpathlineto{\pgfqpoint{0.926979in}{1.642238in}}%
\pgfpathlineto{\pgfqpoint{0.941875in}{1.708160in}}%
\pgfpathlineto{\pgfqpoint{0.954288in}{1.754051in}}%
\pgfpathlineto{\pgfqpoint{0.966701in}{1.791308in}}%
\pgfpathlineto{\pgfqpoint{0.976631in}{1.814761in}}%
\pgfpathlineto{\pgfqpoint{0.986562in}{1.832542in}}%
\pgfpathlineto{\pgfqpoint{0.994009in}{1.842183in}}%
\pgfpathlineto{\pgfqpoint{1.001457in}{1.848704in}}%
\pgfpathlineto{\pgfqpoint{1.008905in}{1.852171in}}%
\pgfpathlineto{\pgfqpoint{1.016353in}{1.852664in}}%
\pgfpathlineto{\pgfqpoint{1.023801in}{1.850284in}}%
\pgfpathlineto{\pgfqpoint{1.031249in}{1.845145in}}%
\pgfpathlineto{\pgfqpoint{1.038696in}{1.837376in}}%
\pgfpathlineto{\pgfqpoint{1.048627in}{1.823173in}}%
\pgfpathlineto{\pgfqpoint{1.058557in}{1.804917in}}%
\pgfpathlineto{\pgfqpoint{1.070970in}{1.777005in}}%
\pgfpathlineto{\pgfqpoint{1.085866in}{1.737142in}}%
\pgfpathlineto{\pgfqpoint{1.103244in}{1.683766in}}%
\pgfpathlineto{\pgfqpoint{1.128070in}{1.599341in}}%
\pgfpathlineto{\pgfqpoint{1.182687in}{1.410944in}}%
\pgfpathlineto{\pgfqpoint{1.202548in}{1.351014in}}%
\pgfpathlineto{\pgfqpoint{1.219926in}{1.305319in}}%
\pgfpathlineto{\pgfqpoint{1.234822in}{1.272030in}}%
\pgfpathlineto{\pgfqpoint{1.247235in}{1.248805in}}%
\pgfpathlineto{\pgfqpoint{1.259648in}{1.229873in}}%
\pgfpathlineto{\pgfqpoint{1.272061in}{1.215321in}}%
\pgfpathlineto{\pgfqpoint{1.281991in}{1.206835in}}%
\pgfpathlineto{\pgfqpoint{1.291922in}{1.201119in}}%
\pgfpathlineto{\pgfqpoint{1.301852in}{1.198105in}}%
\pgfpathlineto{\pgfqpoint{1.311783in}{1.197699in}}%
\pgfpathlineto{\pgfqpoint{1.321713in}{1.199780in}}%
\pgfpathlineto{\pgfqpoint{1.331644in}{1.204204in}}%
\pgfpathlineto{\pgfqpoint{1.341574in}{1.210807in}}%
\pgfpathlineto{\pgfqpoint{1.353987in}{1.221848in}}%
\pgfpathlineto{\pgfqpoint{1.368883in}{1.238669in}}%
\pgfpathlineto{\pgfqpoint{1.386261in}{1.262309in}}%
\pgfpathlineto{\pgfqpoint{1.408604in}{1.297222in}}%
\pgfpathlineto{\pgfqpoint{1.490530in}{1.430769in}}%
\pgfpathlineto{\pgfqpoint{1.510391in}{1.456699in}}%
\pgfpathlineto{\pgfqpoint{1.527769in}{1.475609in}}%
\pgfpathlineto{\pgfqpoint{1.542665in}{1.488704in}}%
\pgfpathlineto{\pgfqpoint{1.557560in}{1.498773in}}%
\pgfpathlineto{\pgfqpoint{1.569973in}{1.504814in}}%
\pgfpathlineto{\pgfqpoint{1.582386in}{1.508734in}}%
\pgfpathlineto{\pgfqpoint{1.594799in}{1.510584in}}%
\pgfpathlineto{\pgfqpoint{1.607212in}{1.510453in}}%
\pgfpathlineto{\pgfqpoint{1.622108in}{1.507847in}}%
\pgfpathlineto{\pgfqpoint{1.637004in}{1.502813in}}%
\pgfpathlineto{\pgfqpoint{1.654382in}{1.494272in}}%
\pgfpathlineto{\pgfqpoint{1.674243in}{1.481648in}}%
\pgfpathlineto{\pgfqpoint{1.699069in}{1.462831in}}%
\pgfpathlineto{\pgfqpoint{1.790925in}{1.389907in}}%
\pgfpathlineto{\pgfqpoint{1.813269in}{1.376605in}}%
\pgfpathlineto{\pgfqpoint{1.833129in}{1.367343in}}%
\pgfpathlineto{\pgfqpoint{1.852990in}{1.360697in}}%
\pgfpathlineto{\pgfqpoint{1.870368in}{1.357077in}}%
\pgfpathlineto{\pgfqpoint{1.887747in}{1.355464in}}%
\pgfpathlineto{\pgfqpoint{1.907607in}{1.355935in}}%
\pgfpathlineto{\pgfqpoint{1.927468in}{1.358631in}}%
\pgfpathlineto{\pgfqpoint{1.949812in}{1.363910in}}%
\pgfpathlineto{\pgfqpoint{1.977120in}{1.372806in}}%
\pgfpathlineto{\pgfqpoint{2.019325in}{1.389417in}}%
\pgfpathlineto{\pgfqpoint{2.071459in}{1.409595in}}%
\pgfpathlineto{\pgfqpoint{2.101251in}{1.418837in}}%
\pgfpathlineto{\pgfqpoint{2.128559in}{1.425061in}}%
\pgfpathlineto{\pgfqpoint{2.155868in}{1.428866in}}%
\pgfpathlineto{\pgfqpoint{2.183176in}{1.430232in}}%
\pgfpathlineto{\pgfqpoint{2.210485in}{1.429347in}}%
\pgfpathlineto{\pgfqpoint{2.242759in}{1.425887in}}%
\pgfpathlineto{\pgfqpoint{2.282481in}{1.419161in}}%
\pgfpathlineto{\pgfqpoint{2.394198in}{1.398735in}}%
\pgfpathlineto{\pgfqpoint{2.431437in}{1.394826in}}%
\pgfpathlineto{\pgfqpoint{2.468676in}{1.393135in}}%
\pgfpathlineto{\pgfqpoint{2.508397in}{1.393638in}}%
\pgfpathlineto{\pgfqpoint{2.555567in}{1.396608in}}%
\pgfpathlineto{\pgfqpoint{2.734314in}{1.410372in}}%
\pgfpathlineto{\pgfqpoint{2.788932in}{1.410752in}}%
\pgfpathlineto{\pgfqpoint{2.853479in}{1.408841in}}%
\pgfpathlineto{\pgfqpoint{3.034709in}{1.402093in}}%
\pgfpathlineto{\pgfqpoint{3.114153in}{1.402298in}}%
\pgfpathlineto{\pgfqpoint{3.225870in}{1.404638in}}%
\pgfpathlineto{\pgfqpoint{3.225870in}{1.404638in}}%
\pgfusepath{stroke}%
\end{pgfscope}%
\begin{pgfscope}%
\pgfpathrectangle{\pgfqpoint{0.619136in}{0.571603in}}{\pgfqpoint{2.730864in}{1.657828in}}%
\pgfusepath{clip}%
\pgfsetrectcap%
\pgfsetroundjoin%
\pgfsetlinewidth{1.505625pt}%
\definecolor{currentstroke}{rgb}{0.737255,0.741176,0.133333}%
\pgfsetstrokecolor{currentstroke}%
\pgfsetdash{}{0pt}%
\pgfpathmoveto{\pgfqpoint{0.743267in}{0.646959in}}%
\pgfpathlineto{\pgfqpoint{0.768093in}{0.914489in}}%
\pgfpathlineto{\pgfqpoint{0.780506in}{1.019088in}}%
\pgfpathlineto{\pgfqpoint{0.792919in}{1.104444in}}%
\pgfpathlineto{\pgfqpoint{0.805332in}{1.173296in}}%
\pgfpathlineto{\pgfqpoint{0.817745in}{1.228352in}}%
\pgfpathlineto{\pgfqpoint{0.830158in}{1.272046in}}%
\pgfpathlineto{\pgfqpoint{0.842571in}{1.306482in}}%
\pgfpathlineto{\pgfqpoint{0.854984in}{1.333435in}}%
\pgfpathlineto{\pgfqpoint{0.867397in}{1.354384in}}%
\pgfpathlineto{\pgfqpoint{0.879810in}{1.370543in}}%
\pgfpathlineto{\pgfqpoint{0.892223in}{1.382904in}}%
\pgfpathlineto{\pgfqpoint{0.904636in}{1.392268in}}%
\pgfpathlineto{\pgfqpoint{0.919531in}{1.400453in}}%
\pgfpathlineto{\pgfqpoint{0.934427in}{1.406124in}}%
\pgfpathlineto{\pgfqpoint{0.951805in}{1.410443in}}%
\pgfpathlineto{\pgfqpoint{0.974149in}{1.413527in}}%
\pgfpathlineto{\pgfqpoint{1.003940in}{1.415050in}}%
\pgfpathlineto{\pgfqpoint{1.048627in}{1.414755in}}%
\pgfpathlineto{\pgfqpoint{1.366400in}{1.408334in}}%
\pgfpathlineto{\pgfqpoint{1.679208in}{1.406702in}}%
\pgfpathlineto{\pgfqpoint{2.419024in}{1.405514in}}%
\pgfpathlineto{\pgfqpoint{3.225870in}{1.405078in}}%
\pgfpathlineto{\pgfqpoint{3.225870in}{1.405078in}}%
\pgfusepath{stroke}%
\end{pgfscope}%
\begin{pgfscope}%
\pgfpathrectangle{\pgfqpoint{0.619136in}{0.571603in}}{\pgfqpoint{2.730864in}{1.657828in}}%
\pgfusepath{clip}%
\pgfsetrectcap%
\pgfsetroundjoin%
\pgfsetlinewidth{1.505625pt}%
\definecolor{currentstroke}{rgb}{0.090196,0.745098,0.811765}%
\pgfsetstrokecolor{currentstroke}%
\pgfsetdash{}{0pt}%
\pgfpathmoveto{\pgfqpoint{0.743267in}{0.646959in}}%
\pgfpathlineto{\pgfqpoint{0.763127in}{0.951741in}}%
\pgfpathlineto{\pgfqpoint{0.773058in}{1.066704in}}%
\pgfpathlineto{\pgfqpoint{0.782988in}{1.157063in}}%
\pgfpathlineto{\pgfqpoint{0.792919in}{1.226896in}}%
\pgfpathlineto{\pgfqpoint{0.802849in}{1.280138in}}%
\pgfpathlineto{\pgfqpoint{0.812779in}{1.320228in}}%
\pgfpathlineto{\pgfqpoint{0.822710in}{1.350043in}}%
\pgfpathlineto{\pgfqpoint{0.832640in}{1.371926in}}%
\pgfpathlineto{\pgfqpoint{0.842571in}{1.387749in}}%
\pgfpathlineto{\pgfqpoint{0.852501in}{1.398990in}}%
\pgfpathlineto{\pgfqpoint{0.862432in}{1.406801in}}%
\pgfpathlineto{\pgfqpoint{0.872362in}{1.412070in}}%
\pgfpathlineto{\pgfqpoint{0.884775in}{1.416103in}}%
\pgfpathlineto{\pgfqpoint{0.899671in}{1.418444in}}%
\pgfpathlineto{\pgfqpoint{0.919531in}{1.419107in}}%
\pgfpathlineto{\pgfqpoint{0.954288in}{1.417476in}}%
\pgfpathlineto{\pgfqpoint{1.056075in}{1.411768in}}%
\pgfpathlineto{\pgfqpoint{1.150413in}{1.409345in}}%
\pgfpathlineto{\pgfqpoint{1.324196in}{1.407470in}}%
\pgfpathlineto{\pgfqpoint{1.699069in}{1.406045in}}%
\pgfpathlineto{\pgfqpoint{2.674732in}{1.405079in}}%
\pgfpathlineto{\pgfqpoint{3.225870in}{1.404886in}}%
\pgfpathlineto{\pgfqpoint{3.225870in}{1.404886in}}%
\pgfusepath{stroke}%
\end{pgfscope}%
\begin{pgfscope}%
\pgfpathrectangle{\pgfqpoint{0.619136in}{0.571603in}}{\pgfqpoint{2.730864in}{1.657828in}}%
\pgfusepath{clip}%
\pgfsetrectcap%
\pgfsetroundjoin%
\pgfsetlinewidth{1.505625pt}%
\definecolor{currentstroke}{rgb}{0.121569,0.466667,0.705882}%
\pgfsetstrokecolor{currentstroke}%
\pgfsetdash{}{0pt}%
\pgfpathmoveto{\pgfqpoint{0.743267in}{0.646959in}}%
\pgfpathlineto{\pgfqpoint{0.748232in}{0.675011in}}%
\pgfpathlineto{\pgfqpoint{0.758162in}{0.756292in}}%
\pgfpathlineto{\pgfqpoint{0.795401in}{1.081486in}}%
\pgfpathlineto{\pgfqpoint{0.810297in}{1.188326in}}%
\pgfpathlineto{\pgfqpoint{0.822710in}{1.263246in}}%
\pgfpathlineto{\pgfqpoint{0.835123in}{1.325441in}}%
\pgfpathlineto{\pgfqpoint{0.847536in}{1.375665in}}%
\pgfpathlineto{\pgfqpoint{0.857466in}{1.407964in}}%
\pgfpathlineto{\pgfqpoint{0.867397in}{1.433964in}}%
\pgfpathlineto{\pgfqpoint{0.877327in}{1.454352in}}%
\pgfpathlineto{\pgfqpoint{0.887258in}{1.469814in}}%
\pgfpathlineto{\pgfqpoint{0.897188in}{1.481020in}}%
\pgfpathlineto{\pgfqpoint{0.907118in}{1.488601in}}%
\pgfpathlineto{\pgfqpoint{0.917049in}{1.493143in}}%
\pgfpathlineto{\pgfqpoint{0.926979in}{1.495182in}}%
\pgfpathlineto{\pgfqpoint{0.936910in}{1.495198in}}%
\pgfpathlineto{\pgfqpoint{0.949323in}{1.493019in}}%
\pgfpathlineto{\pgfqpoint{0.964218in}{1.488103in}}%
\pgfpathlineto{\pgfqpoint{0.986562in}{1.478027in}}%
\pgfpathlineto{\pgfqpoint{1.051109in}{1.447275in}}%
\pgfpathlineto{\pgfqpoint{1.078418in}{1.437374in}}%
\pgfpathlineto{\pgfqpoint{1.105727in}{1.429850in}}%
\pgfpathlineto{\pgfqpoint{1.135518in}{1.424022in}}%
\pgfpathlineto{\pgfqpoint{1.170274in}{1.419636in}}%
\pgfpathlineto{\pgfqpoint{1.214961in}{1.416484in}}%
\pgfpathlineto{\pgfqpoint{1.281991in}{1.414302in}}%
\pgfpathlineto{\pgfqpoint{1.465704in}{1.411472in}}%
\pgfpathlineto{\pgfqpoint{1.758651in}{1.408812in}}%
\pgfpathlineto{\pgfqpoint{2.252689in}{1.406958in}}%
\pgfpathlineto{\pgfqpoint{3.225870in}{1.405680in}}%
\pgfpathlineto{\pgfqpoint{3.225870in}{1.405680in}}%
\pgfusepath{stroke}%
\end{pgfscope}%
\begin{pgfscope}%
\pgfpathrectangle{\pgfqpoint{0.619136in}{0.571603in}}{\pgfqpoint{2.730864in}{1.657828in}}%
\pgfusepath{clip}%
\pgfsetrectcap%
\pgfsetroundjoin%
\pgfsetlinewidth{1.505625pt}%
\definecolor{currentstroke}{rgb}{1.000000,0.498039,0.054902}%
\pgfsetstrokecolor{currentstroke}%
\pgfsetdash{}{0pt}%
\pgfpathmoveto{\pgfqpoint{0.743267in}{0.646959in}}%
\pgfpathlineto{\pgfqpoint{0.748232in}{0.649016in}}%
\pgfpathlineto{\pgfqpoint{0.753197in}{0.654594in}}%
\pgfpathlineto{\pgfqpoint{0.758162in}{0.663383in}}%
\pgfpathlineto{\pgfqpoint{0.765610in}{0.682198in}}%
\pgfpathlineto{\pgfqpoint{0.773058in}{0.707345in}}%
\pgfpathlineto{\pgfqpoint{0.782988in}{0.749955in}}%
\pgfpathlineto{\pgfqpoint{0.795401in}{0.816323in}}%
\pgfpathlineto{\pgfqpoint{0.807814in}{0.895229in}}%
\pgfpathlineto{\pgfqpoint{0.825192in}{1.022365in}}%
\pgfpathlineto{\pgfqpoint{0.847536in}{1.204520in}}%
\pgfpathlineto{\pgfqpoint{0.897188in}{1.615069in}}%
\pgfpathlineto{\pgfqpoint{0.914566in}{1.739415in}}%
\pgfpathlineto{\pgfqpoint{0.929462in}{1.831192in}}%
\pgfpathlineto{\pgfqpoint{0.941875in}{1.895220in}}%
\pgfpathlineto{\pgfqpoint{0.951805in}{1.937428in}}%
\pgfpathlineto{\pgfqpoint{0.961736in}{1.971122in}}%
\pgfpathlineto{\pgfqpoint{0.969183in}{1.990609in}}%
\pgfpathlineto{\pgfqpoint{0.976631in}{2.005042in}}%
\pgfpathlineto{\pgfqpoint{0.984079in}{2.014377in}}%
\pgfpathlineto{\pgfqpoint{0.989044in}{2.017768in}}%
\pgfpathlineto{\pgfqpoint{0.994009in}{2.018904in}}%
\pgfpathlineto{\pgfqpoint{0.998975in}{2.017805in}}%
\pgfpathlineto{\pgfqpoint{1.003940in}{2.014496in}}%
\pgfpathlineto{\pgfqpoint{1.008905in}{2.009013in}}%
\pgfpathlineto{\pgfqpoint{1.016353in}{1.996807in}}%
\pgfpathlineto{\pgfqpoint{1.023801in}{1.979980in}}%
\pgfpathlineto{\pgfqpoint{1.033731in}{1.950721in}}%
\pgfpathlineto{\pgfqpoint{1.043662in}{1.914207in}}%
\pgfpathlineto{\pgfqpoint{1.056075in}{1.859406in}}%
\pgfpathlineto{\pgfqpoint{1.070970in}{1.782216in}}%
\pgfpathlineto{\pgfqpoint{1.088348in}{1.680088in}}%
\pgfpathlineto{\pgfqpoint{1.118140in}{1.488643in}}%
\pgfpathlineto{\pgfqpoint{1.150413in}{1.283731in}}%
\pgfpathlineto{\pgfqpoint{1.170274in}{1.171817in}}%
\pgfpathlineto{\pgfqpoint{1.185170in}{1.099496in}}%
\pgfpathlineto{\pgfqpoint{1.197583in}{1.048508in}}%
\pgfpathlineto{\pgfqpoint{1.209996in}{1.006960in}}%
\pgfpathlineto{\pgfqpoint{1.219926in}{0.980990in}}%
\pgfpathlineto{\pgfqpoint{1.229857in}{0.961749in}}%
\pgfpathlineto{\pgfqpoint{1.237305in}{0.951820in}}%
\pgfpathlineto{\pgfqpoint{1.244752in}{0.945775in}}%
\pgfpathlineto{\pgfqpoint{1.249718in}{0.943898in}}%
\pgfpathlineto{\pgfqpoint{1.254683in}{0.943732in}}%
\pgfpathlineto{\pgfqpoint{1.259648in}{0.945258in}}%
\pgfpathlineto{\pgfqpoint{1.267096in}{0.950668in}}%
\pgfpathlineto{\pgfqpoint{1.274544in}{0.959728in}}%
\pgfpathlineto{\pgfqpoint{1.281991in}{0.972306in}}%
\pgfpathlineto{\pgfqpoint{1.291922in}{0.994271in}}%
\pgfpathlineto{\pgfqpoint{1.301852in}{1.021760in}}%
\pgfpathlineto{\pgfqpoint{1.314265in}{1.063096in}}%
\pgfpathlineto{\pgfqpoint{1.329161in}{1.121406in}}%
\pgfpathlineto{\pgfqpoint{1.346539in}{1.198641in}}%
\pgfpathlineto{\pgfqpoint{1.376330in}{1.343563in}}%
\pgfpathlineto{\pgfqpoint{1.408604in}{1.498789in}}%
\pgfpathlineto{\pgfqpoint{1.428465in}{1.583595in}}%
\pgfpathlineto{\pgfqpoint{1.443361in}{1.638403in}}%
\pgfpathlineto{\pgfqpoint{1.455774in}{1.677042in}}%
\pgfpathlineto{\pgfqpoint{1.468187in}{1.708525in}}%
\pgfpathlineto{\pgfqpoint{1.478117in}{1.728199in}}%
\pgfpathlineto{\pgfqpoint{1.488048in}{1.742768in}}%
\pgfpathlineto{\pgfqpoint{1.495495in}{1.750280in}}%
\pgfpathlineto{\pgfqpoint{1.502943in}{1.754844in}}%
\pgfpathlineto{\pgfqpoint{1.510391in}{1.756470in}}%
\pgfpathlineto{\pgfqpoint{1.517839in}{1.755191in}}%
\pgfpathlineto{\pgfqpoint{1.525287in}{1.751062in}}%
\pgfpathlineto{\pgfqpoint{1.532734in}{1.744162in}}%
\pgfpathlineto{\pgfqpoint{1.540182in}{1.734592in}}%
\pgfpathlineto{\pgfqpoint{1.550113in}{1.717889in}}%
\pgfpathlineto{\pgfqpoint{1.560043in}{1.696992in}}%
\pgfpathlineto{\pgfqpoint{1.572456in}{1.665573in}}%
\pgfpathlineto{\pgfqpoint{1.587352in}{1.621257in}}%
\pgfpathlineto{\pgfqpoint{1.607212in}{1.553716in}}%
\pgfpathlineto{\pgfqpoint{1.639486in}{1.433742in}}%
\pgfpathlineto{\pgfqpoint{1.669278in}{1.325901in}}%
\pgfpathlineto{\pgfqpoint{1.689138in}{1.262542in}}%
\pgfpathlineto{\pgfqpoint{1.704034in}{1.221950in}}%
\pgfpathlineto{\pgfqpoint{1.716447in}{1.193592in}}%
\pgfpathlineto{\pgfqpoint{1.728860in}{1.170759in}}%
\pgfpathlineto{\pgfqpoint{1.738790in}{1.156724in}}%
\pgfpathlineto{\pgfqpoint{1.748721in}{1.146590in}}%
\pgfpathlineto{\pgfqpoint{1.756169in}{1.141591in}}%
\pgfpathlineto{\pgfqpoint{1.763617in}{1.138830in}}%
\pgfpathlineto{\pgfqpoint{1.771064in}{1.138293in}}%
\pgfpathlineto{\pgfqpoint{1.778512in}{1.139952in}}%
\pgfpathlineto{\pgfqpoint{1.785960in}{1.143756in}}%
\pgfpathlineto{\pgfqpoint{1.793408in}{1.149642in}}%
\pgfpathlineto{\pgfqpoint{1.803338in}{1.160584in}}%
\pgfpathlineto{\pgfqpoint{1.813269in}{1.174847in}}%
\pgfpathlineto{\pgfqpoint{1.825682in}{1.196918in}}%
\pgfpathlineto{\pgfqpoint{1.840577in}{1.228792in}}%
\pgfpathlineto{\pgfqpoint{1.857955in}{1.271851in}}%
\pgfpathlineto{\pgfqpoint{1.882781in}{1.340247in}}%
\pgfpathlineto{\pgfqpoint{1.927468in}{1.464601in}}%
\pgfpathlineto{\pgfqpoint{1.947329in}{1.512696in}}%
\pgfpathlineto{\pgfqpoint{1.962225in}{1.543514in}}%
\pgfpathlineto{\pgfqpoint{1.974638in}{1.565047in}}%
\pgfpathlineto{\pgfqpoint{1.987051in}{1.582388in}}%
\pgfpathlineto{\pgfqpoint{1.996981in}{1.593050in}}%
\pgfpathlineto{\pgfqpoint{2.006912in}{1.600751in}}%
\pgfpathlineto{\pgfqpoint{2.016842in}{1.605441in}}%
\pgfpathlineto{\pgfqpoint{2.026772in}{1.607116in}}%
\pgfpathlineto{\pgfqpoint{2.036703in}{1.605813in}}%
\pgfpathlineto{\pgfqpoint{2.046633in}{1.601615in}}%
\pgfpathlineto{\pgfqpoint{2.056564in}{1.594643in}}%
\pgfpathlineto{\pgfqpoint{2.066494in}{1.585057in}}%
\pgfpathlineto{\pgfqpoint{2.078907in}{1.569695in}}%
\pgfpathlineto{\pgfqpoint{2.093803in}{1.546904in}}%
\pgfpathlineto{\pgfqpoint{2.111181in}{1.515446in}}%
\pgfpathlineto{\pgfqpoint{2.133524in}{1.469744in}}%
\pgfpathlineto{\pgfqpoint{2.193107in}{1.344968in}}%
\pgfpathlineto{\pgfqpoint{2.210485in}{1.314504in}}%
\pgfpathlineto{\pgfqpoint{2.225381in}{1.292394in}}%
\pgfpathlineto{\pgfqpoint{2.237794in}{1.277260in}}%
\pgfpathlineto{\pgfqpoint{2.250207in}{1.265403in}}%
\pgfpathlineto{\pgfqpoint{2.262620in}{1.257005in}}%
\pgfpathlineto{\pgfqpoint{2.272550in}{1.252847in}}%
\pgfpathlineto{\pgfqpoint{2.282481in}{1.250981in}}%
\pgfpathlineto{\pgfqpoint{2.292411in}{1.251386in}}%
\pgfpathlineto{\pgfqpoint{2.302341in}{1.254009in}}%
\pgfpathlineto{\pgfqpoint{2.312272in}{1.258764in}}%
\pgfpathlineto{\pgfqpoint{2.324685in}{1.267531in}}%
\pgfpathlineto{\pgfqpoint{2.337098in}{1.279171in}}%
\pgfpathlineto{\pgfqpoint{2.351993in}{1.296448in}}%
\pgfpathlineto{\pgfqpoint{2.369372in}{1.320307in}}%
\pgfpathlineto{\pgfqpoint{2.391715in}{1.354980in}}%
\pgfpathlineto{\pgfqpoint{2.451298in}{1.449691in}}%
\pgfpathlineto{\pgfqpoint{2.471158in}{1.475833in}}%
\pgfpathlineto{\pgfqpoint{2.486054in}{1.492106in}}%
\pgfpathlineto{\pgfqpoint{2.500950in}{1.505025in}}%
\pgfpathlineto{\pgfqpoint{2.513363in}{1.512998in}}%
\pgfpathlineto{\pgfqpoint{2.525776in}{1.518308in}}%
\pgfpathlineto{\pgfqpoint{2.538189in}{1.520903in}}%
\pgfpathlineto{\pgfqpoint{2.550602in}{1.520796in}}%
\pgfpathlineto{\pgfqpoint{2.563015in}{1.518060in}}%
\pgfpathlineto{\pgfqpoint{2.575428in}{1.512825in}}%
\pgfpathlineto{\pgfqpoint{2.587841in}{1.505275in}}%
\pgfpathlineto{\pgfqpoint{2.602736in}{1.493489in}}%
\pgfpathlineto{\pgfqpoint{2.620115in}{1.476587in}}%
\pgfpathlineto{\pgfqpoint{2.642458in}{1.451226in}}%
\pgfpathlineto{\pgfqpoint{2.719419in}{1.359877in}}%
\pgfpathlineto{\pgfqpoint{2.736797in}{1.343891in}}%
\pgfpathlineto{\pgfqpoint{2.751693in}{1.332757in}}%
\pgfpathlineto{\pgfqpoint{2.766588in}{1.324299in}}%
\pgfpathlineto{\pgfqpoint{2.781484in}{1.318709in}}%
\pgfpathlineto{\pgfqpoint{2.793897in}{1.316310in}}%
\pgfpathlineto{\pgfqpoint{2.806310in}{1.315969in}}%
\pgfpathlineto{\pgfqpoint{2.818723in}{1.317638in}}%
\pgfpathlineto{\pgfqpoint{2.833619in}{1.322167in}}%
\pgfpathlineto{\pgfqpoint{2.848514in}{1.329226in}}%
\pgfpathlineto{\pgfqpoint{2.865892in}{1.340244in}}%
\pgfpathlineto{\pgfqpoint{2.885753in}{1.355768in}}%
\pgfpathlineto{\pgfqpoint{2.913062in}{1.380393in}}%
\pgfpathlineto{\pgfqpoint{2.965196in}{1.428293in}}%
\pgfpathlineto{\pgfqpoint{2.987540in}{1.445439in}}%
\pgfpathlineto{\pgfqpoint{3.004918in}{1.456220in}}%
\pgfpathlineto{\pgfqpoint{3.022296in}{1.464328in}}%
\pgfpathlineto{\pgfqpoint{3.037192in}{1.468949in}}%
\pgfpathlineto{\pgfqpoint{3.052088in}{1.471329in}}%
\pgfpathlineto{\pgfqpoint{3.066983in}{1.471463in}}%
\pgfpathlineto{\pgfqpoint{3.081879in}{1.469414in}}%
\pgfpathlineto{\pgfqpoint{3.096774in}{1.465319in}}%
\pgfpathlineto{\pgfqpoint{3.114153in}{1.458220in}}%
\pgfpathlineto{\pgfqpoint{3.134014in}{1.447549in}}%
\pgfpathlineto{\pgfqpoint{3.158840in}{1.431450in}}%
\pgfpathlineto{\pgfqpoint{3.225870in}{1.384762in}}%
\pgfpathlineto{\pgfqpoint{3.225870in}{1.384762in}}%
\pgfusepath{stroke}%
\end{pgfscope}%
\begin{pgfscope}%
\pgfpathrectangle{\pgfqpoint{0.619136in}{0.571603in}}{\pgfqpoint{2.730864in}{1.657828in}}%
\pgfusepath{clip}%
\pgfsetrectcap%
\pgfsetroundjoin%
\pgfsetlinewidth{1.505625pt}%
\definecolor{currentstroke}{rgb}{0.172549,0.627451,0.172549}%
\pgfsetstrokecolor{currentstroke}%
\pgfsetdash{}{0pt}%
\pgfpathmoveto{\pgfqpoint{0.743267in}{0.646959in}}%
\pgfpathlineto{\pgfqpoint{0.745749in}{0.648011in}}%
\pgfpathlineto{\pgfqpoint{0.750714in}{0.653699in}}%
\pgfpathlineto{\pgfqpoint{0.755680in}{0.662909in}}%
\pgfpathlineto{\pgfqpoint{0.763127in}{0.682074in}}%
\pgfpathlineto{\pgfqpoint{0.773058in}{0.715932in}}%
\pgfpathlineto{\pgfqpoint{0.785471in}{0.769115in}}%
\pgfpathlineto{\pgfqpoint{0.800366in}{0.845317in}}%
\pgfpathlineto{\pgfqpoint{0.820227in}{0.961533in}}%
\pgfpathlineto{\pgfqpoint{0.854984in}{1.182924in}}%
\pgfpathlineto{\pgfqpoint{0.887258in}{1.382616in}}%
\pgfpathlineto{\pgfqpoint{0.907118in}{1.492629in}}%
\pgfpathlineto{\pgfqpoint{0.924497in}{1.577140in}}%
\pgfpathlineto{\pgfqpoint{0.939392in}{1.639396in}}%
\pgfpathlineto{\pgfqpoint{0.951805in}{1.683495in}}%
\pgfpathlineto{\pgfqpoint{0.964218in}{1.720225in}}%
\pgfpathlineto{\pgfqpoint{0.976631in}{1.749467in}}%
\pgfpathlineto{\pgfqpoint{0.986562in}{1.767474in}}%
\pgfpathlineto{\pgfqpoint{0.996492in}{1.780765in}}%
\pgfpathlineto{\pgfqpoint{1.003940in}{1.787705in}}%
\pgfpathlineto{\pgfqpoint{1.011388in}{1.792122in}}%
\pgfpathlineto{\pgfqpoint{1.018836in}{1.794094in}}%
\pgfpathlineto{\pgfqpoint{1.026283in}{1.793711in}}%
\pgfpathlineto{\pgfqpoint{1.033731in}{1.791074in}}%
\pgfpathlineto{\pgfqpoint{1.041179in}{1.786293in}}%
\pgfpathlineto{\pgfqpoint{1.051109in}{1.776791in}}%
\pgfpathlineto{\pgfqpoint{1.061040in}{1.763994in}}%
\pgfpathlineto{\pgfqpoint{1.073453in}{1.743853in}}%
\pgfpathlineto{\pgfqpoint{1.088348in}{1.714477in}}%
\pgfpathlineto{\pgfqpoint{1.105727in}{1.674509in}}%
\pgfpathlineto{\pgfqpoint{1.128070in}{1.616995in}}%
\pgfpathlineto{\pgfqpoint{1.200066in}{1.426332in}}%
\pgfpathlineto{\pgfqpoint{1.219926in}{1.381742in}}%
\pgfpathlineto{\pgfqpoint{1.237305in}{1.347884in}}%
\pgfpathlineto{\pgfqpoint{1.252200in}{1.323185in}}%
\pgfpathlineto{\pgfqpoint{1.267096in}{1.302736in}}%
\pgfpathlineto{\pgfqpoint{1.279509in}{1.289030in}}%
\pgfpathlineto{\pgfqpoint{1.291922in}{1.278366in}}%
\pgfpathlineto{\pgfqpoint{1.304335in}{1.270701in}}%
\pgfpathlineto{\pgfqpoint{1.316748in}{1.265944in}}%
\pgfpathlineto{\pgfqpoint{1.329161in}{1.263962in}}%
\pgfpathlineto{\pgfqpoint{1.341574in}{1.264586in}}%
\pgfpathlineto{\pgfqpoint{1.353987in}{1.267616in}}%
\pgfpathlineto{\pgfqpoint{1.366400in}{1.272824in}}%
\pgfpathlineto{\pgfqpoint{1.381296in}{1.281600in}}%
\pgfpathlineto{\pgfqpoint{1.398674in}{1.294750in}}%
\pgfpathlineto{\pgfqpoint{1.418535in}{1.312699in}}%
\pgfpathlineto{\pgfqpoint{1.448326in}{1.343122in}}%
\pgfpathlineto{\pgfqpoint{1.502943in}{1.399404in}}%
\pgfpathlineto{\pgfqpoint{1.527769in}{1.421383in}}%
\pgfpathlineto{\pgfqpoint{1.547630in}{1.436220in}}%
\pgfpathlineto{\pgfqpoint{1.567491in}{1.448227in}}%
\pgfpathlineto{\pgfqpoint{1.584869in}{1.456257in}}%
\pgfpathlineto{\pgfqpoint{1.602247in}{1.461936in}}%
\pgfpathlineto{\pgfqpoint{1.619626in}{1.465311in}}%
\pgfpathlineto{\pgfqpoint{1.637004in}{1.466496in}}%
\pgfpathlineto{\pgfqpoint{1.656865in}{1.465395in}}%
\pgfpathlineto{\pgfqpoint{1.676725in}{1.462000in}}%
\pgfpathlineto{\pgfqpoint{1.699069in}{1.455933in}}%
\pgfpathlineto{\pgfqpoint{1.726377in}{1.446152in}}%
\pgfpathlineto{\pgfqpoint{1.771064in}{1.427368in}}%
\pgfpathlineto{\pgfqpoint{1.820716in}{1.407217in}}%
\pgfpathlineto{\pgfqpoint{1.852990in}{1.396672in}}%
\pgfpathlineto{\pgfqpoint{1.880299in}{1.390020in}}%
\pgfpathlineto{\pgfqpoint{1.907607in}{1.385658in}}%
\pgfpathlineto{\pgfqpoint{1.934916in}{1.383572in}}%
\pgfpathlineto{\pgfqpoint{1.964707in}{1.383665in}}%
\pgfpathlineto{\pgfqpoint{1.996981in}{1.386072in}}%
\pgfpathlineto{\pgfqpoint{2.039185in}{1.391706in}}%
\pgfpathlineto{\pgfqpoint{2.170763in}{1.411124in}}%
\pgfpathlineto{\pgfqpoint{2.212968in}{1.414184in}}%
\pgfpathlineto{\pgfqpoint{2.257655in}{1.415099in}}%
\pgfpathlineto{\pgfqpoint{2.307307in}{1.413772in}}%
\pgfpathlineto{\pgfqpoint{2.381785in}{1.409204in}}%
\pgfpathlineto{\pgfqpoint{2.481089in}{1.403430in}}%
\pgfpathlineto{\pgfqpoint{2.548119in}{1.401850in}}%
\pgfpathlineto{\pgfqpoint{2.625080in}{1.402403in}}%
\pgfpathlineto{\pgfqpoint{2.898166in}{1.406569in}}%
\pgfpathlineto{\pgfqpoint{3.225870in}{1.404468in}}%
\pgfpathlineto{\pgfqpoint{3.225870in}{1.404468in}}%
\pgfusepath{stroke}%
\end{pgfscope}%
\begin{pgfscope}%
\pgfpathrectangle{\pgfqpoint{0.619136in}{0.571603in}}{\pgfqpoint{2.730864in}{1.657828in}}%
\pgfusepath{clip}%
\pgfsetrectcap%
\pgfsetroundjoin%
\pgfsetlinewidth{1.505625pt}%
\definecolor{currentstroke}{rgb}{0.839216,0.152941,0.156863}%
\pgfsetstrokecolor{currentstroke}%
\pgfsetdash{}{0pt}%
\pgfpathmoveto{\pgfqpoint{0.743267in}{0.646959in}}%
\pgfpathlineto{\pgfqpoint{0.765610in}{0.906671in}}%
\pgfpathlineto{\pgfqpoint{0.778023in}{1.019738in}}%
\pgfpathlineto{\pgfqpoint{0.790436in}{1.110647in}}%
\pgfpathlineto{\pgfqpoint{0.802849in}{1.182775in}}%
\pgfpathlineto{\pgfqpoint{0.815262in}{1.239433in}}%
\pgfpathlineto{\pgfqpoint{0.825192in}{1.275610in}}%
\pgfpathlineto{\pgfqpoint{0.835123in}{1.305075in}}%
\pgfpathlineto{\pgfqpoint{0.845053in}{1.328952in}}%
\pgfpathlineto{\pgfqpoint{0.854984in}{1.348199in}}%
\pgfpathlineto{\pgfqpoint{0.867397in}{1.366972in}}%
\pgfpathlineto{\pgfqpoint{0.879810in}{1.381080in}}%
\pgfpathlineto{\pgfqpoint{0.892223in}{1.391567in}}%
\pgfpathlineto{\pgfqpoint{0.904636in}{1.399264in}}%
\pgfpathlineto{\pgfqpoint{0.919531in}{1.405727in}}%
\pgfpathlineto{\pgfqpoint{0.936910in}{1.410511in}}%
\pgfpathlineto{\pgfqpoint{0.956770in}{1.413534in}}%
\pgfpathlineto{\pgfqpoint{0.984079in}{1.415137in}}%
\pgfpathlineto{\pgfqpoint{1.023801in}{1.414942in}}%
\pgfpathlineto{\pgfqpoint{1.351504in}{1.408108in}}%
\pgfpathlineto{\pgfqpoint{1.674243in}{1.406531in}}%
\pgfpathlineto{\pgfqpoint{2.456263in}{1.405398in}}%
\pgfpathlineto{\pgfqpoint{3.225870in}{1.405022in}}%
\pgfpathlineto{\pgfqpoint{3.225870in}{1.405022in}}%
\pgfusepath{stroke}%
\end{pgfscope}%
\begin{pgfscope}%
\pgfpathrectangle{\pgfqpoint{0.619136in}{0.571603in}}{\pgfqpoint{2.730864in}{1.657828in}}%
\pgfusepath{clip}%
\pgfsetrectcap%
\pgfsetroundjoin%
\pgfsetlinewidth{1.505625pt}%
\definecolor{currentstroke}{rgb}{0.580392,0.403922,0.741176}%
\pgfsetstrokecolor{currentstroke}%
\pgfsetdash{}{0pt}%
\pgfpathmoveto{\pgfqpoint{0.743267in}{0.646959in}}%
\pgfpathlineto{\pgfqpoint{0.745749in}{0.649058in}}%
\pgfpathlineto{\pgfqpoint{0.750714in}{0.659133in}}%
\pgfpathlineto{\pgfqpoint{0.758162in}{0.683636in}}%
\pgfpathlineto{\pgfqpoint{0.765610in}{0.716403in}}%
\pgfpathlineto{\pgfqpoint{0.775540in}{0.769732in}}%
\pgfpathlineto{\pgfqpoint{0.790436in}{0.864315in}}%
\pgfpathlineto{\pgfqpoint{0.812779in}{1.024130in}}%
\pgfpathlineto{\pgfqpoint{0.850018in}{1.292162in}}%
\pgfpathlineto{\pgfqpoint{0.869879in}{1.417914in}}%
\pgfpathlineto{\pgfqpoint{0.884775in}{1.499425in}}%
\pgfpathlineto{\pgfqpoint{0.899671in}{1.568266in}}%
\pgfpathlineto{\pgfqpoint{0.912084in}{1.615359in}}%
\pgfpathlineto{\pgfqpoint{0.924497in}{1.652945in}}%
\pgfpathlineto{\pgfqpoint{0.934427in}{1.676235in}}%
\pgfpathlineto{\pgfqpoint{0.944357in}{1.693669in}}%
\pgfpathlineto{\pgfqpoint{0.951805in}{1.703041in}}%
\pgfpathlineto{\pgfqpoint{0.959253in}{1.709379in}}%
\pgfpathlineto{\pgfqpoint{0.966701in}{1.712828in}}%
\pgfpathlineto{\pgfqpoint{0.974149in}{1.713548in}}%
\pgfpathlineto{\pgfqpoint{0.981596in}{1.711715in}}%
\pgfpathlineto{\pgfqpoint{0.989044in}{1.707515in}}%
\pgfpathlineto{\pgfqpoint{0.998975in}{1.698567in}}%
\pgfpathlineto{\pgfqpoint{1.008905in}{1.686231in}}%
\pgfpathlineto{\pgfqpoint{1.021318in}{1.666786in}}%
\pgfpathlineto{\pgfqpoint{1.036214in}{1.638802in}}%
\pgfpathlineto{\pgfqpoint{1.056075in}{1.596209in}}%
\pgfpathlineto{\pgfqpoint{1.120622in}{1.453349in}}%
\pgfpathlineto{\pgfqpoint{1.138000in}{1.421328in}}%
\pgfpathlineto{\pgfqpoint{1.155379in}{1.394045in}}%
\pgfpathlineto{\pgfqpoint{1.170274in}{1.374799in}}%
\pgfpathlineto{\pgfqpoint{1.185170in}{1.359504in}}%
\pgfpathlineto{\pgfqpoint{1.197583in}{1.349759in}}%
\pgfpathlineto{\pgfqpoint{1.209996in}{1.342651in}}%
\pgfpathlineto{\pgfqpoint{1.222409in}{1.338042in}}%
\pgfpathlineto{\pgfqpoint{1.234822in}{1.335749in}}%
\pgfpathlineto{\pgfqpoint{1.247235in}{1.335557in}}%
\pgfpathlineto{\pgfqpoint{1.262131in}{1.337758in}}%
\pgfpathlineto{\pgfqpoint{1.277026in}{1.342188in}}%
\pgfpathlineto{\pgfqpoint{1.294404in}{1.349560in}}%
\pgfpathlineto{\pgfqpoint{1.319231in}{1.362836in}}%
\pgfpathlineto{\pgfqpoint{1.396191in}{1.406410in}}%
\pgfpathlineto{\pgfqpoint{1.421017in}{1.416851in}}%
\pgfpathlineto{\pgfqpoint{1.443361in}{1.423800in}}%
\pgfpathlineto{\pgfqpoint{1.465704in}{1.428348in}}%
\pgfpathlineto{\pgfqpoint{1.488048in}{1.430611in}}%
\pgfpathlineto{\pgfqpoint{1.512874in}{1.430772in}}%
\pgfpathlineto{\pgfqpoint{1.540182in}{1.428670in}}%
\pgfpathlineto{\pgfqpoint{1.577421in}{1.423297in}}%
\pgfpathlineto{\pgfqpoint{1.684173in}{1.406273in}}%
\pgfpathlineto{\pgfqpoint{1.723895in}{1.402872in}}%
\pgfpathlineto{\pgfqpoint{1.766099in}{1.401501in}}%
\pgfpathlineto{\pgfqpoint{1.818234in}{1.402228in}}%
\pgfpathlineto{\pgfqpoint{2.036703in}{1.408154in}}%
\pgfpathlineto{\pgfqpoint{2.163316in}{1.406595in}}%
\pgfpathlineto{\pgfqpoint{2.319720in}{1.405465in}}%
\pgfpathlineto{\pgfqpoint{3.225870in}{1.405077in}}%
\pgfpathlineto{\pgfqpoint{3.225870in}{1.405077in}}%
\pgfusepath{stroke}%
\end{pgfscope}%
\begin{pgfscope}%
\pgfpathrectangle{\pgfqpoint{0.619136in}{0.571603in}}{\pgfqpoint{2.730864in}{1.657828in}}%
\pgfusepath{clip}%
\pgfsetrectcap%
\pgfsetroundjoin%
\pgfsetlinewidth{1.505625pt}%
\definecolor{currentstroke}{rgb}{0.549020,0.337255,0.294118}%
\pgfsetstrokecolor{currentstroke}%
\pgfsetdash{}{0pt}%
\pgfpathmoveto{\pgfqpoint{0.743267in}{0.646959in}}%
\pgfpathlineto{\pgfqpoint{0.745749in}{0.648218in}}%
\pgfpathlineto{\pgfqpoint{0.750714in}{0.655269in}}%
\pgfpathlineto{\pgfqpoint{0.755680in}{0.666888in}}%
\pgfpathlineto{\pgfqpoint{0.763127in}{0.691335in}}%
\pgfpathlineto{\pgfqpoint{0.773058in}{0.734844in}}%
\pgfpathlineto{\pgfqpoint{0.785471in}{0.803343in}}%
\pgfpathlineto{\pgfqpoint{0.800366in}{0.901079in}}%
\pgfpathlineto{\pgfqpoint{0.820227in}{1.048046in}}%
\pgfpathlineto{\pgfqpoint{0.872362in}{1.442224in}}%
\pgfpathlineto{\pgfqpoint{0.889740in}{1.554462in}}%
\pgfpathlineto{\pgfqpoint{0.904636in}{1.637075in}}%
\pgfpathlineto{\pgfqpoint{0.917049in}{1.694945in}}%
\pgfpathlineto{\pgfqpoint{0.929462in}{1.742110in}}%
\pgfpathlineto{\pgfqpoint{0.939392in}{1.771890in}}%
\pgfpathlineto{\pgfqpoint{0.949323in}{1.794546in}}%
\pgfpathlineto{\pgfqpoint{0.956770in}{1.806903in}}%
\pgfpathlineto{\pgfqpoint{0.964218in}{1.815360in}}%
\pgfpathlineto{\pgfqpoint{0.971666in}{1.820020in}}%
\pgfpathlineto{\pgfqpoint{0.976631in}{1.821079in}}%
\pgfpathlineto{\pgfqpoint{0.981596in}{1.820552in}}%
\pgfpathlineto{\pgfqpoint{0.989044in}{1.816898in}}%
\pgfpathlineto{\pgfqpoint{0.996492in}{1.809976in}}%
\pgfpathlineto{\pgfqpoint{1.003940in}{1.799993in}}%
\pgfpathlineto{\pgfqpoint{1.013870in}{1.782309in}}%
\pgfpathlineto{\pgfqpoint{1.023801in}{1.760154in}}%
\pgfpathlineto{\pgfqpoint{1.036214in}{1.727096in}}%
\pgfpathlineto{\pgfqpoint{1.051109in}{1.681182in}}%
\pgfpathlineto{\pgfqpoint{1.073453in}{1.604026in}}%
\pgfpathlineto{\pgfqpoint{1.125587in}{1.420614in}}%
\pgfpathlineto{\pgfqpoint{1.142966in}{1.367971in}}%
\pgfpathlineto{\pgfqpoint{1.157861in}{1.328839in}}%
\pgfpathlineto{\pgfqpoint{1.172757in}{1.296093in}}%
\pgfpathlineto{\pgfqpoint{1.185170in}{1.274061in}}%
\pgfpathlineto{\pgfqpoint{1.197583in}{1.256973in}}%
\pgfpathlineto{\pgfqpoint{1.207513in}{1.246878in}}%
\pgfpathlineto{\pgfqpoint{1.217444in}{1.239916in}}%
\pgfpathlineto{\pgfqpoint{1.227374in}{1.235997in}}%
\pgfpathlineto{\pgfqpoint{1.237305in}{1.234990in}}%
\pgfpathlineto{\pgfqpoint{1.247235in}{1.236727in}}%
\pgfpathlineto{\pgfqpoint{1.257165in}{1.241009in}}%
\pgfpathlineto{\pgfqpoint{1.267096in}{1.247612in}}%
\pgfpathlineto{\pgfqpoint{1.279509in}{1.258750in}}%
\pgfpathlineto{\pgfqpoint{1.294404in}{1.275661in}}%
\pgfpathlineto{\pgfqpoint{1.311783in}{1.299090in}}%
\pgfpathlineto{\pgfqpoint{1.339091in}{1.340486in}}%
\pgfpathlineto{\pgfqpoint{1.378813in}{1.400727in}}%
\pgfpathlineto{\pgfqpoint{1.398674in}{1.427030in}}%
\pgfpathlineto{\pgfqpoint{1.416052in}{1.446576in}}%
\pgfpathlineto{\pgfqpoint{1.430948in}{1.460299in}}%
\pgfpathlineto{\pgfqpoint{1.445843in}{1.470996in}}%
\pgfpathlineto{\pgfqpoint{1.458256in}{1.477530in}}%
\pgfpathlineto{\pgfqpoint{1.470669in}{1.481906in}}%
\pgfpathlineto{\pgfqpoint{1.483082in}{1.484185in}}%
\pgfpathlineto{\pgfqpoint{1.495495in}{1.484470in}}%
\pgfpathlineto{\pgfqpoint{1.510391in}{1.482386in}}%
\pgfpathlineto{\pgfqpoint{1.525287in}{1.477952in}}%
\pgfpathlineto{\pgfqpoint{1.542665in}{1.470291in}}%
\pgfpathlineto{\pgfqpoint{1.562526in}{1.459034in}}%
\pgfpathlineto{\pgfqpoint{1.592317in}{1.439214in}}%
\pgfpathlineto{\pgfqpoint{1.641969in}{1.406017in}}%
\pgfpathlineto{\pgfqpoint{1.666795in}{1.392407in}}%
\pgfpathlineto{\pgfqpoint{1.686656in}{1.383857in}}%
\pgfpathlineto{\pgfqpoint{1.706517in}{1.377666in}}%
\pgfpathlineto{\pgfqpoint{1.726377in}{1.373912in}}%
\pgfpathlineto{\pgfqpoint{1.746238in}{1.372522in}}%
\pgfpathlineto{\pgfqpoint{1.766099in}{1.373291in}}%
\pgfpathlineto{\pgfqpoint{1.788443in}{1.376350in}}%
\pgfpathlineto{\pgfqpoint{1.815751in}{1.382469in}}%
\pgfpathlineto{\pgfqpoint{1.857955in}{1.394629in}}%
\pgfpathlineto{\pgfqpoint{1.907607in}{1.408532in}}%
\pgfpathlineto{\pgfqpoint{1.939881in}{1.415229in}}%
\pgfpathlineto{\pgfqpoint{1.969673in}{1.419051in}}%
\pgfpathlineto{\pgfqpoint{1.999464in}{1.420498in}}%
\pgfpathlineto{\pgfqpoint{2.031738in}{1.419648in}}%
\pgfpathlineto{\pgfqpoint{2.068977in}{1.416312in}}%
\pgfpathlineto{\pgfqpoint{2.150903in}{1.405775in}}%
\pgfpathlineto{\pgfqpoint{2.200555in}{1.400943in}}%
\pgfpathlineto{\pgfqpoint{2.242759in}{1.398999in}}%
\pgfpathlineto{\pgfqpoint{2.289928in}{1.399174in}}%
\pgfpathlineto{\pgfqpoint{2.351993in}{1.401932in}}%
\pgfpathlineto{\pgfqpoint{2.466193in}{1.407323in}}%
\pgfpathlineto{\pgfqpoint{2.533224in}{1.407865in}}%
\pgfpathlineto{\pgfqpoint{2.625080in}{1.406017in}}%
\pgfpathlineto{\pgfqpoint{2.754175in}{1.403710in}}%
\pgfpathlineto{\pgfqpoint{2.870858in}{1.404300in}}%
\pgfpathlineto{\pgfqpoint{3.064501in}{1.405289in}}%
\pgfpathlineto{\pgfqpoint{3.225870in}{1.404568in}}%
\pgfpathlineto{\pgfqpoint{3.225870in}{1.404568in}}%
\pgfusepath{stroke}%
\end{pgfscope}%
\begin{pgfscope}%
\pgfpathrectangle{\pgfqpoint{0.619136in}{0.571603in}}{\pgfqpoint{2.730864in}{1.657828in}}%
\pgfusepath{clip}%
\pgfsetrectcap%
\pgfsetroundjoin%
\pgfsetlinewidth{1.505625pt}%
\definecolor{currentstroke}{rgb}{0.890196,0.466667,0.760784}%
\pgfsetstrokecolor{currentstroke}%
\pgfsetdash{}{0pt}%
\pgfpathmoveto{\pgfqpoint{0.743267in}{0.646959in}}%
\pgfpathlineto{\pgfqpoint{0.748232in}{0.648106in}}%
\pgfpathlineto{\pgfqpoint{0.753197in}{0.651413in}}%
\pgfpathlineto{\pgfqpoint{0.760645in}{0.660256in}}%
\pgfpathlineto{\pgfqpoint{0.768093in}{0.673611in}}%
\pgfpathlineto{\pgfqpoint{0.775540in}{0.691323in}}%
\pgfpathlineto{\pgfqpoint{0.785471in}{0.721440in}}%
\pgfpathlineto{\pgfqpoint{0.795401in}{0.758601in}}%
\pgfpathlineto{\pgfqpoint{0.807814in}{0.814237in}}%
\pgfpathlineto{\pgfqpoint{0.822710in}{0.893089in}}%
\pgfpathlineto{\pgfqpoint{0.840088in}{0.999171in}}%
\pgfpathlineto{\pgfqpoint{0.862432in}{1.152291in}}%
\pgfpathlineto{\pgfqpoint{0.899671in}{1.427848in}}%
\pgfpathlineto{\pgfqpoint{0.931944in}{1.660466in}}%
\pgfpathlineto{\pgfqpoint{0.951805in}{1.789207in}}%
\pgfpathlineto{\pgfqpoint{0.969183in}{1.887849in}}%
\pgfpathlineto{\pgfqpoint{0.984079in}{1.959626in}}%
\pgfpathlineto{\pgfqpoint{0.996492in}{2.009239in}}%
\pgfpathlineto{\pgfqpoint{1.006423in}{2.041723in}}%
\pgfpathlineto{\pgfqpoint{1.016353in}{2.067481in}}%
\pgfpathlineto{\pgfqpoint{1.026283in}{2.086294in}}%
\pgfpathlineto{\pgfqpoint{1.033731in}{2.095754in}}%
\pgfpathlineto{\pgfqpoint{1.041179in}{2.101188in}}%
\pgfpathlineto{\pgfqpoint{1.046144in}{2.102567in}}%
\pgfpathlineto{\pgfqpoint{1.051109in}{2.102154in}}%
\pgfpathlineto{\pgfqpoint{1.056075in}{2.099955in}}%
\pgfpathlineto{\pgfqpoint{1.063522in}{2.093336in}}%
\pgfpathlineto{\pgfqpoint{1.070970in}{2.082785in}}%
\pgfpathlineto{\pgfqpoint{1.078418in}{2.068381in}}%
\pgfpathlineto{\pgfqpoint{1.088348in}{2.043365in}}%
\pgfpathlineto{\pgfqpoint{1.098279in}{2.011987in}}%
\pgfpathlineto{\pgfqpoint{1.110692in}{1.964378in}}%
\pgfpathlineto{\pgfqpoint{1.125587in}{1.896073in}}%
\pgfpathlineto{\pgfqpoint{1.140483in}{1.817188in}}%
\pgfpathlineto{\pgfqpoint{1.160344in}{1.698895in}}%
\pgfpathlineto{\pgfqpoint{1.187653in}{1.520436in}}%
\pgfpathlineto{\pgfqpoint{1.239787in}{1.176459in}}%
\pgfpathlineto{\pgfqpoint{1.259648in}{1.060398in}}%
\pgfpathlineto{\pgfqpoint{1.277026in}{0.971527in}}%
\pgfpathlineto{\pgfqpoint{1.291922in}{0.906897in}}%
\pgfpathlineto{\pgfqpoint{1.304335in}{0.862248in}}%
\pgfpathlineto{\pgfqpoint{1.314265in}{0.833029in}}%
\pgfpathlineto{\pgfqpoint{1.324196in}{0.809875in}}%
\pgfpathlineto{\pgfqpoint{1.334126in}{0.792981in}}%
\pgfpathlineto{\pgfqpoint{1.341574in}{0.784499in}}%
\pgfpathlineto{\pgfqpoint{1.349022in}{0.779644in}}%
\pgfpathlineto{\pgfqpoint{1.353987in}{0.778428in}}%
\pgfpathlineto{\pgfqpoint{1.358952in}{0.778826in}}%
\pgfpathlineto{\pgfqpoint{1.363917in}{0.780831in}}%
\pgfpathlineto{\pgfqpoint{1.371365in}{0.786829in}}%
\pgfpathlineto{\pgfqpoint{1.378813in}{0.796365in}}%
\pgfpathlineto{\pgfqpoint{1.386261in}{0.809368in}}%
\pgfpathlineto{\pgfqpoint{1.396191in}{0.831932in}}%
\pgfpathlineto{\pgfqpoint{1.406122in}{0.860218in}}%
\pgfpathlineto{\pgfqpoint{1.418535in}{0.903115in}}%
\pgfpathlineto{\pgfqpoint{1.433430in}{0.964634in}}%
\pgfpathlineto{\pgfqpoint{1.450808in}{1.048296in}}%
\pgfpathlineto{\pgfqpoint{1.470669in}{1.156170in}}%
\pgfpathlineto{\pgfqpoint{1.500461in}{1.332733in}}%
\pgfpathlineto{\pgfqpoint{1.545147in}{1.598003in}}%
\pgfpathlineto{\pgfqpoint{1.565008in}{1.703711in}}%
\pgfpathlineto{\pgfqpoint{1.582386in}{1.785161in}}%
\pgfpathlineto{\pgfqpoint{1.597282in}{1.844820in}}%
\pgfpathlineto{\pgfqpoint{1.609695in}{1.886385in}}%
\pgfpathlineto{\pgfqpoint{1.622108in}{1.919890in}}%
\pgfpathlineto{\pgfqpoint{1.632039in}{1.940573in}}%
\pgfpathlineto{\pgfqpoint{1.641969in}{1.955620in}}%
\pgfpathlineto{\pgfqpoint{1.649417in}{1.963135in}}%
\pgfpathlineto{\pgfqpoint{1.656865in}{1.967388in}}%
\pgfpathlineto{\pgfqpoint{1.664312in}{1.968369in}}%
\pgfpathlineto{\pgfqpoint{1.671760in}{1.966090in}}%
\pgfpathlineto{\pgfqpoint{1.679208in}{1.960580in}}%
\pgfpathlineto{\pgfqpoint{1.686656in}{1.951889in}}%
\pgfpathlineto{\pgfqpoint{1.694104in}{1.940082in}}%
\pgfpathlineto{\pgfqpoint{1.704034in}{1.919640in}}%
\pgfpathlineto{\pgfqpoint{1.713964in}{1.894056in}}%
\pgfpathlineto{\pgfqpoint{1.726377in}{1.855304in}}%
\pgfpathlineto{\pgfqpoint{1.741273in}{1.799785in}}%
\pgfpathlineto{\pgfqpoint{1.758651in}{1.724344in}}%
\pgfpathlineto{\pgfqpoint{1.778512in}{1.627137in}}%
\pgfpathlineto{\pgfqpoint{1.808303in}{1.468149in}}%
\pgfpathlineto{\pgfqpoint{1.852990in}{1.229501in}}%
\pgfpathlineto{\pgfqpoint{1.872851in}{1.134480in}}%
\pgfpathlineto{\pgfqpoint{1.890229in}{1.061308in}}%
\pgfpathlineto{\pgfqpoint{1.905125in}{1.007748in}}%
\pgfpathlineto{\pgfqpoint{1.917538in}{0.970462in}}%
\pgfpathlineto{\pgfqpoint{1.929951in}{0.940437in}}%
\pgfpathlineto{\pgfqpoint{1.939881in}{0.921930in}}%
\pgfpathlineto{\pgfqpoint{1.949812in}{0.908496in}}%
\pgfpathlineto{\pgfqpoint{1.957260in}{0.901815in}}%
\pgfpathlineto{\pgfqpoint{1.964707in}{0.898070in}}%
\pgfpathlineto{\pgfqpoint{1.972155in}{0.897269in}}%
\pgfpathlineto{\pgfqpoint{1.979603in}{0.899400in}}%
\pgfpathlineto{\pgfqpoint{1.987051in}{0.904437in}}%
\pgfpathlineto{\pgfqpoint{1.994499in}{0.912336in}}%
\pgfpathlineto{\pgfqpoint{2.001946in}{0.923037in}}%
\pgfpathlineto{\pgfqpoint{2.011877in}{0.941529in}}%
\pgfpathlineto{\pgfqpoint{2.021807in}{0.964641in}}%
\pgfpathlineto{\pgfqpoint{2.034220in}{0.999617in}}%
\pgfpathlineto{\pgfqpoint{2.049116in}{1.049685in}}%
\pgfpathlineto{\pgfqpoint{2.066494in}{1.117673in}}%
\pgfpathlineto{\pgfqpoint{2.086355in}{1.205223in}}%
\pgfpathlineto{\pgfqpoint{2.116146in}{1.348325in}}%
\pgfpathlineto{\pgfqpoint{2.160833in}{1.562942in}}%
\pgfpathlineto{\pgfqpoint{2.180694in}{1.648322in}}%
\pgfpathlineto{\pgfqpoint{2.198072in}{1.714029in}}%
\pgfpathlineto{\pgfqpoint{2.212968in}{1.762090in}}%
\pgfpathlineto{\pgfqpoint{2.225381in}{1.795519in}}%
\pgfpathlineto{\pgfqpoint{2.237794in}{1.822408in}}%
\pgfpathlineto{\pgfqpoint{2.247724in}{1.838954in}}%
\pgfpathlineto{\pgfqpoint{2.257655in}{1.850933in}}%
\pgfpathlineto{\pgfqpoint{2.265102in}{1.856862in}}%
\pgfpathlineto{\pgfqpoint{2.272550in}{1.860149in}}%
\pgfpathlineto{\pgfqpoint{2.279998in}{1.860788in}}%
\pgfpathlineto{\pgfqpoint{2.287446in}{1.858788in}}%
\pgfpathlineto{\pgfqpoint{2.294894in}{1.854174in}}%
\pgfpathlineto{\pgfqpoint{2.302341in}{1.846987in}}%
\pgfpathlineto{\pgfqpoint{2.312272in}{1.833498in}}%
\pgfpathlineto{\pgfqpoint{2.322202in}{1.815701in}}%
\pgfpathlineto{\pgfqpoint{2.334615in}{1.787724in}}%
\pgfpathlineto{\pgfqpoint{2.347028in}{1.753851in}}%
\pgfpathlineto{\pgfqpoint{2.361924in}{1.706262in}}%
\pgfpathlineto{\pgfqpoint{2.379302in}{1.642667in}}%
\pgfpathlineto{\pgfqpoint{2.401646in}{1.551380in}}%
\pgfpathlineto{\pgfqpoint{2.443850in}{1.366221in}}%
\pgfpathlineto{\pgfqpoint{2.473641in}{1.240908in}}%
\pgfpathlineto{\pgfqpoint{2.493502in}{1.166411in}}%
\pgfpathlineto{\pgfqpoint{2.510880in}{1.109923in}}%
\pgfpathlineto{\pgfqpoint{2.525776in}{1.069311in}}%
\pgfpathlineto{\pgfqpoint{2.538189in}{1.041641in}}%
\pgfpathlineto{\pgfqpoint{2.548119in}{1.023837in}}%
\pgfpathlineto{\pgfqpoint{2.558050in}{1.010056in}}%
\pgfpathlineto{\pgfqpoint{2.567980in}{1.000411in}}%
\pgfpathlineto{\pgfqpoint{2.575428in}{0.995936in}}%
\pgfpathlineto{\pgfqpoint{2.582876in}{0.993843in}}%
\pgfpathlineto{\pgfqpoint{2.590323in}{0.994131in}}%
\pgfpathlineto{\pgfqpoint{2.597771in}{0.996786in}}%
\pgfpathlineto{\pgfqpoint{2.605219in}{1.001781in}}%
\pgfpathlineto{\pgfqpoint{2.612667in}{1.009076in}}%
\pgfpathlineto{\pgfqpoint{2.622597in}{1.022284in}}%
\pgfpathlineto{\pgfqpoint{2.632528in}{1.039320in}}%
\pgfpathlineto{\pgfqpoint{2.644941in}{1.065688in}}%
\pgfpathlineto{\pgfqpoint{2.657354in}{1.097251in}}%
\pgfpathlineto{\pgfqpoint{2.672249in}{1.141206in}}%
\pgfpathlineto{\pgfqpoint{2.689628in}{1.199490in}}%
\pgfpathlineto{\pgfqpoint{2.711971in}{1.282537in}}%
\pgfpathlineto{\pgfqpoint{2.791414in}{1.586781in}}%
\pgfpathlineto{\pgfqpoint{2.808792in}{1.641932in}}%
\pgfpathlineto{\pgfqpoint{2.823688in}{1.682828in}}%
\pgfpathlineto{\pgfqpoint{2.838584in}{1.716901in}}%
\pgfpathlineto{\pgfqpoint{2.850997in}{1.739563in}}%
\pgfpathlineto{\pgfqpoint{2.860927in}{1.753705in}}%
\pgfpathlineto{\pgfqpoint{2.870858in}{1.764169in}}%
\pgfpathlineto{\pgfqpoint{2.880788in}{1.770873in}}%
\pgfpathlineto{\pgfqpoint{2.888236in}{1.773406in}}%
\pgfpathlineto{\pgfqpoint{2.895684in}{1.773795in}}%
\pgfpathlineto{\pgfqpoint{2.903131in}{1.772049in}}%
\pgfpathlineto{\pgfqpoint{2.910579in}{1.768189in}}%
\pgfpathlineto{\pgfqpoint{2.918027in}{1.762247in}}%
\pgfpathlineto{\pgfqpoint{2.927957in}{1.751167in}}%
\pgfpathlineto{\pgfqpoint{2.937888in}{1.736606in}}%
\pgfpathlineto{\pgfqpoint{2.950301in}{1.713777in}}%
\pgfpathlineto{\pgfqpoint{2.962714in}{1.686191in}}%
\pgfpathlineto{\pgfqpoint{2.977610in}{1.647493in}}%
\pgfpathlineto{\pgfqpoint{2.994988in}{1.595846in}}%
\pgfpathlineto{\pgfqpoint{3.017331in}{1.521802in}}%
\pgfpathlineto{\pgfqpoint{3.066983in}{1.345679in}}%
\pgfpathlineto{\pgfqpoint{3.091809in}{1.262636in}}%
\pgfpathlineto{\pgfqpoint{3.111670in}{1.203462in}}%
\pgfpathlineto{\pgfqpoint{3.129048in}{1.158990in}}%
\pgfpathlineto{\pgfqpoint{3.143944in}{1.127354in}}%
\pgfpathlineto{\pgfqpoint{3.156357in}{1.106080in}}%
\pgfpathlineto{\pgfqpoint{3.166287in}{1.092616in}}%
\pgfpathlineto{\pgfqpoint{3.176218in}{1.082438in}}%
\pgfpathlineto{\pgfqpoint{3.186148in}{1.075631in}}%
\pgfpathlineto{\pgfqpoint{3.193596in}{1.072766in}}%
\pgfpathlineto{\pgfqpoint{3.201044in}{1.071830in}}%
\pgfpathlineto{\pgfqpoint{3.208492in}{1.072819in}}%
\pgfpathlineto{\pgfqpoint{3.215939in}{1.075718in}}%
\pgfpathlineto{\pgfqpoint{3.225870in}{1.082505in}}%
\pgfpathlineto{\pgfqpoint{3.225870in}{1.082505in}}%
\pgfusepath{stroke}%
\end{pgfscope}%
\begin{pgfscope}%
\pgfpathrectangle{\pgfqpoint{0.619136in}{0.571603in}}{\pgfqpoint{2.730864in}{1.657828in}}%
\pgfusepath{clip}%
\pgfsetrectcap%
\pgfsetroundjoin%
\pgfsetlinewidth{1.505625pt}%
\definecolor{currentstroke}{rgb}{0.498039,0.498039,0.498039}%
\pgfsetstrokecolor{currentstroke}%
\pgfsetdash{}{0pt}%
\pgfpathmoveto{\pgfqpoint{0.743267in}{0.646959in}}%
\pgfpathlineto{\pgfqpoint{0.748232in}{0.649364in}}%
\pgfpathlineto{\pgfqpoint{0.753197in}{0.655799in}}%
\pgfpathlineto{\pgfqpoint{0.758162in}{0.665859in}}%
\pgfpathlineto{\pgfqpoint{0.765610in}{0.687249in}}%
\pgfpathlineto{\pgfqpoint{0.773058in}{0.715652in}}%
\pgfpathlineto{\pgfqpoint{0.782988in}{0.763457in}}%
\pgfpathlineto{\pgfqpoint{0.795401in}{0.837302in}}%
\pgfpathlineto{\pgfqpoint{0.810297in}{0.943003in}}%
\pgfpathlineto{\pgfqpoint{0.827675in}{1.083648in}}%
\pgfpathlineto{\pgfqpoint{0.862432in}{1.389516in}}%
\pgfpathlineto{\pgfqpoint{0.887258in}{1.599228in}}%
\pgfpathlineto{\pgfqpoint{0.904636in}{1.728755in}}%
\pgfpathlineto{\pgfqpoint{0.919531in}{1.823272in}}%
\pgfpathlineto{\pgfqpoint{0.931944in}{1.888144in}}%
\pgfpathlineto{\pgfqpoint{0.941875in}{1.930013in}}%
\pgfpathlineto{\pgfqpoint{0.951805in}{1.962455in}}%
\pgfpathlineto{\pgfqpoint{0.959253in}{1.980414in}}%
\pgfpathlineto{\pgfqpoint{0.966701in}{1.992835in}}%
\pgfpathlineto{\pgfqpoint{0.971666in}{1.998030in}}%
\pgfpathlineto{\pgfqpoint{0.976631in}{2.000768in}}%
\pgfpathlineto{\pgfqpoint{0.981596in}{2.001065in}}%
\pgfpathlineto{\pgfqpoint{0.986562in}{1.998952in}}%
\pgfpathlineto{\pgfqpoint{0.991527in}{1.994467in}}%
\pgfpathlineto{\pgfqpoint{0.998975in}{1.983398in}}%
\pgfpathlineto{\pgfqpoint{1.006423in}{1.967298in}}%
\pgfpathlineto{\pgfqpoint{1.013870in}{1.946408in}}%
\pgfpathlineto{\pgfqpoint{1.023801in}{1.911592in}}%
\pgfpathlineto{\pgfqpoint{1.036214in}{1.857973in}}%
\pgfpathlineto{\pgfqpoint{1.051109in}{1.781088in}}%
\pgfpathlineto{\pgfqpoint{1.068488in}{1.678286in}}%
\pgfpathlineto{\pgfqpoint{1.098279in}{1.485005in}}%
\pgfpathlineto{\pgfqpoint{1.128070in}{1.295182in}}%
\pgfpathlineto{\pgfqpoint{1.145448in}{1.196758in}}%
\pgfpathlineto{\pgfqpoint{1.160344in}{1.123900in}}%
\pgfpathlineto{\pgfqpoint{1.172757in}{1.073030in}}%
\pgfpathlineto{\pgfqpoint{1.185170in}{1.032185in}}%
\pgfpathlineto{\pgfqpoint{1.195100in}{1.007215in}}%
\pgfpathlineto{\pgfqpoint{1.202548in}{0.993144in}}%
\pgfpathlineto{\pgfqpoint{1.209996in}{0.983125in}}%
\pgfpathlineto{\pgfqpoint{1.217444in}{0.977172in}}%
\pgfpathlineto{\pgfqpoint{1.222409in}{0.975453in}}%
\pgfpathlineto{\pgfqpoint{1.227374in}{0.975515in}}%
\pgfpathlineto{\pgfqpoint{1.232339in}{0.977336in}}%
\pgfpathlineto{\pgfqpoint{1.239787in}{0.983297in}}%
\pgfpathlineto{\pgfqpoint{1.247235in}{0.993017in}}%
\pgfpathlineto{\pgfqpoint{1.254683in}{1.006335in}}%
\pgfpathlineto{\pgfqpoint{1.264613in}{1.029356in}}%
\pgfpathlineto{\pgfqpoint{1.274544in}{1.057905in}}%
\pgfpathlineto{\pgfqpoint{1.286957in}{1.100438in}}%
\pgfpathlineto{\pgfqpoint{1.301852in}{1.159765in}}%
\pgfpathlineto{\pgfqpoint{1.321713in}{1.248799in}}%
\pgfpathlineto{\pgfqpoint{1.381296in}{1.524481in}}%
\pgfpathlineto{\pgfqpoint{1.398674in}{1.591151in}}%
\pgfpathlineto{\pgfqpoint{1.413569in}{1.638827in}}%
\pgfpathlineto{\pgfqpoint{1.425982in}{1.670789in}}%
\pgfpathlineto{\pgfqpoint{1.435913in}{1.690839in}}%
\pgfpathlineto{\pgfqpoint{1.445843in}{1.705759in}}%
\pgfpathlineto{\pgfqpoint{1.453291in}{1.713512in}}%
\pgfpathlineto{\pgfqpoint{1.460739in}{1.718298in}}%
\pgfpathlineto{\pgfqpoint{1.468187in}{1.720132in}}%
\pgfpathlineto{\pgfqpoint{1.475635in}{1.719050in}}%
\pgfpathlineto{\pgfqpoint{1.483082in}{1.715120in}}%
\pgfpathlineto{\pgfqpoint{1.490530in}{1.708430in}}%
\pgfpathlineto{\pgfqpoint{1.497978in}{1.699094in}}%
\pgfpathlineto{\pgfqpoint{1.507908in}{1.682767in}}%
\pgfpathlineto{\pgfqpoint{1.517839in}{1.662357in}}%
\pgfpathlineto{\pgfqpoint{1.530252in}{1.631766in}}%
\pgfpathlineto{\pgfqpoint{1.545147in}{1.588878in}}%
\pgfpathlineto{\pgfqpoint{1.565008in}{1.524202in}}%
\pgfpathlineto{\pgfqpoint{1.629556in}{1.307378in}}%
\pgfpathlineto{\pgfqpoint{1.644452in}{1.266552in}}%
\pgfpathlineto{\pgfqpoint{1.659347in}{1.232273in}}%
\pgfpathlineto{\pgfqpoint{1.671760in}{1.209433in}}%
\pgfpathlineto{\pgfqpoint{1.681691in}{1.195215in}}%
\pgfpathlineto{\pgfqpoint{1.691621in}{1.184756in}}%
\pgfpathlineto{\pgfqpoint{1.699069in}{1.179426in}}%
\pgfpathlineto{\pgfqpoint{1.706517in}{1.176261in}}%
\pgfpathlineto{\pgfqpoint{1.713964in}{1.175249in}}%
\pgfpathlineto{\pgfqpoint{1.721412in}{1.176358in}}%
\pgfpathlineto{\pgfqpoint{1.728860in}{1.179538in}}%
\pgfpathlineto{\pgfqpoint{1.736308in}{1.184719in}}%
\pgfpathlineto{\pgfqpoint{1.746238in}{1.194593in}}%
\pgfpathlineto{\pgfqpoint{1.756169in}{1.207625in}}%
\pgfpathlineto{\pgfqpoint{1.768582in}{1.227912in}}%
\pgfpathlineto{\pgfqpoint{1.783477in}{1.257242in}}%
\pgfpathlineto{\pgfqpoint{1.800856in}{1.296720in}}%
\pgfpathlineto{\pgfqpoint{1.828164in}{1.365103in}}%
\pgfpathlineto{\pgfqpoint{1.862921in}{1.451148in}}%
\pgfpathlineto{\pgfqpoint{1.880299in}{1.488991in}}%
\pgfpathlineto{\pgfqpoint{1.895194in}{1.516879in}}%
\pgfpathlineto{\pgfqpoint{1.907607in}{1.536250in}}%
\pgfpathlineto{\pgfqpoint{1.920021in}{1.551696in}}%
\pgfpathlineto{\pgfqpoint{1.929951in}{1.561043in}}%
\pgfpathlineto{\pgfqpoint{1.939881in}{1.567622in}}%
\pgfpathlineto{\pgfqpoint{1.949812in}{1.571395in}}%
\pgfpathlineto{\pgfqpoint{1.959742in}{1.572372in}}%
\pgfpathlineto{\pgfqpoint{1.969673in}{1.570609in}}%
\pgfpathlineto{\pgfqpoint{1.979603in}{1.566201in}}%
\pgfpathlineto{\pgfqpoint{1.989533in}{1.559288in}}%
\pgfpathlineto{\pgfqpoint{2.001946in}{1.547394in}}%
\pgfpathlineto{\pgfqpoint{2.014359in}{1.532275in}}%
\pgfpathlineto{\pgfqpoint{2.029255in}{1.510559in}}%
\pgfpathlineto{\pgfqpoint{2.049116in}{1.477093in}}%
\pgfpathlineto{\pgfqpoint{2.088838in}{1.403788in}}%
\pgfpathlineto{\pgfqpoint{2.113664in}{1.360594in}}%
\pgfpathlineto{\pgfqpoint{2.131042in}{1.334329in}}%
\pgfpathlineto{\pgfqpoint{2.145937in}{1.315439in}}%
\pgfpathlineto{\pgfqpoint{2.158350in}{1.302686in}}%
\pgfpathlineto{\pgfqpoint{2.170763in}{1.292905in}}%
\pgfpathlineto{\pgfqpoint{2.183176in}{1.286249in}}%
\pgfpathlineto{\pgfqpoint{2.193107in}{1.283224in}}%
\pgfpathlineto{\pgfqpoint{2.203037in}{1.282243in}}%
\pgfpathlineto{\pgfqpoint{2.212968in}{1.283270in}}%
\pgfpathlineto{\pgfqpoint{2.222898in}{1.286237in}}%
\pgfpathlineto{\pgfqpoint{2.235311in}{1.292524in}}%
\pgfpathlineto{\pgfqpoint{2.247724in}{1.301439in}}%
\pgfpathlineto{\pgfqpoint{2.262620in}{1.315171in}}%
\pgfpathlineto{\pgfqpoint{2.279998in}{1.334579in}}%
\pgfpathlineto{\pgfqpoint{2.302341in}{1.363159in}}%
\pgfpathlineto{\pgfqpoint{2.359441in}{1.438146in}}%
\pgfpathlineto{\pgfqpoint{2.379302in}{1.459540in}}%
\pgfpathlineto{\pgfqpoint{2.394198in}{1.472638in}}%
\pgfpathlineto{\pgfqpoint{2.409093in}{1.482780in}}%
\pgfpathlineto{\pgfqpoint{2.421506in}{1.488786in}}%
\pgfpathlineto{\pgfqpoint{2.433919in}{1.492477in}}%
\pgfpathlineto{\pgfqpoint{2.446332in}{1.493835in}}%
\pgfpathlineto{\pgfqpoint{2.458745in}{1.492898in}}%
\pgfpathlineto{\pgfqpoint{2.471158in}{1.489761in}}%
\pgfpathlineto{\pgfqpoint{2.483571in}{1.484573in}}%
\pgfpathlineto{\pgfqpoint{2.498467in}{1.475918in}}%
\pgfpathlineto{\pgfqpoint{2.515845in}{1.463001in}}%
\pgfpathlineto{\pgfqpoint{2.538189in}{1.443129in}}%
\pgfpathlineto{\pgfqpoint{2.617632in}{1.368936in}}%
\pgfpathlineto{\pgfqpoint{2.635010in}{1.357005in}}%
\pgfpathlineto{\pgfqpoint{2.652388in}{1.347923in}}%
\pgfpathlineto{\pgfqpoint{2.667284in}{1.342654in}}%
\pgfpathlineto{\pgfqpoint{2.682180in}{1.339820in}}%
\pgfpathlineto{\pgfqpoint{2.697075in}{1.339431in}}%
\pgfpathlineto{\pgfqpoint{2.711971in}{1.341410in}}%
\pgfpathlineto{\pgfqpoint{2.726867in}{1.345595in}}%
\pgfpathlineto{\pgfqpoint{2.744245in}{1.352949in}}%
\pgfpathlineto{\pgfqpoint{2.764106in}{1.364021in}}%
\pgfpathlineto{\pgfqpoint{2.788932in}{1.380600in}}%
\pgfpathlineto{\pgfqpoint{2.855962in}{1.426837in}}%
\pgfpathlineto{\pgfqpoint{2.875823in}{1.437392in}}%
\pgfpathlineto{\pgfqpoint{2.893201in}{1.444488in}}%
\pgfpathlineto{\pgfqpoint{2.910579in}{1.449310in}}%
\pgfpathlineto{\pgfqpoint{2.927957in}{1.451722in}}%
\pgfpathlineto{\pgfqpoint{2.945336in}{1.451711in}}%
\pgfpathlineto{\pgfqpoint{2.962714in}{1.449383in}}%
\pgfpathlineto{\pgfqpoint{2.982575in}{1.444166in}}%
\pgfpathlineto{\pgfqpoint{3.004918in}{1.435609in}}%
\pgfpathlineto{\pgfqpoint{3.032227in}{1.422496in}}%
\pgfpathlineto{\pgfqpoint{3.104222in}{1.386519in}}%
\pgfpathlineto{\pgfqpoint{3.126566in}{1.378351in}}%
\pgfpathlineto{\pgfqpoint{3.146427in}{1.373219in}}%
\pgfpathlineto{\pgfqpoint{3.166287in}{1.370332in}}%
\pgfpathlineto{\pgfqpoint{3.186148in}{1.369761in}}%
\pgfpathlineto{\pgfqpoint{3.206009in}{1.371427in}}%
\pgfpathlineto{\pgfqpoint{3.225870in}{1.375113in}}%
\pgfpathlineto{\pgfqpoint{3.225870in}{1.375113in}}%
\pgfusepath{stroke}%
\end{pgfscope}%
\begin{pgfscope}%
\pgfpathrectangle{\pgfqpoint{0.619136in}{0.571603in}}{\pgfqpoint{2.730864in}{1.657828in}}%
\pgfusepath{clip}%
\pgfsetrectcap%
\pgfsetroundjoin%
\pgfsetlinewidth{1.505625pt}%
\definecolor{currentstroke}{rgb}{0.737255,0.741176,0.133333}%
\pgfsetstrokecolor{currentstroke}%
\pgfsetdash{}{0pt}%
\pgfpathmoveto{\pgfqpoint{0.743267in}{0.646959in}}%
\pgfpathlineto{\pgfqpoint{0.755680in}{0.819399in}}%
\pgfpathlineto{\pgfqpoint{0.765610in}{0.929805in}}%
\pgfpathlineto{\pgfqpoint{0.775540in}{1.019815in}}%
\pgfpathlineto{\pgfqpoint{0.785471in}{1.093055in}}%
\pgfpathlineto{\pgfqpoint{0.795401in}{1.152577in}}%
\pgfpathlineto{\pgfqpoint{0.805332in}{1.200905in}}%
\pgfpathlineto{\pgfqpoint{0.815262in}{1.240114in}}%
\pgfpathlineto{\pgfqpoint{0.825192in}{1.271902in}}%
\pgfpathlineto{\pgfqpoint{0.835123in}{1.297658in}}%
\pgfpathlineto{\pgfqpoint{0.847536in}{1.323073in}}%
\pgfpathlineto{\pgfqpoint{0.859949in}{1.342575in}}%
\pgfpathlineto{\pgfqpoint{0.872362in}{1.357525in}}%
\pgfpathlineto{\pgfqpoint{0.884775in}{1.368973in}}%
\pgfpathlineto{\pgfqpoint{0.897188in}{1.377729in}}%
\pgfpathlineto{\pgfqpoint{0.912084in}{1.385551in}}%
\pgfpathlineto{\pgfqpoint{0.929462in}{1.391976in}}%
\pgfpathlineto{\pgfqpoint{0.949323in}{1.396856in}}%
\pgfpathlineto{\pgfqpoint{0.974149in}{1.400575in}}%
\pgfpathlineto{\pgfqpoint{1.008905in}{1.403245in}}%
\pgfpathlineto{\pgfqpoint{1.061040in}{1.404701in}}%
\pgfpathlineto{\pgfqpoint{1.175240in}{1.405045in}}%
\pgfpathlineto{\pgfqpoint{2.716936in}{1.404425in}}%
\pgfpathlineto{\pgfqpoint{3.225870in}{1.404395in}}%
\pgfpathlineto{\pgfqpoint{3.225870in}{1.404395in}}%
\pgfusepath{stroke}%
\end{pgfscope}%
\begin{pgfscope}%
\pgfpathrectangle{\pgfqpoint{0.619136in}{0.571603in}}{\pgfqpoint{2.730864in}{1.657828in}}%
\pgfusepath{clip}%
\pgfsetrectcap%
\pgfsetroundjoin%
\pgfsetlinewidth{1.505625pt}%
\definecolor{currentstroke}{rgb}{0.090196,0.745098,0.811765}%
\pgfsetstrokecolor{currentstroke}%
\pgfsetdash{}{0pt}%
\pgfpathmoveto{\pgfqpoint{0.743267in}{0.646959in}}%
\pgfpathlineto{\pgfqpoint{0.748232in}{0.648537in}}%
\pgfpathlineto{\pgfqpoint{0.753197in}{0.652731in}}%
\pgfpathlineto{\pgfqpoint{0.760645in}{0.663377in}}%
\pgfpathlineto{\pgfqpoint{0.768093in}{0.678853in}}%
\pgfpathlineto{\pgfqpoint{0.778023in}{0.706423in}}%
\pgfpathlineto{\pgfqpoint{0.787953in}{0.741287in}}%
\pgfpathlineto{\pgfqpoint{0.800366in}{0.794106in}}%
\pgfpathlineto{\pgfqpoint{0.815262in}{0.869298in}}%
\pgfpathlineto{\pgfqpoint{0.832640in}{0.970316in}}%
\pgfpathlineto{\pgfqpoint{0.854984in}{1.115224in}}%
\pgfpathlineto{\pgfqpoint{0.934427in}{1.646986in}}%
\pgfpathlineto{\pgfqpoint{0.951805in}{1.743721in}}%
\pgfpathlineto{\pgfqpoint{0.966701in}{1.815951in}}%
\pgfpathlineto{\pgfqpoint{0.981596in}{1.876896in}}%
\pgfpathlineto{\pgfqpoint{0.994009in}{1.918308in}}%
\pgfpathlineto{\pgfqpoint{1.003940in}{1.944976in}}%
\pgfpathlineto{\pgfqpoint{1.013870in}{1.965733in}}%
\pgfpathlineto{\pgfqpoint{1.021318in}{1.977367in}}%
\pgfpathlineto{\pgfqpoint{1.028766in}{1.985611in}}%
\pgfpathlineto{\pgfqpoint{1.036214in}{1.990470in}}%
\pgfpathlineto{\pgfqpoint{1.043662in}{1.991970in}}%
\pgfpathlineto{\pgfqpoint{1.051109in}{1.990154in}}%
\pgfpathlineto{\pgfqpoint{1.058557in}{1.985083in}}%
\pgfpathlineto{\pgfqpoint{1.066005in}{1.976837in}}%
\pgfpathlineto{\pgfqpoint{1.073453in}{1.965512in}}%
\pgfpathlineto{\pgfqpoint{1.083383in}{1.945819in}}%
\pgfpathlineto{\pgfqpoint{1.093314in}{1.921161in}}%
\pgfpathlineto{\pgfqpoint{1.105727in}{1.883893in}}%
\pgfpathlineto{\pgfqpoint{1.120622in}{1.830748in}}%
\pgfpathlineto{\pgfqpoint{1.138000in}{1.759017in}}%
\pgfpathlineto{\pgfqpoint{1.160344in}{1.655379in}}%
\pgfpathlineto{\pgfqpoint{1.197583in}{1.468998in}}%
\pgfpathlineto{\pgfqpoint{1.229857in}{1.311548in}}%
\pgfpathlineto{\pgfqpoint{1.252200in}{1.213730in}}%
\pgfpathlineto{\pgfqpoint{1.269578in}{1.147356in}}%
\pgfpathlineto{\pgfqpoint{1.284474in}{1.098673in}}%
\pgfpathlineto{\pgfqpoint{1.296887in}{1.064554in}}%
\pgfpathlineto{\pgfqpoint{1.309300in}{1.036708in}}%
\pgfpathlineto{\pgfqpoint{1.319231in}{1.019130in}}%
\pgfpathlineto{\pgfqpoint{1.329161in}{1.005827in}}%
\pgfpathlineto{\pgfqpoint{1.339091in}{0.996842in}}%
\pgfpathlineto{\pgfqpoint{1.346539in}{0.992937in}}%
\pgfpathlineto{\pgfqpoint{1.353987in}{0.991443in}}%
\pgfpathlineto{\pgfqpoint{1.361435in}{0.992330in}}%
\pgfpathlineto{\pgfqpoint{1.368883in}{0.995554in}}%
\pgfpathlineto{\pgfqpoint{1.376330in}{1.001061in}}%
\pgfpathlineto{\pgfqpoint{1.386261in}{1.011832in}}%
\pgfpathlineto{\pgfqpoint{1.396191in}{1.026346in}}%
\pgfpathlineto{\pgfqpoint{1.408604in}{1.049403in}}%
\pgfpathlineto{\pgfqpoint{1.421017in}{1.077443in}}%
\pgfpathlineto{\pgfqpoint{1.435913in}{1.116860in}}%
\pgfpathlineto{\pgfqpoint{1.453291in}{1.169412in}}%
\pgfpathlineto{\pgfqpoint{1.475635in}{1.244474in}}%
\pgfpathlineto{\pgfqpoint{1.555078in}{1.519776in}}%
\pgfpathlineto{\pgfqpoint{1.574939in}{1.576769in}}%
\pgfpathlineto{\pgfqpoint{1.589834in}{1.613576in}}%
\pgfpathlineto{\pgfqpoint{1.604730in}{1.644543in}}%
\pgfpathlineto{\pgfqpoint{1.617143in}{1.665513in}}%
\pgfpathlineto{\pgfqpoint{1.629556in}{1.681856in}}%
\pgfpathlineto{\pgfqpoint{1.639486in}{1.691506in}}%
\pgfpathlineto{\pgfqpoint{1.649417in}{1.698075in}}%
\pgfpathlineto{\pgfqpoint{1.659347in}{1.701562in}}%
\pgfpathlineto{\pgfqpoint{1.666795in}{1.702171in}}%
\pgfpathlineto{\pgfqpoint{1.674243in}{1.701088in}}%
\pgfpathlineto{\pgfqpoint{1.684173in}{1.697067in}}%
\pgfpathlineto{\pgfqpoint{1.694104in}{1.690199in}}%
\pgfpathlineto{\pgfqpoint{1.704034in}{1.680612in}}%
\pgfpathlineto{\pgfqpoint{1.716447in}{1.665042in}}%
\pgfpathlineto{\pgfqpoint{1.728860in}{1.645817in}}%
\pgfpathlineto{\pgfqpoint{1.743756in}{1.618484in}}%
\pgfpathlineto{\pgfqpoint{1.761134in}{1.581689in}}%
\pgfpathlineto{\pgfqpoint{1.783477in}{1.528662in}}%
\pgfpathlineto{\pgfqpoint{1.828164in}{1.414585in}}%
\pgfpathlineto{\pgfqpoint{1.857955in}{1.342035in}}%
\pgfpathlineto{\pgfqpoint{1.877816in}{1.298937in}}%
\pgfpathlineto{\pgfqpoint{1.895194in}{1.266127in}}%
\pgfpathlineto{\pgfqpoint{1.910090in}{1.242315in}}%
\pgfpathlineto{\pgfqpoint{1.924986in}{1.222918in}}%
\pgfpathlineto{\pgfqpoint{1.937399in}{1.210344in}}%
\pgfpathlineto{\pgfqpoint{1.949812in}{1.201153in}}%
\pgfpathlineto{\pgfqpoint{1.959742in}{1.196275in}}%
\pgfpathlineto{\pgfqpoint{1.969673in}{1.193601in}}%
\pgfpathlineto{\pgfqpoint{1.979603in}{1.193110in}}%
\pgfpathlineto{\pgfqpoint{1.989533in}{1.194761in}}%
\pgfpathlineto{\pgfqpoint{1.999464in}{1.198491in}}%
\pgfpathlineto{\pgfqpoint{2.011877in}{1.205953in}}%
\pgfpathlineto{\pgfqpoint{2.024290in}{1.216335in}}%
\pgfpathlineto{\pgfqpoint{2.039185in}{1.232309in}}%
\pgfpathlineto{\pgfqpoint{2.054081in}{1.251659in}}%
\pgfpathlineto{\pgfqpoint{2.071459in}{1.277775in}}%
\pgfpathlineto{\pgfqpoint{2.093803in}{1.315499in}}%
\pgfpathlineto{\pgfqpoint{2.138490in}{1.396891in}}%
\pgfpathlineto{\pgfqpoint{2.168281in}{1.448798in}}%
\pgfpathlineto{\pgfqpoint{2.190624in}{1.483274in}}%
\pgfpathlineto{\pgfqpoint{2.208002in}{1.506317in}}%
\pgfpathlineto{\pgfqpoint{2.222898in}{1.522945in}}%
\pgfpathlineto{\pgfqpoint{2.237794in}{1.536390in}}%
\pgfpathlineto{\pgfqpoint{2.250207in}{1.545015in}}%
\pgfpathlineto{\pgfqpoint{2.262620in}{1.551214in}}%
\pgfpathlineto{\pgfqpoint{2.275033in}{1.554955in}}%
\pgfpathlineto{\pgfqpoint{2.287446in}{1.556244in}}%
\pgfpathlineto{\pgfqpoint{2.299859in}{1.555126in}}%
\pgfpathlineto{\pgfqpoint{2.312272in}{1.551682in}}%
\pgfpathlineto{\pgfqpoint{2.324685in}{1.546027in}}%
\pgfpathlineto{\pgfqpoint{2.339580in}{1.536532in}}%
\pgfpathlineto{\pgfqpoint{2.354476in}{1.524374in}}%
\pgfpathlineto{\pgfqpoint{2.371854in}{1.507304in}}%
\pgfpathlineto{\pgfqpoint{2.394198in}{1.481779in}}%
\pgfpathlineto{\pgfqpoint{2.423989in}{1.443880in}}%
\pgfpathlineto{\pgfqpoint{2.476124in}{1.377065in}}%
\pgfpathlineto{\pgfqpoint{2.498467in}{1.351929in}}%
\pgfpathlineto{\pgfqpoint{2.518328in}{1.332768in}}%
\pgfpathlineto{\pgfqpoint{2.535706in}{1.318987in}}%
\pgfpathlineto{\pgfqpoint{2.550602in}{1.309650in}}%
\pgfpathlineto{\pgfqpoint{2.565497in}{1.302745in}}%
\pgfpathlineto{\pgfqpoint{2.580393in}{1.298351in}}%
\pgfpathlineto{\pgfqpoint{2.595289in}{1.296487in}}%
\pgfpathlineto{\pgfqpoint{2.610184in}{1.297113in}}%
\pgfpathlineto{\pgfqpoint{2.625080in}{1.300138in}}%
\pgfpathlineto{\pgfqpoint{2.639975in}{1.305416in}}%
\pgfpathlineto{\pgfqpoint{2.657354in}{1.314167in}}%
\pgfpathlineto{\pgfqpoint{2.674732in}{1.325354in}}%
\pgfpathlineto{\pgfqpoint{2.697075in}{1.342619in}}%
\pgfpathlineto{\pgfqpoint{2.726867in}{1.369014in}}%
\pgfpathlineto{\pgfqpoint{2.796379in}{1.432009in}}%
\pgfpathlineto{\pgfqpoint{2.818723in}{1.448917in}}%
\pgfpathlineto{\pgfqpoint{2.838584in}{1.461431in}}%
\pgfpathlineto{\pgfqpoint{2.855962in}{1.470109in}}%
\pgfpathlineto{\pgfqpoint{2.873340in}{1.476466in}}%
\pgfpathlineto{\pgfqpoint{2.890718in}{1.480396in}}%
\pgfpathlineto{\pgfqpoint{2.908097in}{1.481867in}}%
\pgfpathlineto{\pgfqpoint{2.925475in}{1.480925in}}%
\pgfpathlineto{\pgfqpoint{2.942853in}{1.477688in}}%
\pgfpathlineto{\pgfqpoint{2.960231in}{1.472339in}}%
\pgfpathlineto{\pgfqpoint{2.980092in}{1.463951in}}%
\pgfpathlineto{\pgfqpoint{3.002436in}{1.452140in}}%
\pgfpathlineto{\pgfqpoint{3.032227in}{1.433681in}}%
\pgfpathlineto{\pgfqpoint{3.114153in}{1.381067in}}%
\pgfpathlineto{\pgfqpoint{3.138979in}{1.368436in}}%
\pgfpathlineto{\pgfqpoint{3.161322in}{1.359518in}}%
\pgfpathlineto{\pgfqpoint{3.181183in}{1.353827in}}%
\pgfpathlineto{\pgfqpoint{3.201044in}{1.350388in}}%
\pgfpathlineto{\pgfqpoint{3.220905in}{1.349241in}}%
\pgfpathlineto{\pgfqpoint{3.225870in}{1.349309in}}%
\pgfpathlineto{\pgfqpoint{3.225870in}{1.349309in}}%
\pgfusepath{stroke}%
\end{pgfscope}%
\begin{pgfscope}%
\pgfpathrectangle{\pgfqpoint{0.619136in}{0.571603in}}{\pgfqpoint{2.730864in}{1.657828in}}%
\pgfusepath{clip}%
\pgfsetrectcap%
\pgfsetroundjoin%
\pgfsetlinewidth{1.505625pt}%
\definecolor{currentstroke}{rgb}{0.121569,0.466667,0.705882}%
\pgfsetstrokecolor{currentstroke}%
\pgfsetdash{}{0pt}%
\pgfpathmoveto{\pgfqpoint{0.743267in}{0.646959in}}%
\pgfpathlineto{\pgfqpoint{0.745749in}{0.653694in}}%
\pgfpathlineto{\pgfqpoint{0.750714in}{0.679224in}}%
\pgfpathlineto{\pgfqpoint{0.758162in}{0.732245in}}%
\pgfpathlineto{\pgfqpoint{0.770575in}{0.840461in}}%
\pgfpathlineto{\pgfqpoint{0.812779in}{1.226331in}}%
\pgfpathlineto{\pgfqpoint{0.827675in}{1.336535in}}%
\pgfpathlineto{\pgfqpoint{0.840088in}{1.412509in}}%
\pgfpathlineto{\pgfqpoint{0.852501in}{1.473563in}}%
\pgfpathlineto{\pgfqpoint{0.862432in}{1.511921in}}%
\pgfpathlineto{\pgfqpoint{0.872362in}{1.541482in}}%
\pgfpathlineto{\pgfqpoint{0.882292in}{1.562923in}}%
\pgfpathlineto{\pgfqpoint{0.889740in}{1.574151in}}%
\pgfpathlineto{\pgfqpoint{0.897188in}{1.581623in}}%
\pgfpathlineto{\pgfqpoint{0.904636in}{1.585718in}}%
\pgfpathlineto{\pgfqpoint{0.912084in}{1.586817in}}%
\pgfpathlineto{\pgfqpoint{0.919531in}{1.585301in}}%
\pgfpathlineto{\pgfqpoint{0.926979in}{1.581540in}}%
\pgfpathlineto{\pgfqpoint{0.936910in}{1.573648in}}%
\pgfpathlineto{\pgfqpoint{0.949323in}{1.560264in}}%
\pgfpathlineto{\pgfqpoint{0.966701in}{1.537408in}}%
\pgfpathlineto{\pgfqpoint{1.018836in}{1.465949in}}%
\pgfpathlineto{\pgfqpoint{1.036214in}{1.446834in}}%
\pgfpathlineto{\pgfqpoint{1.051109in}{1.433325in}}%
\pgfpathlineto{\pgfqpoint{1.066005in}{1.422531in}}%
\pgfpathlineto{\pgfqpoint{1.080901in}{1.414332in}}%
\pgfpathlineto{\pgfqpoint{1.095796in}{1.408488in}}%
\pgfpathlineto{\pgfqpoint{1.113174in}{1.404226in}}%
\pgfpathlineto{\pgfqpoint{1.130553in}{1.402185in}}%
\pgfpathlineto{\pgfqpoint{1.152896in}{1.401929in}}%
\pgfpathlineto{\pgfqpoint{1.182687in}{1.404027in}}%
\pgfpathlineto{\pgfqpoint{1.277026in}{1.412216in}}%
\pgfpathlineto{\pgfqpoint{1.324196in}{1.413165in}}%
\pgfpathlineto{\pgfqpoint{1.391226in}{1.411965in}}%
\pgfpathlineto{\pgfqpoint{1.547630in}{1.408763in}}%
\pgfpathlineto{\pgfqpoint{1.805821in}{1.407407in}}%
\pgfpathlineto{\pgfqpoint{2.409093in}{1.405906in}}%
\pgfpathlineto{\pgfqpoint{3.225870in}{1.405197in}}%
\pgfpathlineto{\pgfqpoint{3.225870in}{1.405197in}}%
\pgfusepath{stroke}%
\end{pgfscope}%
\begin{pgfscope}%
\pgfpathrectangle{\pgfqpoint{0.619136in}{0.571603in}}{\pgfqpoint{2.730864in}{1.657828in}}%
\pgfusepath{clip}%
\pgfsetrectcap%
\pgfsetroundjoin%
\pgfsetlinewidth{1.505625pt}%
\definecolor{currentstroke}{rgb}{1.000000,0.498039,0.054902}%
\pgfsetstrokecolor{currentstroke}%
\pgfsetdash{}{0pt}%
\pgfpathmoveto{\pgfqpoint{0.743267in}{0.646959in}}%
\pgfpathlineto{\pgfqpoint{0.748232in}{0.649261in}}%
\pgfpathlineto{\pgfqpoint{0.753197in}{0.655267in}}%
\pgfpathlineto{\pgfqpoint{0.758162in}{0.664535in}}%
\pgfpathlineto{\pgfqpoint{0.765610in}{0.684042in}}%
\pgfpathlineto{\pgfqpoint{0.773058in}{0.709740in}}%
\pgfpathlineto{\pgfqpoint{0.782988in}{0.752730in}}%
\pgfpathlineto{\pgfqpoint{0.795401in}{0.818835in}}%
\pgfpathlineto{\pgfqpoint{0.810297in}{0.913259in}}%
\pgfpathlineto{\pgfqpoint{0.827675in}{1.039092in}}%
\pgfpathlineto{\pgfqpoint{0.854984in}{1.255716in}}%
\pgfpathlineto{\pgfqpoint{0.889740in}{1.529834in}}%
\pgfpathlineto{\pgfqpoint{0.909601in}{1.669202in}}%
\pgfpathlineto{\pgfqpoint{0.924497in}{1.759901in}}%
\pgfpathlineto{\pgfqpoint{0.936910in}{1.824484in}}%
\pgfpathlineto{\pgfqpoint{0.949323in}{1.877920in}}%
\pgfpathlineto{\pgfqpoint{0.959253in}{1.912108in}}%
\pgfpathlineto{\pgfqpoint{0.969183in}{1.938392in}}%
\pgfpathlineto{\pgfqpoint{0.976631in}{1.952824in}}%
\pgfpathlineto{\pgfqpoint{0.984079in}{1.962706in}}%
\pgfpathlineto{\pgfqpoint{0.991527in}{1.968056in}}%
\pgfpathlineto{\pgfqpoint{0.996492in}{1.969130in}}%
\pgfpathlineto{\pgfqpoint{1.001457in}{1.968236in}}%
\pgfpathlineto{\pgfqpoint{1.006423in}{1.965406in}}%
\pgfpathlineto{\pgfqpoint{1.013870in}{1.957614in}}%
\pgfpathlineto{\pgfqpoint{1.021318in}{1.945703in}}%
\pgfpathlineto{\pgfqpoint{1.028766in}{1.929853in}}%
\pgfpathlineto{\pgfqpoint{1.038696in}{1.902956in}}%
\pgfpathlineto{\pgfqpoint{1.051109in}{1.860874in}}%
\pgfpathlineto{\pgfqpoint{1.063522in}{1.810558in}}%
\pgfpathlineto{\pgfqpoint{1.078418in}{1.741259in}}%
\pgfpathlineto{\pgfqpoint{1.098279in}{1.637963in}}%
\pgfpathlineto{\pgfqpoint{1.167792in}{1.265236in}}%
\pgfpathlineto{\pgfqpoint{1.185170in}{1.189164in}}%
\pgfpathlineto{\pgfqpoint{1.200066in}{1.134237in}}%
\pgfpathlineto{\pgfqpoint{1.212479in}{1.096718in}}%
\pgfpathlineto{\pgfqpoint{1.222409in}{1.072506in}}%
\pgfpathlineto{\pgfqpoint{1.232339in}{1.053660in}}%
\pgfpathlineto{\pgfqpoint{1.239787in}{1.043113in}}%
\pgfpathlineto{\pgfqpoint{1.247235in}{1.035660in}}%
\pgfpathlineto{\pgfqpoint{1.254683in}{1.031293in}}%
\pgfpathlineto{\pgfqpoint{1.262131in}{1.029978in}}%
\pgfpathlineto{\pgfqpoint{1.269578in}{1.031658in}}%
\pgfpathlineto{\pgfqpoint{1.277026in}{1.036255in}}%
\pgfpathlineto{\pgfqpoint{1.284474in}{1.043668in}}%
\pgfpathlineto{\pgfqpoint{1.294404in}{1.057722in}}%
\pgfpathlineto{\pgfqpoint{1.304335in}{1.076232in}}%
\pgfpathlineto{\pgfqpoint{1.316748in}{1.105041in}}%
\pgfpathlineto{\pgfqpoint{1.331644in}{1.146787in}}%
\pgfpathlineto{\pgfqpoint{1.349022in}{1.203297in}}%
\pgfpathlineto{\pgfqpoint{1.373848in}{1.293233in}}%
\pgfpathlineto{\pgfqpoint{1.421017in}{1.466238in}}%
\pgfpathlineto{\pgfqpoint{1.440878in}{1.529605in}}%
\pgfpathlineto{\pgfqpoint{1.455774in}{1.570465in}}%
\pgfpathlineto{\pgfqpoint{1.468187in}{1.599304in}}%
\pgfpathlineto{\pgfqpoint{1.480600in}{1.622917in}}%
\pgfpathlineto{\pgfqpoint{1.490530in}{1.637820in}}%
\pgfpathlineto{\pgfqpoint{1.500461in}{1.649063in}}%
\pgfpathlineto{\pgfqpoint{1.510391in}{1.656593in}}%
\pgfpathlineto{\pgfqpoint{1.517839in}{1.659804in}}%
\pgfpathlineto{\pgfqpoint{1.525287in}{1.660946in}}%
\pgfpathlineto{\pgfqpoint{1.532734in}{1.660057in}}%
\pgfpathlineto{\pgfqpoint{1.540182in}{1.657186in}}%
\pgfpathlineto{\pgfqpoint{1.550113in}{1.650392in}}%
\pgfpathlineto{\pgfqpoint{1.560043in}{1.640386in}}%
\pgfpathlineto{\pgfqpoint{1.569973in}{1.627400in}}%
\pgfpathlineto{\pgfqpoint{1.582386in}{1.607387in}}%
\pgfpathlineto{\pgfqpoint{1.597282in}{1.578606in}}%
\pgfpathlineto{\pgfqpoint{1.614660in}{1.539884in}}%
\pgfpathlineto{\pgfqpoint{1.639486in}{1.478605in}}%
\pgfpathlineto{\pgfqpoint{1.684173in}{1.367311in}}%
\pgfpathlineto{\pgfqpoint{1.704034in}{1.323979in}}%
\pgfpathlineto{\pgfqpoint{1.721412in}{1.291678in}}%
\pgfpathlineto{\pgfqpoint{1.736308in}{1.269049in}}%
\pgfpathlineto{\pgfqpoint{1.748721in}{1.254150in}}%
\pgfpathlineto{\pgfqpoint{1.761134in}{1.243061in}}%
\pgfpathlineto{\pgfqpoint{1.771064in}{1.237004in}}%
\pgfpathlineto{\pgfqpoint{1.780995in}{1.233463in}}%
\pgfpathlineto{\pgfqpoint{1.790925in}{1.232416in}}%
\pgfpathlineto{\pgfqpoint{1.800856in}{1.233805in}}%
\pgfpathlineto{\pgfqpoint{1.810786in}{1.237539in}}%
\pgfpathlineto{\pgfqpoint{1.820716in}{1.243498in}}%
\pgfpathlineto{\pgfqpoint{1.833129in}{1.253850in}}%
\pgfpathlineto{\pgfqpoint{1.845542in}{1.267106in}}%
\pgfpathlineto{\pgfqpoint{1.860438in}{1.286291in}}%
\pgfpathlineto{\pgfqpoint{1.880299in}{1.316185in}}%
\pgfpathlineto{\pgfqpoint{1.907607in}{1.362038in}}%
\pgfpathlineto{\pgfqpoint{1.949812in}{1.432769in}}%
\pgfpathlineto{\pgfqpoint{1.969673in}{1.461781in}}%
\pgfpathlineto{\pgfqpoint{1.987051in}{1.483299in}}%
\pgfpathlineto{\pgfqpoint{2.001946in}{1.498283in}}%
\pgfpathlineto{\pgfqpoint{2.014359in}{1.508072in}}%
\pgfpathlineto{\pgfqpoint{2.026772in}{1.515274in}}%
\pgfpathlineto{\pgfqpoint{2.039185in}{1.519825in}}%
\pgfpathlineto{\pgfqpoint{2.051598in}{1.521721in}}%
\pgfpathlineto{\pgfqpoint{2.064011in}{1.521018in}}%
\pgfpathlineto{\pgfqpoint{2.076425in}{1.517824in}}%
\pgfpathlineto{\pgfqpoint{2.088838in}{1.512298in}}%
\pgfpathlineto{\pgfqpoint{2.103733in}{1.502876in}}%
\pgfpathlineto{\pgfqpoint{2.118629in}{1.490815in}}%
\pgfpathlineto{\pgfqpoint{2.138490in}{1.471478in}}%
\pgfpathlineto{\pgfqpoint{2.165798in}{1.441007in}}%
\pgfpathlineto{\pgfqpoint{2.217933in}{1.381965in}}%
\pgfpathlineto{\pgfqpoint{2.237794in}{1.362856in}}%
\pgfpathlineto{\pgfqpoint{2.255172in}{1.348884in}}%
\pgfpathlineto{\pgfqpoint{2.270068in}{1.339321in}}%
\pgfpathlineto{\pgfqpoint{2.284963in}{1.332201in}}%
\pgfpathlineto{\pgfqpoint{2.299859in}{1.327640in}}%
\pgfpathlineto{\pgfqpoint{2.314754in}{1.325669in}}%
\pgfpathlineto{\pgfqpoint{2.329650in}{1.326237in}}%
\pgfpathlineto{\pgfqpoint{2.344546in}{1.329217in}}%
\pgfpathlineto{\pgfqpoint{2.359441in}{1.334413in}}%
\pgfpathlineto{\pgfqpoint{2.376820in}{1.342930in}}%
\pgfpathlineto{\pgfqpoint{2.396680in}{1.355291in}}%
\pgfpathlineto{\pgfqpoint{2.423989in}{1.375365in}}%
\pgfpathlineto{\pgfqpoint{2.488537in}{1.424370in}}%
\pgfpathlineto{\pgfqpoint{2.510880in}{1.437992in}}%
\pgfpathlineto{\pgfqpoint{2.530741in}{1.447459in}}%
\pgfpathlineto{\pgfqpoint{2.548119in}{1.453386in}}%
\pgfpathlineto{\pgfqpoint{2.565497in}{1.456965in}}%
\pgfpathlineto{\pgfqpoint{2.582876in}{1.458161in}}%
\pgfpathlineto{\pgfqpoint{2.600254in}{1.457043in}}%
\pgfpathlineto{\pgfqpoint{2.617632in}{1.453775in}}%
\pgfpathlineto{\pgfqpoint{2.637493in}{1.447723in}}%
\pgfpathlineto{\pgfqpoint{2.659836in}{1.438533in}}%
\pgfpathlineto{\pgfqpoint{2.689628in}{1.423703in}}%
\pgfpathlineto{\pgfqpoint{2.756658in}{1.389496in}}%
\pgfpathlineto{\pgfqpoint{2.781484in}{1.379739in}}%
\pgfpathlineto{\pgfqpoint{2.803827in}{1.373329in}}%
\pgfpathlineto{\pgfqpoint{2.823688in}{1.369776in}}%
\pgfpathlineto{\pgfqpoint{2.843549in}{1.368326in}}%
\pgfpathlineto{\pgfqpoint{2.863410in}{1.368939in}}%
\pgfpathlineto{\pgfqpoint{2.885753in}{1.371903in}}%
\pgfpathlineto{\pgfqpoint{2.910579in}{1.377590in}}%
\pgfpathlineto{\pgfqpoint{2.940370in}{1.386825in}}%
\pgfpathlineto{\pgfqpoint{3.039675in}{1.419823in}}%
\pgfpathlineto{\pgfqpoint{3.066983in}{1.425437in}}%
\pgfpathlineto{\pgfqpoint{3.091809in}{1.428317in}}%
\pgfpathlineto{\pgfqpoint{3.116635in}{1.428976in}}%
\pgfpathlineto{\pgfqpoint{3.141461in}{1.427516in}}%
\pgfpathlineto{\pgfqpoint{3.171253in}{1.423362in}}%
\pgfpathlineto{\pgfqpoint{3.206009in}{1.416140in}}%
\pgfpathlineto{\pgfqpoint{3.225870in}{1.411382in}}%
\pgfpathlineto{\pgfqpoint{3.225870in}{1.411382in}}%
\pgfusepath{stroke}%
\end{pgfscope}%
\begin{pgfscope}%
\pgfpathrectangle{\pgfqpoint{0.619136in}{0.571603in}}{\pgfqpoint{2.730864in}{1.657828in}}%
\pgfusepath{clip}%
\pgfsetrectcap%
\pgfsetroundjoin%
\pgfsetlinewidth{1.505625pt}%
\definecolor{currentstroke}{rgb}{0.172549,0.627451,0.172549}%
\pgfsetstrokecolor{currentstroke}%
\pgfsetdash{}{0pt}%
\pgfpathmoveto{\pgfqpoint{0.743267in}{0.646959in}}%
\pgfpathlineto{\pgfqpoint{0.748232in}{0.648579in}}%
\pgfpathlineto{\pgfqpoint{0.753197in}{0.652779in}}%
\pgfpathlineto{\pgfqpoint{0.760645in}{0.663271in}}%
\pgfpathlineto{\pgfqpoint{0.768093in}{0.678353in}}%
\pgfpathlineto{\pgfqpoint{0.778023in}{0.704981in}}%
\pgfpathlineto{\pgfqpoint{0.787953in}{0.738407in}}%
\pgfpathlineto{\pgfqpoint{0.800366in}{0.788741in}}%
\pgfpathlineto{\pgfqpoint{0.815262in}{0.860011in}}%
\pgfpathlineto{\pgfqpoint{0.832640in}{0.955355in}}%
\pgfpathlineto{\pgfqpoint{0.854984in}{1.091751in}}%
\pgfpathlineto{\pgfqpoint{0.939392in}{1.623208in}}%
\pgfpathlineto{\pgfqpoint{0.956770in}{1.713908in}}%
\pgfpathlineto{\pgfqpoint{0.971666in}{1.782043in}}%
\pgfpathlineto{\pgfqpoint{0.986562in}{1.840132in}}%
\pgfpathlineto{\pgfqpoint{0.998975in}{1.880241in}}%
\pgfpathlineto{\pgfqpoint{1.011388in}{1.912418in}}%
\pgfpathlineto{\pgfqpoint{1.021318in}{1.932291in}}%
\pgfpathlineto{\pgfqpoint{1.031249in}{1.946878in}}%
\pgfpathlineto{\pgfqpoint{1.038696in}{1.954344in}}%
\pgfpathlineto{\pgfqpoint{1.046144in}{1.958845in}}%
\pgfpathlineto{\pgfqpoint{1.053592in}{1.960412in}}%
\pgfpathlineto{\pgfqpoint{1.061040in}{1.959091in}}%
\pgfpathlineto{\pgfqpoint{1.068488in}{1.954939in}}%
\pgfpathlineto{\pgfqpoint{1.075935in}{1.948032in}}%
\pgfpathlineto{\pgfqpoint{1.083383in}{1.938453in}}%
\pgfpathlineto{\pgfqpoint{1.093314in}{1.921701in}}%
\pgfpathlineto{\pgfqpoint{1.103244in}{1.900643in}}%
\pgfpathlineto{\pgfqpoint{1.115657in}{1.868726in}}%
\pgfpathlineto{\pgfqpoint{1.130553in}{1.823086in}}%
\pgfpathlineto{\pgfqpoint{1.147931in}{1.761289in}}%
\pgfpathlineto{\pgfqpoint{1.167792in}{1.682005in}}%
\pgfpathlineto{\pgfqpoint{1.197583in}{1.552828in}}%
\pgfpathlineto{\pgfqpoint{1.244752in}{1.348157in}}%
\pgfpathlineto{\pgfqpoint{1.267096in}{1.261287in}}%
\pgfpathlineto{\pgfqpoint{1.284474in}{1.201557in}}%
\pgfpathlineto{\pgfqpoint{1.299370in}{1.156982in}}%
\pgfpathlineto{\pgfqpoint{1.314265in}{1.119256in}}%
\pgfpathlineto{\pgfqpoint{1.326678in}{1.093433in}}%
\pgfpathlineto{\pgfqpoint{1.339091in}{1.072944in}}%
\pgfpathlineto{\pgfqpoint{1.349022in}{1.060484in}}%
\pgfpathlineto{\pgfqpoint{1.358952in}{1.051549in}}%
\pgfpathlineto{\pgfqpoint{1.368883in}{1.046134in}}%
\pgfpathlineto{\pgfqpoint{1.376330in}{1.044361in}}%
\pgfpathlineto{\pgfqpoint{1.383778in}{1.044519in}}%
\pgfpathlineto{\pgfqpoint{1.391226in}{1.046572in}}%
\pgfpathlineto{\pgfqpoint{1.401156in}{1.052174in}}%
\pgfpathlineto{\pgfqpoint{1.411087in}{1.060926in}}%
\pgfpathlineto{\pgfqpoint{1.421017in}{1.072667in}}%
\pgfpathlineto{\pgfqpoint{1.433430in}{1.091265in}}%
\pgfpathlineto{\pgfqpoint{1.445843in}{1.113841in}}%
\pgfpathlineto{\pgfqpoint{1.460739in}{1.145545in}}%
\pgfpathlineto{\pgfqpoint{1.478117in}{1.187819in}}%
\pgfpathlineto{\pgfqpoint{1.500461in}{1.248308in}}%
\pgfpathlineto{\pgfqpoint{1.545147in}{1.377955in}}%
\pgfpathlineto{\pgfqpoint{1.574939in}{1.460850in}}%
\pgfpathlineto{\pgfqpoint{1.597282in}{1.516472in}}%
\pgfpathlineto{\pgfqpoint{1.614660in}{1.554201in}}%
\pgfpathlineto{\pgfqpoint{1.629556in}{1.581965in}}%
\pgfpathlineto{\pgfqpoint{1.644452in}{1.605058in}}%
\pgfpathlineto{\pgfqpoint{1.656865in}{1.620510in}}%
\pgfpathlineto{\pgfqpoint{1.669278in}{1.632389in}}%
\pgfpathlineto{\pgfqpoint{1.679208in}{1.639276in}}%
\pgfpathlineto{\pgfqpoint{1.689138in}{1.643829in}}%
\pgfpathlineto{\pgfqpoint{1.699069in}{1.646064in}}%
\pgfpathlineto{\pgfqpoint{1.708999in}{1.646018in}}%
\pgfpathlineto{\pgfqpoint{1.718930in}{1.643747in}}%
\pgfpathlineto{\pgfqpoint{1.728860in}{1.639328in}}%
\pgfpathlineto{\pgfqpoint{1.741273in}{1.630928in}}%
\pgfpathlineto{\pgfqpoint{1.753686in}{1.619540in}}%
\pgfpathlineto{\pgfqpoint{1.768582in}{1.602293in}}%
\pgfpathlineto{\pgfqpoint{1.783477in}{1.581615in}}%
\pgfpathlineto{\pgfqpoint{1.800856in}{1.553898in}}%
\pgfpathlineto{\pgfqpoint{1.823199in}{1.514039in}}%
\pgfpathlineto{\pgfqpoint{1.865403in}{1.432826in}}%
\pgfpathlineto{\pgfqpoint{1.897677in}{1.372677in}}%
\pgfpathlineto{\pgfqpoint{1.920021in}{1.335328in}}%
\pgfpathlineto{\pgfqpoint{1.939881in}{1.306522in}}%
\pgfpathlineto{\pgfqpoint{1.957260in}{1.285413in}}%
\pgfpathlineto{\pgfqpoint{1.972155in}{1.270710in}}%
\pgfpathlineto{\pgfqpoint{1.987051in}{1.259337in}}%
\pgfpathlineto{\pgfqpoint{1.999464in}{1.252486in}}%
\pgfpathlineto{\pgfqpoint{2.011877in}{1.248050in}}%
\pgfpathlineto{\pgfqpoint{2.024290in}{1.246020in}}%
\pgfpathlineto{\pgfqpoint{2.036703in}{1.246348in}}%
\pgfpathlineto{\pgfqpoint{2.049116in}{1.248957in}}%
\pgfpathlineto{\pgfqpoint{2.061529in}{1.253739in}}%
\pgfpathlineto{\pgfqpoint{2.076425in}{1.262152in}}%
\pgfpathlineto{\pgfqpoint{2.091320in}{1.273211in}}%
\pgfpathlineto{\pgfqpoint{2.108698in}{1.289021in}}%
\pgfpathlineto{\pgfqpoint{2.128559in}{1.310208in}}%
\pgfpathlineto{\pgfqpoint{2.155868in}{1.343098in}}%
\pgfpathlineto{\pgfqpoint{2.232829in}{1.438602in}}%
\pgfpathlineto{\pgfqpoint{2.255172in}{1.461781in}}%
\pgfpathlineto{\pgfqpoint{2.275033in}{1.479140in}}%
\pgfpathlineto{\pgfqpoint{2.292411in}{1.491422in}}%
\pgfpathlineto{\pgfqpoint{2.309789in}{1.500746in}}%
\pgfpathlineto{\pgfqpoint{2.324685in}{1.506282in}}%
\pgfpathlineto{\pgfqpoint{2.339580in}{1.509514in}}%
\pgfpathlineto{\pgfqpoint{2.354476in}{1.510462in}}%
\pgfpathlineto{\pgfqpoint{2.369372in}{1.509190in}}%
\pgfpathlineto{\pgfqpoint{2.384267in}{1.505806in}}%
\pgfpathlineto{\pgfqpoint{2.401646in}{1.499390in}}%
\pgfpathlineto{\pgfqpoint{2.419024in}{1.490598in}}%
\pgfpathlineto{\pgfqpoint{2.438885in}{1.478090in}}%
\pgfpathlineto{\pgfqpoint{2.463711in}{1.459632in}}%
\pgfpathlineto{\pgfqpoint{2.500950in}{1.428642in}}%
\pgfpathlineto{\pgfqpoint{2.548119in}{1.389647in}}%
\pgfpathlineto{\pgfqpoint{2.575428in}{1.370028in}}%
\pgfpathlineto{\pgfqpoint{2.597771in}{1.356673in}}%
\pgfpathlineto{\pgfqpoint{2.617632in}{1.347266in}}%
\pgfpathlineto{\pgfqpoint{2.637493in}{1.340421in}}%
\pgfpathlineto{\pgfqpoint{2.654871in}{1.336632in}}%
\pgfpathlineto{\pgfqpoint{2.672249in}{1.334918in}}%
\pgfpathlineto{\pgfqpoint{2.689628in}{1.335236in}}%
\pgfpathlineto{\pgfqpoint{2.709488in}{1.337964in}}%
\pgfpathlineto{\pgfqpoint{2.729349in}{1.342995in}}%
\pgfpathlineto{\pgfqpoint{2.751693in}{1.351035in}}%
\pgfpathlineto{\pgfqpoint{2.776519in}{1.362322in}}%
\pgfpathlineto{\pgfqpoint{2.811275in}{1.380807in}}%
\pgfpathlineto{\pgfqpoint{2.883271in}{1.419902in}}%
\pgfpathlineto{\pgfqpoint{2.910579in}{1.432030in}}%
\pgfpathlineto{\pgfqpoint{2.935405in}{1.440744in}}%
\pgfpathlineto{\pgfqpoint{2.957749in}{1.446392in}}%
\pgfpathlineto{\pgfqpoint{2.980092in}{1.449807in}}%
\pgfpathlineto{\pgfqpoint{3.002436in}{1.450954in}}%
\pgfpathlineto{\pgfqpoint{3.024779in}{1.449905in}}%
\pgfpathlineto{\pgfqpoint{3.049605in}{1.446373in}}%
\pgfpathlineto{\pgfqpoint{3.076914in}{1.440023in}}%
\pgfpathlineto{\pgfqpoint{3.109187in}{1.430006in}}%
\pgfpathlineto{\pgfqpoint{3.161322in}{1.410950in}}%
\pgfpathlineto{\pgfqpoint{3.210974in}{1.393523in}}%
\pgfpathlineto{\pgfqpoint{3.225870in}{1.389028in}}%
\pgfpathlineto{\pgfqpoint{3.225870in}{1.389028in}}%
\pgfusepath{stroke}%
\end{pgfscope}%
\begin{pgfscope}%
\pgfpathrectangle{\pgfqpoint{0.619136in}{0.571603in}}{\pgfqpoint{2.730864in}{1.657828in}}%
\pgfusepath{clip}%
\pgfsetrectcap%
\pgfsetroundjoin%
\pgfsetlinewidth{1.505625pt}%
\definecolor{currentstroke}{rgb}{0.839216,0.152941,0.156863}%
\pgfsetstrokecolor{currentstroke}%
\pgfsetdash{}{0pt}%
\pgfpathmoveto{\pgfqpoint{0.743267in}{0.646959in}}%
\pgfpathlineto{\pgfqpoint{0.765610in}{0.937953in}}%
\pgfpathlineto{\pgfqpoint{0.775540in}{1.037368in}}%
\pgfpathlineto{\pgfqpoint{0.785471in}{1.118694in}}%
\pgfpathlineto{\pgfqpoint{0.795401in}{1.184472in}}%
\pgfpathlineto{\pgfqpoint{0.805332in}{1.237203in}}%
\pgfpathlineto{\pgfqpoint{0.815262in}{1.279147in}}%
\pgfpathlineto{\pgfqpoint{0.825192in}{1.312266in}}%
\pgfpathlineto{\pgfqpoint{0.835123in}{1.338227in}}%
\pgfpathlineto{\pgfqpoint{0.845053in}{1.358424in}}%
\pgfpathlineto{\pgfqpoint{0.854984in}{1.374008in}}%
\pgfpathlineto{\pgfqpoint{0.864914in}{1.385923in}}%
\pgfpathlineto{\pgfqpoint{0.874845in}{1.394937in}}%
\pgfpathlineto{\pgfqpoint{0.887258in}{1.403061in}}%
\pgfpathlineto{\pgfqpoint{0.899671in}{1.408564in}}%
\pgfpathlineto{\pgfqpoint{0.914566in}{1.412708in}}%
\pgfpathlineto{\pgfqpoint{0.931944in}{1.415261in}}%
\pgfpathlineto{\pgfqpoint{0.956770in}{1.416405in}}%
\pgfpathlineto{\pgfqpoint{0.996492in}{1.415547in}}%
\pgfpathlineto{\pgfqpoint{1.167792in}{1.409809in}}%
\pgfpathlineto{\pgfqpoint{1.329161in}{1.407834in}}%
\pgfpathlineto{\pgfqpoint{1.666795in}{1.406321in}}%
\pgfpathlineto{\pgfqpoint{2.508397in}{1.405258in}}%
\pgfpathlineto{\pgfqpoint{3.225870in}{1.404952in}}%
\pgfpathlineto{\pgfqpoint{3.225870in}{1.404952in}}%
\pgfusepath{stroke}%
\end{pgfscope}%
\begin{pgfscope}%
\pgfpathrectangle{\pgfqpoint{0.619136in}{0.571603in}}{\pgfqpoint{2.730864in}{1.657828in}}%
\pgfusepath{clip}%
\pgfsetrectcap%
\pgfsetroundjoin%
\pgfsetlinewidth{1.505625pt}%
\definecolor{currentstroke}{rgb}{0.580392,0.403922,0.741176}%
\pgfsetstrokecolor{currentstroke}%
\pgfsetdash{}{0pt}%
\pgfpathmoveto{\pgfqpoint{0.743267in}{0.646959in}}%
\pgfpathlineto{\pgfqpoint{0.745749in}{0.648613in}}%
\pgfpathlineto{\pgfqpoint{0.750714in}{0.657272in}}%
\pgfpathlineto{\pgfqpoint{0.755680in}{0.671035in}}%
\pgfpathlineto{\pgfqpoint{0.763127in}{0.699210in}}%
\pgfpathlineto{\pgfqpoint{0.773058in}{0.748046in}}%
\pgfpathlineto{\pgfqpoint{0.785471in}{0.822962in}}%
\pgfpathlineto{\pgfqpoint{0.802849in}{0.945473in}}%
\pgfpathlineto{\pgfqpoint{0.837605in}{1.214394in}}%
\pgfpathlineto{\pgfqpoint{0.862432in}{1.396647in}}%
\pgfpathlineto{\pgfqpoint{0.879810in}{1.508412in}}%
\pgfpathlineto{\pgfqpoint{0.894705in}{1.590241in}}%
\pgfpathlineto{\pgfqpoint{0.907118in}{1.647291in}}%
\pgfpathlineto{\pgfqpoint{0.919531in}{1.693611in}}%
\pgfpathlineto{\pgfqpoint{0.929462in}{1.722788in}}%
\pgfpathlineto{\pgfqpoint{0.939392in}{1.744989in}}%
\pgfpathlineto{\pgfqpoint{0.946840in}{1.757145in}}%
\pgfpathlineto{\pgfqpoint{0.954288in}{1.765558in}}%
\pgfpathlineto{\pgfqpoint{0.961736in}{1.770358in}}%
\pgfpathlineto{\pgfqpoint{0.969183in}{1.771700in}}%
\pgfpathlineto{\pgfqpoint{0.976631in}{1.769765in}}%
\pgfpathlineto{\pgfqpoint{0.984079in}{1.764753in}}%
\pgfpathlineto{\pgfqpoint{0.991527in}{1.756882in}}%
\pgfpathlineto{\pgfqpoint{1.001457in}{1.742342in}}%
\pgfpathlineto{\pgfqpoint{1.011388in}{1.723712in}}%
\pgfpathlineto{\pgfqpoint{1.023801in}{1.695591in}}%
\pgfpathlineto{\pgfqpoint{1.038696in}{1.656340in}}%
\pgfpathlineto{\pgfqpoint{1.061040in}{1.590465in}}%
\pgfpathlineto{\pgfqpoint{1.108209in}{1.449591in}}%
\pgfpathlineto{\pgfqpoint{1.128070in}{1.398543in}}%
\pgfpathlineto{\pgfqpoint{1.142966in}{1.365755in}}%
\pgfpathlineto{\pgfqpoint{1.157861in}{1.338382in}}%
\pgfpathlineto{\pgfqpoint{1.170274in}{1.319969in}}%
\pgfpathlineto{\pgfqpoint{1.182687in}{1.305636in}}%
\pgfpathlineto{\pgfqpoint{1.192618in}{1.297086in}}%
\pgfpathlineto{\pgfqpoint{1.202548in}{1.291061in}}%
\pgfpathlineto{\pgfqpoint{1.212479in}{1.287461in}}%
\pgfpathlineto{\pgfqpoint{1.222409in}{1.286154in}}%
\pgfpathlineto{\pgfqpoint{1.232339in}{1.286983in}}%
\pgfpathlineto{\pgfqpoint{1.242270in}{1.289766in}}%
\pgfpathlineto{\pgfqpoint{1.254683in}{1.295688in}}%
\pgfpathlineto{\pgfqpoint{1.269578in}{1.305829in}}%
\pgfpathlineto{\pgfqpoint{1.286957in}{1.320890in}}%
\pgfpathlineto{\pgfqpoint{1.309300in}{1.343524in}}%
\pgfpathlineto{\pgfqpoint{1.366400in}{1.403014in}}%
\pgfpathlineto{\pgfqpoint{1.386261in}{1.420130in}}%
\pgfpathlineto{\pgfqpoint{1.403639in}{1.432477in}}%
\pgfpathlineto{\pgfqpoint{1.421017in}{1.442070in}}%
\pgfpathlineto{\pgfqpoint{1.435913in}{1.448012in}}%
\pgfpathlineto{\pgfqpoint{1.450808in}{1.451867in}}%
\pgfpathlineto{\pgfqpoint{1.468187in}{1.453855in}}%
\pgfpathlineto{\pgfqpoint{1.485565in}{1.453382in}}%
\pgfpathlineto{\pgfqpoint{1.502943in}{1.450786in}}%
\pgfpathlineto{\pgfqpoint{1.525287in}{1.444970in}}%
\pgfpathlineto{\pgfqpoint{1.552595in}{1.435316in}}%
\pgfpathlineto{\pgfqpoint{1.637004in}{1.403646in}}%
\pgfpathlineto{\pgfqpoint{1.664312in}{1.396688in}}%
\pgfpathlineto{\pgfqpoint{1.689138in}{1.392563in}}%
\pgfpathlineto{\pgfqpoint{1.713964in}{1.390540in}}%
\pgfpathlineto{\pgfqpoint{1.741273in}{1.390502in}}%
\pgfpathlineto{\pgfqpoint{1.773547in}{1.392786in}}%
\pgfpathlineto{\pgfqpoint{1.820716in}{1.398735in}}%
\pgfpathlineto{\pgfqpoint{1.895194in}{1.408181in}}%
\pgfpathlineto{\pgfqpoint{1.937399in}{1.411195in}}%
\pgfpathlineto{\pgfqpoint{1.979603in}{1.412041in}}%
\pgfpathlineto{\pgfqpoint{2.031738in}{1.410704in}}%
\pgfpathlineto{\pgfqpoint{2.222898in}{1.403425in}}%
\pgfpathlineto{\pgfqpoint{2.309789in}{1.404178in}}%
\pgfpathlineto{\pgfqpoint{2.486054in}{1.406075in}}%
\pgfpathlineto{\pgfqpoint{3.225870in}{1.404806in}}%
\pgfpathlineto{\pgfqpoint{3.225870in}{1.404806in}}%
\pgfusepath{stroke}%
\end{pgfscope}%
\begin{pgfscope}%
\pgfpathrectangle{\pgfqpoint{0.619136in}{0.571603in}}{\pgfqpoint{2.730864in}{1.657828in}}%
\pgfusepath{clip}%
\pgfsetrectcap%
\pgfsetroundjoin%
\pgfsetlinewidth{1.505625pt}%
\definecolor{currentstroke}{rgb}{0.549020,0.337255,0.294118}%
\pgfsetstrokecolor{currentstroke}%
\pgfsetdash{}{0pt}%
\pgfpathmoveto{\pgfqpoint{0.743267in}{0.646959in}}%
\pgfpathlineto{\pgfqpoint{0.745749in}{0.655380in}}%
\pgfpathlineto{\pgfqpoint{0.750714in}{0.683708in}}%
\pgfpathlineto{\pgfqpoint{0.760645in}{0.758457in}}%
\pgfpathlineto{\pgfqpoint{0.785471in}{0.974457in}}%
\pgfpathlineto{\pgfqpoint{0.805332in}{1.137120in}}%
\pgfpathlineto{\pgfqpoint{0.820227in}{1.242189in}}%
\pgfpathlineto{\pgfqpoint{0.832640in}{1.316542in}}%
\pgfpathlineto{\pgfqpoint{0.845053in}{1.378553in}}%
\pgfpathlineto{\pgfqpoint{0.857466in}{1.428593in}}%
\pgfpathlineto{\pgfqpoint{0.867397in}{1.460537in}}%
\pgfpathlineto{\pgfqpoint{0.877327in}{1.485873in}}%
\pgfpathlineto{\pgfqpoint{0.887258in}{1.505214in}}%
\pgfpathlineto{\pgfqpoint{0.897188in}{1.519213in}}%
\pgfpathlineto{\pgfqpoint{0.907118in}{1.528529in}}%
\pgfpathlineto{\pgfqpoint{0.914566in}{1.532837in}}%
\pgfpathlineto{\pgfqpoint{0.922014in}{1.535145in}}%
\pgfpathlineto{\pgfqpoint{0.931944in}{1.535552in}}%
\pgfpathlineto{\pgfqpoint{0.941875in}{1.533424in}}%
\pgfpathlineto{\pgfqpoint{0.954288in}{1.527989in}}%
\pgfpathlineto{\pgfqpoint{0.969183in}{1.518570in}}%
\pgfpathlineto{\pgfqpoint{0.991527in}{1.501160in}}%
\pgfpathlineto{\pgfqpoint{1.038696in}{1.463528in}}%
\pgfpathlineto{\pgfqpoint{1.061040in}{1.448808in}}%
\pgfpathlineto{\pgfqpoint{1.080901in}{1.438135in}}%
\pgfpathlineto{\pgfqpoint{1.100761in}{1.429741in}}%
\pgfpathlineto{\pgfqpoint{1.123105in}{1.422785in}}%
\pgfpathlineto{\pgfqpoint{1.147931in}{1.417649in}}%
\pgfpathlineto{\pgfqpoint{1.175240in}{1.414393in}}%
\pgfpathlineto{\pgfqpoint{1.209996in}{1.412628in}}%
\pgfpathlineto{\pgfqpoint{1.267096in}{1.412381in}}%
\pgfpathlineto{\pgfqpoint{1.408604in}{1.412241in}}%
\pgfpathlineto{\pgfqpoint{1.867886in}{1.408030in}}%
\pgfpathlineto{\pgfqpoint{2.481089in}{1.406339in}}%
\pgfpathlineto{\pgfqpoint{3.225870in}{1.405545in}}%
\pgfpathlineto{\pgfqpoint{3.225870in}{1.405545in}}%
\pgfusepath{stroke}%
\end{pgfscope}%
\begin{pgfscope}%
\pgfpathrectangle{\pgfqpoint{0.619136in}{0.571603in}}{\pgfqpoint{2.730864in}{1.657828in}}%
\pgfusepath{clip}%
\pgfsetrectcap%
\pgfsetroundjoin%
\pgfsetlinewidth{1.505625pt}%
\definecolor{currentstroke}{rgb}{0.890196,0.466667,0.760784}%
\pgfsetstrokecolor{currentstroke}%
\pgfsetdash{}{0pt}%
\pgfpathmoveto{\pgfqpoint{0.743267in}{0.646959in}}%
\pgfpathlineto{\pgfqpoint{0.745749in}{0.648980in}}%
\pgfpathlineto{\pgfqpoint{0.750714in}{0.659264in}}%
\pgfpathlineto{\pgfqpoint{0.755680in}{0.675353in}}%
\pgfpathlineto{\pgfqpoint{0.763127in}{0.707872in}}%
\pgfpathlineto{\pgfqpoint{0.773058in}{0.763464in}}%
\pgfpathlineto{\pgfqpoint{0.785471in}{0.847435in}}%
\pgfpathlineto{\pgfqpoint{0.802849in}{0.982015in}}%
\pgfpathlineto{\pgfqpoint{0.854984in}{1.397200in}}%
\pgfpathlineto{\pgfqpoint{0.872362in}{1.511792in}}%
\pgfpathlineto{\pgfqpoint{0.887258in}{1.593698in}}%
\pgfpathlineto{\pgfqpoint{0.899671in}{1.649180in}}%
\pgfpathlineto{\pgfqpoint{0.909601in}{1.684871in}}%
\pgfpathlineto{\pgfqpoint{0.919531in}{1.712796in}}%
\pgfpathlineto{\pgfqpoint{0.926979in}{1.728713in}}%
\pgfpathlineto{\pgfqpoint{0.934427in}{1.740432in}}%
\pgfpathlineto{\pgfqpoint{0.941875in}{1.748096in}}%
\pgfpathlineto{\pgfqpoint{0.949323in}{1.751881in}}%
\pgfpathlineto{\pgfqpoint{0.954288in}{1.752350in}}%
\pgfpathlineto{\pgfqpoint{0.959253in}{1.751255in}}%
\pgfpathlineto{\pgfqpoint{0.966701in}{1.746836in}}%
\pgfpathlineto{\pgfqpoint{0.974149in}{1.739320in}}%
\pgfpathlineto{\pgfqpoint{0.984079in}{1.724959in}}%
\pgfpathlineto{\pgfqpoint{0.994009in}{1.706270in}}%
\pgfpathlineto{\pgfqpoint{1.006423in}{1.677899in}}%
\pgfpathlineto{\pgfqpoint{1.023801in}{1.631320in}}%
\pgfpathlineto{\pgfqpoint{1.053592in}{1.542549in}}%
\pgfpathlineto{\pgfqpoint{1.080901in}{1.463091in}}%
\pgfpathlineto{\pgfqpoint{1.100761in}{1.412241in}}%
\pgfpathlineto{\pgfqpoint{1.115657in}{1.379732in}}%
\pgfpathlineto{\pgfqpoint{1.130553in}{1.352813in}}%
\pgfpathlineto{\pgfqpoint{1.142966in}{1.334911in}}%
\pgfpathlineto{\pgfqpoint{1.155379in}{1.321191in}}%
\pgfpathlineto{\pgfqpoint{1.165309in}{1.313180in}}%
\pgfpathlineto{\pgfqpoint{1.175240in}{1.307713in}}%
\pgfpathlineto{\pgfqpoint{1.185170in}{1.304661in}}%
\pgfpathlineto{\pgfqpoint{1.195100in}{1.303864in}}%
\pgfpathlineto{\pgfqpoint{1.205031in}{1.305135in}}%
\pgfpathlineto{\pgfqpoint{1.217444in}{1.309311in}}%
\pgfpathlineto{\pgfqpoint{1.229857in}{1.315947in}}%
\pgfpathlineto{\pgfqpoint{1.244752in}{1.326504in}}%
\pgfpathlineto{\pgfqpoint{1.264613in}{1.343653in}}%
\pgfpathlineto{\pgfqpoint{1.339091in}{1.411600in}}%
\pgfpathlineto{\pgfqpoint{1.358952in}{1.424959in}}%
\pgfpathlineto{\pgfqpoint{1.376330in}{1.433901in}}%
\pgfpathlineto{\pgfqpoint{1.393709in}{1.440151in}}%
\pgfpathlineto{\pgfqpoint{1.411087in}{1.443756in}}%
\pgfpathlineto{\pgfqpoint{1.428465in}{1.444904in}}%
\pgfpathlineto{\pgfqpoint{1.445843in}{1.443888in}}%
\pgfpathlineto{\pgfqpoint{1.465704in}{1.440551in}}%
\pgfpathlineto{\pgfqpoint{1.490530in}{1.434027in}}%
\pgfpathlineto{\pgfqpoint{1.535217in}{1.419361in}}%
\pgfpathlineto{\pgfqpoint{1.574939in}{1.407272in}}%
\pgfpathlineto{\pgfqpoint{1.604730in}{1.400574in}}%
\pgfpathlineto{\pgfqpoint{1.632039in}{1.396719in}}%
\pgfpathlineto{\pgfqpoint{1.659347in}{1.395044in}}%
\pgfpathlineto{\pgfqpoint{1.691621in}{1.395469in}}%
\pgfpathlineto{\pgfqpoint{1.731343in}{1.398445in}}%
\pgfpathlineto{\pgfqpoint{1.852990in}{1.409177in}}%
\pgfpathlineto{\pgfqpoint{1.900160in}{1.410229in}}%
\pgfpathlineto{\pgfqpoint{1.957260in}{1.409194in}}%
\pgfpathlineto{\pgfqpoint{2.136007in}{1.404365in}}%
\pgfpathlineto{\pgfqpoint{2.250207in}{1.405142in}}%
\pgfpathlineto{\pgfqpoint{2.411576in}{1.405755in}}%
\pgfpathlineto{\pgfqpoint{2.952783in}{1.404976in}}%
\pgfpathlineto{\pgfqpoint{3.225870in}{1.404839in}}%
\pgfpathlineto{\pgfqpoint{3.225870in}{1.404839in}}%
\pgfusepath{stroke}%
\end{pgfscope}%
\begin{pgfscope}%
\pgfpathrectangle{\pgfqpoint{0.619136in}{0.571603in}}{\pgfqpoint{2.730864in}{1.657828in}}%
\pgfusepath{clip}%
\pgfsetrectcap%
\pgfsetroundjoin%
\pgfsetlinewidth{1.505625pt}%
\definecolor{currentstroke}{rgb}{0.498039,0.498039,0.498039}%
\pgfsetstrokecolor{currentstroke}%
\pgfsetdash{}{0pt}%
\pgfpathmoveto{\pgfqpoint{0.743267in}{0.646959in}}%
\pgfpathlineto{\pgfqpoint{0.745749in}{0.655864in}}%
\pgfpathlineto{\pgfqpoint{0.750714in}{0.685834in}}%
\pgfpathlineto{\pgfqpoint{0.760645in}{0.764762in}}%
\pgfpathlineto{\pgfqpoint{0.812779in}{1.213671in}}%
\pgfpathlineto{\pgfqpoint{0.825192in}{1.295682in}}%
\pgfpathlineto{\pgfqpoint{0.837605in}{1.364210in}}%
\pgfpathlineto{\pgfqpoint{0.850018in}{1.419488in}}%
\pgfpathlineto{\pgfqpoint{0.859949in}{1.454702in}}%
\pgfpathlineto{\pgfqpoint{0.869879in}{1.482542in}}%
\pgfpathlineto{\pgfqpoint{0.879810in}{1.503694in}}%
\pgfpathlineto{\pgfqpoint{0.889740in}{1.518898in}}%
\pgfpathlineto{\pgfqpoint{0.897188in}{1.526854in}}%
\pgfpathlineto{\pgfqpoint{0.904636in}{1.532205in}}%
\pgfpathlineto{\pgfqpoint{0.912084in}{1.535263in}}%
\pgfpathlineto{\pgfqpoint{0.919531in}{1.536323in}}%
\pgfpathlineto{\pgfqpoint{0.929462in}{1.535115in}}%
\pgfpathlineto{\pgfqpoint{0.939392in}{1.531475in}}%
\pgfpathlineto{\pgfqpoint{0.951805in}{1.524355in}}%
\pgfpathlineto{\pgfqpoint{0.969183in}{1.511259in}}%
\pgfpathlineto{\pgfqpoint{1.041179in}{1.452692in}}%
\pgfpathlineto{\pgfqpoint{1.061040in}{1.440757in}}%
\pgfpathlineto{\pgfqpoint{1.080901in}{1.431355in}}%
\pgfpathlineto{\pgfqpoint{1.100761in}{1.424318in}}%
\pgfpathlineto{\pgfqpoint{1.123105in}{1.418829in}}%
\pgfpathlineto{\pgfqpoint{1.147931in}{1.415105in}}%
\pgfpathlineto{\pgfqpoint{1.177722in}{1.412936in}}%
\pgfpathlineto{\pgfqpoint{1.222409in}{1.412224in}}%
\pgfpathlineto{\pgfqpoint{1.428465in}{1.411567in}}%
\pgfpathlineto{\pgfqpoint{1.694104in}{1.408713in}}%
\pgfpathlineto{\pgfqpoint{2.158350in}{1.406844in}}%
\pgfpathlineto{\pgfqpoint{3.131531in}{1.405533in}}%
\pgfpathlineto{\pgfqpoint{3.225870in}{1.405468in}}%
\pgfpathlineto{\pgfqpoint{3.225870in}{1.405468in}}%
\pgfusepath{stroke}%
\end{pgfscope}%
\begin{pgfscope}%
\pgfpathrectangle{\pgfqpoint{0.619136in}{0.571603in}}{\pgfqpoint{2.730864in}{1.657828in}}%
\pgfusepath{clip}%
\pgfsetrectcap%
\pgfsetroundjoin%
\pgfsetlinewidth{1.505625pt}%
\definecolor{currentstroke}{rgb}{0.737255,0.741176,0.133333}%
\pgfsetstrokecolor{currentstroke}%
\pgfsetdash{}{0pt}%
\pgfpathmoveto{\pgfqpoint{0.743267in}{0.646959in}}%
\pgfpathlineto{\pgfqpoint{0.745749in}{0.647907in}}%
\pgfpathlineto{\pgfqpoint{0.750714in}{0.653605in}}%
\pgfpathlineto{\pgfqpoint{0.755680in}{0.663360in}}%
\pgfpathlineto{\pgfqpoint{0.763127in}{0.684480in}}%
\pgfpathlineto{\pgfqpoint{0.770575in}{0.712427in}}%
\pgfpathlineto{\pgfqpoint{0.780506in}{0.758872in}}%
\pgfpathlineto{\pgfqpoint{0.792919in}{0.829308in}}%
\pgfpathlineto{\pgfqpoint{0.807814in}{0.927987in}}%
\pgfpathlineto{\pgfqpoint{0.827675in}{1.075444in}}%
\pgfpathlineto{\pgfqpoint{0.884775in}{1.509342in}}%
\pgfpathlineto{\pgfqpoint{0.902153in}{1.621021in}}%
\pgfpathlineto{\pgfqpoint{0.917049in}{1.702947in}}%
\pgfpathlineto{\pgfqpoint{0.929462in}{1.760051in}}%
\pgfpathlineto{\pgfqpoint{0.941875in}{1.806205in}}%
\pgfpathlineto{\pgfqpoint{0.951805in}{1.834943in}}%
\pgfpathlineto{\pgfqpoint{0.961736in}{1.856300in}}%
\pgfpathlineto{\pgfqpoint{0.969183in}{1.867483in}}%
\pgfpathlineto{\pgfqpoint{0.976631in}{1.874576in}}%
\pgfpathlineto{\pgfqpoint{0.981596in}{1.877071in}}%
\pgfpathlineto{\pgfqpoint{0.986562in}{1.877816in}}%
\pgfpathlineto{\pgfqpoint{0.991527in}{1.876849in}}%
\pgfpathlineto{\pgfqpoint{0.996492in}{1.874212in}}%
\pgfpathlineto{\pgfqpoint{1.003940in}{1.867235in}}%
\pgfpathlineto{\pgfqpoint{1.011388in}{1.856796in}}%
\pgfpathlineto{\pgfqpoint{1.021318in}{1.837851in}}%
\pgfpathlineto{\pgfqpoint{1.031249in}{1.813660in}}%
\pgfpathlineto{\pgfqpoint{1.043662in}{1.776954in}}%
\pgfpathlineto{\pgfqpoint{1.058557in}{1.725071in}}%
\pgfpathlineto{\pgfqpoint{1.078418in}{1.646329in}}%
\pgfpathlineto{\pgfqpoint{1.152896in}{1.339917in}}%
\pgfpathlineto{\pgfqpoint{1.170274in}{1.283130in}}%
\pgfpathlineto{\pgfqpoint{1.185170in}{1.242515in}}%
\pgfpathlineto{\pgfqpoint{1.197583in}{1.214966in}}%
\pgfpathlineto{\pgfqpoint{1.207513in}{1.197258in}}%
\pgfpathlineto{\pgfqpoint{1.217444in}{1.183483in}}%
\pgfpathlineto{\pgfqpoint{1.227374in}{1.173653in}}%
\pgfpathlineto{\pgfqpoint{1.234822in}{1.168843in}}%
\pgfpathlineto{\pgfqpoint{1.242270in}{1.166186in}}%
\pgfpathlineto{\pgfqpoint{1.249718in}{1.165628in}}%
\pgfpathlineto{\pgfqpoint{1.257165in}{1.167095in}}%
\pgfpathlineto{\pgfqpoint{1.264613in}{1.170503in}}%
\pgfpathlineto{\pgfqpoint{1.274544in}{1.177895in}}%
\pgfpathlineto{\pgfqpoint{1.284474in}{1.188295in}}%
\pgfpathlineto{\pgfqpoint{1.296887in}{1.205073in}}%
\pgfpathlineto{\pgfqpoint{1.311783in}{1.229905in}}%
\pgfpathlineto{\pgfqpoint{1.329161in}{1.263866in}}%
\pgfpathlineto{\pgfqpoint{1.353987in}{1.318018in}}%
\pgfpathlineto{\pgfqpoint{1.398674in}{1.416165in}}%
\pgfpathlineto{\pgfqpoint{1.418535in}{1.454108in}}%
\pgfpathlineto{\pgfqpoint{1.435913in}{1.482294in}}%
\pgfpathlineto{\pgfqpoint{1.450808in}{1.502034in}}%
\pgfpathlineto{\pgfqpoint{1.463222in}{1.515088in}}%
\pgfpathlineto{\pgfqpoint{1.475635in}{1.524928in}}%
\pgfpathlineto{\pgfqpoint{1.488048in}{1.531521in}}%
\pgfpathlineto{\pgfqpoint{1.497978in}{1.534482in}}%
\pgfpathlineto{\pgfqpoint{1.507908in}{1.535443in}}%
\pgfpathlineto{\pgfqpoint{1.517839in}{1.534483in}}%
\pgfpathlineto{\pgfqpoint{1.530252in}{1.530745in}}%
\pgfpathlineto{\pgfqpoint{1.542665in}{1.524422in}}%
\pgfpathlineto{\pgfqpoint{1.557560in}{1.513838in}}%
\pgfpathlineto{\pgfqpoint{1.574939in}{1.498088in}}%
\pgfpathlineto{\pgfqpoint{1.594799in}{1.476816in}}%
\pgfpathlineto{\pgfqpoint{1.632039in}{1.432663in}}%
\pgfpathlineto{\pgfqpoint{1.661830in}{1.398641in}}%
\pgfpathlineto{\pgfqpoint{1.681691in}{1.378838in}}%
\pgfpathlineto{\pgfqpoint{1.699069in}{1.364254in}}%
\pgfpathlineto{\pgfqpoint{1.716447in}{1.352681in}}%
\pgfpathlineto{\pgfqpoint{1.731343in}{1.345358in}}%
\pgfpathlineto{\pgfqpoint{1.746238in}{1.340505in}}%
\pgfpathlineto{\pgfqpoint{1.761134in}{1.338105in}}%
\pgfpathlineto{\pgfqpoint{1.776030in}{1.338061in}}%
\pgfpathlineto{\pgfqpoint{1.790925in}{1.340208in}}%
\pgfpathlineto{\pgfqpoint{1.808303in}{1.345179in}}%
\pgfpathlineto{\pgfqpoint{1.828164in}{1.353572in}}%
\pgfpathlineto{\pgfqpoint{1.850508in}{1.365570in}}%
\pgfpathlineto{\pgfqpoint{1.887747in}{1.388569in}}%
\pgfpathlineto{\pgfqpoint{1.927468in}{1.412292in}}%
\pgfpathlineto{\pgfqpoint{1.952294in}{1.424483in}}%
\pgfpathlineto{\pgfqpoint{1.974638in}{1.432855in}}%
\pgfpathlineto{\pgfqpoint{1.994499in}{1.437933in}}%
\pgfpathlineto{\pgfqpoint{2.014359in}{1.440694in}}%
\pgfpathlineto{\pgfqpoint{2.034220in}{1.441195in}}%
\pgfpathlineto{\pgfqpoint{2.054081in}{1.439616in}}%
\pgfpathlineto{\pgfqpoint{2.076425in}{1.435706in}}%
\pgfpathlineto{\pgfqpoint{2.103733in}{1.428591in}}%
\pgfpathlineto{\pgfqpoint{2.145937in}{1.414925in}}%
\pgfpathlineto{\pgfqpoint{2.195589in}{1.399390in}}%
\pgfpathlineto{\pgfqpoint{2.225381in}{1.392353in}}%
\pgfpathlineto{\pgfqpoint{2.252689in}{1.388117in}}%
\pgfpathlineto{\pgfqpoint{2.279998in}{1.386184in}}%
\pgfpathlineto{\pgfqpoint{2.307307in}{1.386468in}}%
\pgfpathlineto{\pgfqpoint{2.337098in}{1.388937in}}%
\pgfpathlineto{\pgfqpoint{2.376820in}{1.394670in}}%
\pgfpathlineto{\pgfqpoint{2.481089in}{1.411134in}}%
\pgfpathlineto{\pgfqpoint{2.518328in}{1.414131in}}%
\pgfpathlineto{\pgfqpoint{2.555567in}{1.414881in}}%
\pgfpathlineto{\pgfqpoint{2.597771in}{1.413320in}}%
\pgfpathlineto{\pgfqpoint{2.654871in}{1.408656in}}%
\pgfpathlineto{\pgfqpoint{2.744245in}{1.401325in}}%
\pgfpathlineto{\pgfqpoint{2.796379in}{1.399523in}}%
\pgfpathlineto{\pgfqpoint{2.850997in}{1.400026in}}%
\pgfpathlineto{\pgfqpoint{2.930440in}{1.403414in}}%
\pgfpathlineto{\pgfqpoint{3.027262in}{1.407050in}}%
\pgfpathlineto{\pgfqpoint{3.096774in}{1.407322in}}%
\pgfpathlineto{\pgfqpoint{3.196079in}{1.405171in}}%
\pgfpathlineto{\pgfqpoint{3.225870in}{1.404428in}}%
\pgfpathlineto{\pgfqpoint{3.225870in}{1.404428in}}%
\pgfusepath{stroke}%
\end{pgfscope}%
\begin{pgfscope}%
\pgfpathrectangle{\pgfqpoint{0.619136in}{0.571603in}}{\pgfqpoint{2.730864in}{1.657828in}}%
\pgfusepath{clip}%
\pgfsetrectcap%
\pgfsetroundjoin%
\pgfsetlinewidth{1.505625pt}%
\definecolor{currentstroke}{rgb}{0.090196,0.745098,0.811765}%
\pgfsetstrokecolor{currentstroke}%
\pgfsetdash{}{0pt}%
\pgfpathmoveto{\pgfqpoint{0.743267in}{0.646959in}}%
\pgfpathlineto{\pgfqpoint{0.745749in}{0.651932in}}%
\pgfpathlineto{\pgfqpoint{0.750714in}{0.670281in}}%
\pgfpathlineto{\pgfqpoint{0.758162in}{0.708069in}}%
\pgfpathlineto{\pgfqpoint{0.770575in}{0.785811in}}%
\pgfpathlineto{\pgfqpoint{0.792919in}{0.944458in}}%
\pgfpathlineto{\pgfqpoint{0.820227in}{1.135350in}}%
\pgfpathlineto{\pgfqpoint{0.837605in}{1.243152in}}%
\pgfpathlineto{\pgfqpoint{0.852501in}{1.323722in}}%
\pgfpathlineto{\pgfqpoint{0.867397in}{1.392314in}}%
\pgfpathlineto{\pgfqpoint{0.879810in}{1.440153in}}%
\pgfpathlineto{\pgfqpoint{0.892223in}{1.479696in}}%
\pgfpathlineto{\pgfqpoint{0.904636in}{1.511330in}}%
\pgfpathlineto{\pgfqpoint{0.914566in}{1.531298in}}%
\pgfpathlineto{\pgfqpoint{0.924497in}{1.546873in}}%
\pgfpathlineto{\pgfqpoint{0.934427in}{1.558408in}}%
\pgfpathlineto{\pgfqpoint{0.944357in}{1.566278in}}%
\pgfpathlineto{\pgfqpoint{0.954288in}{1.570867in}}%
\pgfpathlineto{\pgfqpoint{0.964218in}{1.572559in}}%
\pgfpathlineto{\pgfqpoint{0.974149in}{1.571730in}}%
\pgfpathlineto{\pgfqpoint{0.984079in}{1.568746in}}%
\pgfpathlineto{\pgfqpoint{0.996492in}{1.562515in}}%
\pgfpathlineto{\pgfqpoint{1.011388in}{1.552201in}}%
\pgfpathlineto{\pgfqpoint{1.031249in}{1.535234in}}%
\pgfpathlineto{\pgfqpoint{1.075935in}{1.492531in}}%
\pgfpathlineto{\pgfqpoint{1.103244in}{1.468270in}}%
\pgfpathlineto{\pgfqpoint{1.125587in}{1.451141in}}%
\pgfpathlineto{\pgfqpoint{1.147931in}{1.436954in}}%
\pgfpathlineto{\pgfqpoint{1.167792in}{1.426890in}}%
\pgfpathlineto{\pgfqpoint{1.187653in}{1.419119in}}%
\pgfpathlineto{\pgfqpoint{1.209996in}{1.412852in}}%
\pgfpathlineto{\pgfqpoint{1.232339in}{1.408824in}}%
\pgfpathlineto{\pgfqpoint{1.259648in}{1.406303in}}%
\pgfpathlineto{\pgfqpoint{1.291922in}{1.405724in}}%
\pgfpathlineto{\pgfqpoint{1.339091in}{1.407383in}}%
\pgfpathlineto{\pgfqpoint{1.458256in}{1.412360in}}%
\pgfpathlineto{\pgfqpoint{1.535217in}{1.412676in}}%
\pgfpathlineto{\pgfqpoint{1.696586in}{1.410236in}}%
\pgfpathlineto{\pgfqpoint{1.895194in}{1.408519in}}%
\pgfpathlineto{\pgfqpoint{2.523293in}{1.406601in}}%
\pgfpathlineto{\pgfqpoint{3.225870in}{1.405727in}}%
\pgfpathlineto{\pgfqpoint{3.225870in}{1.405727in}}%
\pgfusepath{stroke}%
\end{pgfscope}%
\begin{pgfscope}%
\pgfpathrectangle{\pgfqpoint{0.619136in}{0.571603in}}{\pgfqpoint{2.730864in}{1.657828in}}%
\pgfusepath{clip}%
\pgfsetrectcap%
\pgfsetroundjoin%
\pgfsetlinewidth{1.505625pt}%
\definecolor{currentstroke}{rgb}{0.121569,0.466667,0.705882}%
\pgfsetstrokecolor{currentstroke}%
\pgfsetdash{}{0pt}%
\pgfpathmoveto{\pgfqpoint{0.743267in}{0.646959in}}%
\pgfpathlineto{\pgfqpoint{0.745749in}{0.658184in}}%
\pgfpathlineto{\pgfqpoint{0.750714in}{0.694818in}}%
\pgfpathlineto{\pgfqpoint{0.760645in}{0.788481in}}%
\pgfpathlineto{\pgfqpoint{0.795401in}{1.134807in}}%
\pgfpathlineto{\pgfqpoint{0.810297in}{1.254791in}}%
\pgfpathlineto{\pgfqpoint{0.822710in}{1.336171in}}%
\pgfpathlineto{\pgfqpoint{0.832640in}{1.389110in}}%
\pgfpathlineto{\pgfqpoint{0.842571in}{1.431810in}}%
\pgfpathlineto{\pgfqpoint{0.852501in}{1.465093in}}%
\pgfpathlineto{\pgfqpoint{0.862432in}{1.489934in}}%
\pgfpathlineto{\pgfqpoint{0.869879in}{1.503659in}}%
\pgfpathlineto{\pgfqpoint{0.877327in}{1.513681in}}%
\pgfpathlineto{\pgfqpoint{0.884775in}{1.520448in}}%
\pgfpathlineto{\pgfqpoint{0.892223in}{1.524391in}}%
\pgfpathlineto{\pgfqpoint{0.899671in}{1.525922in}}%
\pgfpathlineto{\pgfqpoint{0.907118in}{1.525425in}}%
\pgfpathlineto{\pgfqpoint{0.917049in}{1.522219in}}%
\pgfpathlineto{\pgfqpoint{0.929462in}{1.515169in}}%
\pgfpathlineto{\pgfqpoint{0.944357in}{1.503877in}}%
\pgfpathlineto{\pgfqpoint{1.008905in}{1.451182in}}%
\pgfpathlineto{\pgfqpoint{1.028766in}{1.439324in}}%
\pgfpathlineto{\pgfqpoint{1.048627in}{1.430237in}}%
\pgfpathlineto{\pgfqpoint{1.068488in}{1.423661in}}%
\pgfpathlineto{\pgfqpoint{1.090831in}{1.418736in}}%
\pgfpathlineto{\pgfqpoint{1.115657in}{1.415549in}}%
\pgfpathlineto{\pgfqpoint{1.147931in}{1.413684in}}%
\pgfpathlineto{\pgfqpoint{1.205031in}{1.413024in}}%
\pgfpathlineto{\pgfqpoint{1.351504in}{1.411607in}}%
\pgfpathlineto{\pgfqpoint{1.602247in}{1.408716in}}%
\pgfpathlineto{\pgfqpoint{2.021807in}{1.406856in}}%
\pgfpathlineto{\pgfqpoint{2.905614in}{1.405545in}}%
\pgfpathlineto{\pgfqpoint{3.225870in}{1.405331in}}%
\pgfpathlineto{\pgfqpoint{3.225870in}{1.405331in}}%
\pgfusepath{stroke}%
\end{pgfscope}%
\begin{pgfscope}%
\pgfpathrectangle{\pgfqpoint{0.619136in}{0.571603in}}{\pgfqpoint{2.730864in}{1.657828in}}%
\pgfusepath{clip}%
\pgfsetrectcap%
\pgfsetroundjoin%
\pgfsetlinewidth{1.505625pt}%
\definecolor{currentstroke}{rgb}{1.000000,0.498039,0.054902}%
\pgfsetstrokecolor{currentstroke}%
\pgfsetdash{}{0pt}%
\pgfpathmoveto{\pgfqpoint{0.743267in}{0.646959in}}%
\pgfpathlineto{\pgfqpoint{0.745749in}{0.654315in}}%
\pgfpathlineto{\pgfqpoint{0.750714in}{0.679017in}}%
\pgfpathlineto{\pgfqpoint{0.760645in}{0.744459in}}%
\pgfpathlineto{\pgfqpoint{0.780506in}{0.898010in}}%
\pgfpathlineto{\pgfqpoint{0.805332in}{1.087737in}}%
\pgfpathlineto{\pgfqpoint{0.822710in}{1.204141in}}%
\pgfpathlineto{\pgfqpoint{0.837605in}{1.289362in}}%
\pgfpathlineto{\pgfqpoint{0.850018in}{1.349476in}}%
\pgfpathlineto{\pgfqpoint{0.862432in}{1.399772in}}%
\pgfpathlineto{\pgfqpoint{0.874845in}{1.440697in}}%
\pgfpathlineto{\pgfqpoint{0.887258in}{1.472926in}}%
\pgfpathlineto{\pgfqpoint{0.897188in}{1.492998in}}%
\pgfpathlineto{\pgfqpoint{0.907118in}{1.508488in}}%
\pgfpathlineto{\pgfqpoint{0.917049in}{1.519871in}}%
\pgfpathlineto{\pgfqpoint{0.926979in}{1.527624in}}%
\pgfpathlineto{\pgfqpoint{0.936910in}{1.532214in}}%
\pgfpathlineto{\pgfqpoint{0.946840in}{1.534090in}}%
\pgfpathlineto{\pgfqpoint{0.956770in}{1.533676in}}%
\pgfpathlineto{\pgfqpoint{0.969183in}{1.530539in}}%
\pgfpathlineto{\pgfqpoint{0.984079in}{1.523848in}}%
\pgfpathlineto{\pgfqpoint{1.001457in}{1.513323in}}%
\pgfpathlineto{\pgfqpoint{1.031249in}{1.492101in}}%
\pgfpathlineto{\pgfqpoint{1.070970in}{1.464331in}}%
\pgfpathlineto{\pgfqpoint{1.095796in}{1.449699in}}%
\pgfpathlineto{\pgfqpoint{1.118140in}{1.438908in}}%
\pgfpathlineto{\pgfqpoint{1.140483in}{1.430398in}}%
\pgfpathlineto{\pgfqpoint{1.165309in}{1.423398in}}%
\pgfpathlineto{\pgfqpoint{1.192618in}{1.418215in}}%
\pgfpathlineto{\pgfqpoint{1.222409in}{1.414870in}}%
\pgfpathlineto{\pgfqpoint{1.259648in}{1.412957in}}%
\pgfpathlineto{\pgfqpoint{1.319231in}{1.412464in}}%
\pgfpathlineto{\pgfqpoint{1.507908in}{1.411991in}}%
\pgfpathlineto{\pgfqpoint{1.902642in}{1.408470in}}%
\pgfpathlineto{\pgfqpoint{2.495984in}{1.406654in}}%
\pgfpathlineto{\pgfqpoint{3.225870in}{1.405755in}}%
\pgfpathlineto{\pgfqpoint{3.225870in}{1.405755in}}%
\pgfusepath{stroke}%
\end{pgfscope}%
\begin{pgfscope}%
\pgfpathrectangle{\pgfqpoint{0.619136in}{0.571603in}}{\pgfqpoint{2.730864in}{1.657828in}}%
\pgfusepath{clip}%
\pgfsetrectcap%
\pgfsetroundjoin%
\pgfsetlinewidth{1.505625pt}%
\definecolor{currentstroke}{rgb}{0.172549,0.627451,0.172549}%
\pgfsetstrokecolor{currentstroke}%
\pgfsetdash{}{0pt}%
\pgfpathmoveto{\pgfqpoint{0.743267in}{0.646959in}}%
\pgfpathlineto{\pgfqpoint{0.745749in}{0.649184in}}%
\pgfpathlineto{\pgfqpoint{0.750714in}{0.659954in}}%
\pgfpathlineto{\pgfqpoint{0.758162in}{0.686260in}}%
\pgfpathlineto{\pgfqpoint{0.765610in}{0.721500in}}%
\pgfpathlineto{\pgfqpoint{0.775540in}{0.778859in}}%
\pgfpathlineto{\pgfqpoint{0.790436in}{0.880373in}}%
\pgfpathlineto{\pgfqpoint{0.815262in}{1.070148in}}%
\pgfpathlineto{\pgfqpoint{0.847536in}{1.313672in}}%
\pgfpathlineto{\pgfqpoint{0.864914in}{1.428794in}}%
\pgfpathlineto{\pgfqpoint{0.879810in}{1.513935in}}%
\pgfpathlineto{\pgfqpoint{0.892223in}{1.574000in}}%
\pgfpathlineto{\pgfqpoint{0.904636in}{1.623578in}}%
\pgfpathlineto{\pgfqpoint{0.914566in}{1.655559in}}%
\pgfpathlineto{\pgfqpoint{0.924497in}{1.680766in}}%
\pgfpathlineto{\pgfqpoint{0.934427in}{1.699371in}}%
\pgfpathlineto{\pgfqpoint{0.941875in}{1.709154in}}%
\pgfpathlineto{\pgfqpoint{0.949323in}{1.715524in}}%
\pgfpathlineto{\pgfqpoint{0.956770in}{1.718653in}}%
\pgfpathlineto{\pgfqpoint{0.964218in}{1.718731in}}%
\pgfpathlineto{\pgfqpoint{0.971666in}{1.715968in}}%
\pgfpathlineto{\pgfqpoint{0.979114in}{1.710585in}}%
\pgfpathlineto{\pgfqpoint{0.989044in}{1.699736in}}%
\pgfpathlineto{\pgfqpoint{0.998975in}{1.685213in}}%
\pgfpathlineto{\pgfqpoint{1.011388in}{1.662775in}}%
\pgfpathlineto{\pgfqpoint{1.026283in}{1.631051in}}%
\pgfpathlineto{\pgfqpoint{1.048627in}{1.577540in}}%
\pgfpathlineto{\pgfqpoint{1.093314in}{1.469211in}}%
\pgfpathlineto{\pgfqpoint{1.113174in}{1.427571in}}%
\pgfpathlineto{\pgfqpoint{1.130553in}{1.396668in}}%
\pgfpathlineto{\pgfqpoint{1.145448in}{1.374858in}}%
\pgfpathlineto{\pgfqpoint{1.157861in}{1.360144in}}%
\pgfpathlineto{\pgfqpoint{1.170274in}{1.348588in}}%
\pgfpathlineto{\pgfqpoint{1.182687in}{1.340116in}}%
\pgfpathlineto{\pgfqpoint{1.195100in}{1.334577in}}%
\pgfpathlineto{\pgfqpoint{1.207513in}{1.331766in}}%
\pgfpathlineto{\pgfqpoint{1.219926in}{1.331428in}}%
\pgfpathlineto{\pgfqpoint{1.232339in}{1.333274in}}%
\pgfpathlineto{\pgfqpoint{1.247235in}{1.337929in}}%
\pgfpathlineto{\pgfqpoint{1.264613in}{1.346014in}}%
\pgfpathlineto{\pgfqpoint{1.286957in}{1.359243in}}%
\pgfpathlineto{\pgfqpoint{1.366400in}{1.409047in}}%
\pgfpathlineto{\pgfqpoint{1.388743in}{1.419092in}}%
\pgfpathlineto{\pgfqpoint{1.408604in}{1.425686in}}%
\pgfpathlineto{\pgfqpoint{1.428465in}{1.430029in}}%
\pgfpathlineto{\pgfqpoint{1.448326in}{1.432233in}}%
\pgfpathlineto{\pgfqpoint{1.470669in}{1.432445in}}%
\pgfpathlineto{\pgfqpoint{1.495495in}{1.430402in}}%
\pgfpathlineto{\pgfqpoint{1.527769in}{1.425304in}}%
\pgfpathlineto{\pgfqpoint{1.641969in}{1.404920in}}%
\pgfpathlineto{\pgfqpoint{1.679208in}{1.401757in}}%
\pgfpathlineto{\pgfqpoint{1.718930in}{1.400739in}}%
\pgfpathlineto{\pgfqpoint{1.768582in}{1.401940in}}%
\pgfpathlineto{\pgfqpoint{1.942364in}{1.408326in}}%
\pgfpathlineto{\pgfqpoint{2.026772in}{1.407670in}}%
\pgfpathlineto{\pgfqpoint{2.230346in}{1.405353in}}%
\pgfpathlineto{\pgfqpoint{3.225870in}{1.404987in}}%
\pgfpathlineto{\pgfqpoint{3.225870in}{1.404987in}}%
\pgfusepath{stroke}%
\end{pgfscope}%
\begin{pgfscope}%
\pgfpathrectangle{\pgfqpoint{0.619136in}{0.571603in}}{\pgfqpoint{2.730864in}{1.657828in}}%
\pgfusepath{clip}%
\pgfsetrectcap%
\pgfsetroundjoin%
\pgfsetlinewidth{1.505625pt}%
\definecolor{currentstroke}{rgb}{0.839216,0.152941,0.156863}%
\pgfsetstrokecolor{currentstroke}%
\pgfsetdash{}{0pt}%
\pgfpathmoveto{\pgfqpoint{0.743267in}{0.646959in}}%
\pgfpathlineto{\pgfqpoint{0.748232in}{0.649303in}}%
\pgfpathlineto{\pgfqpoint{0.753197in}{0.655559in}}%
\pgfpathlineto{\pgfqpoint{0.758162in}{0.665326in}}%
\pgfpathlineto{\pgfqpoint{0.765610in}{0.686073in}}%
\pgfpathlineto{\pgfqpoint{0.773058in}{0.713607in}}%
\pgfpathlineto{\pgfqpoint{0.782988in}{0.759935in}}%
\pgfpathlineto{\pgfqpoint{0.795401in}{0.831515in}}%
\pgfpathlineto{\pgfqpoint{0.810297in}{0.934066in}}%
\pgfpathlineto{\pgfqpoint{0.827675in}{1.070783in}}%
\pgfpathlineto{\pgfqpoint{0.857466in}{1.326639in}}%
\pgfpathlineto{\pgfqpoint{0.887258in}{1.576760in}}%
\pgfpathlineto{\pgfqpoint{0.904636in}{1.706137in}}%
\pgfpathlineto{\pgfqpoint{0.919531in}{1.801831in}}%
\pgfpathlineto{\pgfqpoint{0.931944in}{1.868643in}}%
\pgfpathlineto{\pgfqpoint{0.941875in}{1.912686in}}%
\pgfpathlineto{\pgfqpoint{0.951805in}{1.947833in}}%
\pgfpathlineto{\pgfqpoint{0.959253in}{1.968150in}}%
\pgfpathlineto{\pgfqpoint{0.966701in}{1.983189in}}%
\pgfpathlineto{\pgfqpoint{0.974149in}{1.992915in}}%
\pgfpathlineto{\pgfqpoint{0.979114in}{1.996452in}}%
\pgfpathlineto{\pgfqpoint{0.984079in}{1.997647in}}%
\pgfpathlineto{\pgfqpoint{0.989044in}{1.996525in}}%
\pgfpathlineto{\pgfqpoint{0.994009in}{1.993120in}}%
\pgfpathlineto{\pgfqpoint{0.998975in}{1.987472in}}%
\pgfpathlineto{\pgfqpoint{1.006423in}{1.974908in}}%
\pgfpathlineto{\pgfqpoint{1.013870in}{1.957617in}}%
\pgfpathlineto{\pgfqpoint{1.023801in}{1.927627in}}%
\pgfpathlineto{\pgfqpoint{1.033731in}{1.890329in}}%
\pgfpathlineto{\pgfqpoint{1.046144in}{1.834598in}}%
\pgfpathlineto{\pgfqpoint{1.061040in}{1.756588in}}%
\pgfpathlineto{\pgfqpoint{1.080901in}{1.638933in}}%
\pgfpathlineto{\pgfqpoint{1.147931in}{1.227670in}}%
\pgfpathlineto{\pgfqpoint{1.162827in}{1.153067in}}%
\pgfpathlineto{\pgfqpoint{1.177722in}{1.090140in}}%
\pgfpathlineto{\pgfqpoint{1.190135in}{1.047893in}}%
\pgfpathlineto{\pgfqpoint{1.200066in}{1.021309in}}%
\pgfpathlineto{\pgfqpoint{1.209996in}{1.001415in}}%
\pgfpathlineto{\pgfqpoint{1.217444in}{0.990972in}}%
\pgfpathlineto{\pgfqpoint{1.224892in}{0.984390in}}%
\pgfpathlineto{\pgfqpoint{1.229857in}{0.982140in}}%
\pgfpathlineto{\pgfqpoint{1.234822in}{0.981587in}}%
\pgfpathlineto{\pgfqpoint{1.239787in}{0.982710in}}%
\pgfpathlineto{\pgfqpoint{1.247235in}{0.987481in}}%
\pgfpathlineto{\pgfqpoint{1.254683in}{0.995850in}}%
\pgfpathlineto{\pgfqpoint{1.262131in}{1.007675in}}%
\pgfpathlineto{\pgfqpoint{1.272061in}{1.028519in}}%
\pgfpathlineto{\pgfqpoint{1.281991in}{1.054726in}}%
\pgfpathlineto{\pgfqpoint{1.294404in}{1.094190in}}%
\pgfpathlineto{\pgfqpoint{1.309300in}{1.149780in}}%
\pgfpathlineto{\pgfqpoint{1.329161in}{1.234090in}}%
\pgfpathlineto{\pgfqpoint{1.398674in}{1.541078in}}%
\pgfpathlineto{\pgfqpoint{1.413569in}{1.594195in}}%
\pgfpathlineto{\pgfqpoint{1.425982in}{1.631967in}}%
\pgfpathlineto{\pgfqpoint{1.438395in}{1.663050in}}%
\pgfpathlineto{\pgfqpoint{1.448326in}{1.682729in}}%
\pgfpathlineto{\pgfqpoint{1.458256in}{1.697585in}}%
\pgfpathlineto{\pgfqpoint{1.465704in}{1.705494in}}%
\pgfpathlineto{\pgfqpoint{1.473152in}{1.710609in}}%
\pgfpathlineto{\pgfqpoint{1.480600in}{1.712941in}}%
\pgfpathlineto{\pgfqpoint{1.488048in}{1.712520in}}%
\pgfpathlineto{\pgfqpoint{1.495495in}{1.709404in}}%
\pgfpathlineto{\pgfqpoint{1.502943in}{1.703670in}}%
\pgfpathlineto{\pgfqpoint{1.510391in}{1.695420in}}%
\pgfpathlineto{\pgfqpoint{1.520321in}{1.680715in}}%
\pgfpathlineto{\pgfqpoint{1.530252in}{1.662090in}}%
\pgfpathlineto{\pgfqpoint{1.542665in}{1.633894in}}%
\pgfpathlineto{\pgfqpoint{1.557560in}{1.593995in}}%
\pgfpathlineto{\pgfqpoint{1.577421in}{1.533229in}}%
\pgfpathlineto{\pgfqpoint{1.649417in}{1.303313in}}%
\pgfpathlineto{\pgfqpoint{1.664312in}{1.265331in}}%
\pgfpathlineto{\pgfqpoint{1.679208in}{1.233651in}}%
\pgfpathlineto{\pgfqpoint{1.691621in}{1.212697in}}%
\pgfpathlineto{\pgfqpoint{1.701551in}{1.199760in}}%
\pgfpathlineto{\pgfqpoint{1.711482in}{1.190353in}}%
\pgfpathlineto{\pgfqpoint{1.721412in}{1.184533in}}%
\pgfpathlineto{\pgfqpoint{1.728860in}{1.182523in}}%
\pgfpathlineto{\pgfqpoint{1.736308in}{1.182511in}}%
\pgfpathlineto{\pgfqpoint{1.743756in}{1.184458in}}%
\pgfpathlineto{\pgfqpoint{1.751203in}{1.188309in}}%
\pgfpathlineto{\pgfqpoint{1.761134in}{1.196283in}}%
\pgfpathlineto{\pgfqpoint{1.771064in}{1.207309in}}%
\pgfpathlineto{\pgfqpoint{1.783477in}{1.224998in}}%
\pgfpathlineto{\pgfqpoint{1.798373in}{1.251196in}}%
\pgfpathlineto{\pgfqpoint{1.815751in}{1.287183in}}%
\pgfpathlineto{\pgfqpoint{1.840577in}{1.344967in}}%
\pgfpathlineto{\pgfqpoint{1.885264in}{1.450310in}}%
\pgfpathlineto{\pgfqpoint{1.905125in}{1.490711in}}%
\pgfpathlineto{\pgfqpoint{1.920021in}{1.516324in}}%
\pgfpathlineto{\pgfqpoint{1.932434in}{1.533988in}}%
\pgfpathlineto{\pgfqpoint{1.944847in}{1.547957in}}%
\pgfpathlineto{\pgfqpoint{1.954777in}{1.556318in}}%
\pgfpathlineto{\pgfqpoint{1.964707in}{1.562104in}}%
\pgfpathlineto{\pgfqpoint{1.974638in}{1.565290in}}%
\pgfpathlineto{\pgfqpoint{1.984568in}{1.565891in}}%
\pgfpathlineto{\pgfqpoint{1.994499in}{1.563963in}}%
\pgfpathlineto{\pgfqpoint{2.004429in}{1.559602in}}%
\pgfpathlineto{\pgfqpoint{2.014359in}{1.552936in}}%
\pgfpathlineto{\pgfqpoint{2.026772in}{1.541614in}}%
\pgfpathlineto{\pgfqpoint{2.039185in}{1.527327in}}%
\pgfpathlineto{\pgfqpoint{2.054081in}{1.506895in}}%
\pgfpathlineto{\pgfqpoint{2.073942in}{1.475478in}}%
\pgfpathlineto{\pgfqpoint{2.111181in}{1.410875in}}%
\pgfpathlineto{\pgfqpoint{2.138490in}{1.365708in}}%
\pgfpathlineto{\pgfqpoint{2.155868in}{1.340622in}}%
\pgfpathlineto{\pgfqpoint{2.170763in}{1.322369in}}%
\pgfpathlineto{\pgfqpoint{2.185659in}{1.307672in}}%
\pgfpathlineto{\pgfqpoint{2.198072in}{1.298416in}}%
\pgfpathlineto{\pgfqpoint{2.210485in}{1.292031in}}%
\pgfpathlineto{\pgfqpoint{2.222898in}{1.288581in}}%
\pgfpathlineto{\pgfqpoint{2.232829in}{1.287930in}}%
\pgfpathlineto{\pgfqpoint{2.242759in}{1.289116in}}%
\pgfpathlineto{\pgfqpoint{2.255172in}{1.293079in}}%
\pgfpathlineto{\pgfqpoint{2.267585in}{1.299623in}}%
\pgfpathlineto{\pgfqpoint{2.279998in}{1.308508in}}%
\pgfpathlineto{\pgfqpoint{2.294894in}{1.321849in}}%
\pgfpathlineto{\pgfqpoint{2.312272in}{1.340372in}}%
\pgfpathlineto{\pgfqpoint{2.337098in}{1.370407in}}%
\pgfpathlineto{\pgfqpoint{2.386750in}{1.431529in}}%
\pgfpathlineto{\pgfqpoint{2.406611in}{1.452236in}}%
\pgfpathlineto{\pgfqpoint{2.423989in}{1.467134in}}%
\pgfpathlineto{\pgfqpoint{2.438885in}{1.477049in}}%
\pgfpathlineto{\pgfqpoint{2.453780in}{1.484064in}}%
\pgfpathlineto{\pgfqpoint{2.466193in}{1.487595in}}%
\pgfpathlineto{\pgfqpoint{2.478606in}{1.488999in}}%
\pgfpathlineto{\pgfqpoint{2.491019in}{1.488309in}}%
\pgfpathlineto{\pgfqpoint{2.503432in}{1.485606in}}%
\pgfpathlineto{\pgfqpoint{2.518328in}{1.479894in}}%
\pgfpathlineto{\pgfqpoint{2.533224in}{1.471774in}}%
\pgfpathlineto{\pgfqpoint{2.550602in}{1.459760in}}%
\pgfpathlineto{\pgfqpoint{2.572945in}{1.441367in}}%
\pgfpathlineto{\pgfqpoint{2.652388in}{1.372503in}}%
\pgfpathlineto{\pgfqpoint{2.672249in}{1.359805in}}%
\pgfpathlineto{\pgfqpoint{2.689628in}{1.351449in}}%
\pgfpathlineto{\pgfqpoint{2.704523in}{1.346565in}}%
\pgfpathlineto{\pgfqpoint{2.719419in}{1.343882in}}%
\pgfpathlineto{\pgfqpoint{2.734314in}{1.343410in}}%
\pgfpathlineto{\pgfqpoint{2.749210in}{1.345081in}}%
\pgfpathlineto{\pgfqpoint{2.766588in}{1.349551in}}%
\pgfpathlineto{\pgfqpoint{2.783966in}{1.356415in}}%
\pgfpathlineto{\pgfqpoint{2.803827in}{1.366637in}}%
\pgfpathlineto{\pgfqpoint{2.831136in}{1.383477in}}%
\pgfpathlineto{\pgfqpoint{2.893201in}{1.423019in}}%
\pgfpathlineto{\pgfqpoint{2.915544in}{1.434238in}}%
\pgfpathlineto{\pgfqpoint{2.935405in}{1.441765in}}%
\pgfpathlineto{\pgfqpoint{2.952783in}{1.446172in}}%
\pgfpathlineto{\pgfqpoint{2.970162in}{1.448416in}}%
\pgfpathlineto{\pgfqpoint{2.987540in}{1.448486in}}%
\pgfpathlineto{\pgfqpoint{3.004918in}{1.446476in}}%
\pgfpathlineto{\pgfqpoint{3.024779in}{1.441874in}}%
\pgfpathlineto{\pgfqpoint{3.047122in}{1.434251in}}%
\pgfpathlineto{\pgfqpoint{3.074431in}{1.422472in}}%
\pgfpathlineto{\pgfqpoint{3.153874in}{1.386639in}}%
\pgfpathlineto{\pgfqpoint{3.178700in}{1.378879in}}%
\pgfpathlineto{\pgfqpoint{3.201044in}{1.374372in}}%
\pgfpathlineto{\pgfqpoint{3.220905in}{1.372526in}}%
\pgfpathlineto{\pgfqpoint{3.225870in}{1.372387in}}%
\pgfpathlineto{\pgfqpoint{3.225870in}{1.372387in}}%
\pgfusepath{stroke}%
\end{pgfscope}%
\begin{pgfscope}%
\pgfpathrectangle{\pgfqpoint{0.619136in}{0.571603in}}{\pgfqpoint{2.730864in}{1.657828in}}%
\pgfusepath{clip}%
\pgfsetrectcap%
\pgfsetroundjoin%
\pgfsetlinewidth{1.505625pt}%
\definecolor{currentstroke}{rgb}{0.580392,0.403922,0.741176}%
\pgfsetstrokecolor{currentstroke}%
\pgfsetdash{}{0pt}%
\pgfpathmoveto{\pgfqpoint{0.743267in}{0.646959in}}%
\pgfpathlineto{\pgfqpoint{0.748232in}{0.648218in}}%
\pgfpathlineto{\pgfqpoint{0.753197in}{0.651657in}}%
\pgfpathlineto{\pgfqpoint{0.760645in}{0.660545in}}%
\pgfpathlineto{\pgfqpoint{0.768093in}{0.673641in}}%
\pgfpathlineto{\pgfqpoint{0.778023in}{0.697252in}}%
\pgfpathlineto{\pgfqpoint{0.787953in}{0.727441in}}%
\pgfpathlineto{\pgfqpoint{0.800366in}{0.773690in}}%
\pgfpathlineto{\pgfqpoint{0.812779in}{0.828482in}}%
\pgfpathlineto{\pgfqpoint{0.827675in}{0.904048in}}%
\pgfpathlineto{\pgfqpoint{0.845053in}{1.003337in}}%
\pgfpathlineto{\pgfqpoint{0.869879in}{1.159897in}}%
\pgfpathlineto{\pgfqpoint{0.946840in}{1.657875in}}%
\pgfpathlineto{\pgfqpoint{0.966701in}{1.766523in}}%
\pgfpathlineto{\pgfqpoint{0.984079in}{1.848754in}}%
\pgfpathlineto{\pgfqpoint{0.998975in}{1.908137in}}%
\pgfpathlineto{\pgfqpoint{1.011388in}{1.949040in}}%
\pgfpathlineto{\pgfqpoint{1.023801in}{1.981678in}}%
\pgfpathlineto{\pgfqpoint{1.033731in}{2.001630in}}%
\pgfpathlineto{\pgfqpoint{1.043662in}{2.015998in}}%
\pgfpathlineto{\pgfqpoint{1.051109in}{2.023080in}}%
\pgfpathlineto{\pgfqpoint{1.058557in}{2.026995in}}%
\pgfpathlineto{\pgfqpoint{1.066005in}{2.027757in}}%
\pgfpathlineto{\pgfqpoint{1.073453in}{2.025397in}}%
\pgfpathlineto{\pgfqpoint{1.080901in}{2.019964in}}%
\pgfpathlineto{\pgfqpoint{1.088348in}{2.011518in}}%
\pgfpathlineto{\pgfqpoint{1.095796in}{2.000136in}}%
\pgfpathlineto{\pgfqpoint{1.105727in}{1.980550in}}%
\pgfpathlineto{\pgfqpoint{1.115657in}{1.956157in}}%
\pgfpathlineto{\pgfqpoint{1.128070in}{1.919351in}}%
\pgfpathlineto{\pgfqpoint{1.142966in}{1.866782in}}%
\pgfpathlineto{\pgfqpoint{1.160344in}{1.795474in}}%
\pgfpathlineto{\pgfqpoint{1.180205in}{1.703556in}}%
\pgfpathlineto{\pgfqpoint{1.207513in}{1.565402in}}%
\pgfpathlineto{\pgfqpoint{1.264613in}{1.272560in}}%
\pgfpathlineto{\pgfqpoint{1.286957in}{1.170926in}}%
\pgfpathlineto{\pgfqpoint{1.304335in}{1.101371in}}%
\pgfpathlineto{\pgfqpoint{1.319231in}{1.049774in}}%
\pgfpathlineto{\pgfqpoint{1.334126in}{1.006498in}}%
\pgfpathlineto{\pgfqpoint{1.346539in}{0.977283in}}%
\pgfpathlineto{\pgfqpoint{1.356470in}{0.958606in}}%
\pgfpathlineto{\pgfqpoint{1.366400in}{0.944217in}}%
\pgfpathlineto{\pgfqpoint{1.376330in}{0.934178in}}%
\pgfpathlineto{\pgfqpoint{1.383778in}{0.929518in}}%
\pgfpathlineto{\pgfqpoint{1.391226in}{0.927314in}}%
\pgfpathlineto{\pgfqpoint{1.398674in}{0.927547in}}%
\pgfpathlineto{\pgfqpoint{1.406122in}{0.930190in}}%
\pgfpathlineto{\pgfqpoint{1.413569in}{0.935200in}}%
\pgfpathlineto{\pgfqpoint{1.421017in}{0.942525in}}%
\pgfpathlineto{\pgfqpoint{1.430948in}{0.955780in}}%
\pgfpathlineto{\pgfqpoint{1.440878in}{0.972857in}}%
\pgfpathlineto{\pgfqpoint{1.453291in}{0.999250in}}%
\pgfpathlineto{\pgfqpoint{1.465704in}{1.030805in}}%
\pgfpathlineto{\pgfqpoint{1.480600in}{1.074721in}}%
\pgfpathlineto{\pgfqpoint{1.497978in}{1.132984in}}%
\pgfpathlineto{\pgfqpoint{1.520321in}{1.216240in}}%
\pgfpathlineto{\pgfqpoint{1.560043in}{1.375467in}}%
\pgfpathlineto{\pgfqpoint{1.594799in}{1.511141in}}%
\pgfpathlineto{\pgfqpoint{1.617143in}{1.589448in}}%
\pgfpathlineto{\pgfqpoint{1.634521in}{1.642969in}}%
\pgfpathlineto{\pgfqpoint{1.649417in}{1.682613in}}%
\pgfpathlineto{\pgfqpoint{1.664312in}{1.715802in}}%
\pgfpathlineto{\pgfqpoint{1.676725in}{1.738150in}}%
\pgfpathlineto{\pgfqpoint{1.689138in}{1.755434in}}%
\pgfpathlineto{\pgfqpoint{1.699069in}{1.765516in}}%
\pgfpathlineto{\pgfqpoint{1.708999in}{1.772223in}}%
\pgfpathlineto{\pgfqpoint{1.716447in}{1.775033in}}%
\pgfpathlineto{\pgfqpoint{1.723895in}{1.775948in}}%
\pgfpathlineto{\pgfqpoint{1.731343in}{1.774984in}}%
\pgfpathlineto{\pgfqpoint{1.738790in}{1.772167in}}%
\pgfpathlineto{\pgfqpoint{1.748721in}{1.765593in}}%
\pgfpathlineto{\pgfqpoint{1.758651in}{1.755899in}}%
\pgfpathlineto{\pgfqpoint{1.768582in}{1.743218in}}%
\pgfpathlineto{\pgfqpoint{1.780995in}{1.723412in}}%
\pgfpathlineto{\pgfqpoint{1.793408in}{1.699549in}}%
\pgfpathlineto{\pgfqpoint{1.808303in}{1.666139in}}%
\pgfpathlineto{\pgfqpoint{1.825682in}{1.621583in}}%
\pgfpathlineto{\pgfqpoint{1.848025in}{1.557597in}}%
\pgfpathlineto{\pgfqpoint{1.882781in}{1.450039in}}%
\pgfpathlineto{\pgfqpoint{1.922503in}{1.328839in}}%
\pgfpathlineto{\pgfqpoint{1.944847in}{1.267522in}}%
\pgfpathlineto{\pgfqpoint{1.962225in}{1.225405in}}%
\pgfpathlineto{\pgfqpoint{1.979603in}{1.189287in}}%
\pgfpathlineto{\pgfqpoint{1.994499in}{1.163732in}}%
\pgfpathlineto{\pgfqpoint{2.006912in}{1.146555in}}%
\pgfpathlineto{\pgfqpoint{2.019325in}{1.133304in}}%
\pgfpathlineto{\pgfqpoint{2.029255in}{1.125607in}}%
\pgfpathlineto{\pgfqpoint{2.039185in}{1.120523in}}%
\pgfpathlineto{\pgfqpoint{2.049116in}{1.118058in}}%
\pgfpathlineto{\pgfqpoint{2.059046in}{1.118193in}}%
\pgfpathlineto{\pgfqpoint{2.068977in}{1.120888in}}%
\pgfpathlineto{\pgfqpoint{2.078907in}{1.126080in}}%
\pgfpathlineto{\pgfqpoint{2.088838in}{1.133683in}}%
\pgfpathlineto{\pgfqpoint{2.101251in}{1.146416in}}%
\pgfpathlineto{\pgfqpoint{2.113664in}{1.162506in}}%
\pgfpathlineto{\pgfqpoint{2.128559in}{1.185836in}}%
\pgfpathlineto{\pgfqpoint{2.145937in}{1.217875in}}%
\pgfpathlineto{\pgfqpoint{2.165798in}{1.259600in}}%
\pgfpathlineto{\pgfqpoint{2.193107in}{1.322940in}}%
\pgfpathlineto{\pgfqpoint{2.257655in}{1.475677in}}%
\pgfpathlineto{\pgfqpoint{2.279998in}{1.521733in}}%
\pgfpathlineto{\pgfqpoint{2.297376in}{1.552937in}}%
\pgfpathlineto{\pgfqpoint{2.312272in}{1.575840in}}%
\pgfpathlineto{\pgfqpoint{2.327167in}{1.594794in}}%
\pgfpathlineto{\pgfqpoint{2.339580in}{1.607361in}}%
\pgfpathlineto{\pgfqpoint{2.351993in}{1.616864in}}%
\pgfpathlineto{\pgfqpoint{2.364406in}{1.623229in}}%
\pgfpathlineto{\pgfqpoint{2.374337in}{1.626041in}}%
\pgfpathlineto{\pgfqpoint{2.384267in}{1.626831in}}%
\pgfpathlineto{\pgfqpoint{2.394198in}{1.625624in}}%
\pgfpathlineto{\pgfqpoint{2.404128in}{1.622460in}}%
\pgfpathlineto{\pgfqpoint{2.416541in}{1.615843in}}%
\pgfpathlineto{\pgfqpoint{2.428954in}{1.606411in}}%
\pgfpathlineto{\pgfqpoint{2.441367in}{1.594347in}}%
\pgfpathlineto{\pgfqpoint{2.456263in}{1.576707in}}%
\pgfpathlineto{\pgfqpoint{2.473641in}{1.552318in}}%
\pgfpathlineto{\pgfqpoint{2.493502in}{1.520373in}}%
\pgfpathlineto{\pgfqpoint{2.520811in}{1.471614in}}%
\pgfpathlineto{\pgfqpoint{2.590323in}{1.344633in}}%
\pgfpathlineto{\pgfqpoint{2.612667in}{1.309586in}}%
\pgfpathlineto{\pgfqpoint{2.630045in}{1.286024in}}%
\pgfpathlineto{\pgfqpoint{2.647423in}{1.266311in}}%
\pgfpathlineto{\pgfqpoint{2.662319in}{1.252808in}}%
\pgfpathlineto{\pgfqpoint{2.674732in}{1.244105in}}%
\pgfpathlineto{\pgfqpoint{2.687145in}{1.237803in}}%
\pgfpathlineto{\pgfqpoint{2.699558in}{1.233945in}}%
\pgfpathlineto{\pgfqpoint{2.711971in}{1.232537in}}%
\pgfpathlineto{\pgfqpoint{2.724384in}{1.233552in}}%
\pgfpathlineto{\pgfqpoint{2.736797in}{1.236931in}}%
\pgfpathlineto{\pgfqpoint{2.749210in}{1.242579in}}%
\pgfpathlineto{\pgfqpoint{2.764106in}{1.252181in}}%
\pgfpathlineto{\pgfqpoint{2.779001in}{1.264612in}}%
\pgfpathlineto{\pgfqpoint{2.796379in}{1.282270in}}%
\pgfpathlineto{\pgfqpoint{2.816240in}{1.305907in}}%
\pgfpathlineto{\pgfqpoint{2.841066in}{1.339232in}}%
\pgfpathlineto{\pgfqpoint{2.932923in}{1.467017in}}%
\pgfpathlineto{\pgfqpoint{2.952783in}{1.489291in}}%
\pgfpathlineto{\pgfqpoint{2.970162in}{1.505784in}}%
\pgfpathlineto{\pgfqpoint{2.987540in}{1.519086in}}%
\pgfpathlineto{\pgfqpoint{3.002436in}{1.527735in}}%
\pgfpathlineto{\pgfqpoint{3.017331in}{1.533720in}}%
\pgfpathlineto{\pgfqpoint{3.032227in}{1.536981in}}%
\pgfpathlineto{\pgfqpoint{3.044640in}{1.537615in}}%
\pgfpathlineto{\pgfqpoint{3.059535in}{1.535915in}}%
\pgfpathlineto{\pgfqpoint{3.074431in}{1.531624in}}%
\pgfpathlineto{\pgfqpoint{3.089327in}{1.524886in}}%
\pgfpathlineto{\pgfqpoint{3.106705in}{1.514190in}}%
\pgfpathlineto{\pgfqpoint{3.124083in}{1.500797in}}%
\pgfpathlineto{\pgfqpoint{3.143944in}{1.482751in}}%
\pgfpathlineto{\pgfqpoint{3.168770in}{1.457158in}}%
\pgfpathlineto{\pgfqpoint{3.220905in}{1.399022in}}%
\pgfpathlineto{\pgfqpoint{3.225870in}{1.393573in}}%
\pgfpathlineto{\pgfqpoint{3.225870in}{1.393573in}}%
\pgfusepath{stroke}%
\end{pgfscope}%
\begin{pgfscope}%
\pgfpathrectangle{\pgfqpoint{0.619136in}{0.571603in}}{\pgfqpoint{2.730864in}{1.657828in}}%
\pgfusepath{clip}%
\pgfsetrectcap%
\pgfsetroundjoin%
\pgfsetlinewidth{1.505625pt}%
\definecolor{currentstroke}{rgb}{0.549020,0.337255,0.294118}%
\pgfsetstrokecolor{currentstroke}%
\pgfsetdash{}{0pt}%
\pgfpathmoveto{\pgfqpoint{0.743267in}{0.646959in}}%
\pgfpathlineto{\pgfqpoint{0.748232in}{0.648670in}}%
\pgfpathlineto{\pgfqpoint{0.753197in}{0.653338in}}%
\pgfpathlineto{\pgfqpoint{0.758162in}{0.660717in}}%
\pgfpathlineto{\pgfqpoint{0.765610in}{0.676569in}}%
\pgfpathlineto{\pgfqpoint{0.773058in}{0.697835in}}%
\pgfpathlineto{\pgfqpoint{0.782988in}{0.734033in}}%
\pgfpathlineto{\pgfqpoint{0.792919in}{0.778472in}}%
\pgfpathlineto{\pgfqpoint{0.805332in}{0.844362in}}%
\pgfpathlineto{\pgfqpoint{0.820227in}{0.936354in}}%
\pgfpathlineto{\pgfqpoint{0.840088in}{1.075798in}}%
\pgfpathlineto{\pgfqpoint{0.869879in}{1.304997in}}%
\pgfpathlineto{\pgfqpoint{0.907118in}{1.588803in}}%
\pgfpathlineto{\pgfqpoint{0.926979in}{1.723436in}}%
\pgfpathlineto{\pgfqpoint{0.941875in}{1.811706in}}%
\pgfpathlineto{\pgfqpoint{0.956770in}{1.886668in}}%
\pgfpathlineto{\pgfqpoint{0.969183in}{1.937714in}}%
\pgfpathlineto{\pgfqpoint{0.979114in}{1.970519in}}%
\pgfpathlineto{\pgfqpoint{0.989044in}{1.995876in}}%
\pgfpathlineto{\pgfqpoint{0.996492in}{2.009894in}}%
\pgfpathlineto{\pgfqpoint{1.003940in}{2.019580in}}%
\pgfpathlineto{\pgfqpoint{1.011388in}{2.024922in}}%
\pgfpathlineto{\pgfqpoint{1.016353in}{2.026080in}}%
\pgfpathlineto{\pgfqpoint{1.021318in}{2.025330in}}%
\pgfpathlineto{\pgfqpoint{1.026283in}{2.022691in}}%
\pgfpathlineto{\pgfqpoint{1.033731in}{2.015247in}}%
\pgfpathlineto{\pgfqpoint{1.041179in}{2.003718in}}%
\pgfpathlineto{\pgfqpoint{1.048627in}{1.988237in}}%
\pgfpathlineto{\pgfqpoint{1.058557in}{1.961730in}}%
\pgfpathlineto{\pgfqpoint{1.068488in}{1.928929in}}%
\pgfpathlineto{\pgfqpoint{1.080901in}{1.879867in}}%
\pgfpathlineto{\pgfqpoint{1.095796in}{1.810688in}}%
\pgfpathlineto{\pgfqpoint{1.113174in}{1.718572in}}%
\pgfpathlineto{\pgfqpoint{1.138000in}{1.572972in}}%
\pgfpathlineto{\pgfqpoint{1.190135in}{1.261857in}}%
\pgfpathlineto{\pgfqpoint{1.209996in}{1.158254in}}%
\pgfpathlineto{\pgfqpoint{1.224892in}{1.090685in}}%
\pgfpathlineto{\pgfqpoint{1.239787in}{1.033614in}}%
\pgfpathlineto{\pgfqpoint{1.252200in}{0.995007in}}%
\pgfpathlineto{\pgfqpoint{1.262131in}{0.970389in}}%
\pgfpathlineto{\pgfqpoint{1.272061in}{0.951564in}}%
\pgfpathlineto{\pgfqpoint{1.279509in}{0.941325in}}%
\pgfpathlineto{\pgfqpoint{1.286957in}{0.934440in}}%
\pgfpathlineto{\pgfqpoint{1.294404in}{0.930911in}}%
\pgfpathlineto{\pgfqpoint{1.301852in}{0.930717in}}%
\pgfpathlineto{\pgfqpoint{1.309300in}{0.933813in}}%
\pgfpathlineto{\pgfqpoint{1.316748in}{0.940133in}}%
\pgfpathlineto{\pgfqpoint{1.324196in}{0.949588in}}%
\pgfpathlineto{\pgfqpoint{1.334126in}{0.966881in}}%
\pgfpathlineto{\pgfqpoint{1.344057in}{0.989243in}}%
\pgfpathlineto{\pgfqpoint{1.356470in}{1.023759in}}%
\pgfpathlineto{\pgfqpoint{1.368883in}{1.064791in}}%
\pgfpathlineto{\pgfqpoint{1.383778in}{1.121310in}}%
\pgfpathlineto{\pgfqpoint{1.403639in}{1.206093in}}%
\pgfpathlineto{\pgfqpoint{1.435913in}{1.355895in}}%
\pgfpathlineto{\pgfqpoint{1.468187in}{1.503269in}}%
\pgfpathlineto{\pgfqpoint{1.488048in}{1.584656in}}%
\pgfpathlineto{\pgfqpoint{1.505426in}{1.646452in}}%
\pgfpathlineto{\pgfqpoint{1.520321in}{1.690791in}}%
\pgfpathlineto{\pgfqpoint{1.532734in}{1.720876in}}%
\pgfpathlineto{\pgfqpoint{1.542665in}{1.740130in}}%
\pgfpathlineto{\pgfqpoint{1.552595in}{1.754930in}}%
\pgfpathlineto{\pgfqpoint{1.562526in}{1.765174in}}%
\pgfpathlineto{\pgfqpoint{1.569973in}{1.769839in}}%
\pgfpathlineto{\pgfqpoint{1.577421in}{1.771920in}}%
\pgfpathlineto{\pgfqpoint{1.584869in}{1.771437in}}%
\pgfpathlineto{\pgfqpoint{1.592317in}{1.768428in}}%
\pgfpathlineto{\pgfqpoint{1.599765in}{1.762949in}}%
\pgfpathlineto{\pgfqpoint{1.607212in}{1.755071in}}%
\pgfpathlineto{\pgfqpoint{1.617143in}{1.740988in}}%
\pgfpathlineto{\pgfqpoint{1.627073in}{1.723044in}}%
\pgfpathlineto{\pgfqpoint{1.639486in}{1.695631in}}%
\pgfpathlineto{\pgfqpoint{1.654382in}{1.656293in}}%
\pgfpathlineto{\pgfqpoint{1.671760in}{1.603139in}}%
\pgfpathlineto{\pgfqpoint{1.694104in}{1.526766in}}%
\pgfpathlineto{\pgfqpoint{1.761134in}{1.291719in}}%
\pgfpathlineto{\pgfqpoint{1.778512in}{1.240551in}}%
\pgfpathlineto{\pgfqpoint{1.793408in}{1.202791in}}%
\pgfpathlineto{\pgfqpoint{1.805821in}{1.176310in}}%
\pgfpathlineto{\pgfqpoint{1.818234in}{1.154795in}}%
\pgfpathlineto{\pgfqpoint{1.828164in}{1.141362in}}%
\pgfpathlineto{\pgfqpoint{1.838095in}{1.131403in}}%
\pgfpathlineto{\pgfqpoint{1.848025in}{1.124978in}}%
\pgfpathlineto{\pgfqpoint{1.855473in}{1.122488in}}%
\pgfpathlineto{\pgfqpoint{1.862921in}{1.121984in}}%
\pgfpathlineto{\pgfqpoint{1.870368in}{1.123443in}}%
\pgfpathlineto{\pgfqpoint{1.877816in}{1.126827in}}%
\pgfpathlineto{\pgfqpoint{1.887747in}{1.134248in}}%
\pgfpathlineto{\pgfqpoint{1.897677in}{1.144850in}}%
\pgfpathlineto{\pgfqpoint{1.907607in}{1.158446in}}%
\pgfpathlineto{\pgfqpoint{1.920021in}{1.179311in}}%
\pgfpathlineto{\pgfqpoint{1.934916in}{1.209362in}}%
\pgfpathlineto{\pgfqpoint{1.952294in}{1.250088in}}%
\pgfpathlineto{\pgfqpoint{1.974638in}{1.308763in}}%
\pgfpathlineto{\pgfqpoint{2.044151in}{1.496144in}}%
\pgfpathlineto{\pgfqpoint{2.061529in}{1.535044in}}%
\pgfpathlineto{\pgfqpoint{2.076425in}{1.563591in}}%
\pgfpathlineto{\pgfqpoint{2.088838in}{1.583483in}}%
\pgfpathlineto{\pgfqpoint{2.101251in}{1.599510in}}%
\pgfpathlineto{\pgfqpoint{2.111181in}{1.609397in}}%
\pgfpathlineto{\pgfqpoint{2.121111in}{1.616596in}}%
\pgfpathlineto{\pgfqpoint{2.131042in}{1.621065in}}%
\pgfpathlineto{\pgfqpoint{2.140972in}{1.622801in}}%
\pgfpathlineto{\pgfqpoint{2.150903in}{1.621831in}}%
\pgfpathlineto{\pgfqpoint{2.160833in}{1.618216in}}%
\pgfpathlineto{\pgfqpoint{2.170763in}{1.612048in}}%
\pgfpathlineto{\pgfqpoint{2.180694in}{1.603451in}}%
\pgfpathlineto{\pgfqpoint{2.193107in}{1.589516in}}%
\pgfpathlineto{\pgfqpoint{2.205520in}{1.572366in}}%
\pgfpathlineto{\pgfqpoint{2.220416in}{1.548109in}}%
\pgfpathlineto{\pgfqpoint{2.240276in}{1.510813in}}%
\pgfpathlineto{\pgfqpoint{2.267585in}{1.453759in}}%
\pgfpathlineto{\pgfqpoint{2.314754in}{1.354432in}}%
\pgfpathlineto{\pgfqpoint{2.334615in}{1.317699in}}%
\pgfpathlineto{\pgfqpoint{2.351993in}{1.290047in}}%
\pgfpathlineto{\pgfqpoint{2.366889in}{1.270404in}}%
\pgfpathlineto{\pgfqpoint{2.379302in}{1.257237in}}%
\pgfpathlineto{\pgfqpoint{2.391715in}{1.247183in}}%
\pgfpathlineto{\pgfqpoint{2.404128in}{1.240362in}}%
\pgfpathlineto{\pgfqpoint{2.414059in}{1.237274in}}%
\pgfpathlineto{\pgfqpoint{2.423989in}{1.236292in}}%
\pgfpathlineto{\pgfqpoint{2.433919in}{1.237391in}}%
\pgfpathlineto{\pgfqpoint{2.443850in}{1.240519in}}%
\pgfpathlineto{\pgfqpoint{2.456263in}{1.247166in}}%
\pgfpathlineto{\pgfqpoint{2.468676in}{1.256667in}}%
\pgfpathlineto{\pgfqpoint{2.483571in}{1.271488in}}%
\pgfpathlineto{\pgfqpoint{2.498467in}{1.289554in}}%
\pgfpathlineto{\pgfqpoint{2.518328in}{1.317674in}}%
\pgfpathlineto{\pgfqpoint{2.543154in}{1.357125in}}%
\pgfpathlineto{\pgfqpoint{2.600254in}{1.449465in}}%
\pgfpathlineto{\pgfqpoint{2.620115in}{1.476979in}}%
\pgfpathlineto{\pgfqpoint{2.637493in}{1.497395in}}%
\pgfpathlineto{\pgfqpoint{2.652388in}{1.511651in}}%
\pgfpathlineto{\pgfqpoint{2.667284in}{1.522582in}}%
\pgfpathlineto{\pgfqpoint{2.679697in}{1.528996in}}%
\pgfpathlineto{\pgfqpoint{2.692110in}{1.532887in}}%
\pgfpathlineto{\pgfqpoint{2.704523in}{1.534237in}}%
\pgfpathlineto{\pgfqpoint{2.716936in}{1.533078in}}%
\pgfpathlineto{\pgfqpoint{2.729349in}{1.529490in}}%
\pgfpathlineto{\pgfqpoint{2.741762in}{1.523595in}}%
\pgfpathlineto{\pgfqpoint{2.756658in}{1.513712in}}%
\pgfpathlineto{\pgfqpoint{2.771553in}{1.501101in}}%
\pgfpathlineto{\pgfqpoint{2.788932in}{1.483505in}}%
\pgfpathlineto{\pgfqpoint{2.811275in}{1.457491in}}%
\pgfpathlineto{\pgfqpoint{2.855962in}{1.400710in}}%
\pgfpathlineto{\pgfqpoint{2.885753in}{1.365042in}}%
\pgfpathlineto{\pgfqpoint{2.905614in}{1.344480in}}%
\pgfpathlineto{\pgfqpoint{2.922992in}{1.329453in}}%
\pgfpathlineto{\pgfqpoint{2.937888in}{1.319158in}}%
\pgfpathlineto{\pgfqpoint{2.952783in}{1.311480in}}%
\pgfpathlineto{\pgfqpoint{2.967679in}{1.306560in}}%
\pgfpathlineto{\pgfqpoint{2.980092in}{1.304612in}}%
\pgfpathlineto{\pgfqpoint{2.992505in}{1.304614in}}%
\pgfpathlineto{\pgfqpoint{3.004918in}{1.306524in}}%
\pgfpathlineto{\pgfqpoint{3.019814in}{1.311216in}}%
\pgfpathlineto{\pgfqpoint{3.034709in}{1.318338in}}%
\pgfpathlineto{\pgfqpoint{3.052088in}{1.329364in}}%
\pgfpathlineto{\pgfqpoint{3.071948in}{1.344926in}}%
\pgfpathlineto{\pgfqpoint{3.096774in}{1.367533in}}%
\pgfpathlineto{\pgfqpoint{3.173735in}{1.440420in}}%
\pgfpathlineto{\pgfqpoint{3.193596in}{1.455396in}}%
\pgfpathlineto{\pgfqpoint{3.210974in}{1.466050in}}%
\pgfpathlineto{\pgfqpoint{3.225870in}{1.473098in}}%
\pgfpathlineto{\pgfqpoint{3.225870in}{1.473098in}}%
\pgfusepath{stroke}%
\end{pgfscope}%
\begin{pgfscope}%
\pgfpathrectangle{\pgfqpoint{0.619136in}{0.571603in}}{\pgfqpoint{2.730864in}{1.657828in}}%
\pgfusepath{clip}%
\pgfsetrectcap%
\pgfsetroundjoin%
\pgfsetlinewidth{1.505625pt}%
\definecolor{currentstroke}{rgb}{0.890196,0.466667,0.760784}%
\pgfsetstrokecolor{currentstroke}%
\pgfsetdash{}{0pt}%
\pgfpathmoveto{\pgfqpoint{0.743267in}{0.646959in}}%
\pgfpathlineto{\pgfqpoint{0.748232in}{0.648225in}}%
\pgfpathlineto{\pgfqpoint{0.753197in}{0.652000in}}%
\pgfpathlineto{\pgfqpoint{0.758162in}{0.658262in}}%
\pgfpathlineto{\pgfqpoint{0.765610in}{0.672258in}}%
\pgfpathlineto{\pgfqpoint{0.773058in}{0.691672in}}%
\pgfpathlineto{\pgfqpoint{0.782988in}{0.725716in}}%
\pgfpathlineto{\pgfqpoint{0.792919in}{0.768651in}}%
\pgfpathlineto{\pgfqpoint{0.805332in}{0.833935in}}%
\pgfpathlineto{\pgfqpoint{0.820227in}{0.927484in}}%
\pgfpathlineto{\pgfqpoint{0.837605in}{1.054024in}}%
\pgfpathlineto{\pgfqpoint{0.859949in}{1.236294in}}%
\pgfpathlineto{\pgfqpoint{0.924497in}{1.778253in}}%
\pgfpathlineto{\pgfqpoint{0.941875in}{1.900448in}}%
\pgfpathlineto{\pgfqpoint{0.956770in}{1.989411in}}%
\pgfpathlineto{\pgfqpoint{0.969183in}{2.050406in}}%
\pgfpathlineto{\pgfqpoint{0.979114in}{2.089695in}}%
\pgfpathlineto{\pgfqpoint{0.989044in}{2.119976in}}%
\pgfpathlineto{\pgfqpoint{0.996492in}{2.136538in}}%
\pgfpathlineto{\pgfqpoint{1.003940in}{2.147693in}}%
\pgfpathlineto{\pgfqpoint{1.008905in}{2.152085in}}%
\pgfpathlineto{\pgfqpoint{1.013870in}{2.154026in}}%
\pgfpathlineto{\pgfqpoint{1.018836in}{2.153510in}}%
\pgfpathlineto{\pgfqpoint{1.023801in}{2.150539in}}%
\pgfpathlineto{\pgfqpoint{1.028766in}{2.145125in}}%
\pgfpathlineto{\pgfqpoint{1.036214in}{2.132466in}}%
\pgfpathlineto{\pgfqpoint{1.043662in}{2.114451in}}%
\pgfpathlineto{\pgfqpoint{1.051109in}{2.091216in}}%
\pgfpathlineto{\pgfqpoint{1.061040in}{2.052423in}}%
\pgfpathlineto{\pgfqpoint{1.070970in}{2.005183in}}%
\pgfpathlineto{\pgfqpoint{1.083383in}{1.935215in}}%
\pgfpathlineto{\pgfqpoint{1.098279in}{1.837181in}}%
\pgfpathlineto{\pgfqpoint{1.115657in}{1.707119in}}%
\pgfpathlineto{\pgfqpoint{1.140483in}{1.502084in}}%
\pgfpathlineto{\pgfqpoint{1.190135in}{1.086929in}}%
\pgfpathlineto{\pgfqpoint{1.207513in}{0.959805in}}%
\pgfpathlineto{\pgfqpoint{1.222409in}{0.864990in}}%
\pgfpathlineto{\pgfqpoint{1.234822in}{0.798096in}}%
\pgfpathlineto{\pgfqpoint{1.247235in}{0.743678in}}%
\pgfpathlineto{\pgfqpoint{1.257165in}{0.709872in}}%
\pgfpathlineto{\pgfqpoint{1.264613in}{0.690487in}}%
\pgfpathlineto{\pgfqpoint{1.272061in}{0.676379in}}%
\pgfpathlineto{\pgfqpoint{1.279509in}{0.667645in}}%
\pgfpathlineto{\pgfqpoint{1.284474in}{0.664841in}}%
\pgfpathlineto{\pgfqpoint{1.289439in}{0.664462in}}%
\pgfpathlineto{\pgfqpoint{1.294404in}{0.666508in}}%
\pgfpathlineto{\pgfqpoint{1.299370in}{0.670972in}}%
\pgfpathlineto{\pgfqpoint{1.304335in}{0.677836in}}%
\pgfpathlineto{\pgfqpoint{1.311783in}{0.692582in}}%
\pgfpathlineto{\pgfqpoint{1.319231in}{0.712565in}}%
\pgfpathlineto{\pgfqpoint{1.329161in}{0.747086in}}%
\pgfpathlineto{\pgfqpoint{1.339091in}{0.790182in}}%
\pgfpathlineto{\pgfqpoint{1.351504in}{0.855229in}}%
\pgfpathlineto{\pgfqpoint{1.366400in}{0.947871in}}%
\pgfpathlineto{\pgfqpoint{1.383778in}{1.072546in}}%
\pgfpathlineto{\pgfqpoint{1.406122in}{1.251280in}}%
\pgfpathlineto{\pgfqpoint{1.468187in}{1.760154in}}%
\pgfpathlineto{\pgfqpoint{1.485565in}{1.880396in}}%
\pgfpathlineto{\pgfqpoint{1.500461in}{1.968309in}}%
\pgfpathlineto{\pgfqpoint{1.512874in}{2.028896in}}%
\pgfpathlineto{\pgfqpoint{1.522804in}{2.068170in}}%
\pgfpathlineto{\pgfqpoint{1.532734in}{2.098715in}}%
\pgfpathlineto{\pgfqpoint{1.540182in}{2.115654in}}%
\pgfpathlineto{\pgfqpoint{1.547630in}{2.127338in}}%
\pgfpathlineto{\pgfqpoint{1.552595in}{2.132166in}}%
\pgfpathlineto{\pgfqpoint{1.557560in}{2.134606in}}%
\pgfpathlineto{\pgfqpoint{1.562526in}{2.134652in}}%
\pgfpathlineto{\pgfqpoint{1.567491in}{2.132304in}}%
\pgfpathlineto{\pgfqpoint{1.572456in}{2.127571in}}%
\pgfpathlineto{\pgfqpoint{1.579904in}{2.116039in}}%
\pgfpathlineto{\pgfqpoint{1.587352in}{2.099270in}}%
\pgfpathlineto{\pgfqpoint{1.594799in}{2.077390in}}%
\pgfpathlineto{\pgfqpoint{1.604730in}{2.040558in}}%
\pgfpathlineto{\pgfqpoint{1.614660in}{1.995429in}}%
\pgfpathlineto{\pgfqpoint{1.627073in}{1.928270in}}%
\pgfpathlineto{\pgfqpoint{1.641969in}{1.833780in}}%
\pgfpathlineto{\pgfqpoint{1.659347in}{1.707960in}}%
\pgfpathlineto{\pgfqpoint{1.684173in}{1.508864in}}%
\pgfpathlineto{\pgfqpoint{1.736308in}{1.084714in}}%
\pgfpathlineto{\pgfqpoint{1.753686in}{0.962018in}}%
\pgfpathlineto{\pgfqpoint{1.768582in}{0.870925in}}%
\pgfpathlineto{\pgfqpoint{1.780995in}{0.806999in}}%
\pgfpathlineto{\pgfqpoint{1.793408in}{0.755360in}}%
\pgfpathlineto{\pgfqpoint{1.803338in}{0.723605in}}%
\pgfpathlineto{\pgfqpoint{1.810786in}{0.705641in}}%
\pgfpathlineto{\pgfqpoint{1.818234in}{0.692838in}}%
\pgfpathlineto{\pgfqpoint{1.825682in}{0.685289in}}%
\pgfpathlineto{\pgfqpoint{1.830647in}{0.683202in}}%
\pgfpathlineto{\pgfqpoint{1.835612in}{0.683479in}}%
\pgfpathlineto{\pgfqpoint{1.840577in}{0.686118in}}%
\pgfpathlineto{\pgfqpoint{1.845542in}{0.691110in}}%
\pgfpathlineto{\pgfqpoint{1.852990in}{0.702968in}}%
\pgfpathlineto{\pgfqpoint{1.860438in}{0.719987in}}%
\pgfpathlineto{\pgfqpoint{1.867886in}{0.742038in}}%
\pgfpathlineto{\pgfqpoint{1.877816in}{0.778975in}}%
\pgfpathlineto{\pgfqpoint{1.887747in}{0.824066in}}%
\pgfpathlineto{\pgfqpoint{1.900160in}{0.890976in}}%
\pgfpathlineto{\pgfqpoint{1.915055in}{0.984883in}}%
\pgfpathlineto{\pgfqpoint{1.932434in}{1.109653in}}%
\pgfpathlineto{\pgfqpoint{1.957260in}{1.306642in}}%
\pgfpathlineto{\pgfqpoint{2.006912in}{1.706418in}}%
\pgfpathlineto{\pgfqpoint{2.026772in}{1.845346in}}%
\pgfpathlineto{\pgfqpoint{2.041668in}{1.934619in}}%
\pgfpathlineto{\pgfqpoint{2.054081in}{1.997098in}}%
\pgfpathlineto{\pgfqpoint{2.066494in}{2.047382in}}%
\pgfpathlineto{\pgfqpoint{2.076425in}{2.078140in}}%
\pgfpathlineto{\pgfqpoint{2.083872in}{2.095416in}}%
\pgfpathlineto{\pgfqpoint{2.091320in}{2.107586in}}%
\pgfpathlineto{\pgfqpoint{2.096285in}{2.112818in}}%
\pgfpathlineto{\pgfqpoint{2.101251in}{2.115726in}}%
\pgfpathlineto{\pgfqpoint{2.106216in}{2.116301in}}%
\pgfpathlineto{\pgfqpoint{2.111181in}{2.114542in}}%
\pgfpathlineto{\pgfqpoint{2.116146in}{2.110457in}}%
\pgfpathlineto{\pgfqpoint{2.121111in}{2.104059in}}%
\pgfpathlineto{\pgfqpoint{2.128559in}{2.090176in}}%
\pgfpathlineto{\pgfqpoint{2.136007in}{2.071249in}}%
\pgfpathlineto{\pgfqpoint{2.145937in}{2.038415in}}%
\pgfpathlineto{\pgfqpoint{2.155868in}{1.997308in}}%
\pgfpathlineto{\pgfqpoint{2.168281in}{1.935132in}}%
\pgfpathlineto{\pgfqpoint{2.183176in}{1.846418in}}%
\pgfpathlineto{\pgfqpoint{2.200555in}{1.726845in}}%
\pgfpathlineto{\pgfqpoint{2.222898in}{1.555171in}}%
\pgfpathlineto{\pgfqpoint{2.284963in}{1.065128in}}%
\pgfpathlineto{\pgfqpoint{2.302341in}{0.949008in}}%
\pgfpathlineto{\pgfqpoint{2.317237in}{0.863967in}}%
\pgfpathlineto{\pgfqpoint{2.329650in}{0.805241in}}%
\pgfpathlineto{\pgfqpoint{2.339580in}{0.767080in}}%
\pgfpathlineto{\pgfqpoint{2.349511in}{0.737299in}}%
\pgfpathlineto{\pgfqpoint{2.356959in}{0.720698in}}%
\pgfpathlineto{\pgfqpoint{2.364406in}{0.709146in}}%
\pgfpathlineto{\pgfqpoint{2.369372in}{0.704293in}}%
\pgfpathlineto{\pgfqpoint{2.374337in}{0.701736in}}%
\pgfpathlineto{\pgfqpoint{2.379302in}{0.701482in}}%
\pgfpathlineto{\pgfqpoint{2.384267in}{0.703532in}}%
\pgfpathlineto{\pgfqpoint{2.389233in}{0.707878in}}%
\pgfpathlineto{\pgfqpoint{2.396680in}{0.718666in}}%
\pgfpathlineto{\pgfqpoint{2.404128in}{0.734499in}}%
\pgfpathlineto{\pgfqpoint{2.411576in}{0.755258in}}%
\pgfpathlineto{\pgfqpoint{2.421506in}{0.790324in}}%
\pgfpathlineto{\pgfqpoint{2.431437in}{0.833397in}}%
\pgfpathlineto{\pgfqpoint{2.443850in}{0.897619in}}%
\pgfpathlineto{\pgfqpoint{2.458745in}{0.988129in}}%
\pgfpathlineto{\pgfqpoint{2.476124in}{1.108827in}}%
\pgfpathlineto{\pgfqpoint{2.500950in}{1.300109in}}%
\pgfpathlineto{\pgfqpoint{2.553084in}{1.708569in}}%
\pgfpathlineto{\pgfqpoint{2.570463in}{1.827012in}}%
\pgfpathlineto{\pgfqpoint{2.585358in}{1.915083in}}%
\pgfpathlineto{\pgfqpoint{2.597771in}{1.976999in}}%
\pgfpathlineto{\pgfqpoint{2.610184in}{2.027133in}}%
\pgfpathlineto{\pgfqpoint{2.620115in}{2.058069in}}%
\pgfpathlineto{\pgfqpoint{2.627562in}{2.075650in}}%
\pgfpathlineto{\pgfqpoint{2.635010in}{2.088271in}}%
\pgfpathlineto{\pgfqpoint{2.642458in}{2.095842in}}%
\pgfpathlineto{\pgfqpoint{2.647423in}{2.098056in}}%
\pgfpathlineto{\pgfqpoint{2.652388in}{2.097996in}}%
\pgfpathlineto{\pgfqpoint{2.657354in}{2.095662in}}%
\pgfpathlineto{\pgfqpoint{2.662319in}{2.091063in}}%
\pgfpathlineto{\pgfqpoint{2.669767in}{2.079956in}}%
\pgfpathlineto{\pgfqpoint{2.677215in}{2.063876in}}%
\pgfpathlineto{\pgfqpoint{2.684662in}{2.042947in}}%
\pgfpathlineto{\pgfqpoint{2.694593in}{2.007773in}}%
\pgfpathlineto{\pgfqpoint{2.704523in}{1.964730in}}%
\pgfpathlineto{\pgfqpoint{2.716936in}{1.900737in}}%
\pgfpathlineto{\pgfqpoint{2.731832in}{1.810778in}}%
\pgfpathlineto{\pgfqpoint{2.749210in}{1.691078in}}%
\pgfpathlineto{\pgfqpoint{2.774036in}{1.501811in}}%
\pgfpathlineto{\pgfqpoint{2.823688in}{1.116832in}}%
\pgfpathlineto{\pgfqpoint{2.843549in}{0.982723in}}%
\pgfpathlineto{\pgfqpoint{2.858445in}{0.896403in}}%
\pgfpathlineto{\pgfqpoint{2.870858in}{0.835882in}}%
\pgfpathlineto{\pgfqpoint{2.883271in}{0.787054in}}%
\pgfpathlineto{\pgfqpoint{2.893201in}{0.757081in}}%
\pgfpathlineto{\pgfqpoint{2.900649in}{0.740164in}}%
\pgfpathlineto{\pgfqpoint{2.908097in}{0.728155in}}%
\pgfpathlineto{\pgfqpoint{2.915544in}{0.721137in}}%
\pgfpathlineto{\pgfqpoint{2.920510in}{0.719257in}}%
\pgfpathlineto{\pgfqpoint{2.925475in}{0.719623in}}%
\pgfpathlineto{\pgfqpoint{2.930440in}{0.722233in}}%
\pgfpathlineto{\pgfqpoint{2.935405in}{0.727077in}}%
\pgfpathlineto{\pgfqpoint{2.942853in}{0.738493in}}%
\pgfpathlineto{\pgfqpoint{2.950301in}{0.754809in}}%
\pgfpathlineto{\pgfqpoint{2.957749in}{0.775899in}}%
\pgfpathlineto{\pgfqpoint{2.967679in}{0.811170in}}%
\pgfpathlineto{\pgfqpoint{2.977610in}{0.854175in}}%
\pgfpathlineto{\pgfqpoint{2.990023in}{0.917929in}}%
\pgfpathlineto{\pgfqpoint{3.004918in}{1.007332in}}%
\pgfpathlineto{\pgfqpoint{3.022296in}{1.126030in}}%
\pgfpathlineto{\pgfqpoint{3.047122in}{1.313291in}}%
\pgfpathlineto{\pgfqpoint{3.096774in}{1.692889in}}%
\pgfpathlineto{\pgfqpoint{3.114153in}{1.809258in}}%
\pgfpathlineto{\pgfqpoint{3.129048in}{1.896117in}}%
\pgfpathlineto{\pgfqpoint{3.141461in}{1.957449in}}%
\pgfpathlineto{\pgfqpoint{3.153874in}{2.007399in}}%
\pgfpathlineto{\pgfqpoint{3.163805in}{2.038478in}}%
\pgfpathlineto{\pgfqpoint{3.171253in}{2.056336in}}%
\pgfpathlineto{\pgfqpoint{3.178700in}{2.069375in}}%
\pgfpathlineto{\pgfqpoint{3.186148in}{2.077503in}}%
\pgfpathlineto{\pgfqpoint{3.191113in}{2.080164in}}%
\pgfpathlineto{\pgfqpoint{3.196079in}{2.080609in}}%
\pgfpathlineto{\pgfqpoint{3.201044in}{2.078837in}}%
\pgfpathlineto{\pgfqpoint{3.206009in}{2.074855in}}%
\pgfpathlineto{\pgfqpoint{3.213457in}{2.064771in}}%
\pgfpathlineto{\pgfqpoint{3.220905in}{2.049826in}}%
\pgfpathlineto{\pgfqpoint{3.225870in}{2.037217in}}%
\pgfpathlineto{\pgfqpoint{3.225870in}{2.037217in}}%
\pgfusepath{stroke}%
\end{pgfscope}%
\begin{pgfscope}%
\pgfpathrectangle{\pgfqpoint{0.619136in}{0.571603in}}{\pgfqpoint{2.730864in}{1.657828in}}%
\pgfusepath{clip}%
\pgfsetrectcap%
\pgfsetroundjoin%
\pgfsetlinewidth{1.505625pt}%
\definecolor{currentstroke}{rgb}{0.498039,0.498039,0.498039}%
\pgfsetstrokecolor{currentstroke}%
\pgfsetdash{}{0pt}%
\pgfpathmoveto{\pgfqpoint{0.743267in}{0.646959in}}%
\pgfpathlineto{\pgfqpoint{0.748232in}{0.648781in}}%
\pgfpathlineto{\pgfqpoint{0.753197in}{0.653643in}}%
\pgfpathlineto{\pgfqpoint{0.758162in}{0.661240in}}%
\pgfpathlineto{\pgfqpoint{0.765610in}{0.677402in}}%
\pgfpathlineto{\pgfqpoint{0.773058in}{0.698909in}}%
\pgfpathlineto{\pgfqpoint{0.782988in}{0.735255in}}%
\pgfpathlineto{\pgfqpoint{0.795401in}{0.791824in}}%
\pgfpathlineto{\pgfqpoint{0.807814in}{0.859201in}}%
\pgfpathlineto{\pgfqpoint{0.822710in}{0.951868in}}%
\pgfpathlineto{\pgfqpoint{0.842571in}{1.090306in}}%
\pgfpathlineto{\pgfqpoint{0.879810in}{1.370736in}}%
\pgfpathlineto{\pgfqpoint{0.909601in}{1.587010in}}%
\pgfpathlineto{\pgfqpoint{0.929462in}{1.714828in}}%
\pgfpathlineto{\pgfqpoint{0.944357in}{1.798063in}}%
\pgfpathlineto{\pgfqpoint{0.959253in}{1.868344in}}%
\pgfpathlineto{\pgfqpoint{0.971666in}{1.915944in}}%
\pgfpathlineto{\pgfqpoint{0.981596in}{1.946382in}}%
\pgfpathlineto{\pgfqpoint{0.991527in}{1.969786in}}%
\pgfpathlineto{\pgfqpoint{0.998975in}{1.982642in}}%
\pgfpathlineto{\pgfqpoint{1.006423in}{1.991446in}}%
\pgfpathlineto{\pgfqpoint{1.013870in}{1.996205in}}%
\pgfpathlineto{\pgfqpoint{1.018836in}{1.997146in}}%
\pgfpathlineto{\pgfqpoint{1.023801in}{1.996319in}}%
\pgfpathlineto{\pgfqpoint{1.028766in}{1.993748in}}%
\pgfpathlineto{\pgfqpoint{1.036214in}{1.986678in}}%
\pgfpathlineto{\pgfqpoint{1.043662in}{1.975853in}}%
\pgfpathlineto{\pgfqpoint{1.051109in}{1.961407in}}%
\pgfpathlineto{\pgfqpoint{1.061040in}{1.936787in}}%
\pgfpathlineto{\pgfqpoint{1.070970in}{1.906438in}}%
\pgfpathlineto{\pgfqpoint{1.083383in}{1.861191in}}%
\pgfpathlineto{\pgfqpoint{1.098279in}{1.797589in}}%
\pgfpathlineto{\pgfqpoint{1.115657in}{1.713137in}}%
\pgfpathlineto{\pgfqpoint{1.140483in}{1.579989in}}%
\pgfpathlineto{\pgfqpoint{1.192618in}{1.295770in}}%
\pgfpathlineto{\pgfqpoint{1.212479in}{1.200746in}}%
\pgfpathlineto{\pgfqpoint{1.229857in}{1.128896in}}%
\pgfpathlineto{\pgfqpoint{1.244752in}{1.077462in}}%
\pgfpathlineto{\pgfqpoint{1.257165in}{1.042584in}}%
\pgfpathlineto{\pgfqpoint{1.267096in}{1.020235in}}%
\pgfpathlineto{\pgfqpoint{1.277026in}{1.002997in}}%
\pgfpathlineto{\pgfqpoint{1.284474in}{0.993478in}}%
\pgfpathlineto{\pgfqpoint{1.291922in}{0.986900in}}%
\pgfpathlineto{\pgfqpoint{1.299370in}{0.983257in}}%
\pgfpathlineto{\pgfqpoint{1.306817in}{0.982525in}}%
\pgfpathlineto{\pgfqpoint{1.314265in}{0.984659in}}%
\pgfpathlineto{\pgfqpoint{1.321713in}{0.989597in}}%
\pgfpathlineto{\pgfqpoint{1.329161in}{0.997257in}}%
\pgfpathlineto{\pgfqpoint{1.339091in}{1.011535in}}%
\pgfpathlineto{\pgfqpoint{1.349022in}{1.030203in}}%
\pgfpathlineto{\pgfqpoint{1.361435in}{1.059219in}}%
\pgfpathlineto{\pgfqpoint{1.373848in}{1.093873in}}%
\pgfpathlineto{\pgfqpoint{1.388743in}{1.141766in}}%
\pgfpathlineto{\pgfqpoint{1.408604in}{1.213828in}}%
\pgfpathlineto{\pgfqpoint{1.440878in}{1.341681in}}%
\pgfpathlineto{\pgfqpoint{1.475635in}{1.477546in}}%
\pgfpathlineto{\pgfqpoint{1.495495in}{1.547094in}}%
\pgfpathlineto{\pgfqpoint{1.512874in}{1.600067in}}%
\pgfpathlineto{\pgfqpoint{1.527769in}{1.638301in}}%
\pgfpathlineto{\pgfqpoint{1.540182in}{1.664483in}}%
\pgfpathlineto{\pgfqpoint{1.552595in}{1.685130in}}%
\pgfpathlineto{\pgfqpoint{1.562526in}{1.697504in}}%
\pgfpathlineto{\pgfqpoint{1.572456in}{1.706120in}}%
\pgfpathlineto{\pgfqpoint{1.579904in}{1.710104in}}%
\pgfpathlineto{\pgfqpoint{1.587352in}{1.711971in}}%
\pgfpathlineto{\pgfqpoint{1.594799in}{1.711744in}}%
\pgfpathlineto{\pgfqpoint{1.602247in}{1.709460in}}%
\pgfpathlineto{\pgfqpoint{1.609695in}{1.705169in}}%
\pgfpathlineto{\pgfqpoint{1.619626in}{1.696434in}}%
\pgfpathlineto{\pgfqpoint{1.629556in}{1.684424in}}%
\pgfpathlineto{\pgfqpoint{1.641969in}{1.665138in}}%
\pgfpathlineto{\pgfqpoint{1.654382in}{1.641565in}}%
\pgfpathlineto{\pgfqpoint{1.669278in}{1.608404in}}%
\pgfpathlineto{\pgfqpoint{1.686656in}{1.564351in}}%
\pgfpathlineto{\pgfqpoint{1.711482in}{1.494872in}}%
\pgfpathlineto{\pgfqpoint{1.766099in}{1.340011in}}%
\pgfpathlineto{\pgfqpoint{1.785960in}{1.291229in}}%
\pgfpathlineto{\pgfqpoint{1.803338in}{1.254643in}}%
\pgfpathlineto{\pgfqpoint{1.818234in}{1.228703in}}%
\pgfpathlineto{\pgfqpoint{1.830647in}{1.211318in}}%
\pgfpathlineto{\pgfqpoint{1.843060in}{1.198013in}}%
\pgfpathlineto{\pgfqpoint{1.852990in}{1.190400in}}%
\pgfpathlineto{\pgfqpoint{1.862921in}{1.185515in}}%
\pgfpathlineto{\pgfqpoint{1.872851in}{1.183356in}}%
\pgfpathlineto{\pgfqpoint{1.882781in}{1.183888in}}%
\pgfpathlineto{\pgfqpoint{1.892712in}{1.187045in}}%
\pgfpathlineto{\pgfqpoint{1.902642in}{1.192729in}}%
\pgfpathlineto{\pgfqpoint{1.912573in}{1.200817in}}%
\pgfpathlineto{\pgfqpoint{1.924986in}{1.214076in}}%
\pgfpathlineto{\pgfqpoint{1.937399in}{1.230511in}}%
\pgfpathlineto{\pgfqpoint{1.952294in}{1.253866in}}%
\pgfpathlineto{\pgfqpoint{1.969673in}{1.285162in}}%
\pgfpathlineto{\pgfqpoint{1.994499in}{1.334936in}}%
\pgfpathlineto{\pgfqpoint{2.054081in}{1.456628in}}%
\pgfpathlineto{\pgfqpoint{2.073942in}{1.491255in}}%
\pgfpathlineto{\pgfqpoint{2.091320in}{1.517008in}}%
\pgfpathlineto{\pgfqpoint{2.106216in}{1.535085in}}%
\pgfpathlineto{\pgfqpoint{2.118629in}{1.547049in}}%
\pgfpathlineto{\pgfqpoint{2.131042in}{1.556039in}}%
\pgfpathlineto{\pgfqpoint{2.143455in}{1.561971in}}%
\pgfpathlineto{\pgfqpoint{2.153385in}{1.564495in}}%
\pgfpathlineto{\pgfqpoint{2.163316in}{1.565058in}}%
\pgfpathlineto{\pgfqpoint{2.173246in}{1.563697in}}%
\pgfpathlineto{\pgfqpoint{2.185659in}{1.559383in}}%
\pgfpathlineto{\pgfqpoint{2.198072in}{1.552317in}}%
\pgfpathlineto{\pgfqpoint{2.210485in}{1.542707in}}%
\pgfpathlineto{\pgfqpoint{2.225381in}{1.528171in}}%
\pgfpathlineto{\pgfqpoint{2.242759in}{1.507705in}}%
\pgfpathlineto{\pgfqpoint{2.262620in}{1.480782in}}%
\pgfpathlineto{\pgfqpoint{2.294894in}{1.432596in}}%
\pgfpathlineto{\pgfqpoint{2.334615in}{1.373943in}}%
\pgfpathlineto{\pgfqpoint{2.356959in}{1.345044in}}%
\pgfpathlineto{\pgfqpoint{2.374337in}{1.325933in}}%
\pgfpathlineto{\pgfqpoint{2.389233in}{1.312387in}}%
\pgfpathlineto{\pgfqpoint{2.404128in}{1.301750in}}%
\pgfpathlineto{\pgfqpoint{2.416541in}{1.295245in}}%
\pgfpathlineto{\pgfqpoint{2.428954in}{1.290952in}}%
\pgfpathlineto{\pgfqpoint{2.441367in}{1.288888in}}%
\pgfpathlineto{\pgfqpoint{2.453780in}{1.289031in}}%
\pgfpathlineto{\pgfqpoint{2.466193in}{1.291317in}}%
\pgfpathlineto{\pgfqpoint{2.478606in}{1.295642in}}%
\pgfpathlineto{\pgfqpoint{2.493502in}{1.303330in}}%
\pgfpathlineto{\pgfqpoint{2.508397in}{1.313461in}}%
\pgfpathlineto{\pgfqpoint{2.525776in}{1.327896in}}%
\pgfpathlineto{\pgfqpoint{2.548119in}{1.349611in}}%
\pgfpathlineto{\pgfqpoint{2.582876in}{1.387171in}}%
\pgfpathlineto{\pgfqpoint{2.622597in}{1.429283in}}%
\pgfpathlineto{\pgfqpoint{2.644941in}{1.449813in}}%
\pgfpathlineto{\pgfqpoint{2.664801in}{1.464985in}}%
\pgfpathlineto{\pgfqpoint{2.682180in}{1.475395in}}%
\pgfpathlineto{\pgfqpoint{2.697075in}{1.481968in}}%
\pgfpathlineto{\pgfqpoint{2.711971in}{1.486259in}}%
\pgfpathlineto{\pgfqpoint{2.726867in}{1.488230in}}%
\pgfpathlineto{\pgfqpoint{2.741762in}{1.487906in}}%
\pgfpathlineto{\pgfqpoint{2.756658in}{1.485371in}}%
\pgfpathlineto{\pgfqpoint{2.771553in}{1.480766in}}%
\pgfpathlineto{\pgfqpoint{2.788932in}{1.473037in}}%
\pgfpathlineto{\pgfqpoint{2.808792in}{1.461563in}}%
\pgfpathlineto{\pgfqpoint{2.833619in}{1.444247in}}%
\pgfpathlineto{\pgfqpoint{2.873340in}{1.413121in}}%
\pgfpathlineto{\pgfqpoint{2.910579in}{1.384850in}}%
\pgfpathlineto{\pgfqpoint{2.935405in}{1.368799in}}%
\pgfpathlineto{\pgfqpoint{2.955266in}{1.358425in}}%
\pgfpathlineto{\pgfqpoint{2.975127in}{1.350676in}}%
\pgfpathlineto{\pgfqpoint{2.992505in}{1.346246in}}%
\pgfpathlineto{\pgfqpoint{3.009883in}{1.344092in}}%
\pgfpathlineto{\pgfqpoint{3.027262in}{1.344204in}}%
\pgfpathlineto{\pgfqpoint{3.044640in}{1.346490in}}%
\pgfpathlineto{\pgfqpoint{3.064501in}{1.351546in}}%
\pgfpathlineto{\pgfqpoint{3.084361in}{1.358869in}}%
\pgfpathlineto{\pgfqpoint{3.109187in}{1.370524in}}%
\pgfpathlineto{\pgfqpoint{3.143944in}{1.389683in}}%
\pgfpathlineto{\pgfqpoint{3.198561in}{1.419939in}}%
\pgfpathlineto{\pgfqpoint{3.225870in}{1.432320in}}%
\pgfpathlineto{\pgfqpoint{3.225870in}{1.432320in}}%
\pgfusepath{stroke}%
\end{pgfscope}%
\begin{pgfscope}%
\pgfpathrectangle{\pgfqpoint{0.619136in}{0.571603in}}{\pgfqpoint{2.730864in}{1.657828in}}%
\pgfusepath{clip}%
\pgfsetrectcap%
\pgfsetroundjoin%
\pgfsetlinewidth{1.505625pt}%
\definecolor{currentstroke}{rgb}{0.737255,0.741176,0.133333}%
\pgfsetstrokecolor{currentstroke}%
\pgfsetdash{}{0pt}%
\pgfpathmoveto{\pgfqpoint{0.743267in}{0.646959in}}%
\pgfpathlineto{\pgfqpoint{0.748232in}{0.649223in}}%
\pgfpathlineto{\pgfqpoint{0.753197in}{0.655165in}}%
\pgfpathlineto{\pgfqpoint{0.758162in}{0.664360in}}%
\pgfpathlineto{\pgfqpoint{0.765610in}{0.683760in}}%
\pgfpathlineto{\pgfqpoint{0.773058in}{0.709371in}}%
\pgfpathlineto{\pgfqpoint{0.782988in}{0.752294in}}%
\pgfpathlineto{\pgfqpoint{0.795401in}{0.818418in}}%
\pgfpathlineto{\pgfqpoint{0.810297in}{0.913040in}}%
\pgfpathlineto{\pgfqpoint{0.827675in}{1.039356in}}%
\pgfpathlineto{\pgfqpoint{0.854984in}{1.257232in}}%
\pgfpathlineto{\pgfqpoint{0.892223in}{1.552024in}}%
\pgfpathlineto{\pgfqpoint{0.912084in}{1.690288in}}%
\pgfpathlineto{\pgfqpoint{0.926979in}{1.779543in}}%
\pgfpathlineto{\pgfqpoint{0.939392in}{1.842545in}}%
\pgfpathlineto{\pgfqpoint{0.951805in}{1.894082in}}%
\pgfpathlineto{\pgfqpoint{0.961736in}{1.926545in}}%
\pgfpathlineto{\pgfqpoint{0.971666in}{1.950938in}}%
\pgfpathlineto{\pgfqpoint{0.979114in}{1.963857in}}%
\pgfpathlineto{\pgfqpoint{0.986562in}{1.972153in}}%
\pgfpathlineto{\pgfqpoint{0.991527in}{1.975129in}}%
\pgfpathlineto{\pgfqpoint{0.996492in}{1.976079in}}%
\pgfpathlineto{\pgfqpoint{1.001457in}{1.975028in}}%
\pgfpathlineto{\pgfqpoint{1.006423in}{1.972006in}}%
\pgfpathlineto{\pgfqpoint{1.013870in}{1.963862in}}%
\pgfpathlineto{\pgfqpoint{1.021318in}{1.951523in}}%
\pgfpathlineto{\pgfqpoint{1.028766in}{1.935172in}}%
\pgfpathlineto{\pgfqpoint{1.038696in}{1.907497in}}%
\pgfpathlineto{\pgfqpoint{1.051109in}{1.864279in}}%
\pgfpathlineto{\pgfqpoint{1.063522in}{1.812669in}}%
\pgfpathlineto{\pgfqpoint{1.078418in}{1.741649in}}%
\pgfpathlineto{\pgfqpoint{1.098279in}{1.635871in}}%
\pgfpathlineto{\pgfqpoint{1.165309in}{1.267017in}}%
\pgfpathlineto{\pgfqpoint{1.182687in}{1.187886in}}%
\pgfpathlineto{\pgfqpoint{1.197583in}{1.130413in}}%
\pgfpathlineto{\pgfqpoint{1.209996in}{1.090906in}}%
\pgfpathlineto{\pgfqpoint{1.219926in}{1.065231in}}%
\pgfpathlineto{\pgfqpoint{1.229857in}{1.045061in}}%
\pgfpathlineto{\pgfqpoint{1.239787in}{1.030526in}}%
\pgfpathlineto{\pgfqpoint{1.247235in}{1.023352in}}%
\pgfpathlineto{\pgfqpoint{1.254683in}{1.019364in}}%
\pgfpathlineto{\pgfqpoint{1.262131in}{1.018526in}}%
\pgfpathlineto{\pgfqpoint{1.269578in}{1.020778in}}%
\pgfpathlineto{\pgfqpoint{1.277026in}{1.026038in}}%
\pgfpathlineto{\pgfqpoint{1.284474in}{1.034201in}}%
\pgfpathlineto{\pgfqpoint{1.294404in}{1.049379in}}%
\pgfpathlineto{\pgfqpoint{1.304335in}{1.069142in}}%
\pgfpathlineto{\pgfqpoint{1.316748in}{1.099669in}}%
\pgfpathlineto{\pgfqpoint{1.331644in}{1.143651in}}%
\pgfpathlineto{\pgfqpoint{1.349022in}{1.202918in}}%
\pgfpathlineto{\pgfqpoint{1.373848in}{1.296833in}}%
\pgfpathlineto{\pgfqpoint{1.418535in}{1.467566in}}%
\pgfpathlineto{\pgfqpoint{1.438395in}{1.534013in}}%
\pgfpathlineto{\pgfqpoint{1.453291in}{1.576985in}}%
\pgfpathlineto{\pgfqpoint{1.465704in}{1.607392in}}%
\pgfpathlineto{\pgfqpoint{1.478117in}{1.632356in}}%
\pgfpathlineto{\pgfqpoint{1.488048in}{1.648162in}}%
\pgfpathlineto{\pgfqpoint{1.497978in}{1.660136in}}%
\pgfpathlineto{\pgfqpoint{1.507908in}{1.668213in}}%
\pgfpathlineto{\pgfqpoint{1.515356in}{1.671710in}}%
\pgfpathlineto{\pgfqpoint{1.522804in}{1.673029in}}%
\pgfpathlineto{\pgfqpoint{1.530252in}{1.672206in}}%
\pgfpathlineto{\pgfqpoint{1.537700in}{1.669292in}}%
\pgfpathlineto{\pgfqpoint{1.545147in}{1.664353in}}%
\pgfpathlineto{\pgfqpoint{1.555078in}{1.654760in}}%
\pgfpathlineto{\pgfqpoint{1.565008in}{1.641944in}}%
\pgfpathlineto{\pgfqpoint{1.577421in}{1.621803in}}%
\pgfpathlineto{\pgfqpoint{1.592317in}{1.592390in}}%
\pgfpathlineto{\pgfqpoint{1.609695in}{1.552317in}}%
\pgfpathlineto{\pgfqpoint{1.634521in}{1.488140in}}%
\pgfpathlineto{\pgfqpoint{1.684173in}{1.357598in}}%
\pgfpathlineto{\pgfqpoint{1.704034in}{1.312481in}}%
\pgfpathlineto{\pgfqpoint{1.718930in}{1.283515in}}%
\pgfpathlineto{\pgfqpoint{1.733825in}{1.259550in}}%
\pgfpathlineto{\pgfqpoint{1.746238in}{1.243796in}}%
\pgfpathlineto{\pgfqpoint{1.756169in}{1.234110in}}%
\pgfpathlineto{\pgfqpoint{1.766099in}{1.227088in}}%
\pgfpathlineto{\pgfqpoint{1.776030in}{1.222755in}}%
\pgfpathlineto{\pgfqpoint{1.785960in}{1.221097in}}%
\pgfpathlineto{\pgfqpoint{1.795890in}{1.222062in}}%
\pgfpathlineto{\pgfqpoint{1.805821in}{1.225563in}}%
\pgfpathlineto{\pgfqpoint{1.815751in}{1.231482in}}%
\pgfpathlineto{\pgfqpoint{1.828164in}{1.242047in}}%
\pgfpathlineto{\pgfqpoint{1.840577in}{1.255799in}}%
\pgfpathlineto{\pgfqpoint{1.855473in}{1.275928in}}%
\pgfpathlineto{\pgfqpoint{1.872851in}{1.303397in}}%
\pgfpathlineto{\pgfqpoint{1.897677in}{1.347454in}}%
\pgfpathlineto{\pgfqpoint{1.949812in}{1.441357in}}%
\pgfpathlineto{\pgfqpoint{1.969673in}{1.471862in}}%
\pgfpathlineto{\pgfqpoint{1.987051in}{1.494201in}}%
\pgfpathlineto{\pgfqpoint{2.001946in}{1.509506in}}%
\pgfpathlineto{\pgfqpoint{2.014359in}{1.519289in}}%
\pgfpathlineto{\pgfqpoint{2.026772in}{1.526240in}}%
\pgfpathlineto{\pgfqpoint{2.039185in}{1.530306in}}%
\pgfpathlineto{\pgfqpoint{2.049116in}{1.531486in}}%
\pgfpathlineto{\pgfqpoint{2.059046in}{1.530859in}}%
\pgfpathlineto{\pgfqpoint{2.071459in}{1.527628in}}%
\pgfpathlineto{\pgfqpoint{2.083872in}{1.521832in}}%
\pgfpathlineto{\pgfqpoint{2.096285in}{1.513686in}}%
\pgfpathlineto{\pgfqpoint{2.111181in}{1.501176in}}%
\pgfpathlineto{\pgfqpoint{2.128559in}{1.483469in}}%
\pgfpathlineto{\pgfqpoint{2.150903in}{1.457194in}}%
\pgfpathlineto{\pgfqpoint{2.225381in}{1.366458in}}%
\pgfpathlineto{\pgfqpoint{2.242759in}{1.349729in}}%
\pgfpathlineto{\pgfqpoint{2.260137in}{1.336133in}}%
\pgfpathlineto{\pgfqpoint{2.275033in}{1.327299in}}%
\pgfpathlineto{\pgfqpoint{2.289928in}{1.321249in}}%
\pgfpathlineto{\pgfqpoint{2.302341in}{1.318389in}}%
\pgfpathlineto{\pgfqpoint{2.314754in}{1.317509in}}%
\pgfpathlineto{\pgfqpoint{2.327167in}{1.318560in}}%
\pgfpathlineto{\pgfqpoint{2.342063in}{1.322239in}}%
\pgfpathlineto{\pgfqpoint{2.356959in}{1.328341in}}%
\pgfpathlineto{\pgfqpoint{2.374337in}{1.338132in}}%
\pgfpathlineto{\pgfqpoint{2.394198in}{1.352158in}}%
\pgfpathlineto{\pgfqpoint{2.421506in}{1.374703in}}%
\pgfpathlineto{\pgfqpoint{2.481089in}{1.425233in}}%
\pgfpathlineto{\pgfqpoint{2.503432in}{1.440750in}}%
\pgfpathlineto{\pgfqpoint{2.523293in}{1.451661in}}%
\pgfpathlineto{\pgfqpoint{2.540671in}{1.458603in}}%
\pgfpathlineto{\pgfqpoint{2.558050in}{1.462925in}}%
\pgfpathlineto{\pgfqpoint{2.572945in}{1.464498in}}%
\pgfpathlineto{\pgfqpoint{2.587841in}{1.464133in}}%
\pgfpathlineto{\pgfqpoint{2.605219in}{1.461384in}}%
\pgfpathlineto{\pgfqpoint{2.622597in}{1.456362in}}%
\pgfpathlineto{\pgfqpoint{2.642458in}{1.448256in}}%
\pgfpathlineto{\pgfqpoint{2.667284in}{1.435441in}}%
\pgfpathlineto{\pgfqpoint{2.707006in}{1.411846in}}%
\pgfpathlineto{\pgfqpoint{2.744245in}{1.390513in}}%
\pgfpathlineto{\pgfqpoint{2.769071in}{1.378741in}}%
\pgfpathlineto{\pgfqpoint{2.791414in}{1.370691in}}%
\pgfpathlineto{\pgfqpoint{2.811275in}{1.365912in}}%
\pgfpathlineto{\pgfqpoint{2.831136in}{1.363520in}}%
\pgfpathlineto{\pgfqpoint{2.850997in}{1.363517in}}%
\pgfpathlineto{\pgfqpoint{2.870858in}{1.365771in}}%
\pgfpathlineto{\pgfqpoint{2.893201in}{1.370691in}}%
\pgfpathlineto{\pgfqpoint{2.918027in}{1.378529in}}%
\pgfpathlineto{\pgfqpoint{2.952783in}{1.392118in}}%
\pgfpathlineto{\pgfqpoint{3.014849in}{1.416657in}}%
\pgfpathlineto{\pgfqpoint{3.042157in}{1.424851in}}%
\pgfpathlineto{\pgfqpoint{3.066983in}{1.429969in}}%
\pgfpathlineto{\pgfqpoint{3.089327in}{1.432425in}}%
\pgfpathlineto{\pgfqpoint{3.111670in}{1.432792in}}%
\pgfpathlineto{\pgfqpoint{3.136496in}{1.430879in}}%
\pgfpathlineto{\pgfqpoint{3.163805in}{1.426345in}}%
\pgfpathlineto{\pgfqpoint{3.196079in}{1.418590in}}%
\pgfpathlineto{\pgfqpoint{3.225870in}{1.410224in}}%
\pgfpathlineto{\pgfqpoint{3.225870in}{1.410224in}}%
\pgfusepath{stroke}%
\end{pgfscope}%
\begin{pgfscope}%
\pgfpathrectangle{\pgfqpoint{0.619136in}{0.571603in}}{\pgfqpoint{2.730864in}{1.657828in}}%
\pgfusepath{clip}%
\pgfsetrectcap%
\pgfsetroundjoin%
\pgfsetlinewidth{1.505625pt}%
\definecolor{currentstroke}{rgb}{0.090196,0.745098,0.811765}%
\pgfsetstrokecolor{currentstroke}%
\pgfsetdash{}{0pt}%
\pgfpathmoveto{\pgfqpoint{0.743267in}{0.646959in}}%
\pgfpathlineto{\pgfqpoint{0.745749in}{0.651210in}}%
\pgfpathlineto{\pgfqpoint{0.750714in}{0.669402in}}%
\pgfpathlineto{\pgfqpoint{0.758162in}{0.710337in}}%
\pgfpathlineto{\pgfqpoint{0.768093in}{0.780852in}}%
\pgfpathlineto{\pgfqpoint{0.782988in}{0.905726in}}%
\pgfpathlineto{\pgfqpoint{0.825192in}{1.272307in}}%
\pgfpathlineto{\pgfqpoint{0.840088in}{1.380884in}}%
\pgfpathlineto{\pgfqpoint{0.852501in}{1.457772in}}%
\pgfpathlineto{\pgfqpoint{0.864914in}{1.521233in}}%
\pgfpathlineto{\pgfqpoint{0.874845in}{1.562155in}}%
\pgfpathlineto{\pgfqpoint{0.884775in}{1.594479in}}%
\pgfpathlineto{\pgfqpoint{0.894705in}{1.618561in}}%
\pgfpathlineto{\pgfqpoint{0.902153in}{1.631520in}}%
\pgfpathlineto{\pgfqpoint{0.909601in}{1.640395in}}%
\pgfpathlineto{\pgfqpoint{0.917049in}{1.645483in}}%
\pgfpathlineto{\pgfqpoint{0.924497in}{1.647100in}}%
\pgfpathlineto{\pgfqpoint{0.931944in}{1.645581in}}%
\pgfpathlineto{\pgfqpoint{0.939392in}{1.641267in}}%
\pgfpathlineto{\pgfqpoint{0.946840in}{1.634504in}}%
\pgfpathlineto{\pgfqpoint{0.956770in}{1.622269in}}%
\pgfpathlineto{\pgfqpoint{0.969183in}{1.602912in}}%
\pgfpathlineto{\pgfqpoint{0.986562in}{1.570813in}}%
\pgfpathlineto{\pgfqpoint{1.041179in}{1.465351in}}%
\pgfpathlineto{\pgfqpoint{1.058557in}{1.438096in}}%
\pgfpathlineto{\pgfqpoint{1.073453in}{1.418734in}}%
\pgfpathlineto{\pgfqpoint{1.088348in}{1.403288in}}%
\pgfpathlineto{\pgfqpoint{1.100761in}{1.393387in}}%
\pgfpathlineto{\pgfqpoint{1.113174in}{1.386056in}}%
\pgfpathlineto{\pgfqpoint{1.125587in}{1.381097in}}%
\pgfpathlineto{\pgfqpoint{1.138000in}{1.378262in}}%
\pgfpathlineto{\pgfqpoint{1.152896in}{1.377264in}}%
\pgfpathlineto{\pgfqpoint{1.170274in}{1.378756in}}%
\pgfpathlineto{\pgfqpoint{1.190135in}{1.382956in}}%
\pgfpathlineto{\pgfqpoint{1.219926in}{1.391890in}}%
\pgfpathlineto{\pgfqpoint{1.269578in}{1.406783in}}%
\pgfpathlineto{\pgfqpoint{1.296887in}{1.412580in}}%
\pgfpathlineto{\pgfqpoint{1.324196in}{1.416103in}}%
\pgfpathlineto{\pgfqpoint{1.353987in}{1.417546in}}%
\pgfpathlineto{\pgfqpoint{1.388743in}{1.416843in}}%
\pgfpathlineto{\pgfqpoint{1.445843in}{1.413005in}}%
\pgfpathlineto{\pgfqpoint{1.522804in}{1.408335in}}%
\pgfpathlineto{\pgfqpoint{1.587352in}{1.406923in}}%
\pgfpathlineto{\pgfqpoint{1.701551in}{1.407286in}}%
\pgfpathlineto{\pgfqpoint{1.870368in}{1.406982in}}%
\pgfpathlineto{\pgfqpoint{2.262620in}{1.405913in}}%
\pgfpathlineto{\pgfqpoint{3.225870in}{1.405055in}}%
\pgfpathlineto{\pgfqpoint{3.225870in}{1.405055in}}%
\pgfusepath{stroke}%
\end{pgfscope}%
\begin{pgfscope}%
\pgfpathrectangle{\pgfqpoint{0.619136in}{0.571603in}}{\pgfqpoint{2.730864in}{1.657828in}}%
\pgfusepath{clip}%
\pgfsetrectcap%
\pgfsetroundjoin%
\pgfsetlinewidth{1.505625pt}%
\definecolor{currentstroke}{rgb}{0.121569,0.466667,0.705882}%
\pgfsetstrokecolor{currentstroke}%
\pgfsetdash{}{0pt}%
\pgfpathmoveto{\pgfqpoint{0.743267in}{0.646959in}}%
\pgfpathlineto{\pgfqpoint{0.748232in}{0.648471in}}%
\pgfpathlineto{\pgfqpoint{0.753197in}{0.652468in}}%
\pgfpathlineto{\pgfqpoint{0.760645in}{0.662581in}}%
\pgfpathlineto{\pgfqpoint{0.768093in}{0.677250in}}%
\pgfpathlineto{\pgfqpoint{0.778023in}{0.703347in}}%
\pgfpathlineto{\pgfqpoint{0.787953in}{0.736322in}}%
\pgfpathlineto{\pgfqpoint{0.800366in}{0.786271in}}%
\pgfpathlineto{\pgfqpoint{0.815262in}{0.857421in}}%
\pgfpathlineto{\pgfqpoint{0.832640in}{0.953174in}}%
\pgfpathlineto{\pgfqpoint{0.854984in}{1.091020in}}%
\pgfpathlineto{\pgfqpoint{0.894705in}{1.354612in}}%
\pgfpathlineto{\pgfqpoint{0.926979in}{1.561916in}}%
\pgfpathlineto{\pgfqpoint{0.946840in}{1.676839in}}%
\pgfpathlineto{\pgfqpoint{0.964218in}{1.765662in}}%
\pgfpathlineto{\pgfqpoint{0.979114in}{1.831334in}}%
\pgfpathlineto{\pgfqpoint{0.991527in}{1.877836in}}%
\pgfpathlineto{\pgfqpoint{1.003940in}{1.916330in}}%
\pgfpathlineto{\pgfqpoint{1.013870in}{1.941115in}}%
\pgfpathlineto{\pgfqpoint{1.023801in}{1.960420in}}%
\pgfpathlineto{\pgfqpoint{1.033731in}{1.974178in}}%
\pgfpathlineto{\pgfqpoint{1.041179in}{1.980845in}}%
\pgfpathlineto{\pgfqpoint{1.048627in}{1.984397in}}%
\pgfpathlineto{\pgfqpoint{1.056075in}{1.984862in}}%
\pgfpathlineto{\pgfqpoint{1.063522in}{1.982287in}}%
\pgfpathlineto{\pgfqpoint{1.070970in}{1.976733in}}%
\pgfpathlineto{\pgfqpoint{1.078418in}{1.968277in}}%
\pgfpathlineto{\pgfqpoint{1.088348in}{1.952645in}}%
\pgfpathlineto{\pgfqpoint{1.098279in}{1.932266in}}%
\pgfpathlineto{\pgfqpoint{1.110692in}{1.900564in}}%
\pgfpathlineto{\pgfqpoint{1.123105in}{1.862547in}}%
\pgfpathlineto{\pgfqpoint{1.138000in}{1.809609in}}%
\pgfpathlineto{\pgfqpoint{1.155379in}{1.739492in}}%
\pgfpathlineto{\pgfqpoint{1.177722in}{1.639720in}}%
\pgfpathlineto{\pgfqpoint{1.262131in}{1.251143in}}%
\pgfpathlineto{\pgfqpoint{1.281991in}{1.175739in}}%
\pgfpathlineto{\pgfqpoint{1.296887in}{1.126818in}}%
\pgfpathlineto{\pgfqpoint{1.311783in}{1.085355in}}%
\pgfpathlineto{\pgfqpoint{1.324196in}{1.056958in}}%
\pgfpathlineto{\pgfqpoint{1.336609in}{1.034443in}}%
\pgfpathlineto{\pgfqpoint{1.346539in}{1.020784in}}%
\pgfpathlineto{\pgfqpoint{1.356470in}{1.011043in}}%
\pgfpathlineto{\pgfqpoint{1.363917in}{1.006315in}}%
\pgfpathlineto{\pgfqpoint{1.371365in}{1.003785in}}%
\pgfpathlineto{\pgfqpoint{1.378813in}{1.003431in}}%
\pgfpathlineto{\pgfqpoint{1.386261in}{1.005218in}}%
\pgfpathlineto{\pgfqpoint{1.393709in}{1.009104in}}%
\pgfpathlineto{\pgfqpoint{1.403639in}{1.017451in}}%
\pgfpathlineto{\pgfqpoint{1.413569in}{1.029273in}}%
\pgfpathlineto{\pgfqpoint{1.423500in}{1.044382in}}%
\pgfpathlineto{\pgfqpoint{1.435913in}{1.067565in}}%
\pgfpathlineto{\pgfqpoint{1.450808in}{1.101061in}}%
\pgfpathlineto{\pgfqpoint{1.468187in}{1.146813in}}%
\pgfpathlineto{\pgfqpoint{1.488048in}{1.205960in}}%
\pgfpathlineto{\pgfqpoint{1.517839in}{1.303073in}}%
\pgfpathlineto{\pgfqpoint{1.567491in}{1.465982in}}%
\pgfpathlineto{\pgfqpoint{1.589834in}{1.531384in}}%
\pgfpathlineto{\pgfqpoint{1.607212in}{1.576164in}}%
\pgfpathlineto{\pgfqpoint{1.622108in}{1.609394in}}%
\pgfpathlineto{\pgfqpoint{1.637004in}{1.637285in}}%
\pgfpathlineto{\pgfqpoint{1.649417in}{1.656147in}}%
\pgfpathlineto{\pgfqpoint{1.661830in}{1.670843in}}%
\pgfpathlineto{\pgfqpoint{1.671760in}{1.679528in}}%
\pgfpathlineto{\pgfqpoint{1.681691in}{1.685456in}}%
\pgfpathlineto{\pgfqpoint{1.691621in}{1.688630in}}%
\pgfpathlineto{\pgfqpoint{1.701551in}{1.689080in}}%
\pgfpathlineto{\pgfqpoint{1.711482in}{1.686858in}}%
\pgfpathlineto{\pgfqpoint{1.721412in}{1.682045in}}%
\pgfpathlineto{\pgfqpoint{1.731343in}{1.674739in}}%
\pgfpathlineto{\pgfqpoint{1.743756in}{1.662291in}}%
\pgfpathlineto{\pgfqpoint{1.756169in}{1.646422in}}%
\pgfpathlineto{\pgfqpoint{1.771064in}{1.623321in}}%
\pgfpathlineto{\pgfqpoint{1.788443in}{1.591575in}}%
\pgfpathlineto{\pgfqpoint{1.808303in}{1.550321in}}%
\pgfpathlineto{\pgfqpoint{1.835612in}{1.488061in}}%
\pgfpathlineto{\pgfqpoint{1.892712in}{1.356390in}}%
\pgfpathlineto{\pgfqpoint{1.915055in}{1.311085in}}%
\pgfpathlineto{\pgfqpoint{1.932434in}{1.280298in}}%
\pgfpathlineto{\pgfqpoint{1.947329in}{1.257629in}}%
\pgfpathlineto{\pgfqpoint{1.962225in}{1.238781in}}%
\pgfpathlineto{\pgfqpoint{1.974638in}{1.226194in}}%
\pgfpathlineto{\pgfqpoint{1.987051in}{1.216559in}}%
\pgfpathlineto{\pgfqpoint{1.999464in}{1.209939in}}%
\pgfpathlineto{\pgfqpoint{2.009394in}{1.206825in}}%
\pgfpathlineto{\pgfqpoint{2.019325in}{1.205638in}}%
\pgfpathlineto{\pgfqpoint{2.029255in}{1.206349in}}%
\pgfpathlineto{\pgfqpoint{2.039185in}{1.208909in}}%
\pgfpathlineto{\pgfqpoint{2.051598in}{1.214611in}}%
\pgfpathlineto{\pgfqpoint{2.064011in}{1.222944in}}%
\pgfpathlineto{\pgfqpoint{2.078907in}{1.236144in}}%
\pgfpathlineto{\pgfqpoint{2.093803in}{1.252463in}}%
\pgfpathlineto{\pgfqpoint{2.111181in}{1.274852in}}%
\pgfpathlineto{\pgfqpoint{2.133524in}{1.307739in}}%
\pgfpathlineto{\pgfqpoint{2.165798in}{1.360015in}}%
\pgfpathlineto{\pgfqpoint{2.212968in}{1.436320in}}%
\pgfpathlineto{\pgfqpoint{2.235311in}{1.468471in}}%
\pgfpathlineto{\pgfqpoint{2.255172in}{1.493303in}}%
\pgfpathlineto{\pgfqpoint{2.272550in}{1.511499in}}%
\pgfpathlineto{\pgfqpoint{2.287446in}{1.524150in}}%
\pgfpathlineto{\pgfqpoint{2.302341in}{1.533893in}}%
\pgfpathlineto{\pgfqpoint{2.314754in}{1.539708in}}%
\pgfpathlineto{\pgfqpoint{2.327167in}{1.543394in}}%
\pgfpathlineto{\pgfqpoint{2.339580in}{1.544956in}}%
\pgfpathlineto{\pgfqpoint{2.351993in}{1.544428in}}%
\pgfpathlineto{\pgfqpoint{2.364406in}{1.541872in}}%
\pgfpathlineto{\pgfqpoint{2.379302in}{1.536261in}}%
\pgfpathlineto{\pgfqpoint{2.394198in}{1.528070in}}%
\pgfpathlineto{\pgfqpoint{2.411576in}{1.515597in}}%
\pgfpathlineto{\pgfqpoint{2.431437in}{1.498078in}}%
\pgfpathlineto{\pgfqpoint{2.453780in}{1.475170in}}%
\pgfpathlineto{\pgfqpoint{2.486054in}{1.438534in}}%
\pgfpathlineto{\pgfqpoint{2.538189in}{1.379364in}}%
\pgfpathlineto{\pgfqpoint{2.563015in}{1.354837in}}%
\pgfpathlineto{\pgfqpoint{2.582876in}{1.338161in}}%
\pgfpathlineto{\pgfqpoint{2.600254in}{1.326160in}}%
\pgfpathlineto{\pgfqpoint{2.617632in}{1.316840in}}%
\pgfpathlineto{\pgfqpoint{2.632528in}{1.311115in}}%
\pgfpathlineto{\pgfqpoint{2.647423in}{1.307536in}}%
\pgfpathlineto{\pgfqpoint{2.662319in}{1.306110in}}%
\pgfpathlineto{\pgfqpoint{2.677215in}{1.306798in}}%
\pgfpathlineto{\pgfqpoint{2.692110in}{1.309516in}}%
\pgfpathlineto{\pgfqpoint{2.709488in}{1.315090in}}%
\pgfpathlineto{\pgfqpoint{2.726867in}{1.323004in}}%
\pgfpathlineto{\pgfqpoint{2.746727in}{1.334513in}}%
\pgfpathlineto{\pgfqpoint{2.771553in}{1.351795in}}%
\pgfpathlineto{\pgfqpoint{2.806310in}{1.379282in}}%
\pgfpathlineto{\pgfqpoint{2.863410in}{1.424665in}}%
\pgfpathlineto{\pgfqpoint{2.888236in}{1.441579in}}%
\pgfpathlineto{\pgfqpoint{2.910579in}{1.454240in}}%
\pgfpathlineto{\pgfqpoint{2.930440in}{1.463050in}}%
\pgfpathlineto{\pgfqpoint{2.950301in}{1.469319in}}%
\pgfpathlineto{\pgfqpoint{2.967679in}{1.472620in}}%
\pgfpathlineto{\pgfqpoint{2.985057in}{1.473861in}}%
\pgfpathlineto{\pgfqpoint{3.002436in}{1.473083in}}%
\pgfpathlineto{\pgfqpoint{3.022296in}{1.469850in}}%
\pgfpathlineto{\pgfqpoint{3.042157in}{1.464334in}}%
\pgfpathlineto{\pgfqpoint{3.064501in}{1.455780in}}%
\pgfpathlineto{\pgfqpoint{3.091809in}{1.442685in}}%
\pgfpathlineto{\pgfqpoint{3.129048in}{1.422017in}}%
\pgfpathlineto{\pgfqpoint{3.188631in}{1.388858in}}%
\pgfpathlineto{\pgfqpoint{3.218422in}{1.375102in}}%
\pgfpathlineto{\pgfqpoint{3.225870in}{1.372144in}}%
\pgfpathlineto{\pgfqpoint{3.225870in}{1.372144in}}%
\pgfusepath{stroke}%
\end{pgfscope}%
\begin{pgfscope}%
\pgfpathrectangle{\pgfqpoint{0.619136in}{0.571603in}}{\pgfqpoint{2.730864in}{1.657828in}}%
\pgfusepath{clip}%
\pgfsetrectcap%
\pgfsetroundjoin%
\pgfsetlinewidth{1.505625pt}%
\definecolor{currentstroke}{rgb}{1.000000,0.498039,0.054902}%
\pgfsetstrokecolor{currentstroke}%
\pgfsetdash{}{0pt}%
\pgfpathmoveto{\pgfqpoint{0.743267in}{0.646959in}}%
\pgfpathlineto{\pgfqpoint{0.745749in}{0.651947in}}%
\pgfpathlineto{\pgfqpoint{0.750714in}{0.671533in}}%
\pgfpathlineto{\pgfqpoint{0.758162in}{0.713332in}}%
\pgfpathlineto{\pgfqpoint{0.768093in}{0.782581in}}%
\pgfpathlineto{\pgfqpoint{0.785471in}{0.921744in}}%
\pgfpathlineto{\pgfqpoint{0.817745in}{1.182450in}}%
\pgfpathlineto{\pgfqpoint{0.835123in}{1.304010in}}%
\pgfpathlineto{\pgfqpoint{0.850018in}{1.392177in}}%
\pgfpathlineto{\pgfqpoint{0.862432in}{1.453289in}}%
\pgfpathlineto{\pgfqpoint{0.874845in}{1.503021in}}%
\pgfpathlineto{\pgfqpoint{0.884775in}{1.534825in}}%
\pgfpathlineto{\pgfqpoint{0.894705in}{1.559879in}}%
\pgfpathlineto{\pgfqpoint{0.904636in}{1.578617in}}%
\pgfpathlineto{\pgfqpoint{0.912084in}{1.588831in}}%
\pgfpathlineto{\pgfqpoint{0.919531in}{1.596014in}}%
\pgfpathlineto{\pgfqpoint{0.926979in}{1.600415in}}%
\pgfpathlineto{\pgfqpoint{0.934427in}{1.602288in}}%
\pgfpathlineto{\pgfqpoint{0.941875in}{1.601889in}}%
\pgfpathlineto{\pgfqpoint{0.949323in}{1.599476in}}%
\pgfpathlineto{\pgfqpoint{0.959253in}{1.593559in}}%
\pgfpathlineto{\pgfqpoint{0.969183in}{1.585084in}}%
\pgfpathlineto{\pgfqpoint{0.981596in}{1.571722in}}%
\pgfpathlineto{\pgfqpoint{0.998975in}{1.549572in}}%
\pgfpathlineto{\pgfqpoint{1.063522in}{1.463570in}}%
\pgfpathlineto{\pgfqpoint{1.080901in}{1.445106in}}%
\pgfpathlineto{\pgfqpoint{1.098279in}{1.429782in}}%
\pgfpathlineto{\pgfqpoint{1.115657in}{1.417682in}}%
\pgfpathlineto{\pgfqpoint{1.133035in}{1.408674in}}%
\pgfpathlineto{\pgfqpoint{1.150413in}{1.402477in}}%
\pgfpathlineto{\pgfqpoint{1.167792in}{1.398712in}}%
\pgfpathlineto{\pgfqpoint{1.187653in}{1.396838in}}%
\pgfpathlineto{\pgfqpoint{1.209996in}{1.397047in}}%
\pgfpathlineto{\pgfqpoint{1.239787in}{1.399712in}}%
\pgfpathlineto{\pgfqpoint{1.353987in}{1.412308in}}%
\pgfpathlineto{\pgfqpoint{1.398674in}{1.413759in}}%
\pgfpathlineto{\pgfqpoint{1.455774in}{1.413179in}}%
\pgfpathlineto{\pgfqpoint{1.696586in}{1.408219in}}%
\pgfpathlineto{\pgfqpoint{2.252689in}{1.406448in}}%
\pgfpathlineto{\pgfqpoint{3.225870in}{1.405331in}}%
\pgfpathlineto{\pgfqpoint{3.225870in}{1.405331in}}%
\pgfusepath{stroke}%
\end{pgfscope}%
\begin{pgfscope}%
\pgfpathrectangle{\pgfqpoint{0.619136in}{0.571603in}}{\pgfqpoint{2.730864in}{1.657828in}}%
\pgfusepath{clip}%
\pgfsetrectcap%
\pgfsetroundjoin%
\pgfsetlinewidth{1.505625pt}%
\definecolor{currentstroke}{rgb}{0.172549,0.627451,0.172549}%
\pgfsetstrokecolor{currentstroke}%
\pgfsetdash{}{0pt}%
\pgfpathmoveto{\pgfqpoint{0.743267in}{0.646959in}}%
\pgfpathlineto{\pgfqpoint{0.745749in}{0.649682in}}%
\pgfpathlineto{\pgfqpoint{0.750714in}{0.662587in}}%
\pgfpathlineto{\pgfqpoint{0.758162in}{0.693639in}}%
\pgfpathlineto{\pgfqpoint{0.768093in}{0.750194in}}%
\pgfpathlineto{\pgfqpoint{0.780506in}{0.837423in}}%
\pgfpathlineto{\pgfqpoint{0.800366in}{0.998044in}}%
\pgfpathlineto{\pgfqpoint{0.837605in}{1.303974in}}%
\pgfpathlineto{\pgfqpoint{0.854984in}{1.427279in}}%
\pgfpathlineto{\pgfqpoint{0.869879in}{1.516711in}}%
\pgfpathlineto{\pgfqpoint{0.882292in}{1.578255in}}%
\pgfpathlineto{\pgfqpoint{0.892223in}{1.618603in}}%
\pgfpathlineto{\pgfqpoint{0.902153in}{1.651004in}}%
\pgfpathlineto{\pgfqpoint{0.912084in}{1.675602in}}%
\pgfpathlineto{\pgfqpoint{0.919531in}{1.689100in}}%
\pgfpathlineto{\pgfqpoint{0.926979in}{1.698541in}}%
\pgfpathlineto{\pgfqpoint{0.934427in}{1.704132in}}%
\pgfpathlineto{\pgfqpoint{0.941875in}{1.706106in}}%
\pgfpathlineto{\pgfqpoint{0.949323in}{1.704723in}}%
\pgfpathlineto{\pgfqpoint{0.956770in}{1.700259in}}%
\pgfpathlineto{\pgfqpoint{0.964218in}{1.693006in}}%
\pgfpathlineto{\pgfqpoint{0.974149in}{1.679517in}}%
\pgfpathlineto{\pgfqpoint{0.986562in}{1.657538in}}%
\pgfpathlineto{\pgfqpoint{1.001457in}{1.625515in}}%
\pgfpathlineto{\pgfqpoint{1.023801in}{1.570741in}}%
\pgfpathlineto{\pgfqpoint{1.063522in}{1.472346in}}%
\pgfpathlineto{\pgfqpoint{1.080901in}{1.434928in}}%
\pgfpathlineto{\pgfqpoint{1.095796in}{1.407342in}}%
\pgfpathlineto{\pgfqpoint{1.110692in}{1.384472in}}%
\pgfpathlineto{\pgfqpoint{1.123105in}{1.369198in}}%
\pgfpathlineto{\pgfqpoint{1.135518in}{1.357373in}}%
\pgfpathlineto{\pgfqpoint{1.147931in}{1.348895in}}%
\pgfpathlineto{\pgfqpoint{1.160344in}{1.343569in}}%
\pgfpathlineto{\pgfqpoint{1.172757in}{1.341130in}}%
\pgfpathlineto{\pgfqpoint{1.185170in}{1.341261in}}%
\pgfpathlineto{\pgfqpoint{1.197583in}{1.343604in}}%
\pgfpathlineto{\pgfqpoint{1.212479in}{1.348805in}}%
\pgfpathlineto{\pgfqpoint{1.229857in}{1.357332in}}%
\pgfpathlineto{\pgfqpoint{1.254683in}{1.372196in}}%
\pgfpathlineto{\pgfqpoint{1.304335in}{1.402315in}}%
\pgfpathlineto{\pgfqpoint{1.326678in}{1.413066in}}%
\pgfpathlineto{\pgfqpoint{1.346539in}{1.420346in}}%
\pgfpathlineto{\pgfqpoint{1.366400in}{1.425334in}}%
\pgfpathlineto{\pgfqpoint{1.386261in}{1.428098in}}%
\pgfpathlineto{\pgfqpoint{1.408604in}{1.428833in}}%
\pgfpathlineto{\pgfqpoint{1.433430in}{1.427270in}}%
\pgfpathlineto{\pgfqpoint{1.465704in}{1.422742in}}%
\pgfpathlineto{\pgfqpoint{1.572456in}{1.405749in}}%
\pgfpathlineto{\pgfqpoint{1.609695in}{1.403077in}}%
\pgfpathlineto{\pgfqpoint{1.651899in}{1.402417in}}%
\pgfpathlineto{\pgfqpoint{1.708999in}{1.404023in}}%
\pgfpathlineto{\pgfqpoint{1.828164in}{1.407784in}}%
\pgfpathlineto{\pgfqpoint{1.910090in}{1.407612in}}%
\pgfpathlineto{\pgfqpoint{2.163316in}{1.405626in}}%
\pgfpathlineto{\pgfqpoint{2.707006in}{1.405238in}}%
\pgfpathlineto{\pgfqpoint{3.225870in}{1.404943in}}%
\pgfpathlineto{\pgfqpoint{3.225870in}{1.404943in}}%
\pgfusepath{stroke}%
\end{pgfscope}%
\begin{pgfscope}%
\pgfpathrectangle{\pgfqpoint{0.619136in}{0.571603in}}{\pgfqpoint{2.730864in}{1.657828in}}%
\pgfusepath{clip}%
\pgfsetrectcap%
\pgfsetroundjoin%
\pgfsetlinewidth{1.505625pt}%
\definecolor{currentstroke}{rgb}{0.839216,0.152941,0.156863}%
\pgfsetstrokecolor{currentstroke}%
\pgfsetdash{}{0pt}%
\pgfpathmoveto{\pgfqpoint{0.743267in}{0.646959in}}%
\pgfpathlineto{\pgfqpoint{0.753197in}{0.779556in}}%
\pgfpathlineto{\pgfqpoint{0.765610in}{0.937075in}}%
\pgfpathlineto{\pgfqpoint{0.778023in}{1.066362in}}%
\pgfpathlineto{\pgfqpoint{0.787953in}{1.149595in}}%
\pgfpathlineto{\pgfqpoint{0.797884in}{1.216893in}}%
\pgfpathlineto{\pgfqpoint{0.807814in}{1.270490in}}%
\pgfpathlineto{\pgfqpoint{0.817745in}{1.312588in}}%
\pgfpathlineto{\pgfqpoint{0.827675in}{1.345199in}}%
\pgfpathlineto{\pgfqpoint{0.837605in}{1.370096in}}%
\pgfpathlineto{\pgfqpoint{0.847536in}{1.388797in}}%
\pgfpathlineto{\pgfqpoint{0.857466in}{1.402578in}}%
\pgfpathlineto{\pgfqpoint{0.867397in}{1.412498in}}%
\pgfpathlineto{\pgfqpoint{0.877327in}{1.419421in}}%
\pgfpathlineto{\pgfqpoint{0.889740in}{1.424911in}}%
\pgfpathlineto{\pgfqpoint{0.902153in}{1.427867in}}%
\pgfpathlineto{\pgfqpoint{0.917049in}{1.429171in}}%
\pgfpathlineto{\pgfqpoint{0.936910in}{1.428642in}}%
\pgfpathlineto{\pgfqpoint{0.974149in}{1.424851in}}%
\pgfpathlineto{\pgfqpoint{1.043662in}{1.417770in}}%
\pgfpathlineto{\pgfqpoint{1.105727in}{1.414005in}}%
\pgfpathlineto{\pgfqpoint{1.192618in}{1.411229in}}%
\pgfpathlineto{\pgfqpoint{1.341574in}{1.408995in}}%
\pgfpathlineto{\pgfqpoint{1.632039in}{1.407163in}}%
\pgfpathlineto{\pgfqpoint{2.260137in}{1.405814in}}%
\pgfpathlineto{\pgfqpoint{3.225870in}{1.405152in}}%
\pgfpathlineto{\pgfqpoint{3.225870in}{1.405152in}}%
\pgfusepath{stroke}%
\end{pgfscope}%
\begin{pgfscope}%
\pgfpathrectangle{\pgfqpoint{0.619136in}{0.571603in}}{\pgfqpoint{2.730864in}{1.657828in}}%
\pgfusepath{clip}%
\pgfsetrectcap%
\pgfsetroundjoin%
\pgfsetlinewidth{1.505625pt}%
\definecolor{currentstroke}{rgb}{0.580392,0.403922,0.741176}%
\pgfsetstrokecolor{currentstroke}%
\pgfsetdash{}{0pt}%
\pgfpathmoveto{\pgfqpoint{0.743267in}{0.646959in}}%
\pgfpathlineto{\pgfqpoint{0.750714in}{0.725760in}}%
\pgfpathlineto{\pgfqpoint{0.770575in}{0.946109in}}%
\pgfpathlineto{\pgfqpoint{0.782988in}{1.061087in}}%
\pgfpathlineto{\pgfqpoint{0.795401in}{1.155731in}}%
\pgfpathlineto{\pgfqpoint{0.807814in}{1.231525in}}%
\pgfpathlineto{\pgfqpoint{0.817745in}{1.280183in}}%
\pgfpathlineto{\pgfqpoint{0.827675in}{1.319633in}}%
\pgfpathlineto{\pgfqpoint{0.837605in}{1.351178in}}%
\pgfpathlineto{\pgfqpoint{0.847536in}{1.376032in}}%
\pgfpathlineto{\pgfqpoint{0.857466in}{1.395294in}}%
\pgfpathlineto{\pgfqpoint{0.867397in}{1.409936in}}%
\pgfpathlineto{\pgfqpoint{0.877327in}{1.420805in}}%
\pgfpathlineto{\pgfqpoint{0.887258in}{1.428627in}}%
\pgfpathlineto{\pgfqpoint{0.897188in}{1.434017in}}%
\pgfpathlineto{\pgfqpoint{0.909601in}{1.438113in}}%
\pgfpathlineto{\pgfqpoint{0.924497in}{1.440206in}}%
\pgfpathlineto{\pgfqpoint{0.941875in}{1.440086in}}%
\pgfpathlineto{\pgfqpoint{0.966701in}{1.437306in}}%
\pgfpathlineto{\pgfqpoint{1.088348in}{1.420261in}}%
\pgfpathlineto{\pgfqpoint{1.145448in}{1.416202in}}%
\pgfpathlineto{\pgfqpoint{1.222409in}{1.413166in}}%
\pgfpathlineto{\pgfqpoint{1.351504in}{1.410613in}}%
\pgfpathlineto{\pgfqpoint{1.589834in}{1.408412in}}%
\pgfpathlineto{\pgfqpoint{2.049116in}{1.406702in}}%
\pgfpathlineto{\pgfqpoint{3.086844in}{1.405488in}}%
\pgfpathlineto{\pgfqpoint{3.225870in}{1.405409in}}%
\pgfpathlineto{\pgfqpoint{3.225870in}{1.405409in}}%
\pgfusepath{stroke}%
\end{pgfscope}%
\begin{pgfscope}%
\pgfpathrectangle{\pgfqpoint{0.619136in}{0.571603in}}{\pgfqpoint{2.730864in}{1.657828in}}%
\pgfusepath{clip}%
\pgfsetrectcap%
\pgfsetroundjoin%
\pgfsetlinewidth{1.505625pt}%
\definecolor{currentstroke}{rgb}{0.549020,0.337255,0.294118}%
\pgfsetstrokecolor{currentstroke}%
\pgfsetdash{}{0pt}%
\pgfpathmoveto{\pgfqpoint{0.743267in}{0.646959in}}%
\pgfpathlineto{\pgfqpoint{0.745749in}{0.659093in}}%
\pgfpathlineto{\pgfqpoint{0.753197in}{0.720161in}}%
\pgfpathlineto{\pgfqpoint{0.768093in}{0.872244in}}%
\pgfpathlineto{\pgfqpoint{0.790436in}{1.096302in}}%
\pgfpathlineto{\pgfqpoint{0.805332in}{1.221513in}}%
\pgfpathlineto{\pgfqpoint{0.817745in}{1.307079in}}%
\pgfpathlineto{\pgfqpoint{0.830158in}{1.375486in}}%
\pgfpathlineto{\pgfqpoint{0.840088in}{1.418579in}}%
\pgfpathlineto{\pgfqpoint{0.850018in}{1.452263in}}%
\pgfpathlineto{\pgfqpoint{0.859949in}{1.477556in}}%
\pgfpathlineto{\pgfqpoint{0.867397in}{1.491671in}}%
\pgfpathlineto{\pgfqpoint{0.874845in}{1.502133in}}%
\pgfpathlineto{\pgfqpoint{0.882292in}{1.509394in}}%
\pgfpathlineto{\pgfqpoint{0.889740in}{1.513884in}}%
\pgfpathlineto{\pgfqpoint{0.897188in}{1.516010in}}%
\pgfpathlineto{\pgfqpoint{0.904636in}{1.516151in}}%
\pgfpathlineto{\pgfqpoint{0.914566in}{1.513850in}}%
\pgfpathlineto{\pgfqpoint{0.926979in}{1.507990in}}%
\pgfpathlineto{\pgfqpoint{0.941875in}{1.498154in}}%
\pgfpathlineto{\pgfqpoint{0.979114in}{1.469424in}}%
\pgfpathlineto{\pgfqpoint{1.003940in}{1.452200in}}%
\pgfpathlineto{\pgfqpoint{1.023801in}{1.440985in}}%
\pgfpathlineto{\pgfqpoint{1.043662in}{1.432224in}}%
\pgfpathlineto{\pgfqpoint{1.063522in}{1.425728in}}%
\pgfpathlineto{\pgfqpoint{1.085866in}{1.420683in}}%
\pgfpathlineto{\pgfqpoint{1.113174in}{1.416957in}}%
\pgfpathlineto{\pgfqpoint{1.147931in}{1.414668in}}%
\pgfpathlineto{\pgfqpoint{1.207513in}{1.413442in}}%
\pgfpathlineto{\pgfqpoint{1.897677in}{1.407280in}}%
\pgfpathlineto{\pgfqpoint{2.625080in}{1.405837in}}%
\pgfpathlineto{\pgfqpoint{3.225870in}{1.405358in}}%
\pgfpathlineto{\pgfqpoint{3.225870in}{1.405358in}}%
\pgfusepath{stroke}%
\end{pgfscope}%
\begin{pgfscope}%
\pgfpathrectangle{\pgfqpoint{0.619136in}{0.571603in}}{\pgfqpoint{2.730864in}{1.657828in}}%
\pgfusepath{clip}%
\pgfsetrectcap%
\pgfsetroundjoin%
\pgfsetlinewidth{1.505625pt}%
\definecolor{currentstroke}{rgb}{0.890196,0.466667,0.760784}%
\pgfsetstrokecolor{currentstroke}%
\pgfsetdash{}{0pt}%
\pgfpathmoveto{\pgfqpoint{0.743267in}{0.646959in}}%
\pgfpathlineto{\pgfqpoint{0.748232in}{0.649098in}}%
\pgfpathlineto{\pgfqpoint{0.753197in}{0.654821in}}%
\pgfpathlineto{\pgfqpoint{0.758162in}{0.663771in}}%
\pgfpathlineto{\pgfqpoint{0.765610in}{0.682816in}}%
\pgfpathlineto{\pgfqpoint{0.773058in}{0.708142in}}%
\pgfpathlineto{\pgfqpoint{0.782988in}{0.750863in}}%
\pgfpathlineto{\pgfqpoint{0.795401in}{0.817103in}}%
\pgfpathlineto{\pgfqpoint{0.810297in}{0.912492in}}%
\pgfpathlineto{\pgfqpoint{0.827675in}{1.040605in}}%
\pgfpathlineto{\pgfqpoint{0.852501in}{1.242334in}}%
\pgfpathlineto{\pgfqpoint{0.892223in}{1.566172in}}%
\pgfpathlineto{\pgfqpoint{0.912084in}{1.708894in}}%
\pgfpathlineto{\pgfqpoint{0.926979in}{1.801088in}}%
\pgfpathlineto{\pgfqpoint{0.939392in}{1.866100in}}%
\pgfpathlineto{\pgfqpoint{0.951805in}{1.919131in}}%
\pgfpathlineto{\pgfqpoint{0.961736in}{1.952350in}}%
\pgfpathlineto{\pgfqpoint{0.971666in}{1.977065in}}%
\pgfpathlineto{\pgfqpoint{0.979114in}{1.989920in}}%
\pgfpathlineto{\pgfqpoint{0.986562in}{1.997880in}}%
\pgfpathlineto{\pgfqpoint{0.991527in}{2.000475in}}%
\pgfpathlineto{\pgfqpoint{0.996492in}{2.000919in}}%
\pgfpathlineto{\pgfqpoint{1.001457in}{1.999234in}}%
\pgfpathlineto{\pgfqpoint{1.006423in}{1.995451in}}%
\pgfpathlineto{\pgfqpoint{1.013870in}{1.985930in}}%
\pgfpathlineto{\pgfqpoint{1.021318in}{1.971933in}}%
\pgfpathlineto{\pgfqpoint{1.028766in}{1.953653in}}%
\pgfpathlineto{\pgfqpoint{1.038696in}{1.923004in}}%
\pgfpathlineto{\pgfqpoint{1.051109in}{1.875474in}}%
\pgfpathlineto{\pgfqpoint{1.063522in}{1.818975in}}%
\pgfpathlineto{\pgfqpoint{1.080901in}{1.727723in}}%
\pgfpathlineto{\pgfqpoint{1.103244in}{1.596301in}}%
\pgfpathlineto{\pgfqpoint{1.157861in}{1.268344in}}%
\pgfpathlineto{\pgfqpoint{1.175240in}{1.178809in}}%
\pgfpathlineto{\pgfqpoint{1.190135in}{1.112770in}}%
\pgfpathlineto{\pgfqpoint{1.202548in}{1.066649in}}%
\pgfpathlineto{\pgfqpoint{1.214961in}{1.029475in}}%
\pgfpathlineto{\pgfqpoint{1.224892in}{1.006563in}}%
\pgfpathlineto{\pgfqpoint{1.234822in}{0.989922in}}%
\pgfpathlineto{\pgfqpoint{1.242270in}{0.981612in}}%
\pgfpathlineto{\pgfqpoint{1.249718in}{0.976885in}}%
\pgfpathlineto{\pgfqpoint{1.254683in}{0.975711in}}%
\pgfpathlineto{\pgfqpoint{1.259648in}{0.976102in}}%
\pgfpathlineto{\pgfqpoint{1.267096in}{0.979580in}}%
\pgfpathlineto{\pgfqpoint{1.274544in}{0.986443in}}%
\pgfpathlineto{\pgfqpoint{1.281991in}{0.996576in}}%
\pgfpathlineto{\pgfqpoint{1.291922in}{1.014925in}}%
\pgfpathlineto{\pgfqpoint{1.301852in}{1.038437in}}%
\pgfpathlineto{\pgfqpoint{1.314265in}{1.074382in}}%
\pgfpathlineto{\pgfqpoint{1.329161in}{1.125769in}}%
\pgfpathlineto{\pgfqpoint{1.346539in}{1.194591in}}%
\pgfpathlineto{\pgfqpoint{1.371365in}{1.303002in}}%
\pgfpathlineto{\pgfqpoint{1.413569in}{1.488075in}}%
\pgfpathlineto{\pgfqpoint{1.433430in}{1.564392in}}%
\pgfpathlineto{\pgfqpoint{1.448326in}{1.613590in}}%
\pgfpathlineto{\pgfqpoint{1.460739in}{1.648229in}}%
\pgfpathlineto{\pgfqpoint{1.473152in}{1.676437in}}%
\pgfpathlineto{\pgfqpoint{1.483082in}{1.694074in}}%
\pgfpathlineto{\pgfqpoint{1.493013in}{1.707164in}}%
\pgfpathlineto{\pgfqpoint{1.500461in}{1.713947in}}%
\pgfpathlineto{\pgfqpoint{1.507908in}{1.718118in}}%
\pgfpathlineto{\pgfqpoint{1.515356in}{1.719689in}}%
\pgfpathlineto{\pgfqpoint{1.522804in}{1.718694in}}%
\pgfpathlineto{\pgfqpoint{1.530252in}{1.715187in}}%
\pgfpathlineto{\pgfqpoint{1.537700in}{1.709241in}}%
\pgfpathlineto{\pgfqpoint{1.547630in}{1.697680in}}%
\pgfpathlineto{\pgfqpoint{1.557560in}{1.682212in}}%
\pgfpathlineto{\pgfqpoint{1.569973in}{1.657864in}}%
\pgfpathlineto{\pgfqpoint{1.582386in}{1.628613in}}%
\pgfpathlineto{\pgfqpoint{1.597282in}{1.588165in}}%
\pgfpathlineto{\pgfqpoint{1.617143in}{1.527646in}}%
\pgfpathlineto{\pgfqpoint{1.686656in}{1.308547in}}%
\pgfpathlineto{\pgfqpoint{1.704034in}{1.264077in}}%
\pgfpathlineto{\pgfqpoint{1.718930in}{1.232226in}}%
\pgfpathlineto{\pgfqpoint{1.731343in}{1.210736in}}%
\pgfpathlineto{\pgfqpoint{1.741273in}{1.197106in}}%
\pgfpathlineto{\pgfqpoint{1.751203in}{1.186774in}}%
\pgfpathlineto{\pgfqpoint{1.761134in}{1.179810in}}%
\pgfpathlineto{\pgfqpoint{1.768582in}{1.176809in}}%
\pgfpathlineto{\pgfqpoint{1.776030in}{1.175706in}}%
\pgfpathlineto{\pgfqpoint{1.783477in}{1.176473in}}%
\pgfpathlineto{\pgfqpoint{1.790925in}{1.179072in}}%
\pgfpathlineto{\pgfqpoint{1.800856in}{1.185291in}}%
\pgfpathlineto{\pgfqpoint{1.810786in}{1.194504in}}%
\pgfpathlineto{\pgfqpoint{1.820716in}{1.206509in}}%
\pgfpathlineto{\pgfqpoint{1.833129in}{1.225074in}}%
\pgfpathlineto{\pgfqpoint{1.848025in}{1.251861in}}%
\pgfpathlineto{\pgfqpoint{1.865403in}{1.288015in}}%
\pgfpathlineto{\pgfqpoint{1.890229in}{1.345395in}}%
\pgfpathlineto{\pgfqpoint{1.934916in}{1.449752in}}%
\pgfpathlineto{\pgfqpoint{1.954777in}{1.490237in}}%
\pgfpathlineto{\pgfqpoint{1.969673in}{1.516290in}}%
\pgfpathlineto{\pgfqpoint{1.984568in}{1.537857in}}%
\pgfpathlineto{\pgfqpoint{1.996981in}{1.552015in}}%
\pgfpathlineto{\pgfqpoint{2.009394in}{1.562474in}}%
\pgfpathlineto{\pgfqpoint{2.019325in}{1.568093in}}%
\pgfpathlineto{\pgfqpoint{2.029255in}{1.571241in}}%
\pgfpathlineto{\pgfqpoint{2.039185in}{1.571932in}}%
\pgfpathlineto{\pgfqpoint{2.049116in}{1.570215in}}%
\pgfpathlineto{\pgfqpoint{2.059046in}{1.566172in}}%
\pgfpathlineto{\pgfqpoint{2.068977in}{1.559915in}}%
\pgfpathlineto{\pgfqpoint{2.081390in}{1.549200in}}%
\pgfpathlineto{\pgfqpoint{2.093803in}{1.535585in}}%
\pgfpathlineto{\pgfqpoint{2.108698in}{1.515969in}}%
\pgfpathlineto{\pgfqpoint{2.128559in}{1.485506in}}%
\pgfpathlineto{\pgfqpoint{2.158350in}{1.434595in}}%
\pgfpathlineto{\pgfqpoint{2.195589in}{1.371469in}}%
\pgfpathlineto{\pgfqpoint{2.215450in}{1.341970in}}%
\pgfpathlineto{\pgfqpoint{2.232829in}{1.320149in}}%
\pgfpathlineto{\pgfqpoint{2.247724in}{1.305049in}}%
\pgfpathlineto{\pgfqpoint{2.260137in}{1.295294in}}%
\pgfpathlineto{\pgfqpoint{2.272550in}{1.288262in}}%
\pgfpathlineto{\pgfqpoint{2.284963in}{1.284030in}}%
\pgfpathlineto{\pgfqpoint{2.297376in}{1.282609in}}%
\pgfpathlineto{\pgfqpoint{2.309789in}{1.283948in}}%
\pgfpathlineto{\pgfqpoint{2.322202in}{1.287933in}}%
\pgfpathlineto{\pgfqpoint{2.334615in}{1.294399in}}%
\pgfpathlineto{\pgfqpoint{2.349511in}{1.305123in}}%
\pgfpathlineto{\pgfqpoint{2.364406in}{1.318644in}}%
\pgfpathlineto{\pgfqpoint{2.384267in}{1.340098in}}%
\pgfpathlineto{\pgfqpoint{2.411576in}{1.373583in}}%
\pgfpathlineto{\pgfqpoint{2.458745in}{1.431908in}}%
\pgfpathlineto{\pgfqpoint{2.478606in}{1.452981in}}%
\pgfpathlineto{\pgfqpoint{2.495984in}{1.468399in}}%
\pgfpathlineto{\pgfqpoint{2.510880in}{1.478924in}}%
\pgfpathlineto{\pgfqpoint{2.525776in}{1.486699in}}%
\pgfpathlineto{\pgfqpoint{2.538189in}{1.490967in}}%
\pgfpathlineto{\pgfqpoint{2.550602in}{1.493183in}}%
\pgfpathlineto{\pgfqpoint{2.563015in}{1.493359in}}%
\pgfpathlineto{\pgfqpoint{2.575428in}{1.491552in}}%
\pgfpathlineto{\pgfqpoint{2.590323in}{1.486907in}}%
\pgfpathlineto{\pgfqpoint{2.605219in}{1.479795in}}%
\pgfpathlineto{\pgfqpoint{2.622597in}{1.468802in}}%
\pgfpathlineto{\pgfqpoint{2.642458in}{1.453422in}}%
\pgfpathlineto{\pgfqpoint{2.669767in}{1.429174in}}%
\pgfpathlineto{\pgfqpoint{2.721901in}{1.382290in}}%
\pgfpathlineto{\pgfqpoint{2.744245in}{1.365545in}}%
\pgfpathlineto{\pgfqpoint{2.761623in}{1.355000in}}%
\pgfpathlineto{\pgfqpoint{2.779001in}{1.347032in}}%
\pgfpathlineto{\pgfqpoint{2.793897in}{1.342440in}}%
\pgfpathlineto{\pgfqpoint{2.808792in}{1.339992in}}%
\pgfpathlineto{\pgfqpoint{2.823688in}{1.339691in}}%
\pgfpathlineto{\pgfqpoint{2.838584in}{1.341470in}}%
\pgfpathlineto{\pgfqpoint{2.855962in}{1.345998in}}%
\pgfpathlineto{\pgfqpoint{2.873340in}{1.352872in}}%
\pgfpathlineto{\pgfqpoint{2.893201in}{1.363091in}}%
\pgfpathlineto{\pgfqpoint{2.920510in}{1.380008in}}%
\pgfpathlineto{\pgfqpoint{2.992505in}{1.426503in}}%
\pgfpathlineto{\pgfqpoint{3.014849in}{1.437581in}}%
\pgfpathlineto{\pgfqpoint{3.034709in}{1.444979in}}%
\pgfpathlineto{\pgfqpoint{3.052088in}{1.449300in}}%
\pgfpathlineto{\pgfqpoint{3.069466in}{1.451501in}}%
\pgfpathlineto{\pgfqpoint{3.086844in}{1.451572in}}%
\pgfpathlineto{\pgfqpoint{3.104222in}{1.449599in}}%
\pgfpathlineto{\pgfqpoint{3.124083in}{1.445062in}}%
\pgfpathlineto{\pgfqpoint{3.146427in}{1.437499in}}%
\pgfpathlineto{\pgfqpoint{3.173735in}{1.425685in}}%
\pgfpathlineto{\pgfqpoint{3.225870in}{1.400283in}}%
\pgfpathlineto{\pgfqpoint{3.225870in}{1.400283in}}%
\pgfusepath{stroke}%
\end{pgfscope}%
\begin{pgfscope}%
\pgfpathrectangle{\pgfqpoint{0.619136in}{0.571603in}}{\pgfqpoint{2.730864in}{1.657828in}}%
\pgfusepath{clip}%
\pgfsetrectcap%
\pgfsetroundjoin%
\pgfsetlinewidth{1.505625pt}%
\definecolor{currentstroke}{rgb}{0.498039,0.498039,0.498039}%
\pgfsetstrokecolor{currentstroke}%
\pgfsetdash{}{0pt}%
\pgfpathmoveto{\pgfqpoint{0.743267in}{0.646959in}}%
\pgfpathlineto{\pgfqpoint{0.745749in}{0.653520in}}%
\pgfpathlineto{\pgfqpoint{0.750714in}{0.677988in}}%
\pgfpathlineto{\pgfqpoint{0.758162in}{0.728375in}}%
\pgfpathlineto{\pgfqpoint{0.770575in}{0.830781in}}%
\pgfpathlineto{\pgfqpoint{0.815262in}{1.217836in}}%
\pgfpathlineto{\pgfqpoint{0.830158in}{1.322515in}}%
\pgfpathlineto{\pgfqpoint{0.842571in}{1.395534in}}%
\pgfpathlineto{\pgfqpoint{0.854984in}{1.455196in}}%
\pgfpathlineto{\pgfqpoint{0.864914in}{1.493508in}}%
\pgfpathlineto{\pgfqpoint{0.874845in}{1.523871in}}%
\pgfpathlineto{\pgfqpoint{0.884775in}{1.546838in}}%
\pgfpathlineto{\pgfqpoint{0.892223in}{1.559603in}}%
\pgfpathlineto{\pgfqpoint{0.899671in}{1.568879in}}%
\pgfpathlineto{\pgfqpoint{0.907118in}{1.574983in}}%
\pgfpathlineto{\pgfqpoint{0.914566in}{1.578236in}}%
\pgfpathlineto{\pgfqpoint{0.922014in}{1.578961in}}%
\pgfpathlineto{\pgfqpoint{0.929462in}{1.577473in}}%
\pgfpathlineto{\pgfqpoint{0.939392in}{1.572578in}}%
\pgfpathlineto{\pgfqpoint{0.949323in}{1.564987in}}%
\pgfpathlineto{\pgfqpoint{0.961736in}{1.552675in}}%
\pgfpathlineto{\pgfqpoint{0.979114in}{1.532133in}}%
\pgfpathlineto{\pgfqpoint{1.031249in}{1.468137in}}%
\pgfpathlineto{\pgfqpoint{1.051109in}{1.448348in}}%
\pgfpathlineto{\pgfqpoint{1.068488in}{1.434248in}}%
\pgfpathlineto{\pgfqpoint{1.085866in}{1.423217in}}%
\pgfpathlineto{\pgfqpoint{1.103244in}{1.415081in}}%
\pgfpathlineto{\pgfqpoint{1.120622in}{1.409514in}}%
\pgfpathlineto{\pgfqpoint{1.140483in}{1.405771in}}%
\pgfpathlineto{\pgfqpoint{1.162827in}{1.404114in}}%
\pgfpathlineto{\pgfqpoint{1.190135in}{1.404517in}}%
\pgfpathlineto{\pgfqpoint{1.237305in}{1.407919in}}%
\pgfpathlineto{\pgfqpoint{1.299370in}{1.411906in}}%
\pgfpathlineto{\pgfqpoint{1.351504in}{1.412937in}}%
\pgfpathlineto{\pgfqpoint{1.425982in}{1.411889in}}%
\pgfpathlineto{\pgfqpoint{1.609695in}{1.408768in}}%
\pgfpathlineto{\pgfqpoint{1.952294in}{1.407143in}}%
\pgfpathlineto{\pgfqpoint{2.672249in}{1.405745in}}%
\pgfpathlineto{\pgfqpoint{3.225870in}{1.405303in}}%
\pgfpathlineto{\pgfqpoint{3.225870in}{1.405303in}}%
\pgfusepath{stroke}%
\end{pgfscope}%
\begin{pgfscope}%
\pgfpathrectangle{\pgfqpoint{0.619136in}{0.571603in}}{\pgfqpoint{2.730864in}{1.657828in}}%
\pgfusepath{clip}%
\pgfsetrectcap%
\pgfsetroundjoin%
\pgfsetlinewidth{1.505625pt}%
\definecolor{currentstroke}{rgb}{0.737255,0.741176,0.133333}%
\pgfsetstrokecolor{currentstroke}%
\pgfsetdash{}{0pt}%
\pgfpathmoveto{\pgfqpoint{0.743267in}{0.646959in}}%
\pgfpathlineto{\pgfqpoint{0.745749in}{0.650700in}}%
\pgfpathlineto{\pgfqpoint{0.750714in}{0.666499in}}%
\pgfpathlineto{\pgfqpoint{0.758162in}{0.701844in}}%
\pgfpathlineto{\pgfqpoint{0.768093in}{0.762732in}}%
\pgfpathlineto{\pgfqpoint{0.782988in}{0.871332in}}%
\pgfpathlineto{\pgfqpoint{0.840088in}{1.306752in}}%
\pgfpathlineto{\pgfqpoint{0.854984in}{1.398332in}}%
\pgfpathlineto{\pgfqpoint{0.867397in}{1.463491in}}%
\pgfpathlineto{\pgfqpoint{0.879810in}{1.517950in}}%
\pgfpathlineto{\pgfqpoint{0.892223in}{1.561648in}}%
\pgfpathlineto{\pgfqpoint{0.902153in}{1.589049in}}%
\pgfpathlineto{\pgfqpoint{0.912084in}{1.610034in}}%
\pgfpathlineto{\pgfqpoint{0.922014in}{1.624986in}}%
\pgfpathlineto{\pgfqpoint{0.929462in}{1.632511in}}%
\pgfpathlineto{\pgfqpoint{0.936910in}{1.637110in}}%
\pgfpathlineto{\pgfqpoint{0.944357in}{1.639007in}}%
\pgfpathlineto{\pgfqpoint{0.951805in}{1.638435in}}%
\pgfpathlineto{\pgfqpoint{0.959253in}{1.635635in}}%
\pgfpathlineto{\pgfqpoint{0.969183in}{1.628851in}}%
\pgfpathlineto{\pgfqpoint{0.979114in}{1.619099in}}%
\pgfpathlineto{\pgfqpoint{0.991527in}{1.603577in}}%
\pgfpathlineto{\pgfqpoint{1.008905in}{1.577441in}}%
\pgfpathlineto{\pgfqpoint{1.038696in}{1.527201in}}%
\pgfpathlineto{\pgfqpoint{1.068488in}{1.478496in}}%
\pgfpathlineto{\pgfqpoint{1.088348in}{1.450223in}}%
\pgfpathlineto{\pgfqpoint{1.105727in}{1.429314in}}%
\pgfpathlineto{\pgfqpoint{1.120622in}{1.414521in}}%
\pgfpathlineto{\pgfqpoint{1.135518in}{1.402667in}}%
\pgfpathlineto{\pgfqpoint{1.150413in}{1.393663in}}%
\pgfpathlineto{\pgfqpoint{1.165309in}{1.387320in}}%
\pgfpathlineto{\pgfqpoint{1.180205in}{1.383374in}}%
\pgfpathlineto{\pgfqpoint{1.195100in}{1.381513in}}%
\pgfpathlineto{\pgfqpoint{1.212479in}{1.381526in}}%
\pgfpathlineto{\pgfqpoint{1.232339in}{1.383748in}}%
\pgfpathlineto{\pgfqpoint{1.259648in}{1.389274in}}%
\pgfpathlineto{\pgfqpoint{1.353987in}{1.410614in}}%
\pgfpathlineto{\pgfqpoint{1.386261in}{1.414595in}}%
\pgfpathlineto{\pgfqpoint{1.421017in}{1.416491in}}%
\pgfpathlineto{\pgfqpoint{1.460739in}{1.416274in}}%
\pgfpathlineto{\pgfqpoint{1.520321in}{1.413406in}}%
\pgfpathlineto{\pgfqpoint{1.624591in}{1.408433in}}%
\pgfpathlineto{\pgfqpoint{1.704034in}{1.407217in}}%
\pgfpathlineto{\pgfqpoint{2.232829in}{1.406310in}}%
\pgfpathlineto{\pgfqpoint{3.225870in}{1.405235in}}%
\pgfpathlineto{\pgfqpoint{3.225870in}{1.405235in}}%
\pgfusepath{stroke}%
\end{pgfscope}%
\begin{pgfscope}%
\pgfpathrectangle{\pgfqpoint{0.619136in}{0.571603in}}{\pgfqpoint{2.730864in}{1.657828in}}%
\pgfusepath{clip}%
\pgfsetrectcap%
\pgfsetroundjoin%
\pgfsetlinewidth{1.505625pt}%
\definecolor{currentstroke}{rgb}{0.090196,0.745098,0.811765}%
\pgfsetstrokecolor{currentstroke}%
\pgfsetdash{}{0pt}%
\pgfpathmoveto{\pgfqpoint{0.743267in}{0.646959in}}%
\pgfpathlineto{\pgfqpoint{0.748232in}{0.648675in}}%
\pgfpathlineto{\pgfqpoint{0.753197in}{0.653248in}}%
\pgfpathlineto{\pgfqpoint{0.758162in}{0.660390in}}%
\pgfpathlineto{\pgfqpoint{0.765610in}{0.675582in}}%
\pgfpathlineto{\pgfqpoint{0.773058in}{0.695801in}}%
\pgfpathlineto{\pgfqpoint{0.782988in}{0.729988in}}%
\pgfpathlineto{\pgfqpoint{0.795401in}{0.783261in}}%
\pgfpathlineto{\pgfqpoint{0.807814in}{0.846830in}}%
\pgfpathlineto{\pgfqpoint{0.822710in}{0.934502in}}%
\pgfpathlineto{\pgfqpoint{0.842571in}{1.066090in}}%
\pgfpathlineto{\pgfqpoint{0.874845in}{1.299165in}}%
\pgfpathlineto{\pgfqpoint{0.909601in}{1.547432in}}%
\pgfpathlineto{\pgfqpoint{0.929462in}{1.675390in}}%
\pgfpathlineto{\pgfqpoint{0.946840in}{1.773708in}}%
\pgfpathlineto{\pgfqpoint{0.961736in}{1.845518in}}%
\pgfpathlineto{\pgfqpoint{0.974149in}{1.895451in}}%
\pgfpathlineto{\pgfqpoint{0.986562in}{1.935690in}}%
\pgfpathlineto{\pgfqpoint{0.996492in}{1.960596in}}%
\pgfpathlineto{\pgfqpoint{1.006423in}{1.978868in}}%
\pgfpathlineto{\pgfqpoint{1.013870in}{1.988177in}}%
\pgfpathlineto{\pgfqpoint{1.021318in}{1.993717in}}%
\pgfpathlineto{\pgfqpoint{1.026283in}{1.995328in}}%
\pgfpathlineto{\pgfqpoint{1.031249in}{1.995289in}}%
\pgfpathlineto{\pgfqpoint{1.036214in}{1.993616in}}%
\pgfpathlineto{\pgfqpoint{1.043662in}{1.988093in}}%
\pgfpathlineto{\pgfqpoint{1.051109in}{1.979039in}}%
\pgfpathlineto{\pgfqpoint{1.058557in}{1.966566in}}%
\pgfpathlineto{\pgfqpoint{1.068488in}{1.944850in}}%
\pgfpathlineto{\pgfqpoint{1.078418in}{1.917661in}}%
\pgfpathlineto{\pgfqpoint{1.090831in}{1.876619in}}%
\pgfpathlineto{\pgfqpoint{1.105727in}{1.818251in}}%
\pgfpathlineto{\pgfqpoint{1.123105in}{1.739837in}}%
\pgfpathlineto{\pgfqpoint{1.145448in}{1.627433in}}%
\pgfpathlineto{\pgfqpoint{1.217444in}{1.254049in}}%
\pgfpathlineto{\pgfqpoint{1.234822in}{1.177984in}}%
\pgfpathlineto{\pgfqpoint{1.249718in}{1.121165in}}%
\pgfpathlineto{\pgfqpoint{1.264613in}{1.073310in}}%
\pgfpathlineto{\pgfqpoint{1.277026in}{1.040932in}}%
\pgfpathlineto{\pgfqpoint{1.286957in}{1.020218in}}%
\pgfpathlineto{\pgfqpoint{1.296887in}{1.004260in}}%
\pgfpathlineto{\pgfqpoint{1.306817in}{0.993130in}}%
\pgfpathlineto{\pgfqpoint{1.314265in}{0.987963in}}%
\pgfpathlineto{\pgfqpoint{1.321713in}{0.985511in}}%
\pgfpathlineto{\pgfqpoint{1.329161in}{0.985744in}}%
\pgfpathlineto{\pgfqpoint{1.336609in}{0.988616in}}%
\pgfpathlineto{\pgfqpoint{1.344057in}{0.994065in}}%
\pgfpathlineto{\pgfqpoint{1.351504in}{1.002015in}}%
\pgfpathlineto{\pgfqpoint{1.361435in}{1.016346in}}%
\pgfpathlineto{\pgfqpoint{1.371365in}{1.034705in}}%
\pgfpathlineto{\pgfqpoint{1.383778in}{1.062870in}}%
\pgfpathlineto{\pgfqpoint{1.398674in}{1.103452in}}%
\pgfpathlineto{\pgfqpoint{1.416052in}{1.158553in}}%
\pgfpathlineto{\pgfqpoint{1.438395in}{1.238287in}}%
\pgfpathlineto{\pgfqpoint{1.517839in}{1.532050in}}%
\pgfpathlineto{\pgfqpoint{1.535217in}{1.584707in}}%
\pgfpathlineto{\pgfqpoint{1.550113in}{1.623499in}}%
\pgfpathlineto{\pgfqpoint{1.562526in}{1.650751in}}%
\pgfpathlineto{\pgfqpoint{1.574939in}{1.673014in}}%
\pgfpathlineto{\pgfqpoint{1.584869in}{1.687062in}}%
\pgfpathlineto{\pgfqpoint{1.594799in}{1.697671in}}%
\pgfpathlineto{\pgfqpoint{1.604730in}{1.704798in}}%
\pgfpathlineto{\pgfqpoint{1.612178in}{1.707855in}}%
\pgfpathlineto{\pgfqpoint{1.619626in}{1.708963in}}%
\pgfpathlineto{\pgfqpoint{1.627073in}{1.708147in}}%
\pgfpathlineto{\pgfqpoint{1.634521in}{1.705444in}}%
\pgfpathlineto{\pgfqpoint{1.644452in}{1.698989in}}%
\pgfpathlineto{\pgfqpoint{1.654382in}{1.689408in}}%
\pgfpathlineto{\pgfqpoint{1.664312in}{1.676876in}}%
\pgfpathlineto{\pgfqpoint{1.676725in}{1.657373in}}%
\pgfpathlineto{\pgfqpoint{1.691621in}{1.628947in}}%
\pgfpathlineto{\pgfqpoint{1.708999in}{1.589990in}}%
\pgfpathlineto{\pgfqpoint{1.728860in}{1.539739in}}%
\pgfpathlineto{\pgfqpoint{1.763617in}{1.444331in}}%
\pgfpathlineto{\pgfqpoint{1.798373in}{1.350869in}}%
\pgfpathlineto{\pgfqpoint{1.818234in}{1.303097in}}%
\pgfpathlineto{\pgfqpoint{1.835612in}{1.266605in}}%
\pgfpathlineto{\pgfqpoint{1.850508in}{1.240109in}}%
\pgfpathlineto{\pgfqpoint{1.862921in}{1.221804in}}%
\pgfpathlineto{\pgfqpoint{1.875334in}{1.207172in}}%
\pgfpathlineto{\pgfqpoint{1.887747in}{1.196367in}}%
\pgfpathlineto{\pgfqpoint{1.897677in}{1.190536in}}%
\pgfpathlineto{\pgfqpoint{1.907607in}{1.187219in}}%
\pgfpathlineto{\pgfqpoint{1.917538in}{1.186398in}}%
\pgfpathlineto{\pgfqpoint{1.927468in}{1.188026in}}%
\pgfpathlineto{\pgfqpoint{1.937399in}{1.192034in}}%
\pgfpathlineto{\pgfqpoint{1.947329in}{1.198324in}}%
\pgfpathlineto{\pgfqpoint{1.959742in}{1.209210in}}%
\pgfpathlineto{\pgfqpoint{1.972155in}{1.223193in}}%
\pgfpathlineto{\pgfqpoint{1.987051in}{1.243597in}}%
\pgfpathlineto{\pgfqpoint{2.004429in}{1.271586in}}%
\pgfpathlineto{\pgfqpoint{2.026772in}{1.312459in}}%
\pgfpathlineto{\pgfqpoint{2.068977in}{1.396134in}}%
\pgfpathlineto{\pgfqpoint{2.098768in}{1.452624in}}%
\pgfpathlineto{\pgfqpoint{2.118629in}{1.485979in}}%
\pgfpathlineto{\pgfqpoint{2.136007in}{1.511101in}}%
\pgfpathlineto{\pgfqpoint{2.150903in}{1.529045in}}%
\pgfpathlineto{\pgfqpoint{2.165798in}{1.543316in}}%
\pgfpathlineto{\pgfqpoint{2.178211in}{1.552229in}}%
\pgfpathlineto{\pgfqpoint{2.190624in}{1.558345in}}%
\pgfpathlineto{\pgfqpoint{2.203037in}{1.561633in}}%
\pgfpathlineto{\pgfqpoint{2.215450in}{1.562113in}}%
\pgfpathlineto{\pgfqpoint{2.227863in}{1.559853in}}%
\pgfpathlineto{\pgfqpoint{2.240276in}{1.554967in}}%
\pgfpathlineto{\pgfqpoint{2.252689in}{1.547611in}}%
\pgfpathlineto{\pgfqpoint{2.267585in}{1.535798in}}%
\pgfpathlineto{\pgfqpoint{2.282481in}{1.521110in}}%
\pgfpathlineto{\pgfqpoint{2.299859in}{1.500953in}}%
\pgfpathlineto{\pgfqpoint{2.322202in}{1.471507in}}%
\pgfpathlineto{\pgfqpoint{2.369372in}{1.404145in}}%
\pgfpathlineto{\pgfqpoint{2.396680in}{1.367273in}}%
\pgfpathlineto{\pgfqpoint{2.416541in}{1.343603in}}%
\pgfpathlineto{\pgfqpoint{2.433919in}{1.325902in}}%
\pgfpathlineto{\pgfqpoint{2.451298in}{1.311526in}}%
\pgfpathlineto{\pgfqpoint{2.466193in}{1.302128in}}%
\pgfpathlineto{\pgfqpoint{2.481089in}{1.295584in}}%
\pgfpathlineto{\pgfqpoint{2.493502in}{1.292365in}}%
\pgfpathlineto{\pgfqpoint{2.505915in}{1.291178in}}%
\pgfpathlineto{\pgfqpoint{2.518328in}{1.291990in}}%
\pgfpathlineto{\pgfqpoint{2.530741in}{1.294732in}}%
\pgfpathlineto{\pgfqpoint{2.545637in}{1.300427in}}%
\pgfpathlineto{\pgfqpoint{2.560532in}{1.308527in}}%
\pgfpathlineto{\pgfqpoint{2.577910in}{1.320643in}}%
\pgfpathlineto{\pgfqpoint{2.597771in}{1.337363in}}%
\pgfpathlineto{\pgfqpoint{2.622597in}{1.361292in}}%
\pgfpathlineto{\pgfqpoint{2.702041in}{1.440192in}}%
\pgfpathlineto{\pgfqpoint{2.721901in}{1.455994in}}%
\pgfpathlineto{\pgfqpoint{2.739280in}{1.467426in}}%
\pgfpathlineto{\pgfqpoint{2.756658in}{1.476305in}}%
\pgfpathlineto{\pgfqpoint{2.774036in}{1.482428in}}%
\pgfpathlineto{\pgfqpoint{2.788932in}{1.485402in}}%
\pgfpathlineto{\pgfqpoint{2.803827in}{1.486268in}}%
\pgfpathlineto{\pgfqpoint{2.818723in}{1.485073in}}%
\pgfpathlineto{\pgfqpoint{2.836101in}{1.481207in}}%
\pgfpathlineto{\pgfqpoint{2.853479in}{1.474900in}}%
\pgfpathlineto{\pgfqpoint{2.873340in}{1.465099in}}%
\pgfpathlineto{\pgfqpoint{2.895684in}{1.451429in}}%
\pgfpathlineto{\pgfqpoint{2.925475in}{1.430357in}}%
\pgfpathlineto{\pgfqpoint{2.990023in}{1.383677in}}%
\pgfpathlineto{\pgfqpoint{3.014849in}{1.368851in}}%
\pgfpathlineto{\pgfqpoint{3.037192in}{1.358242in}}%
\pgfpathlineto{\pgfqpoint{3.057053in}{1.351382in}}%
\pgfpathlineto{\pgfqpoint{3.074431in}{1.347531in}}%
\pgfpathlineto{\pgfqpoint{3.091809in}{1.345748in}}%
\pgfpathlineto{\pgfqpoint{3.109187in}{1.346013in}}%
\pgfpathlineto{\pgfqpoint{3.129048in}{1.348711in}}%
\pgfpathlineto{\pgfqpoint{3.148909in}{1.353740in}}%
\pgfpathlineto{\pgfqpoint{3.171253in}{1.361783in}}%
\pgfpathlineto{\pgfqpoint{3.198561in}{1.374230in}}%
\pgfpathlineto{\pgfqpoint{3.225870in}{1.388426in}}%
\pgfpathlineto{\pgfqpoint{3.225870in}{1.388426in}}%
\pgfusepath{stroke}%
\end{pgfscope}%
\begin{pgfscope}%
\pgfpathrectangle{\pgfqpoint{0.619136in}{0.571603in}}{\pgfqpoint{2.730864in}{1.657828in}}%
\pgfusepath{clip}%
\pgfsetrectcap%
\pgfsetroundjoin%
\pgfsetlinewidth{1.505625pt}%
\definecolor{currentstroke}{rgb}{0.121569,0.466667,0.705882}%
\pgfsetstrokecolor{currentstroke}%
\pgfsetdash{}{0pt}%
\pgfpathmoveto{\pgfqpoint{0.743267in}{0.646959in}}%
\pgfpathlineto{\pgfqpoint{0.760645in}{0.863284in}}%
\pgfpathlineto{\pgfqpoint{0.773058in}{0.985737in}}%
\pgfpathlineto{\pgfqpoint{0.785471in}{1.083143in}}%
\pgfpathlineto{\pgfqpoint{0.797884in}{1.159805in}}%
\pgfpathlineto{\pgfqpoint{0.807814in}{1.208888in}}%
\pgfpathlineto{\pgfqpoint{0.817745in}{1.249016in}}%
\pgfpathlineto{\pgfqpoint{0.827675in}{1.281704in}}%
\pgfpathlineto{\pgfqpoint{0.837605in}{1.308242in}}%
\pgfpathlineto{\pgfqpoint{0.847536in}{1.329714in}}%
\pgfpathlineto{\pgfqpoint{0.859949in}{1.350798in}}%
\pgfpathlineto{\pgfqpoint{0.872362in}{1.366817in}}%
\pgfpathlineto{\pgfqpoint{0.884775in}{1.378912in}}%
\pgfpathlineto{\pgfqpoint{0.897188in}{1.387979in}}%
\pgfpathlineto{\pgfqpoint{0.912084in}{1.395842in}}%
\pgfpathlineto{\pgfqpoint{0.926979in}{1.401274in}}%
\pgfpathlineto{\pgfqpoint{0.946840in}{1.405886in}}%
\pgfpathlineto{\pgfqpoint{0.971666in}{1.408940in}}%
\pgfpathlineto{\pgfqpoint{1.003940in}{1.410427in}}%
\pgfpathlineto{\pgfqpoint{1.058557in}{1.410351in}}%
\pgfpathlineto{\pgfqpoint{1.406122in}{1.406652in}}%
\pgfpathlineto{\pgfqpoint{1.910090in}{1.405493in}}%
\pgfpathlineto{\pgfqpoint{3.225870in}{1.404810in}}%
\pgfpathlineto{\pgfqpoint{3.225870in}{1.404810in}}%
\pgfusepath{stroke}%
\end{pgfscope}%
\begin{pgfscope}%
\pgfpathrectangle{\pgfqpoint{0.619136in}{0.571603in}}{\pgfqpoint{2.730864in}{1.657828in}}%
\pgfusepath{clip}%
\pgfsetrectcap%
\pgfsetroundjoin%
\pgfsetlinewidth{1.505625pt}%
\definecolor{currentstroke}{rgb}{1.000000,0.498039,0.054902}%
\pgfsetstrokecolor{currentstroke}%
\pgfsetdash{}{0pt}%
\pgfpathmoveto{\pgfqpoint{0.743267in}{0.646959in}}%
\pgfpathlineto{\pgfqpoint{0.748232in}{0.649397in}}%
\pgfpathlineto{\pgfqpoint{0.753197in}{0.655351in}}%
\pgfpathlineto{\pgfqpoint{0.760645in}{0.669664in}}%
\pgfpathlineto{\pgfqpoint{0.768093in}{0.689642in}}%
\pgfpathlineto{\pgfqpoint{0.778023in}{0.724025in}}%
\pgfpathlineto{\pgfqpoint{0.790436in}{0.777785in}}%
\pgfpathlineto{\pgfqpoint{0.805332in}{0.855392in}}%
\pgfpathlineto{\pgfqpoint{0.822710in}{0.959746in}}%
\pgfpathlineto{\pgfqpoint{0.847536in}{1.124932in}}%
\pgfpathlineto{\pgfqpoint{0.899671in}{1.476915in}}%
\pgfpathlineto{\pgfqpoint{0.919531in}{1.594871in}}%
\pgfpathlineto{\pgfqpoint{0.936910in}{1.684829in}}%
\pgfpathlineto{\pgfqpoint{0.951805in}{1.750244in}}%
\pgfpathlineto{\pgfqpoint{0.964218in}{1.795693in}}%
\pgfpathlineto{\pgfqpoint{0.976631in}{1.832454in}}%
\pgfpathlineto{\pgfqpoint{0.986562in}{1.855443in}}%
\pgfpathlineto{\pgfqpoint{0.996492in}{1.872674in}}%
\pgfpathlineto{\pgfqpoint{1.003940in}{1.881830in}}%
\pgfpathlineto{\pgfqpoint{1.011388in}{1.887793in}}%
\pgfpathlineto{\pgfqpoint{1.018836in}{1.890617in}}%
\pgfpathlineto{\pgfqpoint{1.026283in}{1.890372in}}%
\pgfpathlineto{\pgfqpoint{1.033731in}{1.887147in}}%
\pgfpathlineto{\pgfqpoint{1.041179in}{1.881049in}}%
\pgfpathlineto{\pgfqpoint{1.048627in}{1.872197in}}%
\pgfpathlineto{\pgfqpoint{1.058557in}{1.856346in}}%
\pgfpathlineto{\pgfqpoint{1.068488in}{1.836194in}}%
\pgfpathlineto{\pgfqpoint{1.080901in}{1.805550in}}%
\pgfpathlineto{\pgfqpoint{1.095796in}{1.761868in}}%
\pgfpathlineto{\pgfqpoint{1.113174in}{1.703300in}}%
\pgfpathlineto{\pgfqpoint{1.138000in}{1.610204in}}%
\pgfpathlineto{\pgfqpoint{1.197583in}{1.381570in}}%
\pgfpathlineto{\pgfqpoint{1.217444in}{1.315199in}}%
\pgfpathlineto{\pgfqpoint{1.234822in}{1.264648in}}%
\pgfpathlineto{\pgfqpoint{1.249718in}{1.227890in}}%
\pgfpathlineto{\pgfqpoint{1.262131in}{1.202325in}}%
\pgfpathlineto{\pgfqpoint{1.274544in}{1.181601in}}%
\pgfpathlineto{\pgfqpoint{1.284474in}{1.168590in}}%
\pgfpathlineto{\pgfqpoint{1.294404in}{1.158775in}}%
\pgfpathlineto{\pgfqpoint{1.304335in}{1.152139in}}%
\pgfpathlineto{\pgfqpoint{1.314265in}{1.148629in}}%
\pgfpathlineto{\pgfqpoint{1.324196in}{1.148161in}}%
\pgfpathlineto{\pgfqpoint{1.334126in}{1.150619in}}%
\pgfpathlineto{\pgfqpoint{1.344057in}{1.155861in}}%
\pgfpathlineto{\pgfqpoint{1.353987in}{1.163716in}}%
\pgfpathlineto{\pgfqpoint{1.366400in}{1.176918in}}%
\pgfpathlineto{\pgfqpoint{1.378813in}{1.193489in}}%
\pgfpathlineto{\pgfqpoint{1.393709in}{1.217179in}}%
\pgfpathlineto{\pgfqpoint{1.411087in}{1.249003in}}%
\pgfpathlineto{\pgfqpoint{1.435913in}{1.299649in}}%
\pgfpathlineto{\pgfqpoint{1.497978in}{1.428771in}}%
\pgfpathlineto{\pgfqpoint{1.517839in}{1.464244in}}%
\pgfpathlineto{\pgfqpoint{1.535217in}{1.491033in}}%
\pgfpathlineto{\pgfqpoint{1.550113in}{1.510324in}}%
\pgfpathlineto{\pgfqpoint{1.565008in}{1.525923in}}%
\pgfpathlineto{\pgfqpoint{1.577421in}{1.535972in}}%
\pgfpathlineto{\pgfqpoint{1.589834in}{1.543289in}}%
\pgfpathlineto{\pgfqpoint{1.602247in}{1.547881in}}%
\pgfpathlineto{\pgfqpoint{1.614660in}{1.549798in}}%
\pgfpathlineto{\pgfqpoint{1.627073in}{1.549137in}}%
\pgfpathlineto{\pgfqpoint{1.639486in}{1.546034in}}%
\pgfpathlineto{\pgfqpoint{1.651899in}{1.540662in}}%
\pgfpathlineto{\pgfqpoint{1.666795in}{1.531507in}}%
\pgfpathlineto{\pgfqpoint{1.681691in}{1.519776in}}%
\pgfpathlineto{\pgfqpoint{1.699069in}{1.503423in}}%
\pgfpathlineto{\pgfqpoint{1.721412in}{1.479328in}}%
\pgfpathlineto{\pgfqpoint{1.773547in}{1.418281in}}%
\pgfpathlineto{\pgfqpoint{1.800856in}{1.388561in}}%
\pgfpathlineto{\pgfqpoint{1.823199in}{1.367514in}}%
\pgfpathlineto{\pgfqpoint{1.843060in}{1.352073in}}%
\pgfpathlineto{\pgfqpoint{1.860438in}{1.341450in}}%
\pgfpathlineto{\pgfqpoint{1.875334in}{1.334634in}}%
\pgfpathlineto{\pgfqpoint{1.890229in}{1.329981in}}%
\pgfpathlineto{\pgfqpoint{1.905125in}{1.327480in}}%
\pgfpathlineto{\pgfqpoint{1.920021in}{1.327068in}}%
\pgfpathlineto{\pgfqpoint{1.934916in}{1.328631in}}%
\pgfpathlineto{\pgfqpoint{1.952294in}{1.332745in}}%
\pgfpathlineto{\pgfqpoint{1.972155in}{1.340085in}}%
\pgfpathlineto{\pgfqpoint{1.994499in}{1.351055in}}%
\pgfpathlineto{\pgfqpoint{2.021807in}{1.367154in}}%
\pgfpathlineto{\pgfqpoint{2.108698in}{1.420540in}}%
\pgfpathlineto{\pgfqpoint{2.133524in}{1.432132in}}%
\pgfpathlineto{\pgfqpoint{2.155868in}{1.440089in}}%
\pgfpathlineto{\pgfqpoint{2.175729in}{1.444987in}}%
\pgfpathlineto{\pgfqpoint{2.195589in}{1.447774in}}%
\pgfpathlineto{\pgfqpoint{2.215450in}{1.448491in}}%
\pgfpathlineto{\pgfqpoint{2.237794in}{1.446984in}}%
\pgfpathlineto{\pgfqpoint{2.260137in}{1.443313in}}%
\pgfpathlineto{\pgfqpoint{2.287446in}{1.436454in}}%
\pgfpathlineto{\pgfqpoint{2.322202in}{1.425206in}}%
\pgfpathlineto{\pgfqpoint{2.409093in}{1.395910in}}%
\pgfpathlineto{\pgfqpoint{2.438885in}{1.388619in}}%
\pgfpathlineto{\pgfqpoint{2.466193in}{1.384033in}}%
\pgfpathlineto{\pgfqpoint{2.493502in}{1.381625in}}%
\pgfpathlineto{\pgfqpoint{2.520811in}{1.381361in}}%
\pgfpathlineto{\pgfqpoint{2.550602in}{1.383275in}}%
\pgfpathlineto{\pgfqpoint{2.585358in}{1.387820in}}%
\pgfpathlineto{\pgfqpoint{2.635010in}{1.396857in}}%
\pgfpathlineto{\pgfqpoint{2.709488in}{1.410343in}}%
\pgfpathlineto{\pgfqpoint{2.749210in}{1.415199in}}%
\pgfpathlineto{\pgfqpoint{2.786449in}{1.417542in}}%
\pgfpathlineto{\pgfqpoint{2.823688in}{1.417689in}}%
\pgfpathlineto{\pgfqpoint{2.865892in}{1.415570in}}%
\pgfpathlineto{\pgfqpoint{2.922992in}{1.410215in}}%
\pgfpathlineto{\pgfqpoint{3.024779in}{1.400373in}}%
\pgfpathlineto{\pgfqpoint{3.076914in}{1.397884in}}%
\pgfpathlineto{\pgfqpoint{3.129048in}{1.397751in}}%
\pgfpathlineto{\pgfqpoint{3.188631in}{1.399929in}}%
\pgfpathlineto{\pgfqpoint{3.225870in}{1.402000in}}%
\pgfpathlineto{\pgfqpoint{3.225870in}{1.402000in}}%
\pgfusepath{stroke}%
\end{pgfscope}%
\begin{pgfscope}%
\pgfpathrectangle{\pgfqpoint{0.619136in}{0.571603in}}{\pgfqpoint{2.730864in}{1.657828in}}%
\pgfusepath{clip}%
\pgfsetrectcap%
\pgfsetroundjoin%
\pgfsetlinewidth{1.505625pt}%
\definecolor{currentstroke}{rgb}{0.172549,0.627451,0.172549}%
\pgfsetstrokecolor{currentstroke}%
\pgfsetdash{}{0pt}%
\pgfpathmoveto{\pgfqpoint{0.743267in}{0.646959in}}%
\pgfpathlineto{\pgfqpoint{0.748232in}{0.649443in}}%
\pgfpathlineto{\pgfqpoint{0.753197in}{0.655346in}}%
\pgfpathlineto{\pgfqpoint{0.760645in}{0.669295in}}%
\pgfpathlineto{\pgfqpoint{0.768093in}{0.688531in}}%
\pgfpathlineto{\pgfqpoint{0.778023in}{0.721321in}}%
\pgfpathlineto{\pgfqpoint{0.790436in}{0.772165in}}%
\pgfpathlineto{\pgfqpoint{0.805332in}{0.845059in}}%
\pgfpathlineto{\pgfqpoint{0.822710in}{0.942594in}}%
\pgfpathlineto{\pgfqpoint{0.847536in}{1.096600in}}%
\pgfpathlineto{\pgfqpoint{0.904636in}{1.456056in}}%
\pgfpathlineto{\pgfqpoint{0.924497in}{1.565707in}}%
\pgfpathlineto{\pgfqpoint{0.941875in}{1.649932in}}%
\pgfpathlineto{\pgfqpoint{0.956770in}{1.711925in}}%
\pgfpathlineto{\pgfqpoint{0.969183in}{1.755731in}}%
\pgfpathlineto{\pgfqpoint{0.981596in}{1.792034in}}%
\pgfpathlineto{\pgfqpoint{0.991527in}{1.815540in}}%
\pgfpathlineto{\pgfqpoint{1.001457in}{1.834081in}}%
\pgfpathlineto{\pgfqpoint{1.011388in}{1.847671in}}%
\pgfpathlineto{\pgfqpoint{1.018836in}{1.854653in}}%
\pgfpathlineto{\pgfqpoint{1.026283in}{1.858930in}}%
\pgfpathlineto{\pgfqpoint{1.033731in}{1.860563in}}%
\pgfpathlineto{\pgfqpoint{1.041179in}{1.859624in}}%
\pgfpathlineto{\pgfqpoint{1.048627in}{1.856199in}}%
\pgfpathlineto{\pgfqpoint{1.056075in}{1.850383in}}%
\pgfpathlineto{\pgfqpoint{1.066005in}{1.839099in}}%
\pgfpathlineto{\pgfqpoint{1.075935in}{1.824040in}}%
\pgfpathlineto{\pgfqpoint{1.088348in}{1.800378in}}%
\pgfpathlineto{\pgfqpoint{1.100761in}{1.771948in}}%
\pgfpathlineto{\pgfqpoint{1.115657in}{1.732513in}}%
\pgfpathlineto{\pgfqpoint{1.135518in}{1.672952in}}%
\pgfpathlineto{\pgfqpoint{1.162827in}{1.583142in}}%
\pgfpathlineto{\pgfqpoint{1.212479in}{1.418933in}}%
\pgfpathlineto{\pgfqpoint{1.234822in}{1.353193in}}%
\pgfpathlineto{\pgfqpoint{1.252200in}{1.308197in}}%
\pgfpathlineto{\pgfqpoint{1.267096in}{1.274718in}}%
\pgfpathlineto{\pgfqpoint{1.281991in}{1.246414in}}%
\pgfpathlineto{\pgfqpoint{1.294404in}{1.227000in}}%
\pgfpathlineto{\pgfqpoint{1.306817in}{1.211491in}}%
\pgfpathlineto{\pgfqpoint{1.319231in}{1.199919in}}%
\pgfpathlineto{\pgfqpoint{1.329161in}{1.193476in}}%
\pgfpathlineto{\pgfqpoint{1.339091in}{1.189492in}}%
\pgfpathlineto{\pgfqpoint{1.349022in}{1.187899in}}%
\pgfpathlineto{\pgfqpoint{1.358952in}{1.188610in}}%
\pgfpathlineto{\pgfqpoint{1.368883in}{1.191518in}}%
\pgfpathlineto{\pgfqpoint{1.378813in}{1.196500in}}%
\pgfpathlineto{\pgfqpoint{1.391226in}{1.205426in}}%
\pgfpathlineto{\pgfqpoint{1.403639in}{1.217069in}}%
\pgfpathlineto{\pgfqpoint{1.418535in}{1.234162in}}%
\pgfpathlineto{\pgfqpoint{1.435913in}{1.257639in}}%
\pgfpathlineto{\pgfqpoint{1.458256in}{1.291871in}}%
\pgfpathlineto{\pgfqpoint{1.502943in}{1.365959in}}%
\pgfpathlineto{\pgfqpoint{1.532734in}{1.413189in}}%
\pgfpathlineto{\pgfqpoint{1.555078in}{1.444664in}}%
\pgfpathlineto{\pgfqpoint{1.574939in}{1.468653in}}%
\pgfpathlineto{\pgfqpoint{1.592317in}{1.486034in}}%
\pgfpathlineto{\pgfqpoint{1.607212in}{1.498018in}}%
\pgfpathlineto{\pgfqpoint{1.622108in}{1.507204in}}%
\pgfpathlineto{\pgfqpoint{1.637004in}{1.513561in}}%
\pgfpathlineto{\pgfqpoint{1.651899in}{1.517124in}}%
\pgfpathlineto{\pgfqpoint{1.666795in}{1.517989in}}%
\pgfpathlineto{\pgfqpoint{1.681691in}{1.516306in}}%
\pgfpathlineto{\pgfqpoint{1.696586in}{1.512273in}}%
\pgfpathlineto{\pgfqpoint{1.713964in}{1.504916in}}%
\pgfpathlineto{\pgfqpoint{1.731343in}{1.495118in}}%
\pgfpathlineto{\pgfqpoint{1.753686in}{1.479701in}}%
\pgfpathlineto{\pgfqpoint{1.783477in}{1.455933in}}%
\pgfpathlineto{\pgfqpoint{1.852990in}{1.399164in}}%
\pgfpathlineto{\pgfqpoint{1.877816in}{1.382281in}}%
\pgfpathlineto{\pgfqpoint{1.900160in}{1.369804in}}%
\pgfpathlineto{\pgfqpoint{1.920021in}{1.361195in}}%
\pgfpathlineto{\pgfqpoint{1.939881in}{1.355075in}}%
\pgfpathlineto{\pgfqpoint{1.959742in}{1.351473in}}%
\pgfpathlineto{\pgfqpoint{1.979603in}{1.350319in}}%
\pgfpathlineto{\pgfqpoint{1.999464in}{1.351453in}}%
\pgfpathlineto{\pgfqpoint{2.021807in}{1.355164in}}%
\pgfpathlineto{\pgfqpoint{2.046633in}{1.361805in}}%
\pgfpathlineto{\pgfqpoint{2.076425in}{1.372332in}}%
\pgfpathlineto{\pgfqpoint{2.126077in}{1.392796in}}%
\pgfpathlineto{\pgfqpoint{2.173246in}{1.411291in}}%
\pgfpathlineto{\pgfqpoint{2.205520in}{1.421424in}}%
\pgfpathlineto{\pgfqpoint{2.232829in}{1.427717in}}%
\pgfpathlineto{\pgfqpoint{2.260137in}{1.431672in}}%
\pgfpathlineto{\pgfqpoint{2.287446in}{1.433262in}}%
\pgfpathlineto{\pgfqpoint{2.314754in}{1.432635in}}%
\pgfpathlineto{\pgfqpoint{2.344546in}{1.429772in}}%
\pgfpathlineto{\pgfqpoint{2.381785in}{1.423788in}}%
\pgfpathlineto{\pgfqpoint{2.443850in}{1.410994in}}%
\pgfpathlineto{\pgfqpoint{2.500950in}{1.400083in}}%
\pgfpathlineto{\pgfqpoint{2.540671in}{1.394697in}}%
\pgfpathlineto{\pgfqpoint{2.577910in}{1.391848in}}%
\pgfpathlineto{\pgfqpoint{2.615149in}{1.391186in}}%
\pgfpathlineto{\pgfqpoint{2.657354in}{1.392757in}}%
\pgfpathlineto{\pgfqpoint{2.709488in}{1.397085in}}%
\pgfpathlineto{\pgfqpoint{2.848514in}{1.409852in}}%
\pgfpathlineto{\pgfqpoint{2.900649in}{1.411762in}}%
\pgfpathlineto{\pgfqpoint{2.955266in}{1.411526in}}%
\pgfpathlineto{\pgfqpoint{3.024779in}{1.408818in}}%
\pgfpathlineto{\pgfqpoint{3.183666in}{1.401843in}}%
\pgfpathlineto{\pgfqpoint{3.225870in}{1.401309in}}%
\pgfpathlineto{\pgfqpoint{3.225870in}{1.401309in}}%
\pgfusepath{stroke}%
\end{pgfscope}%
\begin{pgfscope}%
\pgfpathrectangle{\pgfqpoint{0.619136in}{0.571603in}}{\pgfqpoint{2.730864in}{1.657828in}}%
\pgfusepath{clip}%
\pgfsetrectcap%
\pgfsetroundjoin%
\pgfsetlinewidth{1.505625pt}%
\definecolor{currentstroke}{rgb}{0.839216,0.152941,0.156863}%
\pgfsetstrokecolor{currentstroke}%
\pgfsetdash{}{0pt}%
\pgfpathmoveto{\pgfqpoint{0.743267in}{0.646959in}}%
\pgfpathlineto{\pgfqpoint{0.750714in}{0.720774in}}%
\pgfpathlineto{\pgfqpoint{0.773058in}{0.952212in}}%
\pgfpathlineto{\pgfqpoint{0.785471in}{1.059071in}}%
\pgfpathlineto{\pgfqpoint{0.797884in}{1.148053in}}%
\pgfpathlineto{\pgfqpoint{0.810297in}{1.220417in}}%
\pgfpathlineto{\pgfqpoint{0.822710in}{1.278121in}}%
\pgfpathlineto{\pgfqpoint{0.832640in}{1.315166in}}%
\pgfpathlineto{\pgfqpoint{0.842571in}{1.345268in}}%
\pgfpathlineto{\pgfqpoint{0.852501in}{1.369427in}}%
\pgfpathlineto{\pgfqpoint{0.862432in}{1.388555in}}%
\pgfpathlineto{\pgfqpoint{0.872362in}{1.403467in}}%
\pgfpathlineto{\pgfqpoint{0.882292in}{1.414877in}}%
\pgfpathlineto{\pgfqpoint{0.892223in}{1.423410in}}%
\pgfpathlineto{\pgfqpoint{0.904636in}{1.430835in}}%
\pgfpathlineto{\pgfqpoint{0.917049in}{1.435465in}}%
\pgfpathlineto{\pgfqpoint{0.931944in}{1.438313in}}%
\pgfpathlineto{\pgfqpoint{0.949323in}{1.439067in}}%
\pgfpathlineto{\pgfqpoint{0.971666in}{1.437600in}}%
\pgfpathlineto{\pgfqpoint{1.011388in}{1.432232in}}%
\pgfpathlineto{\pgfqpoint{1.078418in}{1.423306in}}%
\pgfpathlineto{\pgfqpoint{1.133035in}{1.418463in}}%
\pgfpathlineto{\pgfqpoint{1.200066in}{1.414886in}}%
\pgfpathlineto{\pgfqpoint{1.299370in}{1.412065in}}%
\pgfpathlineto{\pgfqpoint{1.473152in}{1.409665in}}%
\pgfpathlineto{\pgfqpoint{1.798373in}{1.407659in}}%
\pgfpathlineto{\pgfqpoint{2.461228in}{1.406153in}}%
\pgfpathlineto{\pgfqpoint{3.225870in}{1.405493in}}%
\pgfpathlineto{\pgfqpoint{3.225870in}{1.405493in}}%
\pgfusepath{stroke}%
\end{pgfscope}%
\begin{pgfscope}%
\pgfpathrectangle{\pgfqpoint{0.619136in}{0.571603in}}{\pgfqpoint{2.730864in}{1.657828in}}%
\pgfusepath{clip}%
\pgfsetrectcap%
\pgfsetroundjoin%
\pgfsetlinewidth{1.505625pt}%
\definecolor{currentstroke}{rgb}{0.580392,0.403922,0.741176}%
\pgfsetstrokecolor{currentstroke}%
\pgfsetdash{}{0pt}%
\pgfpathmoveto{\pgfqpoint{0.743267in}{0.646959in}}%
\pgfpathlineto{\pgfqpoint{0.773058in}{0.959334in}}%
\pgfpathlineto{\pgfqpoint{0.785471in}{1.057460in}}%
\pgfpathlineto{\pgfqpoint{0.797884in}{1.137266in}}%
\pgfpathlineto{\pgfqpoint{0.810297in}{1.201401in}}%
\pgfpathlineto{\pgfqpoint{0.822710in}{1.252447in}}%
\pgfpathlineto{\pgfqpoint{0.835123in}{1.292725in}}%
\pgfpathlineto{\pgfqpoint{0.847536in}{1.324244in}}%
\pgfpathlineto{\pgfqpoint{0.859949in}{1.348702in}}%
\pgfpathlineto{\pgfqpoint{0.872362in}{1.367512in}}%
\pgfpathlineto{\pgfqpoint{0.884775in}{1.381836in}}%
\pgfpathlineto{\pgfqpoint{0.897188in}{1.392623in}}%
\pgfpathlineto{\pgfqpoint{0.909601in}{1.400637in}}%
\pgfpathlineto{\pgfqpoint{0.924497in}{1.407452in}}%
\pgfpathlineto{\pgfqpoint{0.941875in}{1.412567in}}%
\pgfpathlineto{\pgfqpoint{0.961736in}{1.415842in}}%
\pgfpathlineto{\pgfqpoint{0.986562in}{1.417522in}}%
\pgfpathlineto{\pgfqpoint{1.023801in}{1.417448in}}%
\pgfpathlineto{\pgfqpoint{1.113174in}{1.414059in}}%
\pgfpathlineto{\pgfqpoint{1.229857in}{1.410723in}}%
\pgfpathlineto{\pgfqpoint{1.393709in}{1.408540in}}%
\pgfpathlineto{\pgfqpoint{1.713964in}{1.406857in}}%
\pgfpathlineto{\pgfqpoint{2.453780in}{1.405615in}}%
\pgfpathlineto{\pgfqpoint{3.225870in}{1.405160in}}%
\pgfpathlineto{\pgfqpoint{3.225870in}{1.405160in}}%
\pgfusepath{stroke}%
\end{pgfscope}%
\begin{pgfscope}%
\pgfpathrectangle{\pgfqpoint{0.619136in}{0.571603in}}{\pgfqpoint{2.730864in}{1.657828in}}%
\pgfusepath{clip}%
\pgfsetrectcap%
\pgfsetroundjoin%
\pgfsetlinewidth{1.505625pt}%
\definecolor{currentstroke}{rgb}{0.549020,0.337255,0.294118}%
\pgfsetstrokecolor{currentstroke}%
\pgfsetdash{}{0pt}%
\pgfpathmoveto{\pgfqpoint{0.743267in}{0.646959in}}%
\pgfpathlineto{\pgfqpoint{0.748232in}{0.648283in}}%
\pgfpathlineto{\pgfqpoint{0.753197in}{0.652217in}}%
\pgfpathlineto{\pgfqpoint{0.758162in}{0.658726in}}%
\pgfpathlineto{\pgfqpoint{0.765610in}{0.673248in}}%
\pgfpathlineto{\pgfqpoint{0.773058in}{0.693355in}}%
\pgfpathlineto{\pgfqpoint{0.782988in}{0.728554in}}%
\pgfpathlineto{\pgfqpoint{0.792919in}{0.772868in}}%
\pgfpathlineto{\pgfqpoint{0.805332in}{0.840113in}}%
\pgfpathlineto{\pgfqpoint{0.820227in}{0.936232in}}%
\pgfpathlineto{\pgfqpoint{0.837605in}{1.065831in}}%
\pgfpathlineto{\pgfqpoint{0.859949in}{1.251654in}}%
\pgfpathlineto{\pgfqpoint{0.919531in}{1.758740in}}%
\pgfpathlineto{\pgfqpoint{0.936910in}{1.884044in}}%
\pgfpathlineto{\pgfqpoint{0.951805in}{1.975635in}}%
\pgfpathlineto{\pgfqpoint{0.964218in}{2.038688in}}%
\pgfpathlineto{\pgfqpoint{0.974149in}{2.079486in}}%
\pgfpathlineto{\pgfqpoint{0.984079in}{2.111125in}}%
\pgfpathlineto{\pgfqpoint{0.991527in}{2.128590in}}%
\pgfpathlineto{\pgfqpoint{0.998975in}{2.140543in}}%
\pgfpathlineto{\pgfqpoint{1.003940in}{2.145406in}}%
\pgfpathlineto{\pgfqpoint{1.008905in}{2.147767in}}%
\pgfpathlineto{\pgfqpoint{1.013870in}{2.147620in}}%
\pgfpathlineto{\pgfqpoint{1.018836in}{2.144967in}}%
\pgfpathlineto{\pgfqpoint{1.023801in}{2.139820in}}%
\pgfpathlineto{\pgfqpoint{1.031249in}{2.127467in}}%
\pgfpathlineto{\pgfqpoint{1.038696in}{2.109648in}}%
\pgfpathlineto{\pgfqpoint{1.046144in}{2.086506in}}%
\pgfpathlineto{\pgfqpoint{1.056075in}{2.047689in}}%
\pgfpathlineto{\pgfqpoint{1.066005in}{2.000274in}}%
\pgfpathlineto{\pgfqpoint{1.078418in}{1.929922in}}%
\pgfpathlineto{\pgfqpoint{1.093314in}{1.831263in}}%
\pgfpathlineto{\pgfqpoint{1.110692in}{1.700400in}}%
\pgfpathlineto{\pgfqpoint{1.135518in}{1.494482in}}%
\pgfpathlineto{\pgfqpoint{1.182687in}{1.099705in}}%
\pgfpathlineto{\pgfqpoint{1.200066in}{0.972170in}}%
\pgfpathlineto{\pgfqpoint{1.214961in}{0.877143in}}%
\pgfpathlineto{\pgfqpoint{1.227374in}{0.810228in}}%
\pgfpathlineto{\pgfqpoint{1.239787in}{0.755959in}}%
\pgfpathlineto{\pgfqpoint{1.249718in}{0.722408in}}%
\pgfpathlineto{\pgfqpoint{1.257165in}{0.703296in}}%
\pgfpathlineto{\pgfqpoint{1.264613in}{0.689526in}}%
\pgfpathlineto{\pgfqpoint{1.272061in}{0.681198in}}%
\pgfpathlineto{\pgfqpoint{1.277026in}{0.678698in}}%
\pgfpathlineto{\pgfqpoint{1.281991in}{0.678648in}}%
\pgfpathlineto{\pgfqpoint{1.286957in}{0.681046in}}%
\pgfpathlineto{\pgfqpoint{1.291922in}{0.685883in}}%
\pgfpathlineto{\pgfqpoint{1.299370in}{0.697666in}}%
\pgfpathlineto{\pgfqpoint{1.306817in}{0.714795in}}%
\pgfpathlineto{\pgfqpoint{1.314265in}{0.737133in}}%
\pgfpathlineto{\pgfqpoint{1.324196in}{0.774710in}}%
\pgfpathlineto{\pgfqpoint{1.334126in}{0.820707in}}%
\pgfpathlineto{\pgfqpoint{1.346539in}{0.889068in}}%
\pgfpathlineto{\pgfqpoint{1.361435in}{0.985072in}}%
\pgfpathlineto{\pgfqpoint{1.378813in}{1.112573in}}%
\pgfpathlineto{\pgfqpoint{1.403639in}{1.313459in}}%
\pgfpathlineto{\pgfqpoint{1.450808in}{1.699309in}}%
\pgfpathlineto{\pgfqpoint{1.468187in}{1.824193in}}%
\pgfpathlineto{\pgfqpoint{1.483082in}{1.917361in}}%
\pgfpathlineto{\pgfqpoint{1.495495in}{1.983061in}}%
\pgfpathlineto{\pgfqpoint{1.507908in}{2.036444in}}%
\pgfpathlineto{\pgfqpoint{1.517839in}{2.069537in}}%
\pgfpathlineto{\pgfqpoint{1.525287in}{2.088455in}}%
\pgfpathlineto{\pgfqpoint{1.532734in}{2.102160in}}%
\pgfpathlineto{\pgfqpoint{1.540182in}{2.110554in}}%
\pgfpathlineto{\pgfqpoint{1.545147in}{2.113170in}}%
\pgfpathlineto{\pgfqpoint{1.550113in}{2.113393in}}%
\pgfpathlineto{\pgfqpoint{1.555078in}{2.111224in}}%
\pgfpathlineto{\pgfqpoint{1.560043in}{2.106672in}}%
\pgfpathlineto{\pgfqpoint{1.567491in}{2.095418in}}%
\pgfpathlineto{\pgfqpoint{1.574939in}{2.078936in}}%
\pgfpathlineto{\pgfqpoint{1.582386in}{2.057359in}}%
\pgfpathlineto{\pgfqpoint{1.592317in}{2.020963in}}%
\pgfpathlineto{\pgfqpoint{1.602247in}{1.976322in}}%
\pgfpathlineto{\pgfqpoint{1.614660in}{1.909872in}}%
\pgfpathlineto{\pgfqpoint{1.629556in}{1.816425in}}%
\pgfpathlineto{\pgfqpoint{1.646934in}{1.692170in}}%
\pgfpathlineto{\pgfqpoint{1.671760in}{1.496156in}}%
\pgfpathlineto{\pgfqpoint{1.718930in}{1.118973in}}%
\pgfpathlineto{\pgfqpoint{1.738790in}{0.980485in}}%
\pgfpathlineto{\pgfqpoint{1.753686in}{0.891504in}}%
\pgfpathlineto{\pgfqpoint{1.766099in}{0.829288in}}%
\pgfpathlineto{\pgfqpoint{1.776030in}{0.788269in}}%
\pgfpathlineto{\pgfqpoint{1.785960in}{0.755622in}}%
\pgfpathlineto{\pgfqpoint{1.793408in}{0.736892in}}%
\pgfpathlineto{\pgfqpoint{1.800856in}{0.723250in}}%
\pgfpathlineto{\pgfqpoint{1.808303in}{0.714791in}}%
\pgfpathlineto{\pgfqpoint{1.813269in}{0.712062in}}%
\pgfpathlineto{\pgfqpoint{1.818234in}{0.711670in}}%
\pgfpathlineto{\pgfqpoint{1.823199in}{0.713615in}}%
\pgfpathlineto{\pgfqpoint{1.828164in}{0.717888in}}%
\pgfpathlineto{\pgfqpoint{1.835612in}{0.728626in}}%
\pgfpathlineto{\pgfqpoint{1.843060in}{0.744474in}}%
\pgfpathlineto{\pgfqpoint{1.850508in}{0.765308in}}%
\pgfpathlineto{\pgfqpoint{1.860438in}{0.800550in}}%
\pgfpathlineto{\pgfqpoint{1.870368in}{0.843866in}}%
\pgfpathlineto{\pgfqpoint{1.882781in}{0.908448in}}%
\pgfpathlineto{\pgfqpoint{1.897677in}{0.999394in}}%
\pgfpathlineto{\pgfqpoint{1.915055in}{1.120471in}}%
\pgfpathlineto{\pgfqpoint{1.939881in}{1.311713in}}%
\pgfpathlineto{\pgfqpoint{1.987051in}{1.680395in}}%
\pgfpathlineto{\pgfqpoint{2.006912in}{1.816023in}}%
\pgfpathlineto{\pgfqpoint{2.021807in}{1.903287in}}%
\pgfpathlineto{\pgfqpoint{2.034220in}{1.964395in}}%
\pgfpathlineto{\pgfqpoint{2.046633in}{2.013588in}}%
\pgfpathlineto{\pgfqpoint{2.056564in}{2.043676in}}%
\pgfpathlineto{\pgfqpoint{2.064011in}{2.060571in}}%
\pgfpathlineto{\pgfqpoint{2.071459in}{2.072466in}}%
\pgfpathlineto{\pgfqpoint{2.076425in}{2.077576in}}%
\pgfpathlineto{\pgfqpoint{2.081390in}{2.080410in}}%
\pgfpathlineto{\pgfqpoint{2.086355in}{2.080961in}}%
\pgfpathlineto{\pgfqpoint{2.091320in}{2.079229in}}%
\pgfpathlineto{\pgfqpoint{2.096285in}{2.075222in}}%
\pgfpathlineto{\pgfqpoint{2.103733in}{2.064981in}}%
\pgfpathlineto{\pgfqpoint{2.111181in}{2.049743in}}%
\pgfpathlineto{\pgfqpoint{2.118629in}{2.029629in}}%
\pgfpathlineto{\pgfqpoint{2.128559in}{1.995504in}}%
\pgfpathlineto{\pgfqpoint{2.138490in}{1.953473in}}%
\pgfpathlineto{\pgfqpoint{2.150903in}{1.890707in}}%
\pgfpathlineto{\pgfqpoint{2.165798in}{1.802194in}}%
\pgfpathlineto{\pgfqpoint{2.183176in}{1.684211in}}%
\pgfpathlineto{\pgfqpoint{2.208002in}{1.497622in}}%
\pgfpathlineto{\pgfqpoint{2.257655in}{1.119676in}}%
\pgfpathlineto{\pgfqpoint{2.275033in}{1.004417in}}%
\pgfpathlineto{\pgfqpoint{2.289928in}{0.918837in}}%
\pgfpathlineto{\pgfqpoint{2.302341in}{0.858818in}}%
\pgfpathlineto{\pgfqpoint{2.314754in}{0.810405in}}%
\pgfpathlineto{\pgfqpoint{2.324685in}{0.780707in}}%
\pgfpathlineto{\pgfqpoint{2.332133in}{0.763965in}}%
\pgfpathlineto{\pgfqpoint{2.339580in}{0.752103in}}%
\pgfpathlineto{\pgfqpoint{2.347028in}{0.745202in}}%
\pgfpathlineto{\pgfqpoint{2.351993in}{0.743383in}}%
\pgfpathlineto{\pgfqpoint{2.356959in}{0.743795in}}%
\pgfpathlineto{\pgfqpoint{2.361924in}{0.746435in}}%
\pgfpathlineto{\pgfqpoint{2.366889in}{0.751291in}}%
\pgfpathlineto{\pgfqpoint{2.374337in}{0.762691in}}%
\pgfpathlineto{\pgfqpoint{2.381785in}{0.778942in}}%
\pgfpathlineto{\pgfqpoint{2.391715in}{0.807928in}}%
\pgfpathlineto{\pgfqpoint{2.401646in}{0.844901in}}%
\pgfpathlineto{\pgfqpoint{2.414059in}{0.901558in}}%
\pgfpathlineto{\pgfqpoint{2.426472in}{0.968669in}}%
\pgfpathlineto{\pgfqpoint{2.441367in}{1.060967in}}%
\pgfpathlineto{\pgfqpoint{2.461228in}{1.199288in}}%
\pgfpathlineto{\pgfqpoint{2.495984in}{1.461854in}}%
\pgfpathlineto{\pgfqpoint{2.525776in}{1.679810in}}%
\pgfpathlineto{\pgfqpoint{2.543154in}{1.792695in}}%
\pgfpathlineto{\pgfqpoint{2.558050in}{1.876619in}}%
\pgfpathlineto{\pgfqpoint{2.570463in}{1.935564in}}%
\pgfpathlineto{\pgfqpoint{2.582876in}{1.983204in}}%
\pgfpathlineto{\pgfqpoint{2.592806in}{2.012512in}}%
\pgfpathlineto{\pgfqpoint{2.600254in}{2.029098in}}%
\pgfpathlineto{\pgfqpoint{2.607702in}{2.040922in}}%
\pgfpathlineto{\pgfqpoint{2.615149in}{2.047903in}}%
\pgfpathlineto{\pgfqpoint{2.620115in}{2.049840in}}%
\pgfpathlineto{\pgfqpoint{2.625080in}{2.049599in}}%
\pgfpathlineto{\pgfqpoint{2.630045in}{2.047181in}}%
\pgfpathlineto{\pgfqpoint{2.635010in}{2.042596in}}%
\pgfpathlineto{\pgfqpoint{2.642458in}{2.031697in}}%
\pgfpathlineto{\pgfqpoint{2.649906in}{2.016054in}}%
\pgfpathlineto{\pgfqpoint{2.657354in}{1.995790in}}%
\pgfpathlineto{\pgfqpoint{2.667284in}{1.961866in}}%
\pgfpathlineto{\pgfqpoint{2.677215in}{1.920488in}}%
\pgfpathlineto{\pgfqpoint{2.689628in}{1.859160in}}%
\pgfpathlineto{\pgfqpoint{2.704523in}{1.773240in}}%
\pgfpathlineto{\pgfqpoint{2.721901in}{1.659374in}}%
\pgfpathlineto{\pgfqpoint{2.746727in}{1.480364in}}%
\pgfpathlineto{\pgfqpoint{2.793897in}{1.137648in}}%
\pgfpathlineto{\pgfqpoint{2.811275in}{1.027089in}}%
\pgfpathlineto{\pgfqpoint{2.826171in}{0.944790in}}%
\pgfpathlineto{\pgfqpoint{2.838584in}{0.886902in}}%
\pgfpathlineto{\pgfqpoint{2.850997in}{0.840025in}}%
\pgfpathlineto{\pgfqpoint{2.860927in}{0.811105in}}%
\pgfpathlineto{\pgfqpoint{2.868375in}{0.794677in}}%
\pgfpathlineto{\pgfqpoint{2.875823in}{0.782895in}}%
\pgfpathlineto{\pgfqpoint{2.883271in}{0.775842in}}%
\pgfpathlineto{\pgfqpoint{2.888236in}{0.773791in}}%
\pgfpathlineto{\pgfqpoint{2.893201in}{0.773870in}}%
\pgfpathlineto{\pgfqpoint{2.898166in}{0.776075in}}%
\pgfpathlineto{\pgfqpoint{2.903131in}{0.780397in}}%
\pgfpathlineto{\pgfqpoint{2.910579in}{0.790812in}}%
\pgfpathlineto{\pgfqpoint{2.918027in}{0.805866in}}%
\pgfpathlineto{\pgfqpoint{2.925475in}{0.825441in}}%
\pgfpathlineto{\pgfqpoint{2.935405in}{0.858299in}}%
\pgfpathlineto{\pgfqpoint{2.945336in}{0.898458in}}%
\pgfpathlineto{\pgfqpoint{2.957749in}{0.958071in}}%
\pgfpathlineto{\pgfqpoint{2.972644in}{1.041701in}}%
\pgfpathlineto{\pgfqpoint{2.990023in}{1.152665in}}%
\pgfpathlineto{\pgfqpoint{3.014849in}{1.327330in}}%
\pgfpathlineto{\pgfqpoint{3.062018in}{1.662344in}}%
\pgfpathlineto{\pgfqpoint{3.079396in}{1.770621in}}%
\pgfpathlineto{\pgfqpoint{3.094292in}{1.851322in}}%
\pgfpathlineto{\pgfqpoint{3.106705in}{1.908169in}}%
\pgfpathlineto{\pgfqpoint{3.119118in}{1.954292in}}%
\pgfpathlineto{\pgfqpoint{3.129048in}{1.982825in}}%
\pgfpathlineto{\pgfqpoint{3.136496in}{1.999093in}}%
\pgfpathlineto{\pgfqpoint{3.143944in}{2.010828in}}%
\pgfpathlineto{\pgfqpoint{3.151392in}{2.017947in}}%
\pgfpathlineto{\pgfqpoint{3.156357in}{2.020103in}}%
\pgfpathlineto{\pgfqpoint{3.161322in}{2.020180in}}%
\pgfpathlineto{\pgfqpoint{3.166287in}{2.018180in}}%
\pgfpathlineto{\pgfqpoint{3.171253in}{2.014110in}}%
\pgfpathlineto{\pgfqpoint{3.178700in}{2.004163in}}%
\pgfpathlineto{\pgfqpoint{3.186148in}{1.989678in}}%
\pgfpathlineto{\pgfqpoint{3.193596in}{1.970771in}}%
\pgfpathlineto{\pgfqpoint{3.203526in}{1.938948in}}%
\pgfpathlineto{\pgfqpoint{3.213457in}{1.899974in}}%
\pgfpathlineto{\pgfqpoint{3.225870in}{1.842031in}}%
\pgfpathlineto{\pgfqpoint{3.225870in}{1.842031in}}%
\pgfusepath{stroke}%
\end{pgfscope}%
\begin{pgfscope}%
\pgfpathrectangle{\pgfqpoint{0.619136in}{0.571603in}}{\pgfqpoint{2.730864in}{1.657828in}}%
\pgfusepath{clip}%
\pgfsetrectcap%
\pgfsetroundjoin%
\pgfsetlinewidth{1.505625pt}%
\definecolor{currentstroke}{rgb}{0.890196,0.466667,0.760784}%
\pgfsetstrokecolor{currentstroke}%
\pgfsetdash{}{0pt}%
\pgfpathmoveto{\pgfqpoint{0.743267in}{0.646959in}}%
\pgfpathlineto{\pgfqpoint{0.745749in}{0.649688in}}%
\pgfpathlineto{\pgfqpoint{0.750714in}{0.661723in}}%
\pgfpathlineto{\pgfqpoint{0.758162in}{0.689463in}}%
\pgfpathlineto{\pgfqpoint{0.768093in}{0.738557in}}%
\pgfpathlineto{\pgfqpoint{0.780506in}{0.812882in}}%
\pgfpathlineto{\pgfqpoint{0.800366in}{0.948462in}}%
\pgfpathlineto{\pgfqpoint{0.845053in}{1.259410in}}%
\pgfpathlineto{\pgfqpoint{0.862432in}{1.364574in}}%
\pgfpathlineto{\pgfqpoint{0.877327in}{1.443154in}}%
\pgfpathlineto{\pgfqpoint{0.892223in}{1.509783in}}%
\pgfpathlineto{\pgfqpoint{0.904636in}{1.555769in}}%
\pgfpathlineto{\pgfqpoint{0.917049in}{1.593047in}}%
\pgfpathlineto{\pgfqpoint{0.926979in}{1.616728in}}%
\pgfpathlineto{\pgfqpoint{0.936910in}{1.635150in}}%
\pgfpathlineto{\pgfqpoint{0.946840in}{1.648567in}}%
\pgfpathlineto{\pgfqpoint{0.954288in}{1.655529in}}%
\pgfpathlineto{\pgfqpoint{0.961736in}{1.659992in}}%
\pgfpathlineto{\pgfqpoint{0.969183in}{1.662113in}}%
\pgfpathlineto{\pgfqpoint{0.976631in}{1.662056in}}%
\pgfpathlineto{\pgfqpoint{0.984079in}{1.659991in}}%
\pgfpathlineto{\pgfqpoint{0.994009in}{1.654418in}}%
\pgfpathlineto{\pgfqpoint{1.003940in}{1.646004in}}%
\pgfpathlineto{\pgfqpoint{1.016353in}{1.632132in}}%
\pgfpathlineto{\pgfqpoint{1.031249in}{1.611616in}}%
\pgfpathlineto{\pgfqpoint{1.051109in}{1.579840in}}%
\pgfpathlineto{\pgfqpoint{1.123105in}{1.459964in}}%
\pgfpathlineto{\pgfqpoint{1.142966in}{1.433263in}}%
\pgfpathlineto{\pgfqpoint{1.160344in}{1.413655in}}%
\pgfpathlineto{\pgfqpoint{1.175240in}{1.399813in}}%
\pgfpathlineto{\pgfqpoint{1.190135in}{1.388720in}}%
\pgfpathlineto{\pgfqpoint{1.205031in}{1.380293in}}%
\pgfpathlineto{\pgfqpoint{1.219926in}{1.374375in}}%
\pgfpathlineto{\pgfqpoint{1.234822in}{1.370750in}}%
\pgfpathlineto{\pgfqpoint{1.252200in}{1.369074in}}%
\pgfpathlineto{\pgfqpoint{1.269578in}{1.369719in}}%
\pgfpathlineto{\pgfqpoint{1.289439in}{1.372690in}}%
\pgfpathlineto{\pgfqpoint{1.314265in}{1.378746in}}%
\pgfpathlineto{\pgfqpoint{1.358952in}{1.392567in}}%
\pgfpathlineto{\pgfqpoint{1.403639in}{1.405591in}}%
\pgfpathlineto{\pgfqpoint{1.435913in}{1.412657in}}%
\pgfpathlineto{\pgfqpoint{1.465704in}{1.416971in}}%
\pgfpathlineto{\pgfqpoint{1.497978in}{1.419307in}}%
\pgfpathlineto{\pgfqpoint{1.532734in}{1.419542in}}%
\pgfpathlineto{\pgfqpoint{1.577421in}{1.417472in}}%
\pgfpathlineto{\pgfqpoint{1.753686in}{1.406799in}}%
\pgfpathlineto{\pgfqpoint{1.828164in}{1.406089in}}%
\pgfpathlineto{\pgfqpoint{2.324685in}{1.406252in}}%
\pgfpathlineto{\pgfqpoint{3.201044in}{1.405325in}}%
\pgfpathlineto{\pgfqpoint{3.225870in}{1.405309in}}%
\pgfpathlineto{\pgfqpoint{3.225870in}{1.405309in}}%
\pgfusepath{stroke}%
\end{pgfscope}%
\begin{pgfscope}%
\pgfpathrectangle{\pgfqpoint{0.619136in}{0.571603in}}{\pgfqpoint{2.730864in}{1.657828in}}%
\pgfusepath{clip}%
\pgfsetrectcap%
\pgfsetroundjoin%
\pgfsetlinewidth{1.505625pt}%
\definecolor{currentstroke}{rgb}{0.498039,0.498039,0.498039}%
\pgfsetstrokecolor{currentstroke}%
\pgfsetdash{}{0pt}%
\pgfpathmoveto{\pgfqpoint{0.743267in}{0.646959in}}%
\pgfpathlineto{\pgfqpoint{0.748232in}{0.675611in}}%
\pgfpathlineto{\pgfqpoint{0.758162in}{0.756903in}}%
\pgfpathlineto{\pgfqpoint{0.792919in}{1.057623in}}%
\pgfpathlineto{\pgfqpoint{0.807814in}{1.164992in}}%
\pgfpathlineto{\pgfqpoint{0.820227in}{1.240796in}}%
\pgfpathlineto{\pgfqpoint{0.832640in}{1.304212in}}%
\pgfpathlineto{\pgfqpoint{0.845053in}{1.355934in}}%
\pgfpathlineto{\pgfqpoint{0.857466in}{1.397007in}}%
\pgfpathlineto{\pgfqpoint{0.867397in}{1.423015in}}%
\pgfpathlineto{\pgfqpoint{0.877327in}{1.443643in}}%
\pgfpathlineto{\pgfqpoint{0.887258in}{1.459544in}}%
\pgfpathlineto{\pgfqpoint{0.897188in}{1.471349in}}%
\pgfpathlineto{\pgfqpoint{0.907118in}{1.479653in}}%
\pgfpathlineto{\pgfqpoint{0.917049in}{1.485008in}}%
\pgfpathlineto{\pgfqpoint{0.926979in}{1.487917in}}%
\pgfpathlineto{\pgfqpoint{0.939392in}{1.488797in}}%
\pgfpathlineto{\pgfqpoint{0.951805in}{1.487327in}}%
\pgfpathlineto{\pgfqpoint{0.966701in}{1.483369in}}%
\pgfpathlineto{\pgfqpoint{0.989044in}{1.474771in}}%
\pgfpathlineto{\pgfqpoint{1.066005in}{1.442791in}}%
\pgfpathlineto{\pgfqpoint{1.095796in}{1.433859in}}%
\pgfpathlineto{\pgfqpoint{1.125587in}{1.427261in}}%
\pgfpathlineto{\pgfqpoint{1.160344in}{1.422005in}}%
\pgfpathlineto{\pgfqpoint{1.202548in}{1.418135in}}%
\pgfpathlineto{\pgfqpoint{1.259648in}{1.415404in}}%
\pgfpathlineto{\pgfqpoint{1.363917in}{1.413113in}}%
\pgfpathlineto{\pgfqpoint{1.634521in}{1.409842in}}%
\pgfpathlineto{\pgfqpoint{2.011877in}{1.407752in}}%
\pgfpathlineto{\pgfqpoint{2.756658in}{1.406183in}}%
\pgfpathlineto{\pgfqpoint{3.225870in}{1.405733in}}%
\pgfpathlineto{\pgfqpoint{3.225870in}{1.405733in}}%
\pgfusepath{stroke}%
\end{pgfscope}%
\begin{pgfscope}%
\pgfpathrectangle{\pgfqpoint{0.619136in}{0.571603in}}{\pgfqpoint{2.730864in}{1.657828in}}%
\pgfusepath{clip}%
\pgfsetrectcap%
\pgfsetroundjoin%
\pgfsetlinewidth{1.505625pt}%
\definecolor{currentstroke}{rgb}{0.737255,0.741176,0.133333}%
\pgfsetstrokecolor{currentstroke}%
\pgfsetdash{}{0pt}%
\pgfpathmoveto{\pgfqpoint{0.743267in}{0.646959in}}%
\pgfpathlineto{\pgfqpoint{0.745749in}{0.649898in}}%
\pgfpathlineto{\pgfqpoint{0.750714in}{0.663770in}}%
\pgfpathlineto{\pgfqpoint{0.758162in}{0.697033in}}%
\pgfpathlineto{\pgfqpoint{0.768093in}{0.757370in}}%
\pgfpathlineto{\pgfqpoint{0.780506in}{0.849925in}}%
\pgfpathlineto{\pgfqpoint{0.800366in}{1.018767in}}%
\pgfpathlineto{\pgfqpoint{0.832640in}{1.294550in}}%
\pgfpathlineto{\pgfqpoint{0.850018in}{1.423411in}}%
\pgfpathlineto{\pgfqpoint{0.864914in}{1.516317in}}%
\pgfpathlineto{\pgfqpoint{0.877327in}{1.579669in}}%
\pgfpathlineto{\pgfqpoint{0.887258in}{1.620725in}}%
\pgfpathlineto{\pgfqpoint{0.897188in}{1.653205in}}%
\pgfpathlineto{\pgfqpoint{0.907118in}{1.677301in}}%
\pgfpathlineto{\pgfqpoint{0.914566in}{1.690082in}}%
\pgfpathlineto{\pgfqpoint{0.922014in}{1.698555in}}%
\pgfpathlineto{\pgfqpoint{0.929462in}{1.702963in}}%
\pgfpathlineto{\pgfqpoint{0.934427in}{1.703779in}}%
\pgfpathlineto{\pgfqpoint{0.939392in}{1.702999in}}%
\pgfpathlineto{\pgfqpoint{0.946840in}{1.699039in}}%
\pgfpathlineto{\pgfqpoint{0.954288in}{1.692022in}}%
\pgfpathlineto{\pgfqpoint{0.964218in}{1.678494in}}%
\pgfpathlineto{\pgfqpoint{0.974149in}{1.660944in}}%
\pgfpathlineto{\pgfqpoint{0.986562in}{1.634585in}}%
\pgfpathlineto{\pgfqpoint{1.003940in}{1.592170in}}%
\pgfpathlineto{\pgfqpoint{1.058557in}{1.454567in}}%
\pgfpathlineto{\pgfqpoint{1.075935in}{1.418819in}}%
\pgfpathlineto{\pgfqpoint{1.090831in}{1.393400in}}%
\pgfpathlineto{\pgfqpoint{1.103244in}{1.376212in}}%
\pgfpathlineto{\pgfqpoint{1.115657in}{1.362723in}}%
\pgfpathlineto{\pgfqpoint{1.128070in}{1.352855in}}%
\pgfpathlineto{\pgfqpoint{1.138000in}{1.347448in}}%
\pgfpathlineto{\pgfqpoint{1.147931in}{1.344104in}}%
\pgfpathlineto{\pgfqpoint{1.160344in}{1.342569in}}%
\pgfpathlineto{\pgfqpoint{1.172757in}{1.343624in}}%
\pgfpathlineto{\pgfqpoint{1.185170in}{1.346862in}}%
\pgfpathlineto{\pgfqpoint{1.200066in}{1.353031in}}%
\pgfpathlineto{\pgfqpoint{1.219926in}{1.363941in}}%
\pgfpathlineto{\pgfqpoint{1.299370in}{1.410962in}}%
\pgfpathlineto{\pgfqpoint{1.319231in}{1.418803in}}%
\pgfpathlineto{\pgfqpoint{1.339091in}{1.424237in}}%
\pgfpathlineto{\pgfqpoint{1.358952in}{1.427307in}}%
\pgfpathlineto{\pgfqpoint{1.378813in}{1.428244in}}%
\pgfpathlineto{\pgfqpoint{1.401156in}{1.427198in}}%
\pgfpathlineto{\pgfqpoint{1.430948in}{1.423374in}}%
\pgfpathlineto{\pgfqpoint{1.552595in}{1.404761in}}%
\pgfpathlineto{\pgfqpoint{1.589834in}{1.402860in}}%
\pgfpathlineto{\pgfqpoint{1.634521in}{1.402971in}}%
\pgfpathlineto{\pgfqpoint{1.721412in}{1.406188in}}%
\pgfpathlineto{\pgfqpoint{1.798373in}{1.407846in}}%
\pgfpathlineto{\pgfqpoint{1.887747in}{1.407256in}}%
\pgfpathlineto{\pgfqpoint{2.076425in}{1.405639in}}%
\pgfpathlineto{\pgfqpoint{3.225870in}{1.404909in}}%
\pgfpathlineto{\pgfqpoint{3.225870in}{1.404909in}}%
\pgfusepath{stroke}%
\end{pgfscope}%
\begin{pgfscope}%
\pgfpathrectangle{\pgfqpoint{0.619136in}{0.571603in}}{\pgfqpoint{2.730864in}{1.657828in}}%
\pgfusepath{clip}%
\pgfsetrectcap%
\pgfsetroundjoin%
\pgfsetlinewidth{1.505625pt}%
\definecolor{currentstroke}{rgb}{0.090196,0.745098,0.811765}%
\pgfsetstrokecolor{currentstroke}%
\pgfsetdash{}{0pt}%
\pgfpathmoveto{\pgfqpoint{0.743267in}{0.646959in}}%
\pgfpathlineto{\pgfqpoint{0.745749in}{0.654616in}}%
\pgfpathlineto{\pgfqpoint{0.750714in}{0.681048in}}%
\pgfpathlineto{\pgfqpoint{0.760645in}{0.752179in}}%
\pgfpathlineto{\pgfqpoint{0.780506in}{0.920105in}}%
\pgfpathlineto{\pgfqpoint{0.805332in}{1.125484in}}%
\pgfpathlineto{\pgfqpoint{0.820227in}{1.232313in}}%
\pgfpathlineto{\pgfqpoint{0.835123in}{1.322831in}}%
\pgfpathlineto{\pgfqpoint{0.847536in}{1.385151in}}%
\pgfpathlineto{\pgfqpoint{0.859949in}{1.435765in}}%
\pgfpathlineto{\pgfqpoint{0.869879in}{1.468266in}}%
\pgfpathlineto{\pgfqpoint{0.879810in}{1.494180in}}%
\pgfpathlineto{\pgfqpoint{0.889740in}{1.514066in}}%
\pgfpathlineto{\pgfqpoint{0.899671in}{1.528535in}}%
\pgfpathlineto{\pgfqpoint{0.907118in}{1.536210in}}%
\pgfpathlineto{\pgfqpoint{0.914566in}{1.541458in}}%
\pgfpathlineto{\pgfqpoint{0.922014in}{1.544544in}}%
\pgfpathlineto{\pgfqpoint{0.929462in}{1.545721in}}%
\pgfpathlineto{\pgfqpoint{0.939392in}{1.544746in}}%
\pgfpathlineto{\pgfqpoint{0.949323in}{1.541363in}}%
\pgfpathlineto{\pgfqpoint{0.961736in}{1.534509in}}%
\pgfpathlineto{\pgfqpoint{0.976631in}{1.523577in}}%
\pgfpathlineto{\pgfqpoint{1.001457in}{1.501981in}}%
\pgfpathlineto{\pgfqpoint{1.041179in}{1.467464in}}%
\pgfpathlineto{\pgfqpoint{1.063522in}{1.451129in}}%
\pgfpathlineto{\pgfqpoint{1.083383in}{1.439219in}}%
\pgfpathlineto{\pgfqpoint{1.103244in}{1.429823in}}%
\pgfpathlineto{\pgfqpoint{1.123105in}{1.422771in}}%
\pgfpathlineto{\pgfqpoint{1.145448in}{1.417265in}}%
\pgfpathlineto{\pgfqpoint{1.170274in}{1.413558in}}%
\pgfpathlineto{\pgfqpoint{1.200066in}{1.411492in}}%
\pgfpathlineto{\pgfqpoint{1.242270in}{1.411090in}}%
\pgfpathlineto{\pgfqpoint{1.438395in}{1.412035in}}%
\pgfpathlineto{\pgfqpoint{1.798373in}{1.408368in}}%
\pgfpathlineto{\pgfqpoint{2.354476in}{1.406552in}}%
\pgfpathlineto{\pgfqpoint{3.225870in}{1.405530in}}%
\pgfpathlineto{\pgfqpoint{3.225870in}{1.405530in}}%
\pgfusepath{stroke}%
\end{pgfscope}%
\begin{pgfscope}%
\pgfpathrectangle{\pgfqpoint{0.619136in}{0.571603in}}{\pgfqpoint{2.730864in}{1.657828in}}%
\pgfusepath{clip}%
\pgfsetrectcap%
\pgfsetroundjoin%
\pgfsetlinewidth{1.505625pt}%
\definecolor{currentstroke}{rgb}{0.121569,0.466667,0.705882}%
\pgfsetstrokecolor{currentstroke}%
\pgfsetdash{}{0pt}%
\pgfpathmoveto{\pgfqpoint{0.743267in}{0.646959in}}%
\pgfpathlineto{\pgfqpoint{0.745749in}{0.648208in}}%
\pgfpathlineto{\pgfqpoint{0.750714in}{0.655290in}}%
\pgfpathlineto{\pgfqpoint{0.755680in}{0.667041in}}%
\pgfpathlineto{\pgfqpoint{0.763127in}{0.691887in}}%
\pgfpathlineto{\pgfqpoint{0.773058in}{0.736291in}}%
\pgfpathlineto{\pgfqpoint{0.782988in}{0.791280in}}%
\pgfpathlineto{\pgfqpoint{0.795401in}{0.871699in}}%
\pgfpathlineto{\pgfqpoint{0.812779in}{0.999537in}}%
\pgfpathlineto{\pgfqpoint{0.879810in}{1.512264in}}%
\pgfpathlineto{\pgfqpoint{0.894705in}{1.604558in}}%
\pgfpathlineto{\pgfqpoint{0.909601in}{1.682584in}}%
\pgfpathlineto{\pgfqpoint{0.922014in}{1.735535in}}%
\pgfpathlineto{\pgfqpoint{0.931944in}{1.769584in}}%
\pgfpathlineto{\pgfqpoint{0.941875in}{1.796104in}}%
\pgfpathlineto{\pgfqpoint{0.949323in}{1.811055in}}%
\pgfpathlineto{\pgfqpoint{0.956770in}{1.821824in}}%
\pgfpathlineto{\pgfqpoint{0.964218in}{1.828496in}}%
\pgfpathlineto{\pgfqpoint{0.969183in}{1.830726in}}%
\pgfpathlineto{\pgfqpoint{0.974149in}{1.831231in}}%
\pgfpathlineto{\pgfqpoint{0.979114in}{1.830059in}}%
\pgfpathlineto{\pgfqpoint{0.986562in}{1.825279in}}%
\pgfpathlineto{\pgfqpoint{0.994009in}{1.817054in}}%
\pgfpathlineto{\pgfqpoint{1.001457in}{1.805611in}}%
\pgfpathlineto{\pgfqpoint{1.011388in}{1.785772in}}%
\pgfpathlineto{\pgfqpoint{1.023801in}{1.754498in}}%
\pgfpathlineto{\pgfqpoint{1.038696in}{1.709193in}}%
\pgfpathlineto{\pgfqpoint{1.056075in}{1.648569in}}%
\pgfpathlineto{\pgfqpoint{1.090831in}{1.516434in}}%
\pgfpathlineto{\pgfqpoint{1.118140in}{1.416819in}}%
\pgfpathlineto{\pgfqpoint{1.135518in}{1.360844in}}%
\pgfpathlineto{\pgfqpoint{1.150413in}{1.319359in}}%
\pgfpathlineto{\pgfqpoint{1.162827in}{1.290078in}}%
\pgfpathlineto{\pgfqpoint{1.175240in}{1.265991in}}%
\pgfpathlineto{\pgfqpoint{1.185170in}{1.250598in}}%
\pgfpathlineto{\pgfqpoint{1.195100in}{1.238690in}}%
\pgfpathlineto{\pgfqpoint{1.205031in}{1.230243in}}%
\pgfpathlineto{\pgfqpoint{1.214961in}{1.225178in}}%
\pgfpathlineto{\pgfqpoint{1.222409in}{1.223524in}}%
\pgfpathlineto{\pgfqpoint{1.229857in}{1.223633in}}%
\pgfpathlineto{\pgfqpoint{1.239787in}{1.226373in}}%
\pgfpathlineto{\pgfqpoint{1.249718in}{1.231871in}}%
\pgfpathlineto{\pgfqpoint{1.259648in}{1.239868in}}%
\pgfpathlineto{\pgfqpoint{1.272061in}{1.252949in}}%
\pgfpathlineto{\pgfqpoint{1.286957in}{1.272392in}}%
\pgfpathlineto{\pgfqpoint{1.304335in}{1.298891in}}%
\pgfpathlineto{\pgfqpoint{1.334126in}{1.349186in}}%
\pgfpathlineto{\pgfqpoint{1.366400in}{1.402691in}}%
\pgfpathlineto{\pgfqpoint{1.386261in}{1.431563in}}%
\pgfpathlineto{\pgfqpoint{1.403639in}{1.452891in}}%
\pgfpathlineto{\pgfqpoint{1.418535in}{1.467726in}}%
\pgfpathlineto{\pgfqpoint{1.433430in}{1.479120in}}%
\pgfpathlineto{\pgfqpoint{1.445843in}{1.485911in}}%
\pgfpathlineto{\pgfqpoint{1.458256in}{1.490256in}}%
\pgfpathlineto{\pgfqpoint{1.470669in}{1.492229in}}%
\pgfpathlineto{\pgfqpoint{1.483082in}{1.491958in}}%
\pgfpathlineto{\pgfqpoint{1.495495in}{1.489616in}}%
\pgfpathlineto{\pgfqpoint{1.510391in}{1.484371in}}%
\pgfpathlineto{\pgfqpoint{1.527769in}{1.475433in}}%
\pgfpathlineto{\pgfqpoint{1.547630in}{1.462410in}}%
\pgfpathlineto{\pgfqpoint{1.577421in}{1.439687in}}%
\pgfpathlineto{\pgfqpoint{1.622108in}{1.405653in}}%
\pgfpathlineto{\pgfqpoint{1.644452in}{1.391413in}}%
\pgfpathlineto{\pgfqpoint{1.664312in}{1.381278in}}%
\pgfpathlineto{\pgfqpoint{1.681691in}{1.374645in}}%
\pgfpathlineto{\pgfqpoint{1.699069in}{1.370204in}}%
\pgfpathlineto{\pgfqpoint{1.716447in}{1.367936in}}%
\pgfpathlineto{\pgfqpoint{1.733825in}{1.367721in}}%
\pgfpathlineto{\pgfqpoint{1.753686in}{1.369723in}}%
\pgfpathlineto{\pgfqpoint{1.776030in}{1.374345in}}%
\pgfpathlineto{\pgfqpoint{1.803338in}{1.382417in}}%
\pgfpathlineto{\pgfqpoint{1.905125in}{1.415066in}}%
\pgfpathlineto{\pgfqpoint{1.932434in}{1.420133in}}%
\pgfpathlineto{\pgfqpoint{1.959742in}{1.422761in}}%
\pgfpathlineto{\pgfqpoint{1.987051in}{1.423032in}}%
\pgfpathlineto{\pgfqpoint{2.016842in}{1.421039in}}%
\pgfpathlineto{\pgfqpoint{2.054081in}{1.416203in}}%
\pgfpathlineto{\pgfqpoint{2.165798in}{1.400121in}}%
\pgfpathlineto{\pgfqpoint{2.205520in}{1.397639in}}%
\pgfpathlineto{\pgfqpoint{2.245242in}{1.397441in}}%
\pgfpathlineto{\pgfqpoint{2.292411in}{1.399502in}}%
\pgfpathlineto{\pgfqpoint{2.453780in}{1.408492in}}%
\pgfpathlineto{\pgfqpoint{2.513363in}{1.408308in}}%
\pgfpathlineto{\pgfqpoint{2.612667in}{1.405279in}}%
\pgfpathlineto{\pgfqpoint{2.707006in}{1.403310in}}%
\pgfpathlineto{\pgfqpoint{2.801345in}{1.403770in}}%
\pgfpathlineto{\pgfqpoint{3.009883in}{1.405460in}}%
\pgfpathlineto{\pgfqpoint{3.225870in}{1.404351in}}%
\pgfpathlineto{\pgfqpoint{3.225870in}{1.404351in}}%
\pgfusepath{stroke}%
\end{pgfscope}%
\begin{pgfscope}%
\pgfpathrectangle{\pgfqpoint{0.619136in}{0.571603in}}{\pgfqpoint{2.730864in}{1.657828in}}%
\pgfusepath{clip}%
\pgfsetrectcap%
\pgfsetroundjoin%
\pgfsetlinewidth{1.505625pt}%
\definecolor{currentstroke}{rgb}{1.000000,0.498039,0.054902}%
\pgfsetstrokecolor{currentstroke}%
\pgfsetdash{}{0pt}%
\pgfpathmoveto{\pgfqpoint{0.743267in}{0.646959in}}%
\pgfpathlineto{\pgfqpoint{0.745749in}{0.649496in}}%
\pgfpathlineto{\pgfqpoint{0.750714in}{0.660671in}}%
\pgfpathlineto{\pgfqpoint{0.758162in}{0.686433in}}%
\pgfpathlineto{\pgfqpoint{0.768093in}{0.732097in}}%
\pgfpathlineto{\pgfqpoint{0.780506in}{0.801454in}}%
\pgfpathlineto{\pgfqpoint{0.797884in}{0.912236in}}%
\pgfpathlineto{\pgfqpoint{0.854984in}{1.287421in}}%
\pgfpathlineto{\pgfqpoint{0.872362in}{1.384271in}}%
\pgfpathlineto{\pgfqpoint{0.887258in}{1.456470in}}%
\pgfpathlineto{\pgfqpoint{0.902153in}{1.517731in}}%
\pgfpathlineto{\pgfqpoint{0.914566in}{1.560142in}}%
\pgfpathlineto{\pgfqpoint{0.926979in}{1.594706in}}%
\pgfpathlineto{\pgfqpoint{0.936910in}{1.616829in}}%
\pgfpathlineto{\pgfqpoint{0.946840in}{1.634219in}}%
\pgfpathlineto{\pgfqpoint{0.956770in}{1.647094in}}%
\pgfpathlineto{\pgfqpoint{0.966701in}{1.655718in}}%
\pgfpathlineto{\pgfqpoint{0.974149in}{1.659573in}}%
\pgfpathlineto{\pgfqpoint{0.981596in}{1.661339in}}%
\pgfpathlineto{\pgfqpoint{0.989044in}{1.661156in}}%
\pgfpathlineto{\pgfqpoint{0.996492in}{1.659173in}}%
\pgfpathlineto{\pgfqpoint{1.006423in}{1.653986in}}%
\pgfpathlineto{\pgfqpoint{1.016353in}{1.646222in}}%
\pgfpathlineto{\pgfqpoint{1.028766in}{1.633441in}}%
\pgfpathlineto{\pgfqpoint{1.043662in}{1.614498in}}%
\pgfpathlineto{\pgfqpoint{1.063522in}{1.584973in}}%
\pgfpathlineto{\pgfqpoint{1.103244in}{1.520144in}}%
\pgfpathlineto{\pgfqpoint{1.133035in}{1.473893in}}%
\pgfpathlineto{\pgfqpoint{1.152896in}{1.446708in}}%
\pgfpathlineto{\pgfqpoint{1.172757in}{1.423389in}}%
\pgfpathlineto{\pgfqpoint{1.190135in}{1.406489in}}%
\pgfpathlineto{\pgfqpoint{1.207513in}{1.392959in}}%
\pgfpathlineto{\pgfqpoint{1.222409in}{1.383998in}}%
\pgfpathlineto{\pgfqpoint{1.237305in}{1.377357in}}%
\pgfpathlineto{\pgfqpoint{1.252200in}{1.372869in}}%
\pgfpathlineto{\pgfqpoint{1.269578in}{1.370084in}}%
\pgfpathlineto{\pgfqpoint{1.286957in}{1.369584in}}%
\pgfpathlineto{\pgfqpoint{1.306817in}{1.371302in}}%
\pgfpathlineto{\pgfqpoint{1.331644in}{1.376006in}}%
\pgfpathlineto{\pgfqpoint{1.363917in}{1.384664in}}%
\pgfpathlineto{\pgfqpoint{1.443361in}{1.406863in}}%
\pgfpathlineto{\pgfqpoint{1.475635in}{1.413205in}}%
\pgfpathlineto{\pgfqpoint{1.507908in}{1.417310in}}%
\pgfpathlineto{\pgfqpoint{1.542665in}{1.419344in}}%
\pgfpathlineto{\pgfqpoint{1.582386in}{1.419203in}}%
\pgfpathlineto{\pgfqpoint{1.634521in}{1.416495in}}%
\pgfpathlineto{\pgfqpoint{1.780995in}{1.407579in}}%
\pgfpathlineto{\pgfqpoint{1.852990in}{1.406192in}}%
\pgfpathlineto{\pgfqpoint{1.962225in}{1.406664in}}%
\pgfpathlineto{\pgfqpoint{2.153385in}{1.407304in}}%
\pgfpathlineto{\pgfqpoint{2.999953in}{1.405568in}}%
\pgfpathlineto{\pgfqpoint{3.225870in}{1.405396in}}%
\pgfpathlineto{\pgfqpoint{3.225870in}{1.405396in}}%
\pgfusepath{stroke}%
\end{pgfscope}%
\begin{pgfscope}%
\pgfpathrectangle{\pgfqpoint{0.619136in}{0.571603in}}{\pgfqpoint{2.730864in}{1.657828in}}%
\pgfusepath{clip}%
\pgfsetrectcap%
\pgfsetroundjoin%
\pgfsetlinewidth{1.505625pt}%
\definecolor{currentstroke}{rgb}{0.172549,0.627451,0.172549}%
\pgfsetstrokecolor{currentstroke}%
\pgfsetdash{}{0pt}%
\pgfpathmoveto{\pgfqpoint{0.743267in}{0.646959in}}%
\pgfpathlineto{\pgfqpoint{0.748232in}{0.684945in}}%
\pgfpathlineto{\pgfqpoint{0.760645in}{0.817614in}}%
\pgfpathlineto{\pgfqpoint{0.780506in}{1.031090in}}%
\pgfpathlineto{\pgfqpoint{0.792919in}{1.145233in}}%
\pgfpathlineto{\pgfqpoint{0.805332in}{1.240344in}}%
\pgfpathlineto{\pgfqpoint{0.817745in}{1.316476in}}%
\pgfpathlineto{\pgfqpoint{0.827675in}{1.364682in}}%
\pgfpathlineto{\pgfqpoint{0.837605in}{1.402771in}}%
\pgfpathlineto{\pgfqpoint{0.847536in}{1.431968in}}%
\pgfpathlineto{\pgfqpoint{0.857466in}{1.453524in}}%
\pgfpathlineto{\pgfqpoint{0.867397in}{1.468651in}}%
\pgfpathlineto{\pgfqpoint{0.874845in}{1.476463in}}%
\pgfpathlineto{\pgfqpoint{0.882292in}{1.481737in}}%
\pgfpathlineto{\pgfqpoint{0.889740in}{1.484877in}}%
\pgfpathlineto{\pgfqpoint{0.899671in}{1.486366in}}%
\pgfpathlineto{\pgfqpoint{0.909601in}{1.485462in}}%
\pgfpathlineto{\pgfqpoint{0.922014in}{1.481934in}}%
\pgfpathlineto{\pgfqpoint{0.939392in}{1.474270in}}%
\pgfpathlineto{\pgfqpoint{1.011388in}{1.439366in}}%
\pgfpathlineto{\pgfqpoint{1.036214in}{1.431220in}}%
\pgfpathlineto{\pgfqpoint{1.061040in}{1.425340in}}%
\pgfpathlineto{\pgfqpoint{1.090831in}{1.420641in}}%
\pgfpathlineto{\pgfqpoint{1.128070in}{1.417197in}}%
\pgfpathlineto{\pgfqpoint{1.182687in}{1.414664in}}%
\pgfpathlineto{\pgfqpoint{1.299370in}{1.412087in}}%
\pgfpathlineto{\pgfqpoint{1.530252in}{1.409207in}}%
\pgfpathlineto{\pgfqpoint{1.895194in}{1.407260in}}%
\pgfpathlineto{\pgfqpoint{2.647423in}{1.405837in}}%
\pgfpathlineto{\pgfqpoint{3.225870in}{1.405391in}}%
\pgfpathlineto{\pgfqpoint{3.225870in}{1.405391in}}%
\pgfusepath{stroke}%
\end{pgfscope}%
\begin{pgfscope}%
\pgfpathrectangle{\pgfqpoint{0.619136in}{0.571603in}}{\pgfqpoint{2.730864in}{1.657828in}}%
\pgfusepath{clip}%
\pgfsetrectcap%
\pgfsetroundjoin%
\pgfsetlinewidth{1.505625pt}%
\definecolor{currentstroke}{rgb}{0.839216,0.152941,0.156863}%
\pgfsetstrokecolor{currentstroke}%
\pgfsetdash{}{0pt}%
\pgfpathmoveto{\pgfqpoint{0.743267in}{0.646959in}}%
\pgfpathlineto{\pgfqpoint{0.763127in}{0.963266in}}%
\pgfpathlineto{\pgfqpoint{0.773058in}{1.080748in}}%
\pgfpathlineto{\pgfqpoint{0.782988in}{1.171894in}}%
\pgfpathlineto{\pgfqpoint{0.792919in}{1.241328in}}%
\pgfpathlineto{\pgfqpoint{0.802849in}{1.293436in}}%
\pgfpathlineto{\pgfqpoint{0.812779in}{1.331997in}}%
\pgfpathlineto{\pgfqpoint{0.822710in}{1.360130in}}%
\pgfpathlineto{\pgfqpoint{0.832640in}{1.380336in}}%
\pgfpathlineto{\pgfqpoint{0.842571in}{1.394588in}}%
\pgfpathlineto{\pgfqpoint{0.852501in}{1.404419in}}%
\pgfpathlineto{\pgfqpoint{0.862432in}{1.411003in}}%
\pgfpathlineto{\pgfqpoint{0.874845in}{1.416007in}}%
\pgfpathlineto{\pgfqpoint{0.887258in}{1.418588in}}%
\pgfpathlineto{\pgfqpoint{0.904636in}{1.419745in}}%
\pgfpathlineto{\pgfqpoint{0.931944in}{1.418796in}}%
\pgfpathlineto{\pgfqpoint{1.066005in}{1.411079in}}%
\pgfpathlineto{\pgfqpoint{1.170274in}{1.408837in}}%
\pgfpathlineto{\pgfqpoint{1.373848in}{1.407068in}}%
\pgfpathlineto{\pgfqpoint{1.828164in}{1.405758in}}%
\pgfpathlineto{\pgfqpoint{3.096774in}{1.404903in}}%
\pgfpathlineto{\pgfqpoint{3.225870in}{1.404867in}}%
\pgfpathlineto{\pgfqpoint{3.225870in}{1.404867in}}%
\pgfusepath{stroke}%
\end{pgfscope}%
\begin{pgfscope}%
\pgfpathrectangle{\pgfqpoint{0.619136in}{0.571603in}}{\pgfqpoint{2.730864in}{1.657828in}}%
\pgfusepath{clip}%
\pgfsetrectcap%
\pgfsetroundjoin%
\pgfsetlinewidth{1.505625pt}%
\definecolor{currentstroke}{rgb}{0.580392,0.403922,0.741176}%
\pgfsetstrokecolor{currentstroke}%
\pgfsetdash{}{0pt}%
\pgfpathmoveto{\pgfqpoint{0.743267in}{0.646959in}}%
\pgfpathlineto{\pgfqpoint{0.745749in}{0.652075in}}%
\pgfpathlineto{\pgfqpoint{0.750714in}{0.672791in}}%
\pgfpathlineto{\pgfqpoint{0.758162in}{0.717762in}}%
\pgfpathlineto{\pgfqpoint{0.768093in}{0.793015in}}%
\pgfpathlineto{\pgfqpoint{0.785471in}{0.944853in}}%
\pgfpathlineto{\pgfqpoint{0.815262in}{1.206240in}}%
\pgfpathlineto{\pgfqpoint{0.830158in}{1.319701in}}%
\pgfpathlineto{\pgfqpoint{0.845053in}{1.415639in}}%
\pgfpathlineto{\pgfqpoint{0.857466in}{1.480866in}}%
\pgfpathlineto{\pgfqpoint{0.867397in}{1.523246in}}%
\pgfpathlineto{\pgfqpoint{0.877327in}{1.557118in}}%
\pgfpathlineto{\pgfqpoint{0.887258in}{1.582883in}}%
\pgfpathlineto{\pgfqpoint{0.894705in}{1.597215in}}%
\pgfpathlineto{\pgfqpoint{0.902153in}{1.607574in}}%
\pgfpathlineto{\pgfqpoint{0.909601in}{1.614262in}}%
\pgfpathlineto{\pgfqpoint{0.917049in}{1.617602in}}%
\pgfpathlineto{\pgfqpoint{0.924497in}{1.617929in}}%
\pgfpathlineto{\pgfqpoint{0.931944in}{1.615579in}}%
\pgfpathlineto{\pgfqpoint{0.939392in}{1.610890in}}%
\pgfpathlineto{\pgfqpoint{0.949323in}{1.601574in}}%
\pgfpathlineto{\pgfqpoint{0.961736in}{1.586060in}}%
\pgfpathlineto{\pgfqpoint{0.976631in}{1.563566in}}%
\pgfpathlineto{\pgfqpoint{1.043662in}{1.456483in}}%
\pgfpathlineto{\pgfqpoint{1.061040in}{1.435405in}}%
\pgfpathlineto{\pgfqpoint{1.075935in}{1.420762in}}%
\pgfpathlineto{\pgfqpoint{1.090831in}{1.409306in}}%
\pgfpathlineto{\pgfqpoint{1.105727in}{1.400886in}}%
\pgfpathlineto{\pgfqpoint{1.120622in}{1.395223in}}%
\pgfpathlineto{\pgfqpoint{1.135518in}{1.391960in}}%
\pgfpathlineto{\pgfqpoint{1.152896in}{1.390646in}}%
\pgfpathlineto{\pgfqpoint{1.172757in}{1.391629in}}%
\pgfpathlineto{\pgfqpoint{1.197583in}{1.395231in}}%
\pgfpathlineto{\pgfqpoint{1.296887in}{1.412075in}}%
\pgfpathlineto{\pgfqpoint{1.331644in}{1.414372in}}%
\pgfpathlineto{\pgfqpoint{1.371365in}{1.414652in}}%
\pgfpathlineto{\pgfqpoint{1.430948in}{1.412543in}}%
\pgfpathlineto{\pgfqpoint{1.540182in}{1.408644in}}%
\pgfpathlineto{\pgfqpoint{1.641969in}{1.407763in}}%
\pgfpathlineto{\pgfqpoint{3.225870in}{1.405149in}}%
\pgfpathlineto{\pgfqpoint{3.225870in}{1.405149in}}%
\pgfusepath{stroke}%
\end{pgfscope}%
\begin{pgfscope}%
\pgfpathrectangle{\pgfqpoint{0.619136in}{0.571603in}}{\pgfqpoint{2.730864in}{1.657828in}}%
\pgfusepath{clip}%
\pgfsetrectcap%
\pgfsetroundjoin%
\pgfsetlinewidth{1.505625pt}%
\definecolor{currentstroke}{rgb}{0.549020,0.337255,0.294118}%
\pgfsetstrokecolor{currentstroke}%
\pgfsetdash{}{0pt}%
\pgfpathmoveto{\pgfqpoint{0.743267in}{0.646959in}}%
\pgfpathlineto{\pgfqpoint{0.745749in}{0.655603in}}%
\pgfpathlineto{\pgfqpoint{0.750714in}{0.686237in}}%
\pgfpathlineto{\pgfqpoint{0.760645in}{0.769618in}}%
\pgfpathlineto{\pgfqpoint{0.782988in}{0.989834in}}%
\pgfpathlineto{\pgfqpoint{0.802849in}{1.174686in}}%
\pgfpathlineto{\pgfqpoint{0.817745in}{1.291680in}}%
\pgfpathlineto{\pgfqpoint{0.830158in}{1.371945in}}%
\pgfpathlineto{\pgfqpoint{0.842571in}{1.436161in}}%
\pgfpathlineto{\pgfqpoint{0.852501in}{1.476408in}}%
\pgfpathlineto{\pgfqpoint{0.862432in}{1.507463in}}%
\pgfpathlineto{\pgfqpoint{0.872362in}{1.530158in}}%
\pgfpathlineto{\pgfqpoint{0.879810in}{1.542257in}}%
\pgfpathlineto{\pgfqpoint{0.887258in}{1.550598in}}%
\pgfpathlineto{\pgfqpoint{0.894705in}{1.555604in}}%
\pgfpathlineto{\pgfqpoint{0.902153in}{1.557691in}}%
\pgfpathlineto{\pgfqpoint{0.909601in}{1.557264in}}%
\pgfpathlineto{\pgfqpoint{0.917049in}{1.554710in}}%
\pgfpathlineto{\pgfqpoint{0.926979in}{1.548625in}}%
\pgfpathlineto{\pgfqpoint{0.939392in}{1.537812in}}%
\pgfpathlineto{\pgfqpoint{0.956770in}{1.519053in}}%
\pgfpathlineto{\pgfqpoint{1.003940in}{1.465838in}}%
\pgfpathlineto{\pgfqpoint{1.021318in}{1.449809in}}%
\pgfpathlineto{\pgfqpoint{1.038696in}{1.436717in}}%
\pgfpathlineto{\pgfqpoint{1.056075in}{1.426595in}}%
\pgfpathlineto{\pgfqpoint{1.073453in}{1.419220in}}%
\pgfpathlineto{\pgfqpoint{1.090831in}{1.414214in}}%
\pgfpathlineto{\pgfqpoint{1.110692in}{1.410826in}}%
\pgfpathlineto{\pgfqpoint{1.135518in}{1.409108in}}%
\pgfpathlineto{\pgfqpoint{1.167792in}{1.409305in}}%
\pgfpathlineto{\pgfqpoint{1.306817in}{1.412544in}}%
\pgfpathlineto{\pgfqpoint{1.425982in}{1.410519in}}%
\pgfpathlineto{\pgfqpoint{1.597282in}{1.408585in}}%
\pgfpathlineto{\pgfqpoint{2.039185in}{1.406692in}}%
\pgfpathlineto{\pgfqpoint{2.950301in}{1.405425in}}%
\pgfpathlineto{\pgfqpoint{3.225870in}{1.405252in}}%
\pgfpathlineto{\pgfqpoint{3.225870in}{1.405252in}}%
\pgfusepath{stroke}%
\end{pgfscope}%
\begin{pgfscope}%
\pgfpathrectangle{\pgfqpoint{0.619136in}{0.571603in}}{\pgfqpoint{2.730864in}{1.657828in}}%
\pgfusepath{clip}%
\pgfsetrectcap%
\pgfsetroundjoin%
\pgfsetlinewidth{1.505625pt}%
\definecolor{currentstroke}{rgb}{0.890196,0.466667,0.760784}%
\pgfsetstrokecolor{currentstroke}%
\pgfsetdash{}{0pt}%
\pgfpathmoveto{\pgfqpoint{0.743267in}{0.646959in}}%
\pgfpathlineto{\pgfqpoint{0.748232in}{0.675339in}}%
\pgfpathlineto{\pgfqpoint{0.758162in}{0.757081in}}%
\pgfpathlineto{\pgfqpoint{0.795401in}{1.082116in}}%
\pgfpathlineto{\pgfqpoint{0.810297in}{1.188471in}}%
\pgfpathlineto{\pgfqpoint{0.822710in}{1.262946in}}%
\pgfpathlineto{\pgfqpoint{0.835123in}{1.324711in}}%
\pgfpathlineto{\pgfqpoint{0.847536in}{1.374554in}}%
\pgfpathlineto{\pgfqpoint{0.857466in}{1.406596in}}%
\pgfpathlineto{\pgfqpoint{0.867397in}{1.432388in}}%
\pgfpathlineto{\pgfqpoint{0.877327in}{1.452619in}}%
\pgfpathlineto{\pgfqpoint{0.887258in}{1.467977in}}%
\pgfpathlineto{\pgfqpoint{0.897188in}{1.479125in}}%
\pgfpathlineto{\pgfqpoint{0.907118in}{1.486692in}}%
\pgfpathlineto{\pgfqpoint{0.917049in}{1.491260in}}%
\pgfpathlineto{\pgfqpoint{0.926979in}{1.493357in}}%
\pgfpathlineto{\pgfqpoint{0.939392in}{1.493224in}}%
\pgfpathlineto{\pgfqpoint{0.951805in}{1.490777in}}%
\pgfpathlineto{\pgfqpoint{0.969183in}{1.484733in}}%
\pgfpathlineto{\pgfqpoint{0.994009in}{1.473334in}}%
\pgfpathlineto{\pgfqpoint{1.046144in}{1.448984in}}%
\pgfpathlineto{\pgfqpoint{1.073453in}{1.438871in}}%
\pgfpathlineto{\pgfqpoint{1.100761in}{1.431075in}}%
\pgfpathlineto{\pgfqpoint{1.130553in}{1.424954in}}%
\pgfpathlineto{\pgfqpoint{1.165309in}{1.420279in}}%
\pgfpathlineto{\pgfqpoint{1.207513in}{1.417000in}}%
\pgfpathlineto{\pgfqpoint{1.269578in}{1.414656in}}%
\pgfpathlineto{\pgfqpoint{1.411087in}{1.412213in}}%
\pgfpathlineto{\pgfqpoint{1.706517in}{1.409153in}}%
\pgfpathlineto{\pgfqpoint{2.153385in}{1.407216in}}%
\pgfpathlineto{\pgfqpoint{3.074431in}{1.405805in}}%
\pgfpathlineto{\pgfqpoint{3.225870in}{1.405686in}}%
\pgfpathlineto{\pgfqpoint{3.225870in}{1.405686in}}%
\pgfusepath{stroke}%
\end{pgfscope}%
\begin{pgfscope}%
\pgfpathrectangle{\pgfqpoint{0.619136in}{0.571603in}}{\pgfqpoint{2.730864in}{1.657828in}}%
\pgfusepath{clip}%
\pgfsetrectcap%
\pgfsetroundjoin%
\pgfsetlinewidth{1.505625pt}%
\definecolor{currentstroke}{rgb}{0.498039,0.498039,0.498039}%
\pgfsetstrokecolor{currentstroke}%
\pgfsetdash{}{0pt}%
\pgfpathmoveto{\pgfqpoint{0.743267in}{0.646959in}}%
\pgfpathlineto{\pgfqpoint{0.745749in}{0.649385in}}%
\pgfpathlineto{\pgfqpoint{0.750714in}{0.660462in}}%
\pgfpathlineto{\pgfqpoint{0.758162in}{0.686560in}}%
\pgfpathlineto{\pgfqpoint{0.768093in}{0.733541in}}%
\pgfpathlineto{\pgfqpoint{0.780506in}{0.805726in}}%
\pgfpathlineto{\pgfqpoint{0.797884in}{0.922065in}}%
\pgfpathlineto{\pgfqpoint{0.854984in}{1.317957in}}%
\pgfpathlineto{\pgfqpoint{0.872362in}{1.419079in}}%
\pgfpathlineto{\pgfqpoint{0.887258in}{1.493531in}}%
\pgfpathlineto{\pgfqpoint{0.899671in}{1.546148in}}%
\pgfpathlineto{\pgfqpoint{0.912084in}{1.589902in}}%
\pgfpathlineto{\pgfqpoint{0.924497in}{1.624790in}}%
\pgfpathlineto{\pgfqpoint{0.934427in}{1.646455in}}%
\pgfpathlineto{\pgfqpoint{0.944357in}{1.662776in}}%
\pgfpathlineto{\pgfqpoint{0.951805in}{1.671666in}}%
\pgfpathlineto{\pgfqpoint{0.959253in}{1.677828in}}%
\pgfpathlineto{\pgfqpoint{0.966701in}{1.681405in}}%
\pgfpathlineto{\pgfqpoint{0.974149in}{1.682557in}}%
\pgfpathlineto{\pgfqpoint{0.981596in}{1.681448in}}%
\pgfpathlineto{\pgfqpoint{0.989044in}{1.678254in}}%
\pgfpathlineto{\pgfqpoint{0.998975in}{1.671060in}}%
\pgfpathlineto{\pgfqpoint{1.008905in}{1.660911in}}%
\pgfpathlineto{\pgfqpoint{1.021318in}{1.644730in}}%
\pgfpathlineto{\pgfqpoint{1.036214in}{1.621292in}}%
\pgfpathlineto{\pgfqpoint{1.056075in}{1.585484in}}%
\pgfpathlineto{\pgfqpoint{1.123105in}{1.460506in}}%
\pgfpathlineto{\pgfqpoint{1.142966in}{1.430045in}}%
\pgfpathlineto{\pgfqpoint{1.160344in}{1.407567in}}%
\pgfpathlineto{\pgfqpoint{1.175240in}{1.391656in}}%
\pgfpathlineto{\pgfqpoint{1.190135in}{1.378899in}}%
\pgfpathlineto{\pgfqpoint{1.205031in}{1.369236in}}%
\pgfpathlineto{\pgfqpoint{1.219926in}{1.362513in}}%
\pgfpathlineto{\pgfqpoint{1.234822in}{1.358504in}}%
\pgfpathlineto{\pgfqpoint{1.249718in}{1.356925in}}%
\pgfpathlineto{\pgfqpoint{1.264613in}{1.357458in}}%
\pgfpathlineto{\pgfqpoint{1.281991in}{1.360291in}}%
\pgfpathlineto{\pgfqpoint{1.301852in}{1.365768in}}%
\pgfpathlineto{\pgfqpoint{1.329161in}{1.375789in}}%
\pgfpathlineto{\pgfqpoint{1.408604in}{1.406608in}}%
\pgfpathlineto{\pgfqpoint{1.435913in}{1.414222in}}%
\pgfpathlineto{\pgfqpoint{1.463222in}{1.419492in}}%
\pgfpathlineto{\pgfqpoint{1.490530in}{1.422434in}}%
\pgfpathlineto{\pgfqpoint{1.520321in}{1.423294in}}%
\pgfpathlineto{\pgfqpoint{1.555078in}{1.421901in}}%
\pgfpathlineto{\pgfqpoint{1.604730in}{1.417321in}}%
\pgfpathlineto{\pgfqpoint{1.701551in}{1.407988in}}%
\pgfpathlineto{\pgfqpoint{1.756169in}{1.405326in}}%
\pgfpathlineto{\pgfqpoint{1.818234in}{1.404720in}}%
\pgfpathlineto{\pgfqpoint{1.927468in}{1.406473in}}%
\pgfpathlineto{\pgfqpoint{2.049116in}{1.407525in}}%
\pgfpathlineto{\pgfqpoint{2.262620in}{1.406221in}}%
\pgfpathlineto{\pgfqpoint{2.585358in}{1.405811in}}%
\pgfpathlineto{\pgfqpoint{3.225870in}{1.405223in}}%
\pgfpathlineto{\pgfqpoint{3.225870in}{1.405223in}}%
\pgfusepath{stroke}%
\end{pgfscope}%
\begin{pgfscope}%
\pgfpathrectangle{\pgfqpoint{0.619136in}{0.571603in}}{\pgfqpoint{2.730864in}{1.657828in}}%
\pgfusepath{clip}%
\pgfsetrectcap%
\pgfsetroundjoin%
\pgfsetlinewidth{1.505625pt}%
\definecolor{currentstroke}{rgb}{0.737255,0.741176,0.133333}%
\pgfsetstrokecolor{currentstroke}%
\pgfsetdash{}{0pt}%
\pgfpathmoveto{\pgfqpoint{0.743267in}{0.646959in}}%
\pgfpathlineto{\pgfqpoint{0.775540in}{0.972945in}}%
\pgfpathlineto{\pgfqpoint{0.787953in}{1.067286in}}%
\pgfpathlineto{\pgfqpoint{0.800366in}{1.144489in}}%
\pgfpathlineto{\pgfqpoint{0.812779in}{1.206895in}}%
\pgfpathlineto{\pgfqpoint{0.825192in}{1.256829in}}%
\pgfpathlineto{\pgfqpoint{0.837605in}{1.296418in}}%
\pgfpathlineto{\pgfqpoint{0.850018in}{1.327527in}}%
\pgfpathlineto{\pgfqpoint{0.862432in}{1.351751in}}%
\pgfpathlineto{\pgfqpoint{0.874845in}{1.370432in}}%
\pgfpathlineto{\pgfqpoint{0.887258in}{1.384682in}}%
\pgfpathlineto{\pgfqpoint{0.899671in}{1.395418in}}%
\pgfpathlineto{\pgfqpoint{0.912084in}{1.403388in}}%
\pgfpathlineto{\pgfqpoint{0.926979in}{1.410142in}}%
\pgfpathlineto{\pgfqpoint{0.944357in}{1.415164in}}%
\pgfpathlineto{\pgfqpoint{0.964218in}{1.418303in}}%
\pgfpathlineto{\pgfqpoint{0.989044in}{1.419781in}}%
\pgfpathlineto{\pgfqpoint{1.023801in}{1.419455in}}%
\pgfpathlineto{\pgfqpoint{1.110692in}{1.415520in}}%
\pgfpathlineto{\pgfqpoint{1.217444in}{1.411827in}}%
\pgfpathlineto{\pgfqpoint{1.358952in}{1.409405in}}%
\pgfpathlineto{\pgfqpoint{1.619626in}{1.407521in}}%
\pgfpathlineto{\pgfqpoint{2.180694in}{1.406079in}}%
\pgfpathlineto{\pgfqpoint{3.225870in}{1.405248in}}%
\pgfpathlineto{\pgfqpoint{3.225870in}{1.405248in}}%
\pgfusepath{stroke}%
\end{pgfscope}%
\begin{pgfscope}%
\pgfpathrectangle{\pgfqpoint{0.619136in}{0.571603in}}{\pgfqpoint{2.730864in}{1.657828in}}%
\pgfusepath{clip}%
\pgfsetrectcap%
\pgfsetroundjoin%
\pgfsetlinewidth{1.505625pt}%
\definecolor{currentstroke}{rgb}{0.090196,0.745098,0.811765}%
\pgfsetstrokecolor{currentstroke}%
\pgfsetdash{}{0pt}%
\pgfpathmoveto{\pgfqpoint{0.743267in}{0.646959in}}%
\pgfpathlineto{\pgfqpoint{0.748232in}{0.648498in}}%
\pgfpathlineto{\pgfqpoint{0.753197in}{0.652935in}}%
\pgfpathlineto{\pgfqpoint{0.758162in}{0.660162in}}%
\pgfpathlineto{\pgfqpoint{0.765610in}{0.676069in}}%
\pgfpathlineto{\pgfqpoint{0.773058in}{0.697842in}}%
\pgfpathlineto{\pgfqpoint{0.782988in}{0.735557in}}%
\pgfpathlineto{\pgfqpoint{0.792919in}{0.782566in}}%
\pgfpathlineto{\pgfqpoint{0.805332in}{0.853202in}}%
\pgfpathlineto{\pgfqpoint{0.820227in}{0.953065in}}%
\pgfpathlineto{\pgfqpoint{0.837605in}{1.086065in}}%
\pgfpathlineto{\pgfqpoint{0.862432in}{1.295410in}}%
\pgfpathlineto{\pgfqpoint{0.907118in}{1.674558in}}%
\pgfpathlineto{\pgfqpoint{0.924497in}{1.804158in}}%
\pgfpathlineto{\pgfqpoint{0.939392in}{1.900713in}}%
\pgfpathlineto{\pgfqpoint{0.951805in}{1.968712in}}%
\pgfpathlineto{\pgfqpoint{0.964218in}{2.023907in}}%
\pgfpathlineto{\pgfqpoint{0.974149in}{2.058109in}}%
\pgfpathlineto{\pgfqpoint{0.981596in}{2.077675in}}%
\pgfpathlineto{\pgfqpoint{0.989044in}{2.091883in}}%
\pgfpathlineto{\pgfqpoint{0.996492in}{2.100654in}}%
\pgfpathlineto{\pgfqpoint{1.001457in}{2.103459in}}%
\pgfpathlineto{\pgfqpoint{1.006423in}{2.103828in}}%
\pgfpathlineto{\pgfqpoint{1.011388in}{2.101770in}}%
\pgfpathlineto{\pgfqpoint{1.016353in}{2.097301in}}%
\pgfpathlineto{\pgfqpoint{1.023801in}{2.086132in}}%
\pgfpathlineto{\pgfqpoint{1.031249in}{2.069714in}}%
\pgfpathlineto{\pgfqpoint{1.038696in}{2.048203in}}%
\pgfpathlineto{\pgfqpoint{1.048627in}{2.011947in}}%
\pgfpathlineto{\pgfqpoint{1.058557in}{1.967562in}}%
\pgfpathlineto{\pgfqpoint{1.070970in}{1.901692in}}%
\pgfpathlineto{\pgfqpoint{1.085866in}{1.809466in}}%
\pgfpathlineto{\pgfqpoint{1.103244in}{1.687569in}}%
\pgfpathlineto{\pgfqpoint{1.130553in}{1.477447in}}%
\pgfpathlineto{\pgfqpoint{1.170274in}{1.172482in}}%
\pgfpathlineto{\pgfqpoint{1.190135in}{1.038636in}}%
\pgfpathlineto{\pgfqpoint{1.205031in}{0.952518in}}%
\pgfpathlineto{\pgfqpoint{1.217444in}{0.892183in}}%
\pgfpathlineto{\pgfqpoint{1.229857in}{0.843536in}}%
\pgfpathlineto{\pgfqpoint{1.239787in}{0.813675in}}%
\pgfpathlineto{\pgfqpoint{1.247235in}{0.796803in}}%
\pgfpathlineto{\pgfqpoint{1.254683in}{0.784784in}}%
\pgfpathlineto{\pgfqpoint{1.262131in}{0.777685in}}%
\pgfpathlineto{\pgfqpoint{1.267096in}{0.775700in}}%
\pgfpathlineto{\pgfqpoint{1.272061in}{0.775913in}}%
\pgfpathlineto{\pgfqpoint{1.277026in}{0.778313in}}%
\pgfpathlineto{\pgfqpoint{1.281991in}{0.782885in}}%
\pgfpathlineto{\pgfqpoint{1.289439in}{0.793760in}}%
\pgfpathlineto{\pgfqpoint{1.296887in}{0.809350in}}%
\pgfpathlineto{\pgfqpoint{1.306817in}{0.837210in}}%
\pgfpathlineto{\pgfqpoint{1.316748in}{0.872734in}}%
\pgfpathlineto{\pgfqpoint{1.329161in}{0.927068in}}%
\pgfpathlineto{\pgfqpoint{1.344057in}{1.005106in}}%
\pgfpathlineto{\pgfqpoint{1.361435in}{1.110534in}}%
\pgfpathlineto{\pgfqpoint{1.383778in}{1.261580in}}%
\pgfpathlineto{\pgfqpoint{1.440878in}{1.654867in}}%
\pgfpathlineto{\pgfqpoint{1.458256in}{1.756501in}}%
\pgfpathlineto{\pgfqpoint{1.473152in}{1.830826in}}%
\pgfpathlineto{\pgfqpoint{1.485565in}{1.882055in}}%
\pgfpathlineto{\pgfqpoint{1.495495in}{1.915281in}}%
\pgfpathlineto{\pgfqpoint{1.505426in}{1.941164in}}%
\pgfpathlineto{\pgfqpoint{1.512874in}{1.955571in}}%
\pgfpathlineto{\pgfqpoint{1.520321in}{1.965587in}}%
\pgfpathlineto{\pgfqpoint{1.527769in}{1.971159in}}%
\pgfpathlineto{\pgfqpoint{1.532734in}{1.972394in}}%
\pgfpathlineto{\pgfqpoint{1.537700in}{1.971649in}}%
\pgfpathlineto{\pgfqpoint{1.542665in}{1.968934in}}%
\pgfpathlineto{\pgfqpoint{1.547630in}{1.964266in}}%
\pgfpathlineto{\pgfqpoint{1.555078in}{1.953655in}}%
\pgfpathlineto{\pgfqpoint{1.562526in}{1.938814in}}%
\pgfpathlineto{\pgfqpoint{1.572456in}{1.912690in}}%
\pgfpathlineto{\pgfqpoint{1.582386in}{1.879715in}}%
\pgfpathlineto{\pgfqpoint{1.594799in}{1.829650in}}%
\pgfpathlineto{\pgfqpoint{1.609695in}{1.758177in}}%
\pgfpathlineto{\pgfqpoint{1.627073in}{1.662108in}}%
\pgfpathlineto{\pgfqpoint{1.649417in}{1.525130in}}%
\pgfpathlineto{\pgfqpoint{1.701551in}{1.199414in}}%
\pgfpathlineto{\pgfqpoint{1.718930in}{1.105115in}}%
\pgfpathlineto{\pgfqpoint{1.733825in}{1.035245in}}%
\pgfpathlineto{\pgfqpoint{1.746238in}{0.986340in}}%
\pgfpathlineto{\pgfqpoint{1.758651in}{0.946961in}}%
\pgfpathlineto{\pgfqpoint{1.768582in}{0.922835in}}%
\pgfpathlineto{\pgfqpoint{1.776030in}{0.909238in}}%
\pgfpathlineto{\pgfqpoint{1.783477in}{0.899592in}}%
\pgfpathlineto{\pgfqpoint{1.790925in}{0.893949in}}%
\pgfpathlineto{\pgfqpoint{1.795890in}{0.892424in}}%
\pgfpathlineto{\pgfqpoint{1.800856in}{0.892686in}}%
\pgfpathlineto{\pgfqpoint{1.805821in}{0.894728in}}%
\pgfpathlineto{\pgfqpoint{1.813269in}{0.901095in}}%
\pgfpathlineto{\pgfqpoint{1.820716in}{0.911362in}}%
\pgfpathlineto{\pgfqpoint{1.828164in}{0.925425in}}%
\pgfpathlineto{\pgfqpoint{1.838095in}{0.949849in}}%
\pgfpathlineto{\pgfqpoint{1.848025in}{0.980394in}}%
\pgfpathlineto{\pgfqpoint{1.860438in}{1.026459in}}%
\pgfpathlineto{\pgfqpoint{1.875334in}{1.091848in}}%
\pgfpathlineto{\pgfqpoint{1.892712in}{1.179310in}}%
\pgfpathlineto{\pgfqpoint{1.917538in}{1.317702in}}%
\pgfpathlineto{\pgfqpoint{1.964707in}{1.583588in}}%
\pgfpathlineto{\pgfqpoint{1.982086in}{1.669339in}}%
\pgfpathlineto{\pgfqpoint{1.996981in}{1.733125in}}%
\pgfpathlineto{\pgfqpoint{2.009394in}{1.777971in}}%
\pgfpathlineto{\pgfqpoint{2.021807in}{1.814298in}}%
\pgfpathlineto{\pgfqpoint{2.031738in}{1.836744in}}%
\pgfpathlineto{\pgfqpoint{2.041668in}{1.853017in}}%
\pgfpathlineto{\pgfqpoint{2.049116in}{1.861060in}}%
\pgfpathlineto{\pgfqpoint{2.056564in}{1.865485in}}%
\pgfpathlineto{\pgfqpoint{2.061529in}{1.866418in}}%
\pgfpathlineto{\pgfqpoint{2.066494in}{1.865740in}}%
\pgfpathlineto{\pgfqpoint{2.071459in}{1.863459in}}%
\pgfpathlineto{\pgfqpoint{2.078907in}{1.857066in}}%
\pgfpathlineto{\pgfqpoint{2.086355in}{1.847170in}}%
\pgfpathlineto{\pgfqpoint{2.093803in}{1.833870in}}%
\pgfpathlineto{\pgfqpoint{2.103733in}{1.811057in}}%
\pgfpathlineto{\pgfqpoint{2.113664in}{1.782777in}}%
\pgfpathlineto{\pgfqpoint{2.126077in}{1.740406in}}%
\pgfpathlineto{\pgfqpoint{2.140972in}{1.680594in}}%
\pgfpathlineto{\pgfqpoint{2.158350in}{1.600974in}}%
\pgfpathlineto{\pgfqpoint{2.183176in}{1.475603in}}%
\pgfpathlineto{\pgfqpoint{2.227863in}{1.248116in}}%
\pgfpathlineto{\pgfqpoint{2.247724in}{1.159826in}}%
\pgfpathlineto{\pgfqpoint{2.262620in}{1.103118in}}%
\pgfpathlineto{\pgfqpoint{2.275033in}{1.063463in}}%
\pgfpathlineto{\pgfqpoint{2.287446in}{1.031572in}}%
\pgfpathlineto{\pgfqpoint{2.297376in}{1.012069in}}%
\pgfpathlineto{\pgfqpoint{2.304824in}{1.001104in}}%
\pgfpathlineto{\pgfqpoint{2.312272in}{0.993357in}}%
\pgfpathlineto{\pgfqpoint{2.319720in}{0.988867in}}%
\pgfpathlineto{\pgfqpoint{2.327167in}{0.987653in}}%
\pgfpathlineto{\pgfqpoint{2.334615in}{0.989703in}}%
\pgfpathlineto{\pgfqpoint{2.342063in}{0.994985in}}%
\pgfpathlineto{\pgfqpoint{2.349511in}{1.003436in}}%
\pgfpathlineto{\pgfqpoint{2.356959in}{1.014974in}}%
\pgfpathlineto{\pgfqpoint{2.366889in}{1.034967in}}%
\pgfpathlineto{\pgfqpoint{2.376820in}{1.059934in}}%
\pgfpathlineto{\pgfqpoint{2.389233in}{1.097543in}}%
\pgfpathlineto{\pgfqpoint{2.404128in}{1.150878in}}%
\pgfpathlineto{\pgfqpoint{2.421506in}{1.222158in}}%
\pgfpathlineto{\pgfqpoint{2.446332in}{1.334853in}}%
\pgfpathlineto{\pgfqpoint{2.493502in}{1.551108in}}%
\pgfpathlineto{\pgfqpoint{2.510880in}{1.620771in}}%
\pgfpathlineto{\pgfqpoint{2.525776in}{1.672549in}}%
\pgfpathlineto{\pgfqpoint{2.538189in}{1.708920in}}%
\pgfpathlineto{\pgfqpoint{2.550602in}{1.738346in}}%
\pgfpathlineto{\pgfqpoint{2.560532in}{1.756498in}}%
\pgfpathlineto{\pgfqpoint{2.570463in}{1.769622in}}%
\pgfpathlineto{\pgfqpoint{2.577910in}{1.776077in}}%
\pgfpathlineto{\pgfqpoint{2.585358in}{1.779588in}}%
\pgfpathlineto{\pgfqpoint{2.592806in}{1.780144in}}%
\pgfpathlineto{\pgfqpoint{2.600254in}{1.777760in}}%
\pgfpathlineto{\pgfqpoint{2.607702in}{1.772469in}}%
\pgfpathlineto{\pgfqpoint{2.615149in}{1.764331in}}%
\pgfpathlineto{\pgfqpoint{2.625080in}{1.749193in}}%
\pgfpathlineto{\pgfqpoint{2.635010in}{1.729385in}}%
\pgfpathlineto{\pgfqpoint{2.647423in}{1.698537in}}%
\pgfpathlineto{\pgfqpoint{2.659836in}{1.661618in}}%
\pgfpathlineto{\pgfqpoint{2.674732in}{1.610514in}}%
\pgfpathlineto{\pgfqpoint{2.694593in}{1.533627in}}%
\pgfpathlineto{\pgfqpoint{2.729349in}{1.387574in}}%
\pgfpathlineto{\pgfqpoint{2.756658in}{1.276302in}}%
\pgfpathlineto{\pgfqpoint{2.776519in}{1.204580in}}%
\pgfpathlineto{\pgfqpoint{2.791414in}{1.158553in}}%
\pgfpathlineto{\pgfqpoint{2.803827in}{1.126398in}}%
\pgfpathlineto{\pgfqpoint{2.816240in}{1.100570in}}%
\pgfpathlineto{\pgfqpoint{2.826171in}{1.084804in}}%
\pgfpathlineto{\pgfqpoint{2.836101in}{1.073595in}}%
\pgfpathlineto{\pgfqpoint{2.843549in}{1.068253in}}%
\pgfpathlineto{\pgfqpoint{2.850997in}{1.065570in}}%
\pgfpathlineto{\pgfqpoint{2.858445in}{1.065552in}}%
\pgfpathlineto{\pgfqpoint{2.865892in}{1.068182in}}%
\pgfpathlineto{\pgfqpoint{2.873340in}{1.073425in}}%
\pgfpathlineto{\pgfqpoint{2.880788in}{1.081226in}}%
\pgfpathlineto{\pgfqpoint{2.890718in}{1.095470in}}%
\pgfpathlineto{\pgfqpoint{2.900649in}{1.113892in}}%
\pgfpathlineto{\pgfqpoint{2.913062in}{1.142349in}}%
\pgfpathlineto{\pgfqpoint{2.927957in}{1.183556in}}%
\pgfpathlineto{\pgfqpoint{2.945336in}{1.239606in}}%
\pgfpathlineto{\pgfqpoint{2.967679in}{1.320420in}}%
\pgfpathlineto{\pgfqpoint{3.027262in}{1.541409in}}%
\pgfpathlineto{\pgfqpoint{3.044640in}{1.595868in}}%
\pgfpathlineto{\pgfqpoint{3.059535in}{1.635591in}}%
\pgfpathlineto{\pgfqpoint{3.071948in}{1.662886in}}%
\pgfpathlineto{\pgfqpoint{3.081879in}{1.680524in}}%
\pgfpathlineto{\pgfqpoint{3.091809in}{1.694193in}}%
\pgfpathlineto{\pgfqpoint{3.101740in}{1.703732in}}%
\pgfpathlineto{\pgfqpoint{3.109187in}{1.708115in}}%
\pgfpathlineto{\pgfqpoint{3.116635in}{1.710097in}}%
\pgfpathlineto{\pgfqpoint{3.124083in}{1.709676in}}%
\pgfpathlineto{\pgfqpoint{3.131531in}{1.706871in}}%
\pgfpathlineto{\pgfqpoint{3.138979in}{1.701717in}}%
\pgfpathlineto{\pgfqpoint{3.146427in}{1.694267in}}%
\pgfpathlineto{\pgfqpoint{3.156357in}{1.680889in}}%
\pgfpathlineto{\pgfqpoint{3.166287in}{1.663776in}}%
\pgfpathlineto{\pgfqpoint{3.178700in}{1.637542in}}%
\pgfpathlineto{\pgfqpoint{3.193596in}{1.599791in}}%
\pgfpathlineto{\pgfqpoint{3.210974in}{1.548708in}}%
\pgfpathlineto{\pgfqpoint{3.225870in}{1.500519in}}%
\pgfpathlineto{\pgfqpoint{3.225870in}{1.500519in}}%
\pgfusepath{stroke}%
\end{pgfscope}%
\begin{pgfscope}%
\pgfpathrectangle{\pgfqpoint{0.619136in}{0.571603in}}{\pgfqpoint{2.730864in}{1.657828in}}%
\pgfusepath{clip}%
\pgfsetrectcap%
\pgfsetroundjoin%
\pgfsetlinewidth{1.505625pt}%
\definecolor{currentstroke}{rgb}{0.121569,0.466667,0.705882}%
\pgfsetstrokecolor{currentstroke}%
\pgfsetdash{}{0pt}%
\pgfpathmoveto{\pgfqpoint{0.743267in}{0.646959in}}%
\pgfpathlineto{\pgfqpoint{0.745749in}{0.648359in}}%
\pgfpathlineto{\pgfqpoint{0.750714in}{0.655398in}}%
\pgfpathlineto{\pgfqpoint{0.758162in}{0.673052in}}%
\pgfpathlineto{\pgfqpoint{0.765610in}{0.697219in}}%
\pgfpathlineto{\pgfqpoint{0.775540in}{0.737441in}}%
\pgfpathlineto{\pgfqpoint{0.787953in}{0.797762in}}%
\pgfpathlineto{\pgfqpoint{0.805332in}{0.895574in}}%
\pgfpathlineto{\pgfqpoint{0.830158in}{1.050282in}}%
\pgfpathlineto{\pgfqpoint{0.872362in}{1.314213in}}%
\pgfpathlineto{\pgfqpoint{0.892223in}{1.425320in}}%
\pgfpathlineto{\pgfqpoint{0.909601in}{1.511065in}}%
\pgfpathlineto{\pgfqpoint{0.924497in}{1.574587in}}%
\pgfpathlineto{\pgfqpoint{0.939392in}{1.628127in}}%
\pgfpathlineto{\pgfqpoint{0.951805in}{1.664838in}}%
\pgfpathlineto{\pgfqpoint{0.964218in}{1.694299in}}%
\pgfpathlineto{\pgfqpoint{0.974149in}{1.712700in}}%
\pgfpathlineto{\pgfqpoint{0.984079in}{1.726613in}}%
\pgfpathlineto{\pgfqpoint{0.994009in}{1.736185in}}%
\pgfpathlineto{\pgfqpoint{1.001457in}{1.740625in}}%
\pgfpathlineto{\pgfqpoint{1.008905in}{1.742818in}}%
\pgfpathlineto{\pgfqpoint{1.016353in}{1.742866in}}%
\pgfpathlineto{\pgfqpoint{1.023801in}{1.740879in}}%
\pgfpathlineto{\pgfqpoint{1.031249in}{1.736974in}}%
\pgfpathlineto{\pgfqpoint{1.041179in}{1.729000in}}%
\pgfpathlineto{\pgfqpoint{1.051109in}{1.718145in}}%
\pgfpathlineto{\pgfqpoint{1.063522in}{1.701010in}}%
\pgfpathlineto{\pgfqpoint{1.078418in}{1.676061in}}%
\pgfpathlineto{\pgfqpoint{1.095796in}{1.642299in}}%
\pgfpathlineto{\pgfqpoint{1.120622in}{1.588614in}}%
\pgfpathlineto{\pgfqpoint{1.177722in}{1.463093in}}%
\pgfpathlineto{\pgfqpoint{1.200066in}{1.420322in}}%
\pgfpathlineto{\pgfqpoint{1.217444in}{1.391342in}}%
\pgfpathlineto{\pgfqpoint{1.234822in}{1.366669in}}%
\pgfpathlineto{\pgfqpoint{1.249718in}{1.349175in}}%
\pgfpathlineto{\pgfqpoint{1.264613in}{1.335133in}}%
\pgfpathlineto{\pgfqpoint{1.279509in}{1.324522in}}%
\pgfpathlineto{\pgfqpoint{1.291922in}{1.318227in}}%
\pgfpathlineto{\pgfqpoint{1.304335in}{1.314144in}}%
\pgfpathlineto{\pgfqpoint{1.316748in}{1.312145in}}%
\pgfpathlineto{\pgfqpoint{1.331644in}{1.312279in}}%
\pgfpathlineto{\pgfqpoint{1.346539in}{1.314890in}}%
\pgfpathlineto{\pgfqpoint{1.361435in}{1.319644in}}%
\pgfpathlineto{\pgfqpoint{1.378813in}{1.327433in}}%
\pgfpathlineto{\pgfqpoint{1.401156in}{1.340120in}}%
\pgfpathlineto{\pgfqpoint{1.433430in}{1.361526in}}%
\pgfpathlineto{\pgfqpoint{1.488048in}{1.398073in}}%
\pgfpathlineto{\pgfqpoint{1.515356in}{1.413467in}}%
\pgfpathlineto{\pgfqpoint{1.537700in}{1.423740in}}%
\pgfpathlineto{\pgfqpoint{1.560043in}{1.431646in}}%
\pgfpathlineto{\pgfqpoint{1.582386in}{1.437106in}}%
\pgfpathlineto{\pgfqpoint{1.604730in}{1.440187in}}%
\pgfpathlineto{\pgfqpoint{1.627073in}{1.441077in}}%
\pgfpathlineto{\pgfqpoint{1.651899in}{1.439835in}}%
\pgfpathlineto{\pgfqpoint{1.681691in}{1.435894in}}%
\pgfpathlineto{\pgfqpoint{1.718930in}{1.428505in}}%
\pgfpathlineto{\pgfqpoint{1.828164in}{1.405334in}}%
\pgfpathlineto{\pgfqpoint{1.865403in}{1.400364in}}%
\pgfpathlineto{\pgfqpoint{1.902642in}{1.397572in}}%
\pgfpathlineto{\pgfqpoint{1.942364in}{1.396821in}}%
\pgfpathlineto{\pgfqpoint{1.989533in}{1.398231in}}%
\pgfpathlineto{\pgfqpoint{2.068977in}{1.403317in}}%
\pgfpathlineto{\pgfqpoint{2.155868in}{1.408145in}}%
\pgfpathlineto{\pgfqpoint{2.222898in}{1.409541in}}%
\pgfpathlineto{\pgfqpoint{2.302341in}{1.408804in}}%
\pgfpathlineto{\pgfqpoint{2.558050in}{1.404692in}}%
\pgfpathlineto{\pgfqpoint{3.225870in}{1.405078in}}%
\pgfpathlineto{\pgfqpoint{3.225870in}{1.405078in}}%
\pgfusepath{stroke}%
\end{pgfscope}%
\begin{pgfscope}%
\pgfpathrectangle{\pgfqpoint{0.619136in}{0.571603in}}{\pgfqpoint{2.730864in}{1.657828in}}%
\pgfusepath{clip}%
\pgfsetrectcap%
\pgfsetroundjoin%
\pgfsetlinewidth{1.505625pt}%
\definecolor{currentstroke}{rgb}{1.000000,0.498039,0.054902}%
\pgfsetstrokecolor{currentstroke}%
\pgfsetdash{}{0pt}%
\pgfpathmoveto{\pgfqpoint{0.743267in}{0.646959in}}%
\pgfpathlineto{\pgfqpoint{0.755680in}{0.822915in}}%
\pgfpathlineto{\pgfqpoint{0.765610in}{0.935919in}}%
\pgfpathlineto{\pgfqpoint{0.775540in}{1.027777in}}%
\pgfpathlineto{\pgfqpoint{0.785471in}{1.102185in}}%
\pgfpathlineto{\pgfqpoint{0.795401in}{1.162322in}}%
\pgfpathlineto{\pgfqpoint{0.805332in}{1.210842in}}%
\pgfpathlineto{\pgfqpoint{0.815262in}{1.249933in}}%
\pgfpathlineto{\pgfqpoint{0.825192in}{1.281386in}}%
\pgfpathlineto{\pgfqpoint{0.835123in}{1.306661in}}%
\pgfpathlineto{\pgfqpoint{0.845053in}{1.326949in}}%
\pgfpathlineto{\pgfqpoint{0.857466in}{1.346744in}}%
\pgfpathlineto{\pgfqpoint{0.869879in}{1.361731in}}%
\pgfpathlineto{\pgfqpoint{0.882292in}{1.373052in}}%
\pgfpathlineto{\pgfqpoint{0.894705in}{1.381584in}}%
\pgfpathlineto{\pgfqpoint{0.909601in}{1.389073in}}%
\pgfpathlineto{\pgfqpoint{0.926979in}{1.395082in}}%
\pgfpathlineto{\pgfqpoint{0.946840in}{1.399506in}}%
\pgfpathlineto{\pgfqpoint{0.971666in}{1.402728in}}%
\pgfpathlineto{\pgfqpoint{1.006423in}{1.404858in}}%
\pgfpathlineto{\pgfqpoint{1.063522in}{1.405820in}}%
\pgfpathlineto{\pgfqpoint{1.237305in}{1.405440in}}%
\pgfpathlineto{\pgfqpoint{1.870368in}{1.404719in}}%
\pgfpathlineto{\pgfqpoint{3.225870in}{1.404470in}}%
\pgfpathlineto{\pgfqpoint{3.225870in}{1.404470in}}%
\pgfusepath{stroke}%
\end{pgfscope}%
\begin{pgfscope}%
\pgfpathrectangle{\pgfqpoint{0.619136in}{0.571603in}}{\pgfqpoint{2.730864in}{1.657828in}}%
\pgfusepath{clip}%
\pgfsetrectcap%
\pgfsetroundjoin%
\pgfsetlinewidth{1.505625pt}%
\definecolor{currentstroke}{rgb}{0.172549,0.627451,0.172549}%
\pgfsetstrokecolor{currentstroke}%
\pgfsetdash{}{0pt}%
\pgfpathmoveto{\pgfqpoint{0.743267in}{0.646959in}}%
\pgfpathlineto{\pgfqpoint{0.748232in}{0.681808in}}%
\pgfpathlineto{\pgfqpoint{0.763127in}{0.819066in}}%
\pgfpathlineto{\pgfqpoint{0.782988in}{0.997286in}}%
\pgfpathlineto{\pgfqpoint{0.797884in}{1.111468in}}%
\pgfpathlineto{\pgfqpoint{0.810297in}{1.191659in}}%
\pgfpathlineto{\pgfqpoint{0.822710in}{1.258529in}}%
\pgfpathlineto{\pgfqpoint{0.835123in}{1.313061in}}%
\pgfpathlineto{\pgfqpoint{0.847536in}{1.356557in}}%
\pgfpathlineto{\pgfqpoint{0.857466in}{1.384365in}}%
\pgfpathlineto{\pgfqpoint{0.867397in}{1.406754in}}%
\pgfpathlineto{\pgfqpoint{0.877327in}{1.424428in}}%
\pgfpathlineto{\pgfqpoint{0.887258in}{1.438052in}}%
\pgfpathlineto{\pgfqpoint{0.897188in}{1.448234in}}%
\pgfpathlineto{\pgfqpoint{0.907118in}{1.455528in}}%
\pgfpathlineto{\pgfqpoint{0.917049in}{1.460429in}}%
\pgfpathlineto{\pgfqpoint{0.929462in}{1.463848in}}%
\pgfpathlineto{\pgfqpoint{0.941875in}{1.464936in}}%
\pgfpathlineto{\pgfqpoint{0.956770in}{1.464011in}}%
\pgfpathlineto{\pgfqpoint{0.976631in}{1.460328in}}%
\pgfpathlineto{\pgfqpoint{1.008905in}{1.451600in}}%
\pgfpathlineto{\pgfqpoint{1.063522in}{1.436841in}}%
\pgfpathlineto{\pgfqpoint{1.100761in}{1.429235in}}%
\pgfpathlineto{\pgfqpoint{1.140483in}{1.423496in}}%
\pgfpathlineto{\pgfqpoint{1.187653in}{1.419113in}}%
\pgfpathlineto{\pgfqpoint{1.249718in}{1.415806in}}%
\pgfpathlineto{\pgfqpoint{1.346539in}{1.413138in}}%
\pgfpathlineto{\pgfqpoint{1.537700in}{1.410444in}}%
\pgfpathlineto{\pgfqpoint{1.867886in}{1.408195in}}%
\pgfpathlineto{\pgfqpoint{2.491019in}{1.406518in}}%
\pgfpathlineto{\pgfqpoint{3.225870in}{1.405727in}}%
\pgfpathlineto{\pgfqpoint{3.225870in}{1.405727in}}%
\pgfusepath{stroke}%
\end{pgfscope}%
\begin{pgfscope}%
\pgfpathrectangle{\pgfqpoint{0.619136in}{0.571603in}}{\pgfqpoint{2.730864in}{1.657828in}}%
\pgfusepath{clip}%
\pgfsetrectcap%
\pgfsetroundjoin%
\pgfsetlinewidth{1.505625pt}%
\definecolor{currentstroke}{rgb}{0.839216,0.152941,0.156863}%
\pgfsetstrokecolor{currentstroke}%
\pgfsetdash{}{0pt}%
\pgfpathmoveto{\pgfqpoint{0.743267in}{0.646959in}}%
\pgfpathlineto{\pgfqpoint{0.745749in}{0.654067in}}%
\pgfpathlineto{\pgfqpoint{0.750714in}{0.680088in}}%
\pgfpathlineto{\pgfqpoint{0.758162in}{0.732992in}}%
\pgfpathlineto{\pgfqpoint{0.770575in}{0.839162in}}%
\pgfpathlineto{\pgfqpoint{0.810297in}{1.191616in}}%
\pgfpathlineto{\pgfqpoint{0.825192in}{1.300587in}}%
\pgfpathlineto{\pgfqpoint{0.837605in}{1.376771in}}%
\pgfpathlineto{\pgfqpoint{0.850018in}{1.439131in}}%
\pgfpathlineto{\pgfqpoint{0.859949in}{1.479261in}}%
\pgfpathlineto{\pgfqpoint{0.869879in}{1.511163in}}%
\pgfpathlineto{\pgfqpoint{0.879810in}{1.535423in}}%
\pgfpathlineto{\pgfqpoint{0.887258in}{1.549018in}}%
\pgfpathlineto{\pgfqpoint{0.894705in}{1.559025in}}%
\pgfpathlineto{\pgfqpoint{0.902153in}{1.565777in}}%
\pgfpathlineto{\pgfqpoint{0.909601in}{1.569612in}}%
\pgfpathlineto{\pgfqpoint{0.917049in}{1.570866in}}%
\pgfpathlineto{\pgfqpoint{0.924497in}{1.569869in}}%
\pgfpathlineto{\pgfqpoint{0.931944in}{1.566940in}}%
\pgfpathlineto{\pgfqpoint{0.941875in}{1.560552in}}%
\pgfpathlineto{\pgfqpoint{0.954288in}{1.549511in}}%
\pgfpathlineto{\pgfqpoint{0.971666in}{1.530403in}}%
\pgfpathlineto{\pgfqpoint{1.031249in}{1.461564in}}%
\pgfpathlineto{\pgfqpoint{1.048627in}{1.445794in}}%
\pgfpathlineto{\pgfqpoint{1.066005in}{1.432946in}}%
\pgfpathlineto{\pgfqpoint{1.083383in}{1.423009in}}%
\pgfpathlineto{\pgfqpoint{1.100761in}{1.415763in}}%
\pgfpathlineto{\pgfqpoint{1.118140in}{1.410864in}}%
\pgfpathlineto{\pgfqpoint{1.138000in}{1.407611in}}%
\pgfpathlineto{\pgfqpoint{1.160344in}{1.406192in}}%
\pgfpathlineto{\pgfqpoint{1.190135in}{1.406624in}}%
\pgfpathlineto{\pgfqpoint{1.351504in}{1.412732in}}%
\pgfpathlineto{\pgfqpoint{1.448326in}{1.411061in}}%
\pgfpathlineto{\pgfqpoint{1.607212in}{1.408803in}}%
\pgfpathlineto{\pgfqpoint{1.979603in}{1.407046in}}%
\pgfpathlineto{\pgfqpoint{2.754175in}{1.405666in}}%
\pgfpathlineto{\pgfqpoint{3.225870in}{1.405309in}}%
\pgfpathlineto{\pgfqpoint{3.225870in}{1.405309in}}%
\pgfusepath{stroke}%
\end{pgfscope}%
\begin{pgfscope}%
\pgfpathrectangle{\pgfqpoint{0.619136in}{0.571603in}}{\pgfqpoint{2.730864in}{1.657828in}}%
\pgfusepath{clip}%
\pgfsetrectcap%
\pgfsetroundjoin%
\pgfsetlinewidth{1.505625pt}%
\definecolor{currentstroke}{rgb}{0.580392,0.403922,0.741176}%
\pgfsetstrokecolor{currentstroke}%
\pgfsetdash{}{0pt}%
\pgfpathmoveto{\pgfqpoint{0.743267in}{0.646959in}}%
\pgfpathlineto{\pgfqpoint{0.748232in}{0.648833in}}%
\pgfpathlineto{\pgfqpoint{0.753197in}{0.653771in}}%
\pgfpathlineto{\pgfqpoint{0.758162in}{0.661437in}}%
\pgfpathlineto{\pgfqpoint{0.765610in}{0.677659in}}%
\pgfpathlineto{\pgfqpoint{0.773058in}{0.699150in}}%
\pgfpathlineto{\pgfqpoint{0.782988in}{0.735327in}}%
\pgfpathlineto{\pgfqpoint{0.795401in}{0.791420in}}%
\pgfpathlineto{\pgfqpoint{0.807814in}{0.858012in}}%
\pgfpathlineto{\pgfqpoint{0.822710in}{0.949339in}}%
\pgfpathlineto{\pgfqpoint{0.842571in}{1.085405in}}%
\pgfpathlineto{\pgfqpoint{0.882292in}{1.378581in}}%
\pgfpathlineto{\pgfqpoint{0.909601in}{1.572054in}}%
\pgfpathlineto{\pgfqpoint{0.929462in}{1.697433in}}%
\pgfpathlineto{\pgfqpoint{0.944357in}{1.779331in}}%
\pgfpathlineto{\pgfqpoint{0.959253in}{1.848817in}}%
\pgfpathlineto{\pgfqpoint{0.971666in}{1.896228in}}%
\pgfpathlineto{\pgfqpoint{0.981596in}{1.926853in}}%
\pgfpathlineto{\pgfqpoint{0.991527in}{1.950759in}}%
\pgfpathlineto{\pgfqpoint{0.998975in}{1.964203in}}%
\pgfpathlineto{\pgfqpoint{1.006423in}{1.973778in}}%
\pgfpathlineto{\pgfqpoint{1.013870in}{1.979490in}}%
\pgfpathlineto{\pgfqpoint{1.018836in}{1.981166in}}%
\pgfpathlineto{\pgfqpoint{1.023801in}{1.981153in}}%
\pgfpathlineto{\pgfqpoint{1.028766in}{1.979473in}}%
\pgfpathlineto{\pgfqpoint{1.036214in}{1.973881in}}%
\pgfpathlineto{\pgfqpoint{1.043662in}{1.964696in}}%
\pgfpathlineto{\pgfqpoint{1.051109in}{1.952044in}}%
\pgfpathlineto{\pgfqpoint{1.061040in}{1.930032in}}%
\pgfpathlineto{\pgfqpoint{1.070970in}{1.902513in}}%
\pgfpathlineto{\pgfqpoint{1.083383in}{1.861063in}}%
\pgfpathlineto{\pgfqpoint{1.098279in}{1.802297in}}%
\pgfpathlineto{\pgfqpoint{1.115657in}{1.723687in}}%
\pgfpathlineto{\pgfqpoint{1.138000in}{1.611709in}}%
\pgfpathlineto{\pgfqpoint{1.202548in}{1.281233in}}%
\pgfpathlineto{\pgfqpoint{1.222409in}{1.195050in}}%
\pgfpathlineto{\pgfqpoint{1.237305in}{1.139399in}}%
\pgfpathlineto{\pgfqpoint{1.252200in}{1.092760in}}%
\pgfpathlineto{\pgfqpoint{1.264613in}{1.061422in}}%
\pgfpathlineto{\pgfqpoint{1.274544in}{1.041546in}}%
\pgfpathlineto{\pgfqpoint{1.284474in}{1.026417in}}%
\pgfpathlineto{\pgfqpoint{1.291922in}{1.018224in}}%
\pgfpathlineto{\pgfqpoint{1.299370in}{1.012739in}}%
\pgfpathlineto{\pgfqpoint{1.306817in}{1.009948in}}%
\pgfpathlineto{\pgfqpoint{1.314265in}{1.009819in}}%
\pgfpathlineto{\pgfqpoint{1.321713in}{1.012304in}}%
\pgfpathlineto{\pgfqpoint{1.329161in}{1.017338in}}%
\pgfpathlineto{\pgfqpoint{1.336609in}{1.024841in}}%
\pgfpathlineto{\pgfqpoint{1.346539in}{1.038518in}}%
\pgfpathlineto{\pgfqpoint{1.356470in}{1.056148in}}%
\pgfpathlineto{\pgfqpoint{1.368883in}{1.083282in}}%
\pgfpathlineto{\pgfqpoint{1.383778in}{1.122424in}}%
\pgfpathlineto{\pgfqpoint{1.401156in}{1.175526in}}%
\pgfpathlineto{\pgfqpoint{1.423500in}{1.252127in}}%
\pgfpathlineto{\pgfqpoint{1.495495in}{1.506714in}}%
\pgfpathlineto{\pgfqpoint{1.512874in}{1.558059in}}%
\pgfpathlineto{\pgfqpoint{1.527769in}{1.596110in}}%
\pgfpathlineto{\pgfqpoint{1.542665in}{1.627820in}}%
\pgfpathlineto{\pgfqpoint{1.555078in}{1.648966in}}%
\pgfpathlineto{\pgfqpoint{1.565008in}{1.662250in}}%
\pgfpathlineto{\pgfqpoint{1.574939in}{1.672220in}}%
\pgfpathlineto{\pgfqpoint{1.584869in}{1.678840in}}%
\pgfpathlineto{\pgfqpoint{1.592317in}{1.681608in}}%
\pgfpathlineto{\pgfqpoint{1.599765in}{1.682511in}}%
\pgfpathlineto{\pgfqpoint{1.607212in}{1.681575in}}%
\pgfpathlineto{\pgfqpoint{1.614660in}{1.678843in}}%
\pgfpathlineto{\pgfqpoint{1.624591in}{1.672492in}}%
\pgfpathlineto{\pgfqpoint{1.634521in}{1.663187in}}%
\pgfpathlineto{\pgfqpoint{1.644452in}{1.651108in}}%
\pgfpathlineto{\pgfqpoint{1.656865in}{1.632429in}}%
\pgfpathlineto{\pgfqpoint{1.671760in}{1.605377in}}%
\pgfpathlineto{\pgfqpoint{1.689138in}{1.568558in}}%
\pgfpathlineto{\pgfqpoint{1.711482in}{1.515286in}}%
\pgfpathlineto{\pgfqpoint{1.785960in}{1.331777in}}%
\pgfpathlineto{\pgfqpoint{1.803338in}{1.296326in}}%
\pgfpathlineto{\pgfqpoint{1.818234in}{1.270160in}}%
\pgfpathlineto{\pgfqpoint{1.833129in}{1.248463in}}%
\pgfpathlineto{\pgfqpoint{1.845542in}{1.234089in}}%
\pgfpathlineto{\pgfqpoint{1.857955in}{1.223256in}}%
\pgfpathlineto{\pgfqpoint{1.867886in}{1.217201in}}%
\pgfpathlineto{\pgfqpoint{1.877816in}{1.213484in}}%
\pgfpathlineto{\pgfqpoint{1.887747in}{1.212089in}}%
\pgfpathlineto{\pgfqpoint{1.897677in}{1.212977in}}%
\pgfpathlineto{\pgfqpoint{1.907607in}{1.216081in}}%
\pgfpathlineto{\pgfqpoint{1.917538in}{1.221311in}}%
\pgfpathlineto{\pgfqpoint{1.929951in}{1.230662in}}%
\pgfpathlineto{\pgfqpoint{1.942364in}{1.242891in}}%
\pgfpathlineto{\pgfqpoint{1.957260in}{1.260924in}}%
\pgfpathlineto{\pgfqpoint{1.974638in}{1.285821in}}%
\pgfpathlineto{\pgfqpoint{1.996981in}{1.322299in}}%
\pgfpathlineto{\pgfqpoint{2.081390in}{1.465600in}}%
\pgfpathlineto{\pgfqpoint{2.098768in}{1.488978in}}%
\pgfpathlineto{\pgfqpoint{2.113664in}{1.505885in}}%
\pgfpathlineto{\pgfqpoint{2.128559in}{1.519546in}}%
\pgfpathlineto{\pgfqpoint{2.140972in}{1.528277in}}%
\pgfpathlineto{\pgfqpoint{2.153385in}{1.534502in}}%
\pgfpathlineto{\pgfqpoint{2.165798in}{1.538186in}}%
\pgfpathlineto{\pgfqpoint{2.178211in}{1.539338in}}%
\pgfpathlineto{\pgfqpoint{2.190624in}{1.538016in}}%
\pgfpathlineto{\pgfqpoint{2.203037in}{1.534317in}}%
\pgfpathlineto{\pgfqpoint{2.215450in}{1.528383in}}%
\pgfpathlineto{\pgfqpoint{2.230346in}{1.518560in}}%
\pgfpathlineto{\pgfqpoint{2.245242in}{1.506135in}}%
\pgfpathlineto{\pgfqpoint{2.262620in}{1.488912in}}%
\pgfpathlineto{\pgfqpoint{2.284963in}{1.463589in}}%
\pgfpathlineto{\pgfqpoint{2.371854in}{1.360931in}}%
\pgfpathlineto{\pgfqpoint{2.391715in}{1.342754in}}%
\pgfpathlineto{\pgfqpoint{2.409093in}{1.329884in}}%
\pgfpathlineto{\pgfqpoint{2.423989in}{1.321379in}}%
\pgfpathlineto{\pgfqpoint{2.438885in}{1.315353in}}%
\pgfpathlineto{\pgfqpoint{2.453780in}{1.311870in}}%
\pgfpathlineto{\pgfqpoint{2.468676in}{1.310930in}}%
\pgfpathlineto{\pgfqpoint{2.483571in}{1.312463in}}%
\pgfpathlineto{\pgfqpoint{2.498467in}{1.316338in}}%
\pgfpathlineto{\pgfqpoint{2.513363in}{1.322370in}}%
\pgfpathlineto{\pgfqpoint{2.530741in}{1.331816in}}%
\pgfpathlineto{\pgfqpoint{2.550602in}{1.345237in}}%
\pgfpathlineto{\pgfqpoint{2.575428in}{1.364838in}}%
\pgfpathlineto{\pgfqpoint{2.662319in}{1.436125in}}%
\pgfpathlineto{\pgfqpoint{2.684662in}{1.449990in}}%
\pgfpathlineto{\pgfqpoint{2.704523in}{1.459492in}}%
\pgfpathlineto{\pgfqpoint{2.721901in}{1.465370in}}%
\pgfpathlineto{\pgfqpoint{2.739280in}{1.468853in}}%
\pgfpathlineto{\pgfqpoint{2.756658in}{1.469920in}}%
\pgfpathlineto{\pgfqpoint{2.774036in}{1.468637in}}%
\pgfpathlineto{\pgfqpoint{2.791414in}{1.465149in}}%
\pgfpathlineto{\pgfqpoint{2.811275in}{1.458741in}}%
\pgfpathlineto{\pgfqpoint{2.833619in}{1.448956in}}%
\pgfpathlineto{\pgfqpoint{2.860927in}{1.434308in}}%
\pgfpathlineto{\pgfqpoint{2.957749in}{1.379559in}}%
\pgfpathlineto{\pgfqpoint{2.980092in}{1.370524in}}%
\pgfpathlineto{\pgfqpoint{3.002436in}{1.363946in}}%
\pgfpathlineto{\pgfqpoint{3.022296in}{1.360366in}}%
\pgfpathlineto{\pgfqpoint{3.042157in}{1.358984in}}%
\pgfpathlineto{\pgfqpoint{3.062018in}{1.359759in}}%
\pgfpathlineto{\pgfqpoint{3.084361in}{1.363030in}}%
\pgfpathlineto{\pgfqpoint{3.106705in}{1.368523in}}%
\pgfpathlineto{\pgfqpoint{3.134014in}{1.377606in}}%
\pgfpathlineto{\pgfqpoint{3.173735in}{1.393581in}}%
\pgfpathlineto{\pgfqpoint{3.225870in}{1.414708in}}%
\pgfpathlineto{\pgfqpoint{3.225870in}{1.414708in}}%
\pgfusepath{stroke}%
\end{pgfscope}%
\begin{pgfscope}%
\pgfpathrectangle{\pgfqpoint{0.619136in}{0.571603in}}{\pgfqpoint{2.730864in}{1.657828in}}%
\pgfusepath{clip}%
\pgfsetrectcap%
\pgfsetroundjoin%
\pgfsetlinewidth{1.505625pt}%
\definecolor{currentstroke}{rgb}{0.549020,0.337255,0.294118}%
\pgfsetstrokecolor{currentstroke}%
\pgfsetdash{}{0pt}%
\pgfpathmoveto{\pgfqpoint{0.743267in}{0.646959in}}%
\pgfpathlineto{\pgfqpoint{0.748232in}{0.677239in}}%
\pgfpathlineto{\pgfqpoint{0.758162in}{0.765799in}}%
\pgfpathlineto{\pgfqpoint{0.792919in}{1.094552in}}%
\pgfpathlineto{\pgfqpoint{0.807814in}{1.208849in}}%
\pgfpathlineto{\pgfqpoint{0.820227in}{1.287419in}}%
\pgfpathlineto{\pgfqpoint{0.832640in}{1.351091in}}%
\pgfpathlineto{\pgfqpoint{0.842571in}{1.392014in}}%
\pgfpathlineto{\pgfqpoint{0.852501in}{1.424847in}}%
\pgfpathlineto{\pgfqpoint{0.862432in}{1.450442in}}%
\pgfpathlineto{\pgfqpoint{0.872362in}{1.469684in}}%
\pgfpathlineto{\pgfqpoint{0.882292in}{1.483453in}}%
\pgfpathlineto{\pgfqpoint{0.892223in}{1.492591in}}%
\pgfpathlineto{\pgfqpoint{0.902153in}{1.497884in}}%
\pgfpathlineto{\pgfqpoint{0.912084in}{1.500050in}}%
\pgfpathlineto{\pgfqpoint{0.922014in}{1.499732in}}%
\pgfpathlineto{\pgfqpoint{0.934427in}{1.496696in}}%
\pgfpathlineto{\pgfqpoint{0.949323in}{1.490400in}}%
\pgfpathlineto{\pgfqpoint{0.971666in}{1.478104in}}%
\pgfpathlineto{\pgfqpoint{1.018836in}{1.451559in}}%
\pgfpathlineto{\pgfqpoint{1.043662in}{1.440433in}}%
\pgfpathlineto{\pgfqpoint{1.068488in}{1.431869in}}%
\pgfpathlineto{\pgfqpoint{1.093314in}{1.425651in}}%
\pgfpathlineto{\pgfqpoint{1.123105in}{1.420674in}}%
\pgfpathlineto{\pgfqpoint{1.157861in}{1.417270in}}%
\pgfpathlineto{\pgfqpoint{1.207513in}{1.414908in}}%
\pgfpathlineto{\pgfqpoint{1.309300in}{1.412916in}}%
\pgfpathlineto{\pgfqpoint{1.624591in}{1.409150in}}%
\pgfpathlineto{\pgfqpoint{2.029255in}{1.407216in}}%
\pgfpathlineto{\pgfqpoint{2.863410in}{1.405802in}}%
\pgfpathlineto{\pgfqpoint{3.225870in}{1.405519in}}%
\pgfpathlineto{\pgfqpoint{3.225870in}{1.405519in}}%
\pgfusepath{stroke}%
\end{pgfscope}%
\begin{pgfscope}%
\pgfpathrectangle{\pgfqpoint{0.619136in}{0.571603in}}{\pgfqpoint{2.730864in}{1.657828in}}%
\pgfusepath{clip}%
\pgfsetrectcap%
\pgfsetroundjoin%
\pgfsetlinewidth{1.505625pt}%
\definecolor{currentstroke}{rgb}{0.890196,0.466667,0.760784}%
\pgfsetstrokecolor{currentstroke}%
\pgfsetdash{}{0pt}%
\pgfpathmoveto{\pgfqpoint{0.743267in}{0.646959in}}%
\pgfpathlineto{\pgfqpoint{0.768093in}{1.001714in}}%
\pgfpathlineto{\pgfqpoint{0.778023in}{1.105459in}}%
\pgfpathlineto{\pgfqpoint{0.787953in}{1.187500in}}%
\pgfpathlineto{\pgfqpoint{0.797884in}{1.251241in}}%
\pgfpathlineto{\pgfqpoint{0.807814in}{1.300016in}}%
\pgfpathlineto{\pgfqpoint{0.817745in}{1.336798in}}%
\pgfpathlineto{\pgfqpoint{0.827675in}{1.364120in}}%
\pgfpathlineto{\pgfqpoint{0.837605in}{1.384080in}}%
\pgfpathlineto{\pgfqpoint{0.847536in}{1.398380in}}%
\pgfpathlineto{\pgfqpoint{0.857466in}{1.408380in}}%
\pgfpathlineto{\pgfqpoint{0.867397in}{1.415153in}}%
\pgfpathlineto{\pgfqpoint{0.877327in}{1.419535in}}%
\pgfpathlineto{\pgfqpoint{0.889740in}{1.422620in}}%
\pgfpathlineto{\pgfqpoint{0.904636in}{1.424003in}}%
\pgfpathlineto{\pgfqpoint{0.926979in}{1.423474in}}%
\pgfpathlineto{\pgfqpoint{0.974149in}{1.419237in}}%
\pgfpathlineto{\pgfqpoint{1.036214in}{1.414483in}}%
\pgfpathlineto{\pgfqpoint{1.108209in}{1.411443in}}%
\pgfpathlineto{\pgfqpoint{1.222409in}{1.409158in}}%
\pgfpathlineto{\pgfqpoint{1.443361in}{1.407306in}}%
\pgfpathlineto{\pgfqpoint{1.920021in}{1.405920in}}%
\pgfpathlineto{\pgfqpoint{3.186148in}{1.404998in}}%
\pgfpathlineto{\pgfqpoint{3.225870in}{1.404985in}}%
\pgfpathlineto{\pgfqpoint{3.225870in}{1.404985in}}%
\pgfusepath{stroke}%
\end{pgfscope}%
\begin{pgfscope}%
\pgfpathrectangle{\pgfqpoint{0.619136in}{0.571603in}}{\pgfqpoint{2.730864in}{1.657828in}}%
\pgfusepath{clip}%
\pgfsetrectcap%
\pgfsetroundjoin%
\pgfsetlinewidth{1.505625pt}%
\definecolor{currentstroke}{rgb}{0.498039,0.498039,0.498039}%
\pgfsetstrokecolor{currentstroke}%
\pgfsetdash{}{0pt}%
\pgfpathmoveto{\pgfqpoint{0.743267in}{0.646959in}}%
\pgfpathlineto{\pgfqpoint{0.745749in}{0.648441in}}%
\pgfpathlineto{\pgfqpoint{0.750714in}{0.656653in}}%
\pgfpathlineto{\pgfqpoint{0.755680in}{0.670104in}}%
\pgfpathlineto{\pgfqpoint{0.763127in}{0.698247in}}%
\pgfpathlineto{\pgfqpoint{0.773058in}{0.747994in}}%
\pgfpathlineto{\pgfqpoint{0.785471in}{0.825623in}}%
\pgfpathlineto{\pgfqpoint{0.800366in}{0.935072in}}%
\pgfpathlineto{\pgfqpoint{0.825192in}{1.138312in}}%
\pgfpathlineto{\pgfqpoint{0.857466in}{1.400591in}}%
\pgfpathlineto{\pgfqpoint{0.874845in}{1.524986in}}%
\pgfpathlineto{\pgfqpoint{0.889740in}{1.616589in}}%
\pgfpathlineto{\pgfqpoint{0.902153in}{1.680485in}}%
\pgfpathlineto{\pgfqpoint{0.914566in}{1.732118in}}%
\pgfpathlineto{\pgfqpoint{0.924497in}{1.764283in}}%
\pgfpathlineto{\pgfqpoint{0.934427in}{1.788261in}}%
\pgfpathlineto{\pgfqpoint{0.941875in}{1.800932in}}%
\pgfpathlineto{\pgfqpoint{0.949323in}{1.809152in}}%
\pgfpathlineto{\pgfqpoint{0.954288in}{1.812227in}}%
\pgfpathlineto{\pgfqpoint{0.959253in}{1.813435in}}%
\pgfpathlineto{\pgfqpoint{0.964218in}{1.812832in}}%
\pgfpathlineto{\pgfqpoint{0.969183in}{1.810483in}}%
\pgfpathlineto{\pgfqpoint{0.976631in}{1.803836in}}%
\pgfpathlineto{\pgfqpoint{0.984079in}{1.793671in}}%
\pgfpathlineto{\pgfqpoint{0.994009in}{1.775127in}}%
\pgfpathlineto{\pgfqpoint{1.003940in}{1.751533in}}%
\pgfpathlineto{\pgfqpoint{1.016353in}{1.716085in}}%
\pgfpathlineto{\pgfqpoint{1.031249in}{1.666824in}}%
\pgfpathlineto{\pgfqpoint{1.056075in}{1.575259in}}%
\pgfpathlineto{\pgfqpoint{1.093314in}{1.437646in}}%
\pgfpathlineto{\pgfqpoint{1.110692in}{1.381149in}}%
\pgfpathlineto{\pgfqpoint{1.125587in}{1.339241in}}%
\pgfpathlineto{\pgfqpoint{1.138000in}{1.309707in}}%
\pgfpathlineto{\pgfqpoint{1.150413in}{1.285491in}}%
\pgfpathlineto{\pgfqpoint{1.160344in}{1.270089in}}%
\pgfpathlineto{\pgfqpoint{1.170274in}{1.258251in}}%
\pgfpathlineto{\pgfqpoint{1.180205in}{1.249939in}}%
\pgfpathlineto{\pgfqpoint{1.190135in}{1.245053in}}%
\pgfpathlineto{\pgfqpoint{1.197583in}{1.243548in}}%
\pgfpathlineto{\pgfqpoint{1.205031in}{1.243804in}}%
\pgfpathlineto{\pgfqpoint{1.214961in}{1.246720in}}%
\pgfpathlineto{\pgfqpoint{1.224892in}{1.252337in}}%
\pgfpathlineto{\pgfqpoint{1.234822in}{1.260368in}}%
\pgfpathlineto{\pgfqpoint{1.247235in}{1.273326in}}%
\pgfpathlineto{\pgfqpoint{1.262131in}{1.292309in}}%
\pgfpathlineto{\pgfqpoint{1.281991in}{1.321575in}}%
\pgfpathlineto{\pgfqpoint{1.341574in}{1.412579in}}%
\pgfpathlineto{\pgfqpoint{1.358952in}{1.434311in}}%
\pgfpathlineto{\pgfqpoint{1.373848in}{1.449842in}}%
\pgfpathlineto{\pgfqpoint{1.388743in}{1.462167in}}%
\pgfpathlineto{\pgfqpoint{1.401156in}{1.469866in}}%
\pgfpathlineto{\pgfqpoint{1.413569in}{1.475206in}}%
\pgfpathlineto{\pgfqpoint{1.425982in}{1.478238in}}%
\pgfpathlineto{\pgfqpoint{1.438395in}{1.479071in}}%
\pgfpathlineto{\pgfqpoint{1.450808in}{1.477868in}}%
\pgfpathlineto{\pgfqpoint{1.465704in}{1.474025in}}%
\pgfpathlineto{\pgfqpoint{1.480600in}{1.467956in}}%
\pgfpathlineto{\pgfqpoint{1.500461in}{1.457199in}}%
\pgfpathlineto{\pgfqpoint{1.527769in}{1.439380in}}%
\pgfpathlineto{\pgfqpoint{1.577421in}{1.406437in}}%
\pgfpathlineto{\pgfqpoint{1.599765in}{1.394332in}}%
\pgfpathlineto{\pgfqpoint{1.619626in}{1.385938in}}%
\pgfpathlineto{\pgfqpoint{1.639486in}{1.380059in}}%
\pgfpathlineto{\pgfqpoint{1.659347in}{1.376760in}}%
\pgfpathlineto{\pgfqpoint{1.679208in}{1.375917in}}%
\pgfpathlineto{\pgfqpoint{1.699069in}{1.377260in}}%
\pgfpathlineto{\pgfqpoint{1.721412in}{1.380900in}}%
\pgfpathlineto{\pgfqpoint{1.751203in}{1.388156in}}%
\pgfpathlineto{\pgfqpoint{1.840577in}{1.411713in}}%
\pgfpathlineto{\pgfqpoint{1.870368in}{1.416352in}}%
\pgfpathlineto{\pgfqpoint{1.897677in}{1.418453in}}%
\pgfpathlineto{\pgfqpoint{1.927468in}{1.418512in}}%
\pgfpathlineto{\pgfqpoint{1.962225in}{1.416216in}}%
\pgfpathlineto{\pgfqpoint{2.014359in}{1.410158in}}%
\pgfpathlineto{\pgfqpoint{2.081390in}{1.402742in}}%
\pgfpathlineto{\pgfqpoint{2.126077in}{1.400197in}}%
\pgfpathlineto{\pgfqpoint{2.170763in}{1.399940in}}%
\pgfpathlineto{\pgfqpoint{2.227863in}{1.402001in}}%
\pgfpathlineto{\pgfqpoint{2.354476in}{1.407206in}}%
\pgfpathlineto{\pgfqpoint{2.423989in}{1.407223in}}%
\pgfpathlineto{\pgfqpoint{2.707006in}{1.404327in}}%
\pgfpathlineto{\pgfqpoint{2.920510in}{1.405131in}}%
\pgfpathlineto{\pgfqpoint{3.225870in}{1.404666in}}%
\pgfpathlineto{\pgfqpoint{3.225870in}{1.404666in}}%
\pgfusepath{stroke}%
\end{pgfscope}%
\begin{pgfscope}%
\pgfpathrectangle{\pgfqpoint{0.619136in}{0.571603in}}{\pgfqpoint{2.730864in}{1.657828in}}%
\pgfusepath{clip}%
\pgfsetrectcap%
\pgfsetroundjoin%
\pgfsetlinewidth{1.505625pt}%
\definecolor{currentstroke}{rgb}{0.737255,0.741176,0.133333}%
\pgfsetstrokecolor{currentstroke}%
\pgfsetdash{}{0pt}%
\pgfpathmoveto{\pgfqpoint{0.743267in}{0.646959in}}%
\pgfpathlineto{\pgfqpoint{0.745749in}{0.650744in}}%
\pgfpathlineto{\pgfqpoint{0.750714in}{0.667227in}}%
\pgfpathlineto{\pgfqpoint{0.758162in}{0.704769in}}%
\pgfpathlineto{\pgfqpoint{0.768093in}{0.770179in}}%
\pgfpathlineto{\pgfqpoint{0.782988in}{0.887619in}}%
\pgfpathlineto{\pgfqpoint{0.835123in}{1.318485in}}%
\pgfpathlineto{\pgfqpoint{0.850018in}{1.418142in}}%
\pgfpathlineto{\pgfqpoint{0.862432in}{1.488198in}}%
\pgfpathlineto{\pgfqpoint{0.874845in}{1.545681in}}%
\pgfpathlineto{\pgfqpoint{0.884775in}{1.582533in}}%
\pgfpathlineto{\pgfqpoint{0.894705in}{1.611445in}}%
\pgfpathlineto{\pgfqpoint{0.904636in}{1.632757in}}%
\pgfpathlineto{\pgfqpoint{0.912084in}{1.644033in}}%
\pgfpathlineto{\pgfqpoint{0.919531in}{1.651536in}}%
\pgfpathlineto{\pgfqpoint{0.926979in}{1.655531in}}%
\pgfpathlineto{\pgfqpoint{0.934427in}{1.656303in}}%
\pgfpathlineto{\pgfqpoint{0.941875in}{1.654149in}}%
\pgfpathlineto{\pgfqpoint{0.949323in}{1.649377in}}%
\pgfpathlineto{\pgfqpoint{0.959253in}{1.639475in}}%
\pgfpathlineto{\pgfqpoint{0.969183in}{1.626200in}}%
\pgfpathlineto{\pgfqpoint{0.981596in}{1.605945in}}%
\pgfpathlineto{\pgfqpoint{0.998975in}{1.573080in}}%
\pgfpathlineto{\pgfqpoint{1.053592in}{1.466055in}}%
\pgfpathlineto{\pgfqpoint{1.070970in}{1.438064in}}%
\pgfpathlineto{\pgfqpoint{1.085866in}{1.417942in}}%
\pgfpathlineto{\pgfqpoint{1.100761in}{1.401658in}}%
\pgfpathlineto{\pgfqpoint{1.113174in}{1.391032in}}%
\pgfpathlineto{\pgfqpoint{1.125587in}{1.382989in}}%
\pgfpathlineto{\pgfqpoint{1.138000in}{1.377363in}}%
\pgfpathlineto{\pgfqpoint{1.150413in}{1.373937in}}%
\pgfpathlineto{\pgfqpoint{1.165309in}{1.372364in}}%
\pgfpathlineto{\pgfqpoint{1.180205in}{1.373103in}}%
\pgfpathlineto{\pgfqpoint{1.197583in}{1.376216in}}%
\pgfpathlineto{\pgfqpoint{1.219926in}{1.382599in}}%
\pgfpathlineto{\pgfqpoint{1.309300in}{1.410929in}}%
\pgfpathlineto{\pgfqpoint{1.336609in}{1.415793in}}%
\pgfpathlineto{\pgfqpoint{1.363917in}{1.418291in}}%
\pgfpathlineto{\pgfqpoint{1.393709in}{1.418726in}}%
\pgfpathlineto{\pgfqpoint{1.430948in}{1.416961in}}%
\pgfpathlineto{\pgfqpoint{1.587352in}{1.406911in}}%
\pgfpathlineto{\pgfqpoint{1.656865in}{1.406460in}}%
\pgfpathlineto{\pgfqpoint{1.932434in}{1.406849in}}%
\pgfpathlineto{\pgfqpoint{2.272550in}{1.405942in}}%
\pgfpathlineto{\pgfqpoint{3.225870in}{1.405070in}}%
\pgfpathlineto{\pgfqpoint{3.225870in}{1.405070in}}%
\pgfusepath{stroke}%
\end{pgfscope}%
\begin{pgfscope}%
\pgfpathrectangle{\pgfqpoint{0.619136in}{0.571603in}}{\pgfqpoint{2.730864in}{1.657828in}}%
\pgfusepath{clip}%
\pgfsetrectcap%
\pgfsetroundjoin%
\pgfsetlinewidth{1.505625pt}%
\definecolor{currentstroke}{rgb}{0.090196,0.745098,0.811765}%
\pgfsetstrokecolor{currentstroke}%
\pgfsetdash{}{0pt}%
\pgfpathmoveto{\pgfqpoint{0.743267in}{0.646959in}}%
\pgfpathlineto{\pgfqpoint{0.745749in}{0.652686in}}%
\pgfpathlineto{\pgfqpoint{0.750714in}{0.675271in}}%
\pgfpathlineto{\pgfqpoint{0.758162in}{0.723441in}}%
\pgfpathlineto{\pgfqpoint{0.770575in}{0.824345in}}%
\pgfpathlineto{\pgfqpoint{0.825192in}{1.299385in}}%
\pgfpathlineto{\pgfqpoint{0.837605in}{1.382951in}}%
\pgfpathlineto{\pgfqpoint{0.850018in}{1.452652in}}%
\pgfpathlineto{\pgfqpoint{0.859949in}{1.498230in}}%
\pgfpathlineto{\pgfqpoint{0.869879in}{1.534947in}}%
\pgfpathlineto{\pgfqpoint{0.879810in}{1.563217in}}%
\pgfpathlineto{\pgfqpoint{0.887258in}{1.579223in}}%
\pgfpathlineto{\pgfqpoint{0.894705in}{1.591098in}}%
\pgfpathlineto{\pgfqpoint{0.902153in}{1.599164in}}%
\pgfpathlineto{\pgfqpoint{0.909601in}{1.603760in}}%
\pgfpathlineto{\pgfqpoint{0.917049in}{1.605239in}}%
\pgfpathlineto{\pgfqpoint{0.924497in}{1.603958in}}%
\pgfpathlineto{\pgfqpoint{0.931944in}{1.600270in}}%
\pgfpathlineto{\pgfqpoint{0.941875in}{1.592204in}}%
\pgfpathlineto{\pgfqpoint{0.954288in}{1.578161in}}%
\pgfpathlineto{\pgfqpoint{0.969183in}{1.557348in}}%
\pgfpathlineto{\pgfqpoint{1.001457in}{1.506648in}}%
\pgfpathlineto{\pgfqpoint{1.023801in}{1.473446in}}%
\pgfpathlineto{\pgfqpoint{1.041179in}{1.451094in}}%
\pgfpathlineto{\pgfqpoint{1.056075in}{1.434991in}}%
\pgfpathlineto{\pgfqpoint{1.070970in}{1.421891in}}%
\pgfpathlineto{\pgfqpoint{1.085866in}{1.411757in}}%
\pgfpathlineto{\pgfqpoint{1.100761in}{1.404395in}}%
\pgfpathlineto{\pgfqpoint{1.115657in}{1.399508in}}%
\pgfpathlineto{\pgfqpoint{1.133035in}{1.396451in}}%
\pgfpathlineto{\pgfqpoint{1.150413in}{1.395652in}}%
\pgfpathlineto{\pgfqpoint{1.172757in}{1.396965in}}%
\pgfpathlineto{\pgfqpoint{1.205031in}{1.401358in}}%
\pgfpathlineto{\pgfqpoint{1.269578in}{1.410461in}}%
\pgfpathlineto{\pgfqpoint{1.306817in}{1.413209in}}%
\pgfpathlineto{\pgfqpoint{1.349022in}{1.413946in}}%
\pgfpathlineto{\pgfqpoint{1.408604in}{1.412458in}}%
\pgfpathlineto{\pgfqpoint{1.542665in}{1.408671in}}%
\pgfpathlineto{\pgfqpoint{1.684173in}{1.407851in}}%
\pgfpathlineto{\pgfqpoint{2.890718in}{1.405378in}}%
\pgfpathlineto{\pgfqpoint{3.225870in}{1.405168in}}%
\pgfpathlineto{\pgfqpoint{3.225870in}{1.405168in}}%
\pgfusepath{stroke}%
\end{pgfscope}%
\begin{pgfscope}%
\pgfpathrectangle{\pgfqpoint{0.619136in}{0.571603in}}{\pgfqpoint{2.730864in}{1.657828in}}%
\pgfusepath{clip}%
\pgfsetrectcap%
\pgfsetroundjoin%
\pgfsetlinewidth{1.505625pt}%
\definecolor{currentstroke}{rgb}{0.121569,0.466667,0.705882}%
\pgfsetstrokecolor{currentstroke}%
\pgfsetdash{}{0pt}%
\pgfpathmoveto{\pgfqpoint{0.743267in}{0.646959in}}%
\pgfpathlineto{\pgfqpoint{0.745749in}{0.647776in}}%
\pgfpathlineto{\pgfqpoint{0.750714in}{0.653057in}}%
\pgfpathlineto{\pgfqpoint{0.755680in}{0.662454in}}%
\pgfpathlineto{\pgfqpoint{0.763127in}{0.683382in}}%
\pgfpathlineto{\pgfqpoint{0.770575in}{0.711696in}}%
\pgfpathlineto{\pgfqpoint{0.780506in}{0.759639in}}%
\pgfpathlineto{\pgfqpoint{0.792919in}{0.833653in}}%
\pgfpathlineto{\pgfqpoint{0.807814in}{0.939019in}}%
\pgfpathlineto{\pgfqpoint{0.827675in}{1.098751in}}%
\pgfpathlineto{\pgfqpoint{0.887258in}{1.593029in}}%
\pgfpathlineto{\pgfqpoint{0.904636in}{1.711567in}}%
\pgfpathlineto{\pgfqpoint{0.917049in}{1.783174in}}%
\pgfpathlineto{\pgfqpoint{0.929462in}{1.842327in}}%
\pgfpathlineto{\pgfqpoint{0.939392in}{1.880024in}}%
\pgfpathlineto{\pgfqpoint{0.949323in}{1.908811in}}%
\pgfpathlineto{\pgfqpoint{0.956770in}{1.924456in}}%
\pgfpathlineto{\pgfqpoint{0.964218in}{1.934990in}}%
\pgfpathlineto{\pgfqpoint{0.969183in}{1.939190in}}%
\pgfpathlineto{\pgfqpoint{0.974149in}{1.941159in}}%
\pgfpathlineto{\pgfqpoint{0.979114in}{1.940927in}}%
\pgfpathlineto{\pgfqpoint{0.984079in}{1.938536in}}%
\pgfpathlineto{\pgfqpoint{0.989044in}{1.934033in}}%
\pgfpathlineto{\pgfqpoint{0.996492in}{1.923446in}}%
\pgfpathlineto{\pgfqpoint{1.003940in}{1.908457in}}%
\pgfpathlineto{\pgfqpoint{1.013870in}{1.882075in}}%
\pgfpathlineto{\pgfqpoint{1.023801in}{1.849019in}}%
\pgfpathlineto{\pgfqpoint{1.036214in}{1.799503in}}%
\pgfpathlineto{\pgfqpoint{1.051109in}{1.730272in}}%
\pgfpathlineto{\pgfqpoint{1.070970in}{1.626403in}}%
\pgfpathlineto{\pgfqpoint{1.130553in}{1.306304in}}%
\pgfpathlineto{\pgfqpoint{1.147931in}{1.229093in}}%
\pgfpathlineto{\pgfqpoint{1.162827in}{1.173714in}}%
\pgfpathlineto{\pgfqpoint{1.175240in}{1.136317in}}%
\pgfpathlineto{\pgfqpoint{1.185170in}{1.112560in}}%
\pgfpathlineto{\pgfqpoint{1.195100in}{1.094485in}}%
\pgfpathlineto{\pgfqpoint{1.202548in}{1.084711in}}%
\pgfpathlineto{\pgfqpoint{1.209996in}{1.078181in}}%
\pgfpathlineto{\pgfqpoint{1.217444in}{1.074864in}}%
\pgfpathlineto{\pgfqpoint{1.224892in}{1.074703in}}%
\pgfpathlineto{\pgfqpoint{1.232339in}{1.077609in}}%
\pgfpathlineto{\pgfqpoint{1.239787in}{1.083472in}}%
\pgfpathlineto{\pgfqpoint{1.247235in}{1.092154in}}%
\pgfpathlineto{\pgfqpoint{1.257165in}{1.107840in}}%
\pgfpathlineto{\pgfqpoint{1.267096in}{1.127825in}}%
\pgfpathlineto{\pgfqpoint{1.279509in}{1.158108in}}%
\pgfpathlineto{\pgfqpoint{1.294404in}{1.200840in}}%
\pgfpathlineto{\pgfqpoint{1.314265in}{1.265460in}}%
\pgfpathlineto{\pgfqpoint{1.376330in}{1.474381in}}%
\pgfpathlineto{\pgfqpoint{1.393709in}{1.522364in}}%
\pgfpathlineto{\pgfqpoint{1.408604in}{1.556610in}}%
\pgfpathlineto{\pgfqpoint{1.421017in}{1.579592in}}%
\pgfpathlineto{\pgfqpoint{1.430948in}{1.594077in}}%
\pgfpathlineto{\pgfqpoint{1.440878in}{1.604974in}}%
\pgfpathlineto{\pgfqpoint{1.450808in}{1.612234in}}%
\pgfpathlineto{\pgfqpoint{1.458256in}{1.615301in}}%
\pgfpathlineto{\pgfqpoint{1.465704in}{1.616356in}}%
\pgfpathlineto{\pgfqpoint{1.473152in}{1.615447in}}%
\pgfpathlineto{\pgfqpoint{1.480600in}{1.612636in}}%
\pgfpathlineto{\pgfqpoint{1.490530in}{1.606065in}}%
\pgfpathlineto{\pgfqpoint{1.500461in}{1.596478in}}%
\pgfpathlineto{\pgfqpoint{1.512874in}{1.580660in}}%
\pgfpathlineto{\pgfqpoint{1.525287in}{1.561143in}}%
\pgfpathlineto{\pgfqpoint{1.540182in}{1.533772in}}%
\pgfpathlineto{\pgfqpoint{1.562526in}{1.487216in}}%
\pgfpathlineto{\pgfqpoint{1.617143in}{1.370078in}}%
\pgfpathlineto{\pgfqpoint{1.634521in}{1.338610in}}%
\pgfpathlineto{\pgfqpoint{1.649417in}{1.315791in}}%
\pgfpathlineto{\pgfqpoint{1.661830in}{1.300184in}}%
\pgfpathlineto{\pgfqpoint{1.674243in}{1.287950in}}%
\pgfpathlineto{\pgfqpoint{1.686656in}{1.279245in}}%
\pgfpathlineto{\pgfqpoint{1.696586in}{1.274863in}}%
\pgfpathlineto{\pgfqpoint{1.706517in}{1.272765in}}%
\pgfpathlineto{\pgfqpoint{1.716447in}{1.272902in}}%
\pgfpathlineto{\pgfqpoint{1.726377in}{1.275188in}}%
\pgfpathlineto{\pgfqpoint{1.736308in}{1.279508in}}%
\pgfpathlineto{\pgfqpoint{1.748721in}{1.287544in}}%
\pgfpathlineto{\pgfqpoint{1.761134in}{1.298190in}}%
\pgfpathlineto{\pgfqpoint{1.776030in}{1.313861in}}%
\pgfpathlineto{\pgfqpoint{1.795890in}{1.338438in}}%
\pgfpathlineto{\pgfqpoint{1.830647in}{1.386265in}}%
\pgfpathlineto{\pgfqpoint{1.860438in}{1.425751in}}%
\pgfpathlineto{\pgfqpoint{1.880299in}{1.448419in}}%
\pgfpathlineto{\pgfqpoint{1.897677in}{1.464685in}}%
\pgfpathlineto{\pgfqpoint{1.912573in}{1.475498in}}%
\pgfpathlineto{\pgfqpoint{1.924986in}{1.482116in}}%
\pgfpathlineto{\pgfqpoint{1.937399in}{1.486482in}}%
\pgfpathlineto{\pgfqpoint{1.949812in}{1.488589in}}%
\pgfpathlineto{\pgfqpoint{1.962225in}{1.488485in}}%
\pgfpathlineto{\pgfqpoint{1.974638in}{1.486274in}}%
\pgfpathlineto{\pgfqpoint{1.989533in}{1.481056in}}%
\pgfpathlineto{\pgfqpoint{2.004429in}{1.473366in}}%
\pgfpathlineto{\pgfqpoint{2.021807in}{1.461824in}}%
\pgfpathlineto{\pgfqpoint{2.044151in}{1.444070in}}%
\pgfpathlineto{\pgfqpoint{2.121111in}{1.380104in}}%
\pgfpathlineto{\pgfqpoint{2.140972in}{1.367944in}}%
\pgfpathlineto{\pgfqpoint{2.158350in}{1.359933in}}%
\pgfpathlineto{\pgfqpoint{2.175729in}{1.354629in}}%
\pgfpathlineto{\pgfqpoint{2.190624in}{1.352308in}}%
\pgfpathlineto{\pgfqpoint{2.205520in}{1.352013in}}%
\pgfpathlineto{\pgfqpoint{2.222898in}{1.354095in}}%
\pgfpathlineto{\pgfqpoint{2.240276in}{1.358529in}}%
\pgfpathlineto{\pgfqpoint{2.260137in}{1.366003in}}%
\pgfpathlineto{\pgfqpoint{2.284963in}{1.377971in}}%
\pgfpathlineto{\pgfqpoint{2.374337in}{1.423843in}}%
\pgfpathlineto{\pgfqpoint{2.396680in}{1.431372in}}%
\pgfpathlineto{\pgfqpoint{2.416541in}{1.435768in}}%
\pgfpathlineto{\pgfqpoint{2.436402in}{1.437870in}}%
\pgfpathlineto{\pgfqpoint{2.456263in}{1.437708in}}%
\pgfpathlineto{\pgfqpoint{2.476124in}{1.435455in}}%
\pgfpathlineto{\pgfqpoint{2.498467in}{1.430790in}}%
\pgfpathlineto{\pgfqpoint{2.525776in}{1.422821in}}%
\pgfpathlineto{\pgfqpoint{2.627562in}{1.390523in}}%
\pgfpathlineto{\pgfqpoint{2.652388in}{1.386045in}}%
\pgfpathlineto{\pgfqpoint{2.677215in}{1.383789in}}%
\pgfpathlineto{\pgfqpoint{2.702041in}{1.383776in}}%
\pgfpathlineto{\pgfqpoint{2.729349in}{1.386111in}}%
\pgfpathlineto{\pgfqpoint{2.761623in}{1.391334in}}%
\pgfpathlineto{\pgfqpoint{2.816240in}{1.403044in}}%
\pgfpathlineto{\pgfqpoint{2.860927in}{1.411724in}}%
\pgfpathlineto{\pgfqpoint{2.893201in}{1.415902in}}%
\pgfpathlineto{\pgfqpoint{2.925475in}{1.417753in}}%
\pgfpathlineto{\pgfqpoint{2.957749in}{1.417241in}}%
\pgfpathlineto{\pgfqpoint{2.994988in}{1.414189in}}%
\pgfpathlineto{\pgfqpoint{3.049605in}{1.407041in}}%
\pgfpathlineto{\pgfqpoint{3.111670in}{1.399374in}}%
\pgfpathlineto{\pgfqpoint{3.151392in}{1.396646in}}%
\pgfpathlineto{\pgfqpoint{3.191113in}{1.396189in}}%
\pgfpathlineto{\pgfqpoint{3.225870in}{1.397504in}}%
\pgfpathlineto{\pgfqpoint{3.225870in}{1.397504in}}%
\pgfusepath{stroke}%
\end{pgfscope}%
\begin{pgfscope}%
\pgfpathrectangle{\pgfqpoint{0.619136in}{0.571603in}}{\pgfqpoint{2.730864in}{1.657828in}}%
\pgfusepath{clip}%
\pgfsetrectcap%
\pgfsetroundjoin%
\pgfsetlinewidth{1.505625pt}%
\definecolor{currentstroke}{rgb}{1.000000,0.498039,0.054902}%
\pgfsetstrokecolor{currentstroke}%
\pgfsetdash{}{0pt}%
\pgfpathmoveto{\pgfqpoint{0.743267in}{0.646959in}}%
\pgfpathlineto{\pgfqpoint{0.745749in}{0.648797in}}%
\pgfpathlineto{\pgfqpoint{0.750714in}{0.657468in}}%
\pgfpathlineto{\pgfqpoint{0.758162in}{0.678353in}}%
\pgfpathlineto{\pgfqpoint{0.768093in}{0.716690in}}%
\pgfpathlineto{\pgfqpoint{0.780506in}{0.776811in}}%
\pgfpathlineto{\pgfqpoint{0.795401in}{0.861250in}}%
\pgfpathlineto{\pgfqpoint{0.820227in}{1.017958in}}%
\pgfpathlineto{\pgfqpoint{0.859949in}{1.270268in}}%
\pgfpathlineto{\pgfqpoint{0.879810in}{1.383006in}}%
\pgfpathlineto{\pgfqpoint{0.897188in}{1.469598in}}%
\pgfpathlineto{\pgfqpoint{0.912084in}{1.533429in}}%
\pgfpathlineto{\pgfqpoint{0.926979in}{1.586974in}}%
\pgfpathlineto{\pgfqpoint{0.939392in}{1.623539in}}%
\pgfpathlineto{\pgfqpoint{0.951805in}{1.652802in}}%
\pgfpathlineto{\pgfqpoint{0.961736in}{1.671062in}}%
\pgfpathlineto{\pgfqpoint{0.971666in}{1.684899in}}%
\pgfpathlineto{\pgfqpoint{0.981596in}{1.694498in}}%
\pgfpathlineto{\pgfqpoint{0.989044in}{1.699052in}}%
\pgfpathlineto{\pgfqpoint{0.996492in}{1.701454in}}%
\pgfpathlineto{\pgfqpoint{1.003940in}{1.701822in}}%
\pgfpathlineto{\pgfqpoint{1.011388in}{1.700277in}}%
\pgfpathlineto{\pgfqpoint{1.021318in}{1.695463in}}%
\pgfpathlineto{\pgfqpoint{1.031249in}{1.687793in}}%
\pgfpathlineto{\pgfqpoint{1.043662in}{1.674682in}}%
\pgfpathlineto{\pgfqpoint{1.056075in}{1.658256in}}%
\pgfpathlineto{\pgfqpoint{1.073453in}{1.630903in}}%
\pgfpathlineto{\pgfqpoint{1.095796in}{1.590737in}}%
\pgfpathlineto{\pgfqpoint{1.165309in}{1.461619in}}%
\pgfpathlineto{\pgfqpoint{1.185170in}{1.430439in}}%
\pgfpathlineto{\pgfqpoint{1.202548in}{1.406788in}}%
\pgfpathlineto{\pgfqpoint{1.219926in}{1.386886in}}%
\pgfpathlineto{\pgfqpoint{1.234822in}{1.372935in}}%
\pgfpathlineto{\pgfqpoint{1.249718in}{1.361856in}}%
\pgfpathlineto{\pgfqpoint{1.264613in}{1.353576in}}%
\pgfpathlineto{\pgfqpoint{1.279509in}{1.347957in}}%
\pgfpathlineto{\pgfqpoint{1.294404in}{1.344804in}}%
\pgfpathlineto{\pgfqpoint{1.309300in}{1.343885in}}%
\pgfpathlineto{\pgfqpoint{1.326678in}{1.345282in}}%
\pgfpathlineto{\pgfqpoint{1.344057in}{1.348909in}}%
\pgfpathlineto{\pgfqpoint{1.366400in}{1.356099in}}%
\pgfpathlineto{\pgfqpoint{1.393709in}{1.367436in}}%
\pgfpathlineto{\pgfqpoint{1.485565in}{1.407557in}}%
\pgfpathlineto{\pgfqpoint{1.512874in}{1.416162in}}%
\pgfpathlineto{\pgfqpoint{1.540182in}{1.422369in}}%
\pgfpathlineto{\pgfqpoint{1.567491in}{1.426149in}}%
\pgfpathlineto{\pgfqpoint{1.594799in}{1.427688in}}%
\pgfpathlineto{\pgfqpoint{1.624591in}{1.427207in}}%
\pgfpathlineto{\pgfqpoint{1.661830in}{1.424271in}}%
\pgfpathlineto{\pgfqpoint{1.721412in}{1.416878in}}%
\pgfpathlineto{\pgfqpoint{1.795890in}{1.408110in}}%
\pgfpathlineto{\pgfqpoint{1.848025in}{1.404371in}}%
\pgfpathlineto{\pgfqpoint{1.900160in}{1.402912in}}%
\pgfpathlineto{\pgfqpoint{1.964707in}{1.403465in}}%
\pgfpathlineto{\pgfqpoint{2.210485in}{1.407825in}}%
\pgfpathlineto{\pgfqpoint{2.411576in}{1.406094in}}%
\pgfpathlineto{\pgfqpoint{2.627562in}{1.405714in}}%
\pgfpathlineto{\pgfqpoint{3.225870in}{1.405289in}}%
\pgfpathlineto{\pgfqpoint{3.225870in}{1.405289in}}%
\pgfusepath{stroke}%
\end{pgfscope}%
\begin{pgfscope}%
\pgfpathrectangle{\pgfqpoint{0.619136in}{0.571603in}}{\pgfqpoint{2.730864in}{1.657828in}}%
\pgfusepath{clip}%
\pgfsetrectcap%
\pgfsetroundjoin%
\pgfsetlinewidth{1.505625pt}%
\definecolor{currentstroke}{rgb}{0.172549,0.627451,0.172549}%
\pgfsetstrokecolor{currentstroke}%
\pgfsetdash{}{0pt}%
\pgfpathmoveto{\pgfqpoint{0.743267in}{0.646959in}}%
\pgfpathlineto{\pgfqpoint{0.748232in}{0.692936in}}%
\pgfpathlineto{\pgfqpoint{0.782988in}{1.067768in}}%
\pgfpathlineto{\pgfqpoint{0.795401in}{1.169750in}}%
\pgfpathlineto{\pgfqpoint{0.807814in}{1.251734in}}%
\pgfpathlineto{\pgfqpoint{0.817745in}{1.304170in}}%
\pgfpathlineto{\pgfqpoint{0.827675in}{1.346282in}}%
\pgfpathlineto{\pgfqpoint{0.837605in}{1.379422in}}%
\pgfpathlineto{\pgfqpoint{0.847536in}{1.404912in}}%
\pgfpathlineto{\pgfqpoint{0.857466in}{1.423995in}}%
\pgfpathlineto{\pgfqpoint{0.867397in}{1.437798in}}%
\pgfpathlineto{\pgfqpoint{0.877327in}{1.447318in}}%
\pgfpathlineto{\pgfqpoint{0.887258in}{1.453425in}}%
\pgfpathlineto{\pgfqpoint{0.897188in}{1.456860in}}%
\pgfpathlineto{\pgfqpoint{0.909601in}{1.458334in}}%
\pgfpathlineto{\pgfqpoint{0.924497in}{1.457240in}}%
\pgfpathlineto{\pgfqpoint{0.944357in}{1.452948in}}%
\pgfpathlineto{\pgfqpoint{1.043662in}{1.427185in}}%
\pgfpathlineto{\pgfqpoint{1.080901in}{1.421736in}}%
\pgfpathlineto{\pgfqpoint{1.125587in}{1.417680in}}%
\pgfpathlineto{\pgfqpoint{1.187653in}{1.414562in}}%
\pgfpathlineto{\pgfqpoint{1.291922in}{1.411902in}}%
\pgfpathlineto{\pgfqpoint{1.488048in}{1.409382in}}%
\pgfpathlineto{\pgfqpoint{1.835612in}{1.407405in}}%
\pgfpathlineto{\pgfqpoint{2.548119in}{1.405956in}}%
\pgfpathlineto{\pgfqpoint{3.225870in}{1.405416in}}%
\pgfpathlineto{\pgfqpoint{3.225870in}{1.405416in}}%
\pgfusepath{stroke}%
\end{pgfscope}%
\begin{pgfscope}%
\pgfpathrectangle{\pgfqpoint{0.619136in}{0.571603in}}{\pgfqpoint{2.730864in}{1.657828in}}%
\pgfusepath{clip}%
\pgfsetrectcap%
\pgfsetroundjoin%
\pgfsetlinewidth{1.505625pt}%
\definecolor{currentstroke}{rgb}{0.839216,0.152941,0.156863}%
\pgfsetstrokecolor{currentstroke}%
\pgfsetdash{}{0pt}%
\pgfpathmoveto{\pgfqpoint{0.743267in}{0.646959in}}%
\pgfpathlineto{\pgfqpoint{0.745749in}{0.649025in}}%
\pgfpathlineto{\pgfqpoint{0.750714in}{0.659627in}}%
\pgfpathlineto{\pgfqpoint{0.755680in}{0.676285in}}%
\pgfpathlineto{\pgfqpoint{0.763127in}{0.710044in}}%
\pgfpathlineto{\pgfqpoint{0.773058in}{0.767858in}}%
\pgfpathlineto{\pgfqpoint{0.785471in}{0.855232in}}%
\pgfpathlineto{\pgfqpoint{0.802849in}{0.995080in}}%
\pgfpathlineto{\pgfqpoint{0.850018in}{1.385319in}}%
\pgfpathlineto{\pgfqpoint{0.867397in}{1.506271in}}%
\pgfpathlineto{\pgfqpoint{0.882292in}{1.592867in}}%
\pgfpathlineto{\pgfqpoint{0.894705in}{1.651472in}}%
\pgfpathlineto{\pgfqpoint{0.904636in}{1.689064in}}%
\pgfpathlineto{\pgfqpoint{0.914566in}{1.718326in}}%
\pgfpathlineto{\pgfqpoint{0.922014in}{1.734871in}}%
\pgfpathlineto{\pgfqpoint{0.929462in}{1.746903in}}%
\pgfpathlineto{\pgfqpoint{0.936910in}{1.754575in}}%
\pgfpathlineto{\pgfqpoint{0.941875in}{1.757362in}}%
\pgfpathlineto{\pgfqpoint{0.946840in}{1.758362in}}%
\pgfpathlineto{\pgfqpoint{0.951805in}{1.757648in}}%
\pgfpathlineto{\pgfqpoint{0.959253in}{1.753534in}}%
\pgfpathlineto{\pgfqpoint{0.966701in}{1.746020in}}%
\pgfpathlineto{\pgfqpoint{0.974149in}{1.735408in}}%
\pgfpathlineto{\pgfqpoint{0.984079in}{1.716991in}}%
\pgfpathlineto{\pgfqpoint{0.996492in}{1.688206in}}%
\pgfpathlineto{\pgfqpoint{1.011388in}{1.647255in}}%
\pgfpathlineto{\pgfqpoint{1.033731in}{1.578172in}}%
\pgfpathlineto{\pgfqpoint{1.070970in}{1.462125in}}%
\pgfpathlineto{\pgfqpoint{1.088348in}{1.414563in}}%
\pgfpathlineto{\pgfqpoint{1.103244in}{1.379415in}}%
\pgfpathlineto{\pgfqpoint{1.115657in}{1.354726in}}%
\pgfpathlineto{\pgfqpoint{1.128070in}{1.334521in}}%
\pgfpathlineto{\pgfqpoint{1.140483in}{1.318909in}}%
\pgfpathlineto{\pgfqpoint{1.150413in}{1.309702in}}%
\pgfpathlineto{\pgfqpoint{1.160344in}{1.303327in}}%
\pgfpathlineto{\pgfqpoint{1.170274in}{1.299653in}}%
\pgfpathlineto{\pgfqpoint{1.180205in}{1.298508in}}%
\pgfpathlineto{\pgfqpoint{1.190135in}{1.299686in}}%
\pgfpathlineto{\pgfqpoint{1.200066in}{1.302953in}}%
\pgfpathlineto{\pgfqpoint{1.212479in}{1.309589in}}%
\pgfpathlineto{\pgfqpoint{1.227374in}{1.320600in}}%
\pgfpathlineto{\pgfqpoint{1.244752in}{1.336482in}}%
\pgfpathlineto{\pgfqpoint{1.274544in}{1.367386in}}%
\pgfpathlineto{\pgfqpoint{1.306817in}{1.399883in}}%
\pgfpathlineto{\pgfqpoint{1.326678in}{1.416815in}}%
\pgfpathlineto{\pgfqpoint{1.344057in}{1.428834in}}%
\pgfpathlineto{\pgfqpoint{1.358952in}{1.436795in}}%
\pgfpathlineto{\pgfqpoint{1.373848in}{1.442517in}}%
\pgfpathlineto{\pgfqpoint{1.388743in}{1.446038in}}%
\pgfpathlineto{\pgfqpoint{1.403639in}{1.447485in}}%
\pgfpathlineto{\pgfqpoint{1.421017in}{1.446822in}}%
\pgfpathlineto{\pgfqpoint{1.440878in}{1.443475in}}%
\pgfpathlineto{\pgfqpoint{1.463222in}{1.437258in}}%
\pgfpathlineto{\pgfqpoint{1.497978in}{1.424834in}}%
\pgfpathlineto{\pgfqpoint{1.545147in}{1.408198in}}%
\pgfpathlineto{\pgfqpoint{1.572456in}{1.400904in}}%
\pgfpathlineto{\pgfqpoint{1.597282in}{1.396423in}}%
\pgfpathlineto{\pgfqpoint{1.622108in}{1.394069in}}%
\pgfpathlineto{\pgfqpoint{1.649417in}{1.393710in}}%
\pgfpathlineto{\pgfqpoint{1.681691in}{1.395622in}}%
\pgfpathlineto{\pgfqpoint{1.731343in}{1.401180in}}%
\pgfpathlineto{\pgfqpoint{1.795890in}{1.408046in}}%
\pgfpathlineto{\pgfqpoint{1.840577in}{1.410401in}}%
\pgfpathlineto{\pgfqpoint{1.887747in}{1.410500in}}%
\pgfpathlineto{\pgfqpoint{1.954777in}{1.408056in}}%
\pgfpathlineto{\pgfqpoint{2.056564in}{1.404476in}}%
\pgfpathlineto{\pgfqpoint{2.136007in}{1.404237in}}%
\pgfpathlineto{\pgfqpoint{2.411576in}{1.405530in}}%
\pgfpathlineto{\pgfqpoint{2.677215in}{1.405027in}}%
\pgfpathlineto{\pgfqpoint{3.225870in}{1.404804in}}%
\pgfpathlineto{\pgfqpoint{3.225870in}{1.404804in}}%
\pgfusepath{stroke}%
\end{pgfscope}%
\begin{pgfscope}%
\pgfpathrectangle{\pgfqpoint{0.619136in}{0.571603in}}{\pgfqpoint{2.730864in}{1.657828in}}%
\pgfusepath{clip}%
\pgfsetrectcap%
\pgfsetroundjoin%
\pgfsetlinewidth{1.505625pt}%
\definecolor{currentstroke}{rgb}{0.580392,0.403922,0.741176}%
\pgfsetstrokecolor{currentstroke}%
\pgfsetdash{}{0pt}%
\pgfpathmoveto{\pgfqpoint{0.743267in}{0.646959in}}%
\pgfpathlineto{\pgfqpoint{0.748232in}{0.704834in}}%
\pgfpathlineto{\pgfqpoint{0.770575in}{0.994004in}}%
\pgfpathlineto{\pgfqpoint{0.782988in}{1.122027in}}%
\pgfpathlineto{\pgfqpoint{0.792919in}{1.204432in}}%
\pgfpathlineto{\pgfqpoint{0.802849in}{1.270541in}}%
\pgfpathlineto{\pgfqpoint{0.812779in}{1.322374in}}%
\pgfpathlineto{\pgfqpoint{0.822710in}{1.362089in}}%
\pgfpathlineto{\pgfqpoint{0.832640in}{1.391768in}}%
\pgfpathlineto{\pgfqpoint{0.842571in}{1.413305in}}%
\pgfpathlineto{\pgfqpoint{0.852501in}{1.428365in}}%
\pgfpathlineto{\pgfqpoint{0.862432in}{1.438366in}}%
\pgfpathlineto{\pgfqpoint{0.872362in}{1.444495in}}%
\pgfpathlineto{\pgfqpoint{0.882292in}{1.447723in}}%
\pgfpathlineto{\pgfqpoint{0.894705in}{1.448855in}}%
\pgfpathlineto{\pgfqpoint{0.909601in}{1.447460in}}%
\pgfpathlineto{\pgfqpoint{0.931944in}{1.442542in}}%
\pgfpathlineto{\pgfqpoint{0.996492in}{1.426985in}}%
\pgfpathlineto{\pgfqpoint{1.031249in}{1.421522in}}%
\pgfpathlineto{\pgfqpoint{1.073453in}{1.417333in}}%
\pgfpathlineto{\pgfqpoint{1.130553in}{1.414157in}}%
\pgfpathlineto{\pgfqpoint{1.224892in}{1.411488in}}%
\pgfpathlineto{\pgfqpoint{1.398674in}{1.409094in}}%
\pgfpathlineto{\pgfqpoint{1.716447in}{1.407205in}}%
\pgfpathlineto{\pgfqpoint{2.386750in}{1.405825in}}%
\pgfpathlineto{\pgfqpoint{3.225870in}{1.405224in}}%
\pgfpathlineto{\pgfqpoint{3.225870in}{1.405224in}}%
\pgfusepath{stroke}%
\end{pgfscope}%
\begin{pgfscope}%
\pgfpathrectangle{\pgfqpoint{0.619136in}{0.571603in}}{\pgfqpoint{2.730864in}{1.657828in}}%
\pgfusepath{clip}%
\pgfsetrectcap%
\pgfsetroundjoin%
\pgfsetlinewidth{1.505625pt}%
\definecolor{currentstroke}{rgb}{0.549020,0.337255,0.294118}%
\pgfsetstrokecolor{currentstroke}%
\pgfsetdash{}{0pt}%
\pgfpathmoveto{\pgfqpoint{0.743267in}{0.646959in}}%
\pgfpathlineto{\pgfqpoint{0.745749in}{0.648774in}}%
\pgfpathlineto{\pgfqpoint{0.750714in}{0.657862in}}%
\pgfpathlineto{\pgfqpoint{0.755680in}{0.671968in}}%
\pgfpathlineto{\pgfqpoint{0.763127in}{0.700356in}}%
\pgfpathlineto{\pgfqpoint{0.773058in}{0.748799in}}%
\pgfpathlineto{\pgfqpoint{0.785471in}{0.822104in}}%
\pgfpathlineto{\pgfqpoint{0.802849in}{0.940471in}}%
\pgfpathlineto{\pgfqpoint{0.869879in}{1.416711in}}%
\pgfpathlineto{\pgfqpoint{0.887258in}{1.516160in}}%
\pgfpathlineto{\pgfqpoint{0.902153in}{1.587801in}}%
\pgfpathlineto{\pgfqpoint{0.914566in}{1.637093in}}%
\pgfpathlineto{\pgfqpoint{0.926979in}{1.676629in}}%
\pgfpathlineto{\pgfqpoint{0.936910in}{1.701228in}}%
\pgfpathlineto{\pgfqpoint{0.946840in}{1.719696in}}%
\pgfpathlineto{\pgfqpoint{0.954288in}{1.729642in}}%
\pgfpathlineto{\pgfqpoint{0.961736in}{1.736364in}}%
\pgfpathlineto{\pgfqpoint{0.969183in}{1.739998in}}%
\pgfpathlineto{\pgfqpoint{0.976631in}{1.740697in}}%
\pgfpathlineto{\pgfqpoint{0.984079in}{1.738632in}}%
\pgfpathlineto{\pgfqpoint{0.991527in}{1.733988in}}%
\pgfpathlineto{\pgfqpoint{0.998975in}{1.726959in}}%
\pgfpathlineto{\pgfqpoint{1.008905in}{1.714232in}}%
\pgfpathlineto{\pgfqpoint{1.021318in}{1.693639in}}%
\pgfpathlineto{\pgfqpoint{1.036214in}{1.663378in}}%
\pgfpathlineto{\pgfqpoint{1.053592in}{1.622598in}}%
\pgfpathlineto{\pgfqpoint{1.085866in}{1.539696in}}%
\pgfpathlineto{\pgfqpoint{1.115657in}{1.465347in}}%
\pgfpathlineto{\pgfqpoint{1.135518in}{1.421461in}}%
\pgfpathlineto{\pgfqpoint{1.152896in}{1.388471in}}%
\pgfpathlineto{\pgfqpoint{1.167792in}{1.364821in}}%
\pgfpathlineto{\pgfqpoint{1.182687in}{1.345716in}}%
\pgfpathlineto{\pgfqpoint{1.195100in}{1.333322in}}%
\pgfpathlineto{\pgfqpoint{1.207513in}{1.324088in}}%
\pgfpathlineto{\pgfqpoint{1.219926in}{1.317898in}}%
\pgfpathlineto{\pgfqpoint{1.232339in}{1.314577in}}%
\pgfpathlineto{\pgfqpoint{1.244752in}{1.313901in}}%
\pgfpathlineto{\pgfqpoint{1.257165in}{1.315607in}}%
\pgfpathlineto{\pgfqpoint{1.269578in}{1.319404in}}%
\pgfpathlineto{\pgfqpoint{1.284474in}{1.326281in}}%
\pgfpathlineto{\pgfqpoint{1.301852in}{1.336804in}}%
\pgfpathlineto{\pgfqpoint{1.326678in}{1.354857in}}%
\pgfpathlineto{\pgfqpoint{1.391226in}{1.403529in}}%
\pgfpathlineto{\pgfqpoint{1.413569in}{1.416995in}}%
\pgfpathlineto{\pgfqpoint{1.433430in}{1.426505in}}%
\pgfpathlineto{\pgfqpoint{1.453291in}{1.433491in}}%
\pgfpathlineto{\pgfqpoint{1.473152in}{1.437937in}}%
\pgfpathlineto{\pgfqpoint{1.493013in}{1.439979in}}%
\pgfpathlineto{\pgfqpoint{1.512874in}{1.439869in}}%
\pgfpathlineto{\pgfqpoint{1.535217in}{1.437601in}}%
\pgfpathlineto{\pgfqpoint{1.562526in}{1.432536in}}%
\pgfpathlineto{\pgfqpoint{1.604730in}{1.422082in}}%
\pgfpathlineto{\pgfqpoint{1.661830in}{1.408231in}}%
\pgfpathlineto{\pgfqpoint{1.696586in}{1.402144in}}%
\pgfpathlineto{\pgfqpoint{1.728860in}{1.398683in}}%
\pgfpathlineto{\pgfqpoint{1.763617in}{1.397266in}}%
\pgfpathlineto{\pgfqpoint{1.803338in}{1.398038in}}%
\pgfpathlineto{\pgfqpoint{1.860438in}{1.401738in}}%
\pgfpathlineto{\pgfqpoint{1.957260in}{1.408056in}}%
\pgfpathlineto{\pgfqpoint{2.016842in}{1.409405in}}%
\pgfpathlineto{\pgfqpoint{2.088838in}{1.408569in}}%
\pgfpathlineto{\pgfqpoint{2.294894in}{1.404774in}}%
\pgfpathlineto{\pgfqpoint{2.483571in}{1.405674in}}%
\pgfpathlineto{\pgfqpoint{2.711971in}{1.405306in}}%
\pgfpathlineto{\pgfqpoint{3.206009in}{1.404997in}}%
\pgfpathlineto{\pgfqpoint{3.225870in}{1.404983in}}%
\pgfpathlineto{\pgfqpoint{3.225870in}{1.404983in}}%
\pgfusepath{stroke}%
\end{pgfscope}%
\begin{pgfscope}%
\pgfpathrectangle{\pgfqpoint{0.619136in}{0.571603in}}{\pgfqpoint{2.730864in}{1.657828in}}%
\pgfusepath{clip}%
\pgfsetrectcap%
\pgfsetroundjoin%
\pgfsetlinewidth{1.505625pt}%
\definecolor{currentstroke}{rgb}{0.890196,0.466667,0.760784}%
\pgfsetstrokecolor{currentstroke}%
\pgfsetdash{}{0pt}%
\pgfpathmoveto{\pgfqpoint{0.743267in}{0.646959in}}%
\pgfpathlineto{\pgfqpoint{0.745749in}{0.652324in}}%
\pgfpathlineto{\pgfqpoint{0.750714in}{0.671244in}}%
\pgfpathlineto{\pgfqpoint{0.758162in}{0.709166in}}%
\pgfpathlineto{\pgfqpoint{0.770575in}{0.785509in}}%
\pgfpathlineto{\pgfqpoint{0.802849in}{1.006287in}}%
\pgfpathlineto{\pgfqpoint{0.825192in}{1.150113in}}%
\pgfpathlineto{\pgfqpoint{0.842571in}{1.248466in}}%
\pgfpathlineto{\pgfqpoint{0.857466in}{1.321421in}}%
\pgfpathlineto{\pgfqpoint{0.872362in}{1.383357in}}%
\pgfpathlineto{\pgfqpoint{0.884775in}{1.426598in}}%
\pgfpathlineto{\pgfqpoint{0.897188in}{1.462505in}}%
\pgfpathlineto{\pgfqpoint{0.909601in}{1.491494in}}%
\pgfpathlineto{\pgfqpoint{0.922014in}{1.514082in}}%
\pgfpathlineto{\pgfqpoint{0.931944in}{1.527936in}}%
\pgfpathlineto{\pgfqpoint{0.941875in}{1.538387in}}%
\pgfpathlineto{\pgfqpoint{0.951805in}{1.545762in}}%
\pgfpathlineto{\pgfqpoint{0.961736in}{1.550388in}}%
\pgfpathlineto{\pgfqpoint{0.971666in}{1.552585in}}%
\pgfpathlineto{\pgfqpoint{0.981596in}{1.552669in}}%
\pgfpathlineto{\pgfqpoint{0.994009in}{1.550254in}}%
\pgfpathlineto{\pgfqpoint{1.006423in}{1.545550in}}%
\pgfpathlineto{\pgfqpoint{1.021318in}{1.537591in}}%
\pgfpathlineto{\pgfqpoint{1.041179in}{1.524302in}}%
\pgfpathlineto{\pgfqpoint{1.078418in}{1.495875in}}%
\pgfpathlineto{\pgfqpoint{1.115657in}{1.468530in}}%
\pgfpathlineto{\pgfqpoint{1.140483in}{1.452871in}}%
\pgfpathlineto{\pgfqpoint{1.165309in}{1.439871in}}%
\pgfpathlineto{\pgfqpoint{1.187653in}{1.430526in}}%
\pgfpathlineto{\pgfqpoint{1.212479in}{1.422619in}}%
\pgfpathlineto{\pgfqpoint{1.237305in}{1.417021in}}%
\pgfpathlineto{\pgfqpoint{1.264613in}{1.413067in}}%
\pgfpathlineto{\pgfqpoint{1.296887in}{1.410651in}}%
\pgfpathlineto{\pgfqpoint{1.339091in}{1.409874in}}%
\pgfpathlineto{\pgfqpoint{1.416052in}{1.411227in}}%
\pgfpathlineto{\pgfqpoint{1.525287in}{1.412591in}}%
\pgfpathlineto{\pgfqpoint{1.654382in}{1.411540in}}%
\pgfpathlineto{\pgfqpoint{1.957260in}{1.408762in}}%
\pgfpathlineto{\pgfqpoint{2.540671in}{1.406882in}}%
\pgfpathlineto{\pgfqpoint{3.225870in}{1.405938in}}%
\pgfpathlineto{\pgfqpoint{3.225870in}{1.405938in}}%
\pgfusepath{stroke}%
\end{pgfscope}%
\begin{pgfscope}%
\pgfpathrectangle{\pgfqpoint{0.619136in}{0.571603in}}{\pgfqpoint{2.730864in}{1.657828in}}%
\pgfusepath{clip}%
\pgfsetrectcap%
\pgfsetroundjoin%
\pgfsetlinewidth{1.505625pt}%
\definecolor{currentstroke}{rgb}{0.498039,0.498039,0.498039}%
\pgfsetstrokecolor{currentstroke}%
\pgfsetdash{}{0pt}%
\pgfpathmoveto{\pgfqpoint{0.743267in}{0.646959in}}%
\pgfpathlineto{\pgfqpoint{0.745749in}{0.651078in}}%
\pgfpathlineto{\pgfqpoint{0.750714in}{0.668361in}}%
\pgfpathlineto{\pgfqpoint{0.758162in}{0.706821in}}%
\pgfpathlineto{\pgfqpoint{0.768093in}{0.772665in}}%
\pgfpathlineto{\pgfqpoint{0.782988in}{0.889044in}}%
\pgfpathlineto{\pgfqpoint{0.830158in}{1.271385in}}%
\pgfpathlineto{\pgfqpoint{0.845053in}{1.371916in}}%
\pgfpathlineto{\pgfqpoint{0.859949in}{1.456626in}}%
\pgfpathlineto{\pgfqpoint{0.872362in}{1.514250in}}%
\pgfpathlineto{\pgfqpoint{0.882292in}{1.551774in}}%
\pgfpathlineto{\pgfqpoint{0.892223in}{1.581848in}}%
\pgfpathlineto{\pgfqpoint{0.902153in}{1.604789in}}%
\pgfpathlineto{\pgfqpoint{0.909601in}{1.617573in}}%
\pgfpathlineto{\pgfqpoint{0.917049in}{1.626807in}}%
\pgfpathlineto{\pgfqpoint{0.924497in}{1.632733in}}%
\pgfpathlineto{\pgfqpoint{0.931944in}{1.635609in}}%
\pgfpathlineto{\pgfqpoint{0.939392in}{1.635706in}}%
\pgfpathlineto{\pgfqpoint{0.946840in}{1.633299in}}%
\pgfpathlineto{\pgfqpoint{0.954288in}{1.628669in}}%
\pgfpathlineto{\pgfqpoint{0.964218in}{1.619519in}}%
\pgfpathlineto{\pgfqpoint{0.976631in}{1.604201in}}%
\pgfpathlineto{\pgfqpoint{0.991527in}{1.581702in}}%
\pgfpathlineto{\pgfqpoint{1.016353in}{1.538995in}}%
\pgfpathlineto{\pgfqpoint{1.051109in}{1.479644in}}%
\pgfpathlineto{\pgfqpoint{1.070970in}{1.450394in}}%
\pgfpathlineto{\pgfqpoint{1.088348in}{1.428961in}}%
\pgfpathlineto{\pgfqpoint{1.103244in}{1.413991in}}%
\pgfpathlineto{\pgfqpoint{1.118140in}{1.402197in}}%
\pgfpathlineto{\pgfqpoint{1.133035in}{1.393454in}}%
\pgfpathlineto{\pgfqpoint{1.147931in}{1.387524in}}%
\pgfpathlineto{\pgfqpoint{1.162827in}{1.384090in}}%
\pgfpathlineto{\pgfqpoint{1.177722in}{1.382782in}}%
\pgfpathlineto{\pgfqpoint{1.195100in}{1.383423in}}%
\pgfpathlineto{\pgfqpoint{1.217444in}{1.386730in}}%
\pgfpathlineto{\pgfqpoint{1.249718in}{1.394188in}}%
\pgfpathlineto{\pgfqpoint{1.306817in}{1.407647in}}%
\pgfpathlineto{\pgfqpoint{1.339091in}{1.412784in}}%
\pgfpathlineto{\pgfqpoint{1.371365in}{1.415585in}}%
\pgfpathlineto{\pgfqpoint{1.406122in}{1.416318in}}%
\pgfpathlineto{\pgfqpoint{1.450808in}{1.414934in}}%
\pgfpathlineto{\pgfqpoint{1.617143in}{1.407718in}}%
\pgfpathlineto{\pgfqpoint{1.716447in}{1.407324in}}%
\pgfpathlineto{\pgfqpoint{1.987051in}{1.406856in}}%
\pgfpathlineto{\pgfqpoint{2.493502in}{1.405793in}}%
\pgfpathlineto{\pgfqpoint{3.225870in}{1.405176in}}%
\pgfpathlineto{\pgfqpoint{3.225870in}{1.405176in}}%
\pgfusepath{stroke}%
\end{pgfscope}%
\begin{pgfscope}%
\pgfpathrectangle{\pgfqpoint{0.619136in}{0.571603in}}{\pgfqpoint{2.730864in}{1.657828in}}%
\pgfusepath{clip}%
\pgfsetrectcap%
\pgfsetroundjoin%
\pgfsetlinewidth{1.505625pt}%
\definecolor{currentstroke}{rgb}{0.737255,0.741176,0.133333}%
\pgfsetstrokecolor{currentstroke}%
\pgfsetdash{}{0pt}%
\pgfpathmoveto{\pgfqpoint{0.743267in}{0.646959in}}%
\pgfpathlineto{\pgfqpoint{0.748232in}{0.649480in}}%
\pgfpathlineto{\pgfqpoint{0.753197in}{0.656009in}}%
\pgfpathlineto{\pgfqpoint{0.758162in}{0.666042in}}%
\pgfpathlineto{\pgfqpoint{0.765610in}{0.687078in}}%
\pgfpathlineto{\pgfqpoint{0.773058in}{0.714691in}}%
\pgfpathlineto{\pgfqpoint{0.782988in}{0.760709in}}%
\pgfpathlineto{\pgfqpoint{0.795401in}{0.831135in}}%
\pgfpathlineto{\pgfqpoint{0.810297in}{0.931130in}}%
\pgfpathlineto{\pgfqpoint{0.830158in}{1.083205in}}%
\pgfpathlineto{\pgfqpoint{0.899671in}{1.636935in}}%
\pgfpathlineto{\pgfqpoint{0.914566in}{1.733264in}}%
\pgfpathlineto{\pgfqpoint{0.929462in}{1.814666in}}%
\pgfpathlineto{\pgfqpoint{0.941875in}{1.869596in}}%
\pgfpathlineto{\pgfqpoint{0.951805in}{1.904469in}}%
\pgfpathlineto{\pgfqpoint{0.961736in}{1.930969in}}%
\pgfpathlineto{\pgfqpoint{0.969183in}{1.945260in}}%
\pgfpathlineto{\pgfqpoint{0.976631in}{1.954749in}}%
\pgfpathlineto{\pgfqpoint{0.981596in}{1.958421in}}%
\pgfpathlineto{\pgfqpoint{0.986562in}{1.959989in}}%
\pgfpathlineto{\pgfqpoint{0.991527in}{1.959481in}}%
\pgfpathlineto{\pgfqpoint{0.996492in}{1.956931in}}%
\pgfpathlineto{\pgfqpoint{1.001457in}{1.952379in}}%
\pgfpathlineto{\pgfqpoint{1.008905in}{1.941905in}}%
\pgfpathlineto{\pgfqpoint{1.016353in}{1.927228in}}%
\pgfpathlineto{\pgfqpoint{1.026283in}{1.901503in}}%
\pgfpathlineto{\pgfqpoint{1.036214in}{1.869299in}}%
\pgfpathlineto{\pgfqpoint{1.048627in}{1.820973in}}%
\pgfpathlineto{\pgfqpoint{1.063522in}{1.753096in}}%
\pgfpathlineto{\pgfqpoint{1.083383in}{1.650367in}}%
\pgfpathlineto{\pgfqpoint{1.155379in}{1.263074in}}%
\pgfpathlineto{\pgfqpoint{1.170274in}{1.198318in}}%
\pgfpathlineto{\pgfqpoint{1.185170in}{1.143567in}}%
\pgfpathlineto{\pgfqpoint{1.197583in}{1.106563in}}%
\pgfpathlineto{\pgfqpoint{1.207513in}{1.083010in}}%
\pgfpathlineto{\pgfqpoint{1.217444in}{1.065034in}}%
\pgfpathlineto{\pgfqpoint{1.224892in}{1.055270in}}%
\pgfpathlineto{\pgfqpoint{1.232339in}{1.048701in}}%
\pgfpathlineto{\pgfqpoint{1.239787in}{1.045306in}}%
\pgfpathlineto{\pgfqpoint{1.247235in}{1.045037in}}%
\pgfpathlineto{\pgfqpoint{1.254683in}{1.047822in}}%
\pgfpathlineto{\pgfqpoint{1.262131in}{1.053564in}}%
\pgfpathlineto{\pgfqpoint{1.269578in}{1.062144in}}%
\pgfpathlineto{\pgfqpoint{1.279509in}{1.077754in}}%
\pgfpathlineto{\pgfqpoint{1.289439in}{1.097773in}}%
\pgfpathlineto{\pgfqpoint{1.301852in}{1.128325in}}%
\pgfpathlineto{\pgfqpoint{1.316748in}{1.171825in}}%
\pgfpathlineto{\pgfqpoint{1.334126in}{1.229711in}}%
\pgfpathlineto{\pgfqpoint{1.363917in}{1.338380in}}%
\pgfpathlineto{\pgfqpoint{1.396191in}{1.454279in}}%
\pgfpathlineto{\pgfqpoint{1.416052in}{1.517256in}}%
\pgfpathlineto{\pgfqpoint{1.430948in}{1.557814in}}%
\pgfpathlineto{\pgfqpoint{1.443361in}{1.586349in}}%
\pgfpathlineto{\pgfqpoint{1.455774in}{1.609589in}}%
\pgfpathlineto{\pgfqpoint{1.465704in}{1.624140in}}%
\pgfpathlineto{\pgfqpoint{1.475635in}{1.634985in}}%
\pgfpathlineto{\pgfqpoint{1.485565in}{1.642076in}}%
\pgfpathlineto{\pgfqpoint{1.493013in}{1.644938in}}%
\pgfpathlineto{\pgfqpoint{1.500461in}{1.645720in}}%
\pgfpathlineto{\pgfqpoint{1.507908in}{1.644466in}}%
\pgfpathlineto{\pgfqpoint{1.515356in}{1.641234in}}%
\pgfpathlineto{\pgfqpoint{1.525287in}{1.633977in}}%
\pgfpathlineto{\pgfqpoint{1.535217in}{1.623551in}}%
\pgfpathlineto{\pgfqpoint{1.547630in}{1.606454in}}%
\pgfpathlineto{\pgfqpoint{1.560043in}{1.585379in}}%
\pgfpathlineto{\pgfqpoint{1.574939in}{1.555750in}}%
\pgfpathlineto{\pgfqpoint{1.594799in}{1.510876in}}%
\pgfpathlineto{\pgfqpoint{1.666795in}{1.341617in}}%
\pgfpathlineto{\pgfqpoint{1.684173in}{1.309019in}}%
\pgfpathlineto{\pgfqpoint{1.699069in}{1.285900in}}%
\pgfpathlineto{\pgfqpoint{1.711482in}{1.270473in}}%
\pgfpathlineto{\pgfqpoint{1.723895in}{1.258772in}}%
\pgfpathlineto{\pgfqpoint{1.733825in}{1.252181in}}%
\pgfpathlineto{\pgfqpoint{1.743756in}{1.248076in}}%
\pgfpathlineto{\pgfqpoint{1.753686in}{1.246441in}}%
\pgfpathlineto{\pgfqpoint{1.763617in}{1.247221in}}%
\pgfpathlineto{\pgfqpoint{1.773547in}{1.250328in}}%
\pgfpathlineto{\pgfqpoint{1.783477in}{1.255641in}}%
\pgfpathlineto{\pgfqpoint{1.795890in}{1.265158in}}%
\pgfpathlineto{\pgfqpoint{1.808303in}{1.277542in}}%
\pgfpathlineto{\pgfqpoint{1.823199in}{1.295622in}}%
\pgfpathlineto{\pgfqpoint{1.843060in}{1.323917in}}%
\pgfpathlineto{\pgfqpoint{1.872851in}{1.371320in}}%
\pgfpathlineto{\pgfqpoint{1.910090in}{1.429817in}}%
\pgfpathlineto{\pgfqpoint{1.929951in}{1.456914in}}%
\pgfpathlineto{\pgfqpoint{1.947329in}{1.476796in}}%
\pgfpathlineto{\pgfqpoint{1.962225in}{1.490425in}}%
\pgfpathlineto{\pgfqpoint{1.974638in}{1.499137in}}%
\pgfpathlineto{\pgfqpoint{1.987051in}{1.505325in}}%
\pgfpathlineto{\pgfqpoint{1.999464in}{1.508946in}}%
\pgfpathlineto{\pgfqpoint{2.011877in}{1.510016in}}%
\pgfpathlineto{\pgfqpoint{2.024290in}{1.508611in}}%
\pgfpathlineto{\pgfqpoint{2.036703in}{1.504860in}}%
\pgfpathlineto{\pgfqpoint{2.049116in}{1.498941in}}%
\pgfpathlineto{\pgfqpoint{2.064011in}{1.489293in}}%
\pgfpathlineto{\pgfqpoint{2.081390in}{1.475104in}}%
\pgfpathlineto{\pgfqpoint{2.103733in}{1.453480in}}%
\pgfpathlineto{\pgfqpoint{2.185659in}{1.370382in}}%
\pgfpathlineto{\pgfqpoint{2.203037in}{1.357206in}}%
\pgfpathlineto{\pgfqpoint{2.220416in}{1.346885in}}%
\pgfpathlineto{\pgfqpoint{2.235311in}{1.340549in}}%
\pgfpathlineto{\pgfqpoint{2.250207in}{1.336641in}}%
\pgfpathlineto{\pgfqpoint{2.265102in}{1.335175in}}%
\pgfpathlineto{\pgfqpoint{2.279998in}{1.336086in}}%
\pgfpathlineto{\pgfqpoint{2.294894in}{1.339231in}}%
\pgfpathlineto{\pgfqpoint{2.312272in}{1.345442in}}%
\pgfpathlineto{\pgfqpoint{2.332133in}{1.355356in}}%
\pgfpathlineto{\pgfqpoint{2.354476in}{1.369177in}}%
\pgfpathlineto{\pgfqpoint{2.394198in}{1.397013in}}%
\pgfpathlineto{\pgfqpoint{2.428954in}{1.420234in}}%
\pgfpathlineto{\pgfqpoint{2.451298in}{1.432633in}}%
\pgfpathlineto{\pgfqpoint{2.471158in}{1.441222in}}%
\pgfpathlineto{\pgfqpoint{2.488537in}{1.446555in}}%
\pgfpathlineto{\pgfqpoint{2.505915in}{1.449713in}}%
\pgfpathlineto{\pgfqpoint{2.523293in}{1.450674in}}%
\pgfpathlineto{\pgfqpoint{2.540671in}{1.449513in}}%
\pgfpathlineto{\pgfqpoint{2.560532in}{1.445806in}}%
\pgfpathlineto{\pgfqpoint{2.582876in}{1.439060in}}%
\pgfpathlineto{\pgfqpoint{2.610184in}{1.428090in}}%
\pgfpathlineto{\pgfqpoint{2.709488in}{1.385237in}}%
\pgfpathlineto{\pgfqpoint{2.731832in}{1.379193in}}%
\pgfpathlineto{\pgfqpoint{2.754175in}{1.375434in}}%
\pgfpathlineto{\pgfqpoint{2.776519in}{1.374077in}}%
\pgfpathlineto{\pgfqpoint{2.798862in}{1.375054in}}%
\pgfpathlineto{\pgfqpoint{2.823688in}{1.378588in}}%
\pgfpathlineto{\pgfqpoint{2.850997in}{1.384841in}}%
\pgfpathlineto{\pgfqpoint{2.890718in}{1.396544in}}%
\pgfpathlineto{\pgfqpoint{2.947818in}{1.413292in}}%
\pgfpathlineto{\pgfqpoint{2.977610in}{1.419711in}}%
\pgfpathlineto{\pgfqpoint{3.004918in}{1.423371in}}%
\pgfpathlineto{\pgfqpoint{3.032227in}{1.424673in}}%
\pgfpathlineto{\pgfqpoint{3.059535in}{1.423675in}}%
\pgfpathlineto{\pgfqpoint{3.089327in}{1.420322in}}%
\pgfpathlineto{\pgfqpoint{3.126566in}{1.413795in}}%
\pgfpathlineto{\pgfqpoint{3.223387in}{1.395653in}}%
\pgfpathlineto{\pgfqpoint{3.225870in}{1.395324in}}%
\pgfpathlineto{\pgfqpoint{3.225870in}{1.395324in}}%
\pgfusepath{stroke}%
\end{pgfscope}%
\begin{pgfscope}%
\pgfpathrectangle{\pgfqpoint{0.619136in}{0.571603in}}{\pgfqpoint{2.730864in}{1.657828in}}%
\pgfusepath{clip}%
\pgfsetrectcap%
\pgfsetroundjoin%
\pgfsetlinewidth{1.505625pt}%
\definecolor{currentstroke}{rgb}{0.090196,0.745098,0.811765}%
\pgfsetstrokecolor{currentstroke}%
\pgfsetdash{}{0pt}%
\pgfpathmoveto{\pgfqpoint{0.743267in}{0.646959in}}%
\pgfpathlineto{\pgfqpoint{0.753197in}{0.822403in}}%
\pgfpathlineto{\pgfqpoint{0.763127in}{0.961117in}}%
\pgfpathlineto{\pgfqpoint{0.773058in}{1.068355in}}%
\pgfpathlineto{\pgfqpoint{0.782988in}{1.150704in}}%
\pgfpathlineto{\pgfqpoint{0.792919in}{1.213670in}}%
\pgfpathlineto{\pgfqpoint{0.802849in}{1.261653in}}%
\pgfpathlineto{\pgfqpoint{0.812779in}{1.298112in}}%
\pgfpathlineto{\pgfqpoint{0.822710in}{1.325735in}}%
\pgfpathlineto{\pgfqpoint{0.832640in}{1.346606in}}%
\pgfpathlineto{\pgfqpoint{0.842571in}{1.362329in}}%
\pgfpathlineto{\pgfqpoint{0.852501in}{1.374136in}}%
\pgfpathlineto{\pgfqpoint{0.862432in}{1.382971in}}%
\pgfpathlineto{\pgfqpoint{0.874845in}{1.390919in}}%
\pgfpathlineto{\pgfqpoint{0.887258in}{1.396382in}}%
\pgfpathlineto{\pgfqpoint{0.902153in}{1.400688in}}%
\pgfpathlineto{\pgfqpoint{0.922014in}{1.403991in}}%
\pgfpathlineto{\pgfqpoint{0.949323in}{1.406020in}}%
\pgfpathlineto{\pgfqpoint{0.994009in}{1.406758in}}%
\pgfpathlineto{\pgfqpoint{1.175240in}{1.405729in}}%
\pgfpathlineto{\pgfqpoint{1.629556in}{1.404900in}}%
\pgfpathlineto{\pgfqpoint{3.225870in}{1.404488in}}%
\pgfpathlineto{\pgfqpoint{3.225870in}{1.404488in}}%
\pgfusepath{stroke}%
\end{pgfscope}%
\begin{pgfscope}%
\pgfsetrectcap%
\pgfsetmiterjoin%
\pgfsetlinewidth{0.803000pt}%
\definecolor{currentstroke}{rgb}{0.000000,0.000000,0.000000}%
\pgfsetstrokecolor{currentstroke}%
\pgfsetdash{}{0pt}%
\pgfpathmoveto{\pgfqpoint{0.619136in}{0.571603in}}%
\pgfpathlineto{\pgfqpoint{0.619136in}{2.229431in}}%
\pgfusepath{stroke}%
\end{pgfscope}%
\begin{pgfscope}%
\pgfsetrectcap%
\pgfsetmiterjoin%
\pgfsetlinewidth{0.803000pt}%
\definecolor{currentstroke}{rgb}{0.000000,0.000000,0.000000}%
\pgfsetstrokecolor{currentstroke}%
\pgfsetdash{}{0pt}%
\pgfpathmoveto{\pgfqpoint{3.350000in}{0.571603in}}%
\pgfpathlineto{\pgfqpoint{3.350000in}{2.229431in}}%
\pgfusepath{stroke}%
\end{pgfscope}%
\begin{pgfscope}%
\pgfsetrectcap%
\pgfsetmiterjoin%
\pgfsetlinewidth{0.803000pt}%
\definecolor{currentstroke}{rgb}{0.000000,0.000000,0.000000}%
\pgfsetstrokecolor{currentstroke}%
\pgfsetdash{}{0pt}%
\pgfpathmoveto{\pgfqpoint{0.619136in}{0.571603in}}%
\pgfpathlineto{\pgfqpoint{3.350000in}{0.571603in}}%
\pgfusepath{stroke}%
\end{pgfscope}%
\begin{pgfscope}%
\pgfsetrectcap%
\pgfsetmiterjoin%
\pgfsetlinewidth{0.803000pt}%
\definecolor{currentstroke}{rgb}{0.000000,0.000000,0.000000}%
\pgfsetstrokecolor{currentstroke}%
\pgfsetdash{}{0pt}%
\pgfpathmoveto{\pgfqpoint{0.619136in}{2.229431in}}%
\pgfpathlineto{\pgfqpoint{3.350000in}{2.229431in}}%
\pgfusepath{stroke}%
\end{pgfscope}%
\end{pgfpicture}%
\makeatother%
\endgroup%

\vspace*{-5pt}
\caption{First 100 fractional order step responses generated and used for the
neural network.}
\label{fig:accuracy}
\end{figure}


\begin{figure}
\centering
%% Creator: Matplotlib, PGF backend
%%
%% To include the figure in your LaTeX document, write
%%   \input{<filename>.pgf}
%%
%% Make sure the required packages are loaded in your preamble
%%   \usepackage{pgf}
%%
%% Also ensure that all the required font packages are loaded; for instance,
%% the lmodern package is sometimes necessary when using math font.
%%   \usepackage{lmodern}
%%
%% Figures using additional raster images can only be included by \input if
%% they are in the same directory as the main LaTeX file. For loading figures
%% from other directories you can use the `import` package
%%   \usepackage{import}
%%
%% and then include the figures with
%%   \import{<path to file>}{<filename>.pgf}
%%
%% Matplotlib used the following preamble
%%   \def\mathdefault#1{#1}
%%   \everymath=\expandafter{\the\everymath\displaystyle}
%%   
%%   \usepackage{fontspec}
%%   \setmainfont{DejaVuSerif.ttf}[Path=\detokenize{/Users/billgoodwine/research/step/steps/lib/python3.11/site-packages/matplotlib/mpl-data/fonts/ttf/}]
%%   \setsansfont{DejaVuSans.ttf}[Path=\detokenize{/Users/billgoodwine/research/step/steps/lib/python3.11/site-packages/matplotlib/mpl-data/fonts/ttf/}]
%%   \setmonofont{DejaVuSansMono.ttf}[Path=\detokenize{/Users/billgoodwine/research/step/steps/lib/python3.11/site-packages/matplotlib/mpl-data/fonts/ttf/}]
%%   \makeatletter\@ifpackageloaded{underscore}{}{\usepackage[strings]{underscore}}\makeatother
%%
\begingroup%
\makeatletter%
\begin{pgfpicture}%
\pgfpathrectangle{\pgfpointorigin}{\pgfqpoint{3.500000in}{2.379431in}}%
\pgfusepath{use as bounding box, clip}%
\begin{pgfscope}%
\pgfsetbuttcap%
\pgfsetmiterjoin%
\definecolor{currentfill}{rgb}{1.000000,1.000000,1.000000}%
\pgfsetfillcolor{currentfill}%
\pgfsetlinewidth{0.000000pt}%
\definecolor{currentstroke}{rgb}{1.000000,1.000000,1.000000}%
\pgfsetstrokecolor{currentstroke}%
\pgfsetdash{}{0pt}%
\pgfpathmoveto{\pgfqpoint{0.000000in}{0.000000in}}%
\pgfpathlineto{\pgfqpoint{3.500000in}{0.000000in}}%
\pgfpathlineto{\pgfqpoint{3.500000in}{2.379431in}}%
\pgfpathlineto{\pgfqpoint{0.000000in}{2.379431in}}%
\pgfpathlineto{\pgfqpoint{0.000000in}{0.000000in}}%
\pgfpathclose%
\pgfusepath{fill}%
\end{pgfscope}%
\begin{pgfscope}%
\pgfsetbuttcap%
\pgfsetmiterjoin%
\definecolor{currentfill}{rgb}{1.000000,1.000000,1.000000}%
\pgfsetfillcolor{currentfill}%
\pgfsetlinewidth{0.000000pt}%
\definecolor{currentstroke}{rgb}{0.000000,0.000000,0.000000}%
\pgfsetstrokecolor{currentstroke}%
\pgfsetstrokeopacity{0.000000}%
\pgfsetdash{}{0pt}%
\pgfpathmoveto{\pgfqpoint{0.684105in}{0.571603in}}%
\pgfpathlineto{\pgfqpoint{3.350000in}{0.571603in}}%
\pgfpathlineto{\pgfqpoint{3.350000in}{2.229431in}}%
\pgfpathlineto{\pgfqpoint{0.684105in}{2.229431in}}%
\pgfpathlineto{\pgfqpoint{0.684105in}{0.571603in}}%
\pgfpathclose%
\pgfusepath{fill}%
\end{pgfscope}%
\begin{pgfscope}%
\pgfpathrectangle{\pgfqpoint{0.684105in}{0.571603in}}{\pgfqpoint{2.665895in}{1.657828in}}%
\pgfusepath{clip}%
\pgfsetbuttcap%
\pgfsetroundjoin%
\definecolor{currentfill}{rgb}{0.121569,0.466667,0.705882}%
\pgfsetfillcolor{currentfill}%
\pgfsetlinewidth{1.003750pt}%
\definecolor{currentstroke}{rgb}{0.121569,0.466667,0.705882}%
\pgfsetstrokecolor{currentstroke}%
\pgfsetdash{}{0pt}%
\pgfsys@defobject{currentmarker}{\pgfqpoint{-0.020833in}{-0.020833in}}{\pgfqpoint{0.020833in}{0.020833in}}{%
\pgfpathmoveto{\pgfqpoint{0.000000in}{-0.020833in}}%
\pgfpathcurveto{\pgfqpoint{0.005525in}{-0.020833in}}{\pgfqpoint{0.010825in}{-0.018638in}}{\pgfqpoint{0.014731in}{-0.014731in}}%
\pgfpathcurveto{\pgfqpoint{0.018638in}{-0.010825in}}{\pgfqpoint{0.020833in}{-0.005525in}}{\pgfqpoint{0.020833in}{0.000000in}}%
\pgfpathcurveto{\pgfqpoint{0.020833in}{0.005525in}}{\pgfqpoint{0.018638in}{0.010825in}}{\pgfqpoint{0.014731in}{0.014731in}}%
\pgfpathcurveto{\pgfqpoint{0.010825in}{0.018638in}}{\pgfqpoint{0.005525in}{0.020833in}}{\pgfqpoint{0.000000in}{0.020833in}}%
\pgfpathcurveto{\pgfqpoint{-0.005525in}{0.020833in}}{\pgfqpoint{-0.010825in}{0.018638in}}{\pgfqpoint{-0.014731in}{0.014731in}}%
\pgfpathcurveto{\pgfqpoint{-0.018638in}{0.010825in}}{\pgfqpoint{-0.020833in}{0.005525in}}{\pgfqpoint{-0.020833in}{0.000000in}}%
\pgfpathcurveto{\pgfqpoint{-0.020833in}{-0.005525in}}{\pgfqpoint{-0.018638in}{-0.010825in}}{\pgfqpoint{-0.014731in}{-0.014731in}}%
\pgfpathcurveto{\pgfqpoint{-0.010825in}{-0.018638in}}{\pgfqpoint{-0.005525in}{-0.020833in}}{\pgfqpoint{0.000000in}{-0.020833in}}%
\pgfpathlineto{\pgfqpoint{0.000000in}{-0.020833in}}%
\pgfpathclose%
\pgfusepath{stroke,fill}%
}%
\begin{pgfscope}%
\pgfsys@transformshift{1.567365in}{1.077684in}%
\pgfsys@useobject{currentmarker}{}%
\end{pgfscope}%
\begin{pgfscope}%
\pgfsys@transformshift{2.959389in}{1.593215in}%
\pgfsys@useobject{currentmarker}{}%
\end{pgfscope}%
\begin{pgfscope}%
\pgfsys@transformshift{0.840495in}{0.906986in}%
\pgfsys@useobject{currentmarker}{}%
\end{pgfscope}%
\begin{pgfscope}%
\pgfsys@transformshift{1.738378in}{1.179837in}%
\pgfsys@useobject{currentmarker}{}%
\end{pgfscope}%
\begin{pgfscope}%
\pgfsys@transformshift{2.675084in}{1.527935in}%
\pgfsys@useobject{currentmarker}{}%
\end{pgfscope}%
\begin{pgfscope}%
\pgfsys@transformshift{1.115880in}{1.010666in}%
\pgfsys@useobject{currentmarker}{}%
\end{pgfscope}%
\begin{pgfscope}%
\pgfsys@transformshift{1.421089in}{1.087489in}%
\pgfsys@useobject{currentmarker}{}%
\end{pgfscope}%
\begin{pgfscope}%
\pgfsys@transformshift{0.807598in}{0.751107in}%
\pgfsys@useobject{currentmarker}{}%
\end{pgfscope}%
\begin{pgfscope}%
\pgfsys@transformshift{1.345197in}{1.050566in}%
\pgfsys@useobject{currentmarker}{}%
\end{pgfscope}%
\begin{pgfscope}%
\pgfsys@transformshift{1.161759in}{0.971614in}%
\pgfsys@useobject{currentmarker}{}%
\end{pgfscope}%
\begin{pgfscope}%
\pgfsys@transformshift{3.013476in}{1.665015in}%
\pgfsys@useobject{currentmarker}{}%
\end{pgfscope}%
\begin{pgfscope}%
\pgfsys@transformshift{2.072313in}{1.448358in}%
\pgfsys@useobject{currentmarker}{}%
\end{pgfscope}%
\begin{pgfscope}%
\pgfsys@transformshift{1.286060in}{0.930990in}%
\pgfsys@useobject{currentmarker}{}%
\end{pgfscope}%
\begin{pgfscope}%
\pgfsys@transformshift{2.018877in}{1.449486in}%
\pgfsys@useobject{currentmarker}{}%
\end{pgfscope}%
\begin{pgfscope}%
\pgfsys@transformshift{1.307739in}{1.066504in}%
\pgfsys@useobject{currentmarker}{}%
\end{pgfscope}%
\begin{pgfscope}%
\pgfsys@transformshift{1.760125in}{1.321614in}%
\pgfsys@useobject{currentmarker}{}%
\end{pgfscope}%
\begin{pgfscope}%
\pgfsys@transformshift{2.814476in}{1.622389in}%
\pgfsys@useobject{currentmarker}{}%
\end{pgfscope}%
\begin{pgfscope}%
\pgfsys@transformshift{1.525797in}{1.146606in}%
\pgfsys@useobject{currentmarker}{}%
\end{pgfscope}%
\begin{pgfscope}%
\pgfsys@transformshift{2.339688in}{1.404921in}%
\pgfsys@useobject{currentmarker}{}%
\end{pgfscope}%
\begin{pgfscope}%
\pgfsys@transformshift{2.063288in}{1.418848in}%
\pgfsys@useobject{currentmarker}{}%
\end{pgfscope}%
\begin{pgfscope}%
\pgfsys@transformshift{1.445145in}{1.135714in}%
\pgfsys@useobject{currentmarker}{}%
\end{pgfscope}%
\begin{pgfscope}%
\pgfsys@transformshift{2.811153in}{1.619860in}%
\pgfsys@useobject{currentmarker}{}%
\end{pgfscope}%
\begin{pgfscope}%
\pgfsys@transformshift{1.425543in}{1.134101in}%
\pgfsys@useobject{currentmarker}{}%
\end{pgfscope}%
\begin{pgfscope}%
\pgfsys@transformshift{0.836567in}{0.785960in}%
\pgfsys@useobject{currentmarker}{}%
\end{pgfscope}%
\begin{pgfscope}%
\pgfsys@transformshift{1.719327in}{1.318701in}%
\pgfsys@useobject{currentmarker}{}%
\end{pgfscope}%
\begin{pgfscope}%
\pgfsys@transformshift{2.668219in}{1.614251in}%
\pgfsys@useobject{currentmarker}{}%
\end{pgfscope}%
\begin{pgfscope}%
\pgfsys@transformshift{1.782855in}{1.227933in}%
\pgfsys@useobject{currentmarker}{}%
\end{pgfscope}%
\begin{pgfscope}%
\pgfsys@transformshift{1.126208in}{0.843908in}%
\pgfsys@useobject{currentmarker}{}%
\end{pgfscope}%
\begin{pgfscope}%
\pgfsys@transformshift{2.352174in}{1.554265in}%
\pgfsys@useobject{currentmarker}{}%
\end{pgfscope}%
\begin{pgfscope}%
\pgfsys@transformshift{1.190625in}{0.918058in}%
\pgfsys@useobject{currentmarker}{}%
\end{pgfscope}%
\begin{pgfscope}%
\pgfsys@transformshift{2.024265in}{1.275283in}%
\pgfsys@useobject{currentmarker}{}%
\end{pgfscope}%
\begin{pgfscope}%
\pgfsys@transformshift{0.842506in}{0.764372in}%
\pgfsys@useobject{currentmarker}{}%
\end{pgfscope}%
\begin{pgfscope}%
\pgfsys@transformshift{2.671512in}{1.555093in}%
\pgfsys@useobject{currentmarker}{}%
\end{pgfscope}%
\begin{pgfscope}%
\pgfsys@transformshift{1.304092in}{1.046475in}%
\pgfsys@useobject{currentmarker}{}%
\end{pgfscope}%
\begin{pgfscope}%
\pgfsys@transformshift{2.207700in}{1.437729in}%
\pgfsys@useobject{currentmarker}{}%
\end{pgfscope}%
\begin{pgfscope}%
\pgfsys@transformshift{1.650535in}{1.196571in}%
\pgfsys@useobject{currentmarker}{}%
\end{pgfscope}%
\begin{pgfscope}%
\pgfsys@transformshift{1.207885in}{1.064333in}%
\pgfsys@useobject{currentmarker}{}%
\end{pgfscope}%
\begin{pgfscope}%
\pgfsys@transformshift{2.089094in}{1.142621in}%
\pgfsys@useobject{currentmarker}{}%
\end{pgfscope}%
\begin{pgfscope}%
\pgfsys@transformshift{0.929408in}{0.906972in}%
\pgfsys@useobject{currentmarker}{}%
\end{pgfscope}%
\begin{pgfscope}%
\pgfsys@transformshift{1.110384in}{1.014528in}%
\pgfsys@useobject{currentmarker}{}%
\end{pgfscope}%
\begin{pgfscope}%
\pgfsys@transformshift{1.354175in}{1.131861in}%
\pgfsys@useobject{currentmarker}{}%
\end{pgfscope}%
\begin{pgfscope}%
\pgfsys@transformshift{2.925048in}{1.673958in}%
\pgfsys@useobject{currentmarker}{}%
\end{pgfscope}%
\begin{pgfscope}%
\pgfsys@transformshift{1.268486in}{0.993019in}%
\pgfsys@useobject{currentmarker}{}%
\end{pgfscope}%
\begin{pgfscope}%
\pgfsys@transformshift{1.680776in}{1.115658in}%
\pgfsys@useobject{currentmarker}{}%
\end{pgfscope}%
\begin{pgfscope}%
\pgfsys@transformshift{1.133552in}{0.980015in}%
\pgfsys@useobject{currentmarker}{}%
\end{pgfscope}%
\begin{pgfscope}%
\pgfsys@transformshift{2.169374in}{1.567403in}%
\pgfsys@useobject{currentmarker}{}%
\end{pgfscope}%
\begin{pgfscope}%
\pgfsys@transformshift{1.636936in}{1.259875in}%
\pgfsys@useobject{currentmarker}{}%
\end{pgfscope}%
\begin{pgfscope}%
\pgfsys@transformshift{2.901254in}{1.650313in}%
\pgfsys@useobject{currentmarker}{}%
\end{pgfscope}%
\begin{pgfscope}%
\pgfsys@transformshift{2.816904in}{1.627145in}%
\pgfsys@useobject{currentmarker}{}%
\end{pgfscope}%
\begin{pgfscope}%
\pgfsys@transformshift{1.729149in}{1.134160in}%
\pgfsys@useobject{currentmarker}{}%
\end{pgfscope}%
\begin{pgfscope}%
\pgfsys@transformshift{2.673372in}{1.598492in}%
\pgfsys@useobject{currentmarker}{}%
\end{pgfscope}%
\begin{pgfscope}%
\pgfsys@transformshift{1.289867in}{1.034845in}%
\pgfsys@useobject{currentmarker}{}%
\end{pgfscope}%
\begin{pgfscope}%
\pgfsys@transformshift{1.389420in}{0.941433in}%
\pgfsys@useobject{currentmarker}{}%
\end{pgfscope}%
\begin{pgfscope}%
\pgfsys@transformshift{0.958965in}{0.801131in}%
\pgfsys@useobject{currentmarker}{}%
\end{pgfscope}%
\begin{pgfscope}%
\pgfsys@transformshift{0.836173in}{0.840863in}%
\pgfsys@useobject{currentmarker}{}%
\end{pgfscope}%
\begin{pgfscope}%
\pgfsys@transformshift{2.054474in}{1.346277in}%
\pgfsys@useobject{currentmarker}{}%
\end{pgfscope}%
\begin{pgfscope}%
\pgfsys@transformshift{1.284970in}{1.083617in}%
\pgfsys@useobject{currentmarker}{}%
\end{pgfscope}%
\begin{pgfscope}%
\pgfsys@transformshift{1.740338in}{1.207640in}%
\pgfsys@useobject{currentmarker}{}%
\end{pgfscope}%
\begin{pgfscope}%
\pgfsys@transformshift{2.140099in}{1.408513in}%
\pgfsys@useobject{currentmarker}{}%
\end{pgfscope}%
\begin{pgfscope}%
\pgfsys@transformshift{2.421243in}{1.472526in}%
\pgfsys@useobject{currentmarker}{}%
\end{pgfscope}%
\begin{pgfscope}%
\pgfsys@transformshift{0.988899in}{0.910606in}%
\pgfsys@useobject{currentmarker}{}%
\end{pgfscope}%
\begin{pgfscope}%
\pgfsys@transformshift{1.505555in}{1.215001in}%
\pgfsys@useobject{currentmarker}{}%
\end{pgfscope}%
\begin{pgfscope}%
\pgfsys@transformshift{1.202576in}{0.938240in}%
\pgfsys@useobject{currentmarker}{}%
\end{pgfscope}%
\begin{pgfscope}%
\pgfsys@transformshift{2.310979in}{1.384622in}%
\pgfsys@useobject{currentmarker}{}%
\end{pgfscope}%
\begin{pgfscope}%
\pgfsys@transformshift{2.949856in}{1.630704in}%
\pgfsys@useobject{currentmarker}{}%
\end{pgfscope}%
\begin{pgfscope}%
\pgfsys@transformshift{1.896210in}{1.362635in}%
\pgfsys@useobject{currentmarker}{}%
\end{pgfscope}%
\begin{pgfscope}%
\pgfsys@transformshift{3.165436in}{1.849335in}%
\pgfsys@useobject{currentmarker}{}%
\end{pgfscope}%
\begin{pgfscope}%
\pgfsys@transformshift{1.609636in}{1.063509in}%
\pgfsys@useobject{currentmarker}{}%
\end{pgfscope}%
\begin{pgfscope}%
\pgfsys@transformshift{0.816383in}{0.840234in}%
\pgfsys@useobject{currentmarker}{}%
\end{pgfscope}%
\begin{pgfscope}%
\pgfsys@transformshift{1.777421in}{1.152435in}%
\pgfsys@useobject{currentmarker}{}%
\end{pgfscope}%
\begin{pgfscope}%
\pgfsys@transformshift{2.650195in}{1.543529in}%
\pgfsys@useobject{currentmarker}{}%
\end{pgfscope}%
\begin{pgfscope}%
\pgfsys@transformshift{2.464943in}{1.467514in}%
\pgfsys@useobject{currentmarker}{}%
\end{pgfscope}%
\begin{pgfscope}%
\pgfsys@transformshift{0.999395in}{0.869771in}%
\pgfsys@useobject{currentmarker}{}%
\end{pgfscope}%
\begin{pgfscope}%
\pgfsys@transformshift{2.362313in}{1.361778in}%
\pgfsys@useobject{currentmarker}{}%
\end{pgfscope}%
\begin{pgfscope}%
\pgfsys@transformshift{2.772194in}{1.577346in}%
\pgfsys@useobject{currentmarker}{}%
\end{pgfscope}%
\begin{pgfscope}%
\pgfsys@transformshift{2.862506in}{1.786498in}%
\pgfsys@useobject{currentmarker}{}%
\end{pgfscope}%
\begin{pgfscope}%
\pgfsys@transformshift{1.455663in}{1.078972in}%
\pgfsys@useobject{currentmarker}{}%
\end{pgfscope}%
\begin{pgfscope}%
\pgfsys@transformshift{2.086718in}{1.401298in}%
\pgfsys@useobject{currentmarker}{}%
\end{pgfscope}%
\begin{pgfscope}%
\pgfsys@transformshift{2.686702in}{1.621251in}%
\pgfsys@useobject{currentmarker}{}%
\end{pgfscope}%
\begin{pgfscope}%
\pgfsys@transformshift{1.217554in}{1.001951in}%
\pgfsys@useobject{currentmarker}{}%
\end{pgfscope}%
\begin{pgfscope}%
\pgfsys@transformshift{0.964615in}{0.936929in}%
\pgfsys@useobject{currentmarker}{}%
\end{pgfscope}%
\begin{pgfscope}%
\pgfsys@transformshift{1.696916in}{1.128941in}%
\pgfsys@useobject{currentmarker}{}%
\end{pgfscope}%
\begin{pgfscope}%
\pgfsys@transformshift{1.050033in}{0.980850in}%
\pgfsys@useobject{currentmarker}{}%
\end{pgfscope}%
\begin{pgfscope}%
\pgfsys@transformshift{1.709383in}{1.242367in}%
\pgfsys@useobject{currentmarker}{}%
\end{pgfscope}%
\begin{pgfscope}%
\pgfsys@transformshift{2.682348in}{1.516188in}%
\pgfsys@useobject{currentmarker}{}%
\end{pgfscope}%
\begin{pgfscope}%
\pgfsys@transformshift{2.200127in}{1.267879in}%
\pgfsys@useobject{currentmarker}{}%
\end{pgfscope}%
\begin{pgfscope}%
\pgfsys@transformshift{0.902200in}{0.658534in}%
\pgfsys@useobject{currentmarker}{}%
\end{pgfscope}%
\begin{pgfscope}%
\pgfsys@transformshift{1.488621in}{1.136607in}%
\pgfsys@useobject{currentmarker}{}%
\end{pgfscope}%
\begin{pgfscope}%
\pgfsys@transformshift{2.510675in}{1.340431in}%
\pgfsys@useobject{currentmarker}{}%
\end{pgfscope}%
\begin{pgfscope}%
\pgfsys@transformshift{1.437892in}{1.125871in}%
\pgfsys@useobject{currentmarker}{}%
\end{pgfscope}%
\begin{pgfscope}%
\pgfsys@transformshift{2.241522in}{1.431878in}%
\pgfsys@useobject{currentmarker}{}%
\end{pgfscope}%
\begin{pgfscope}%
\pgfsys@transformshift{2.344124in}{1.580348in}%
\pgfsys@useobject{currentmarker}{}%
\end{pgfscope}%
\begin{pgfscope}%
\pgfsys@transformshift{1.569816in}{1.008754in}%
\pgfsys@useobject{currentmarker}{}%
\end{pgfscope}%
\begin{pgfscope}%
\pgfsys@transformshift{1.621629in}{1.261440in}%
\pgfsys@useobject{currentmarker}{}%
\end{pgfscope}%
\begin{pgfscope}%
\pgfsys@transformshift{3.010707in}{1.682468in}%
\pgfsys@useobject{currentmarker}{}%
\end{pgfscope}%
\begin{pgfscope}%
\pgfsys@transformshift{1.642624in}{1.207176in}%
\pgfsys@useobject{currentmarker}{}%
\end{pgfscope}%
\begin{pgfscope}%
\pgfsys@transformshift{0.806910in}{0.813555in}%
\pgfsys@useobject{currentmarker}{}%
\end{pgfscope}%
\begin{pgfscope}%
\pgfsys@transformshift{1.681471in}{1.245743in}%
\pgfsys@useobject{currentmarker}{}%
\end{pgfscope}%
\begin{pgfscope}%
\pgfsys@transformshift{2.611949in}{1.513547in}%
\pgfsys@useobject{currentmarker}{}%
\end{pgfscope}%
\begin{pgfscope}%
\pgfsys@transformshift{0.950945in}{0.811608in}%
\pgfsys@useobject{currentmarker}{}%
\end{pgfscope}%
\begin{pgfscope}%
\pgfsys@transformshift{1.057483in}{0.859948in}%
\pgfsys@useobject{currentmarker}{}%
\end{pgfscope}%
\begin{pgfscope}%
\pgfsys@transformshift{2.122966in}{1.353347in}%
\pgfsys@useobject{currentmarker}{}%
\end{pgfscope}%
\begin{pgfscope}%
\pgfsys@transformshift{2.679059in}{1.563082in}%
\pgfsys@useobject{currentmarker}{}%
\end{pgfscope}%
\begin{pgfscope}%
\pgfsys@transformshift{2.121185in}{1.223641in}%
\pgfsys@useobject{currentmarker}{}%
\end{pgfscope}%
\begin{pgfscope}%
\pgfsys@transformshift{1.359588in}{0.954613in}%
\pgfsys@useobject{currentmarker}{}%
\end{pgfscope}%
\begin{pgfscope}%
\pgfsys@transformshift{2.446023in}{1.611080in}%
\pgfsys@useobject{currentmarker}{}%
\end{pgfscope}%
\begin{pgfscope}%
\pgfsys@transformshift{2.950086in}{1.688210in}%
\pgfsys@useobject{currentmarker}{}%
\end{pgfscope}%
\begin{pgfscope}%
\pgfsys@transformshift{1.167247in}{0.990886in}%
\pgfsys@useobject{currentmarker}{}%
\end{pgfscope}%
\begin{pgfscope}%
\pgfsys@transformshift{0.952619in}{0.810714in}%
\pgfsys@useobject{currentmarker}{}%
\end{pgfscope}%
\begin{pgfscope}%
\pgfsys@transformshift{0.859989in}{0.834450in}%
\pgfsys@useobject{currentmarker}{}%
\end{pgfscope}%
\begin{pgfscope}%
\pgfsys@transformshift{1.389550in}{1.088912in}%
\pgfsys@useobject{currentmarker}{}%
\end{pgfscope}%
\begin{pgfscope}%
\pgfsys@transformshift{1.946839in}{1.389340in}%
\pgfsys@useobject{currentmarker}{}%
\end{pgfscope}%
\begin{pgfscope}%
\pgfsys@transformshift{1.289951in}{0.981548in}%
\pgfsys@useobject{currentmarker}{}%
\end{pgfscope}%
\begin{pgfscope}%
\pgfsys@transformshift{3.019721in}{1.600594in}%
\pgfsys@useobject{currentmarker}{}%
\end{pgfscope}%
\begin{pgfscope}%
\pgfsys@transformshift{1.253466in}{0.988280in}%
\pgfsys@useobject{currentmarker}{}%
\end{pgfscope}%
\begin{pgfscope}%
\pgfsys@transformshift{2.671749in}{1.577837in}%
\pgfsys@useobject{currentmarker}{}%
\end{pgfscope}%
\begin{pgfscope}%
\pgfsys@transformshift{2.918198in}{1.597451in}%
\pgfsys@useobject{currentmarker}{}%
\end{pgfscope}%
\begin{pgfscope}%
\pgfsys@transformshift{3.079971in}{1.559688in}%
\pgfsys@useobject{currentmarker}{}%
\end{pgfscope}%
\begin{pgfscope}%
\pgfsys@transformshift{2.100853in}{1.437716in}%
\pgfsys@useobject{currentmarker}{}%
\end{pgfscope}%
\begin{pgfscope}%
\pgfsys@transformshift{2.106788in}{1.245493in}%
\pgfsys@useobject{currentmarker}{}%
\end{pgfscope}%
\begin{pgfscope}%
\pgfsys@transformshift{0.823887in}{0.869371in}%
\pgfsys@useobject{currentmarker}{}%
\end{pgfscope}%
\begin{pgfscope}%
\pgfsys@transformshift{1.509479in}{1.102857in}%
\pgfsys@useobject{currentmarker}{}%
\end{pgfscope}%
\begin{pgfscope}%
\pgfsys@transformshift{2.512579in}{1.494155in}%
\pgfsys@useobject{currentmarker}{}%
\end{pgfscope}%
\begin{pgfscope}%
\pgfsys@transformshift{2.134270in}{1.108019in}%
\pgfsys@useobject{currentmarker}{}%
\end{pgfscope}%
\begin{pgfscope}%
\pgfsys@transformshift{2.648471in}{1.595438in}%
\pgfsys@useobject{currentmarker}{}%
\end{pgfscope}%
\begin{pgfscope}%
\pgfsys@transformshift{1.701616in}{1.133739in}%
\pgfsys@useobject{currentmarker}{}%
\end{pgfscope}%
\begin{pgfscope}%
\pgfsys@transformshift{1.616120in}{1.213887in}%
\pgfsys@useobject{currentmarker}{}%
\end{pgfscope}%
\begin{pgfscope}%
\pgfsys@transformshift{1.173458in}{0.792435in}%
\pgfsys@useobject{currentmarker}{}%
\end{pgfscope}%
\begin{pgfscope}%
\pgfsys@transformshift{1.505800in}{1.201002in}%
\pgfsys@useobject{currentmarker}{}%
\end{pgfscope}%
\begin{pgfscope}%
\pgfsys@transformshift{1.214038in}{0.900297in}%
\pgfsys@useobject{currentmarker}{}%
\end{pgfscope}%
\begin{pgfscope}%
\pgfsys@transformshift{1.615127in}{1.070336in}%
\pgfsys@useobject{currentmarker}{}%
\end{pgfscope}%
\begin{pgfscope}%
\pgfsys@transformshift{1.978694in}{1.374633in}%
\pgfsys@useobject{currentmarker}{}%
\end{pgfscope}%
\begin{pgfscope}%
\pgfsys@transformshift{2.164379in}{1.447710in}%
\pgfsys@useobject{currentmarker}{}%
\end{pgfscope}%
\begin{pgfscope}%
\pgfsys@transformshift{1.492677in}{1.032201in}%
\pgfsys@useobject{currentmarker}{}%
\end{pgfscope}%
\begin{pgfscope}%
\pgfsys@transformshift{1.316994in}{0.961511in}%
\pgfsys@useobject{currentmarker}{}%
\end{pgfscope}%
\begin{pgfscope}%
\pgfsys@transformshift{1.632700in}{1.262400in}%
\pgfsys@useobject{currentmarker}{}%
\end{pgfscope}%
\begin{pgfscope}%
\pgfsys@transformshift{1.383198in}{1.014042in}%
\pgfsys@useobject{currentmarker}{}%
\end{pgfscope}%
\begin{pgfscope}%
\pgfsys@transformshift{2.213774in}{1.469202in}%
\pgfsys@useobject{currentmarker}{}%
\end{pgfscope}%
\begin{pgfscope}%
\pgfsys@transformshift{2.398950in}{1.553600in}%
\pgfsys@useobject{currentmarker}{}%
\end{pgfscope}%
\begin{pgfscope}%
\pgfsys@transformshift{1.862984in}{1.293391in}%
\pgfsys@useobject{currentmarker}{}%
\end{pgfscope}%
\begin{pgfscope}%
\pgfsys@transformshift{2.957937in}{1.690171in}%
\pgfsys@useobject{currentmarker}{}%
\end{pgfscope}%
\begin{pgfscope}%
\pgfsys@transformshift{3.186930in}{2.018239in}%
\pgfsys@useobject{currentmarker}{}%
\end{pgfscope}%
\begin{pgfscope}%
\pgfsys@transformshift{1.187882in}{0.988083in}%
\pgfsys@useobject{currentmarker}{}%
\end{pgfscope}%
\begin{pgfscope}%
\pgfsys@transformshift{2.702375in}{1.555571in}%
\pgfsys@useobject{currentmarker}{}%
\end{pgfscope}%
\begin{pgfscope}%
\pgfsys@transformshift{1.050634in}{0.944379in}%
\pgfsys@useobject{currentmarker}{}%
\end{pgfscope}%
\begin{pgfscope}%
\pgfsys@transformshift{0.937313in}{0.789489in}%
\pgfsys@useobject{currentmarker}{}%
\end{pgfscope}%
\begin{pgfscope}%
\pgfsys@transformshift{2.694545in}{1.603705in}%
\pgfsys@useobject{currentmarker}{}%
\end{pgfscope}%
\begin{pgfscope}%
\pgfsys@transformshift{2.840471in}{1.819849in}%
\pgfsys@useobject{currentmarker}{}%
\end{pgfscope}%
\begin{pgfscope}%
\pgfsys@transformshift{2.929115in}{1.767342in}%
\pgfsys@useobject{currentmarker}{}%
\end{pgfscope}%
\begin{pgfscope}%
\pgfsys@transformshift{2.620007in}{1.594706in}%
\pgfsys@useobject{currentmarker}{}%
\end{pgfscope}%
\begin{pgfscope}%
\pgfsys@transformshift{2.978678in}{1.691867in}%
\pgfsys@useobject{currentmarker}{}%
\end{pgfscope}%
\begin{pgfscope}%
\pgfsys@transformshift{2.408823in}{1.374944in}%
\pgfsys@useobject{currentmarker}{}%
\end{pgfscope}%
\begin{pgfscope}%
\pgfsys@transformshift{3.186530in}{1.765704in}%
\pgfsys@useobject{currentmarker}{}%
\end{pgfscope}%
\begin{pgfscope}%
\pgfsys@transformshift{2.550688in}{1.609924in}%
\pgfsys@useobject{currentmarker}{}%
\end{pgfscope}%
\begin{pgfscope}%
\pgfsys@transformshift{3.100102in}{1.491223in}%
\pgfsys@useobject{currentmarker}{}%
\end{pgfscope}%
\begin{pgfscope}%
\pgfsys@transformshift{0.845337in}{0.794540in}%
\pgfsys@useobject{currentmarker}{}%
\end{pgfscope}%
\begin{pgfscope}%
\pgfsys@transformshift{1.153574in}{1.023942in}%
\pgfsys@useobject{currentmarker}{}%
\end{pgfscope}%
\begin{pgfscope}%
\pgfsys@transformshift{3.064883in}{1.720703in}%
\pgfsys@useobject{currentmarker}{}%
\end{pgfscope}%
\begin{pgfscope}%
\pgfsys@transformshift{2.950610in}{1.713074in}%
\pgfsys@useobject{currentmarker}{}%
\end{pgfscope}%
\begin{pgfscope}%
\pgfsys@transformshift{0.976472in}{0.832781in}%
\pgfsys@useobject{currentmarker}{}%
\end{pgfscope}%
\begin{pgfscope}%
\pgfsys@transformshift{2.758437in}{1.633376in}%
\pgfsys@useobject{currentmarker}{}%
\end{pgfscope}%
\begin{pgfscope}%
\pgfsys@transformshift{1.404805in}{0.926698in}%
\pgfsys@useobject{currentmarker}{}%
\end{pgfscope}%
\begin{pgfscope}%
\pgfsys@transformshift{0.954587in}{0.787761in}%
\pgfsys@useobject{currentmarker}{}%
\end{pgfscope}%
\begin{pgfscope}%
\pgfsys@transformshift{2.025126in}{1.327738in}%
\pgfsys@useobject{currentmarker}{}%
\end{pgfscope}%
\begin{pgfscope}%
\pgfsys@transformshift{0.865581in}{0.858974in}%
\pgfsys@useobject{currentmarker}{}%
\end{pgfscope}%
\begin{pgfscope}%
\pgfsys@transformshift{1.826630in}{1.302716in}%
\pgfsys@useobject{currentmarker}{}%
\end{pgfscope}%
\begin{pgfscope}%
\pgfsys@transformshift{1.857593in}{1.397327in}%
\pgfsys@useobject{currentmarker}{}%
\end{pgfscope}%
\begin{pgfscope}%
\pgfsys@transformshift{3.225187in}{2.000181in}%
\pgfsys@useobject{currentmarker}{}%
\end{pgfscope}%
\begin{pgfscope}%
\pgfsys@transformshift{2.843100in}{1.558920in}%
\pgfsys@useobject{currentmarker}{}%
\end{pgfscope}%
\begin{pgfscope}%
\pgfsys@transformshift{1.274760in}{0.950031in}%
\pgfsys@useobject{currentmarker}{}%
\end{pgfscope}%
\begin{pgfscope}%
\pgfsys@transformshift{1.974168in}{1.145858in}%
\pgfsys@useobject{currentmarker}{}%
\end{pgfscope}%
\begin{pgfscope}%
\pgfsys@transformshift{1.276995in}{0.955391in}%
\pgfsys@useobject{currentmarker}{}%
\end{pgfscope}%
\begin{pgfscope}%
\pgfsys@transformshift{0.968849in}{0.876294in}%
\pgfsys@useobject{currentmarker}{}%
\end{pgfscope}%
\begin{pgfscope}%
\pgfsys@transformshift{1.514865in}{1.098906in}%
\pgfsys@useobject{currentmarker}{}%
\end{pgfscope}%
\begin{pgfscope}%
\pgfsys@transformshift{1.454755in}{1.072736in}%
\pgfsys@useobject{currentmarker}{}%
\end{pgfscope}%
\begin{pgfscope}%
\pgfsys@transformshift{1.138821in}{0.817527in}%
\pgfsys@useobject{currentmarker}{}%
\end{pgfscope}%
\begin{pgfscope}%
\pgfsys@transformshift{1.290615in}{0.998018in}%
\pgfsys@useobject{currentmarker}{}%
\end{pgfscope}%
\begin{pgfscope}%
\pgfsys@transformshift{3.091496in}{1.522451in}%
\pgfsys@useobject{currentmarker}{}%
\end{pgfscope}%
\begin{pgfscope}%
\pgfsys@transformshift{1.628406in}{1.166259in}%
\pgfsys@useobject{currentmarker}{}%
\end{pgfscope}%
\begin{pgfscope}%
\pgfsys@transformshift{2.165946in}{1.554281in}%
\pgfsys@useobject{currentmarker}{}%
\end{pgfscope}%
\begin{pgfscope}%
\pgfsys@transformshift{2.563035in}{1.419992in}%
\pgfsys@useobject{currentmarker}{}%
\end{pgfscope}%
\begin{pgfscope}%
\pgfsys@transformshift{3.036686in}{1.876784in}%
\pgfsys@useobject{currentmarker}{}%
\end{pgfscope}%
\begin{pgfscope}%
\pgfsys@transformshift{2.077588in}{1.369478in}%
\pgfsys@useobject{currentmarker}{}%
\end{pgfscope}%
\begin{pgfscope}%
\pgfsys@transformshift{2.885877in}{1.713206in}%
\pgfsys@useobject{currentmarker}{}%
\end{pgfscope}%
\begin{pgfscope}%
\pgfsys@transformshift{1.374466in}{1.118666in}%
\pgfsys@useobject{currentmarker}{}%
\end{pgfscope}%
\begin{pgfscope}%
\pgfsys@transformshift{2.637375in}{1.682287in}%
\pgfsys@useobject{currentmarker}{}%
\end{pgfscope}%
\begin{pgfscope}%
\pgfsys@transformshift{0.805282in}{0.804152in}%
\pgfsys@useobject{currentmarker}{}%
\end{pgfscope}%
\begin{pgfscope}%
\pgfsys@transformshift{1.089758in}{0.895162in}%
\pgfsys@useobject{currentmarker}{}%
\end{pgfscope}%
\begin{pgfscope}%
\pgfsys@transformshift{2.085087in}{1.443483in}%
\pgfsys@useobject{currentmarker}{}%
\end{pgfscope}%
\begin{pgfscope}%
\pgfsys@transformshift{1.376350in}{1.075451in}%
\pgfsys@useobject{currentmarker}{}%
\end{pgfscope}%
\begin{pgfscope}%
\pgfsys@transformshift{1.313195in}{1.052453in}%
\pgfsys@useobject{currentmarker}{}%
\end{pgfscope}%
\begin{pgfscope}%
\pgfsys@transformshift{3.072509in}{1.735337in}%
\pgfsys@useobject{currentmarker}{}%
\end{pgfscope}%
\begin{pgfscope}%
\pgfsys@transformshift{1.621288in}{1.163535in}%
\pgfsys@useobject{currentmarker}{}%
\end{pgfscope}%
\begin{pgfscope}%
\pgfsys@transformshift{1.273442in}{1.036314in}%
\pgfsys@useobject{currentmarker}{}%
\end{pgfscope}%
\begin{pgfscope}%
\pgfsys@transformshift{0.816771in}{0.859335in}%
\pgfsys@useobject{currentmarker}{}%
\end{pgfscope}%
\begin{pgfscope}%
\pgfsys@transformshift{1.655115in}{1.216458in}%
\pgfsys@useobject{currentmarker}{}%
\end{pgfscope}%
\begin{pgfscope}%
\pgfsys@transformshift{2.864914in}{1.530982in}%
\pgfsys@useobject{currentmarker}{}%
\end{pgfscope}%
\begin{pgfscope}%
\pgfsys@transformshift{2.429883in}{1.522263in}%
\pgfsys@useobject{currentmarker}{}%
\end{pgfscope}%
\begin{pgfscope}%
\pgfsys@transformshift{2.848879in}{1.638179in}%
\pgfsys@useobject{currentmarker}{}%
\end{pgfscope}%
\begin{pgfscope}%
\pgfsys@transformshift{1.659778in}{1.219641in}%
\pgfsys@useobject{currentmarker}{}%
\end{pgfscope}%
\begin{pgfscope}%
\pgfsys@transformshift{1.617943in}{0.963482in}%
\pgfsys@useobject{currentmarker}{}%
\end{pgfscope}%
\begin{pgfscope}%
\pgfsys@transformshift{2.687400in}{1.462962in}%
\pgfsys@useobject{currentmarker}{}%
\end{pgfscope}%
\begin{pgfscope}%
\pgfsys@transformshift{1.111626in}{0.868441in}%
\pgfsys@useobject{currentmarker}{}%
\end{pgfscope}%
\begin{pgfscope}%
\pgfsys@transformshift{1.957412in}{1.226425in}%
\pgfsys@useobject{currentmarker}{}%
\end{pgfscope}%
\begin{pgfscope}%
\pgfsys@transformshift{3.146714in}{1.740034in}%
\pgfsys@useobject{currentmarker}{}%
\end{pgfscope}%
\begin{pgfscope}%
\pgfsys@transformshift{3.087372in}{1.731185in}%
\pgfsys@useobject{currentmarker}{}%
\end{pgfscope}%
\begin{pgfscope}%
\pgfsys@transformshift{1.502472in}{1.193774in}%
\pgfsys@useobject{currentmarker}{}%
\end{pgfscope}%
\begin{pgfscope}%
\pgfsys@transformshift{1.005349in}{0.884913in}%
\pgfsys@useobject{currentmarker}{}%
\end{pgfscope}%
\begin{pgfscope}%
\pgfsys@transformshift{1.259470in}{1.062108in}%
\pgfsys@useobject{currentmarker}{}%
\end{pgfscope}%
\begin{pgfscope}%
\pgfsys@transformshift{3.187662in}{2.003663in}%
\pgfsys@useobject{currentmarker}{}%
\end{pgfscope}%
\begin{pgfscope}%
\pgfsys@transformshift{1.370480in}{1.166722in}%
\pgfsys@useobject{currentmarker}{}%
\end{pgfscope}%
\begin{pgfscope}%
\pgfsys@transformshift{1.492792in}{1.148236in}%
\pgfsys@useobject{currentmarker}{}%
\end{pgfscope}%
\begin{pgfscope}%
\pgfsys@transformshift{1.307987in}{1.003846in}%
\pgfsys@useobject{currentmarker}{}%
\end{pgfscope}%
\begin{pgfscope}%
\pgfsys@transformshift{2.730322in}{1.543405in}%
\pgfsys@useobject{currentmarker}{}%
\end{pgfscope}%
\begin{pgfscope}%
\pgfsys@transformshift{1.531859in}{1.133758in}%
\pgfsys@useobject{currentmarker}{}%
\end{pgfscope}%
\begin{pgfscope}%
\pgfsys@transformshift{0.869375in}{0.867213in}%
\pgfsys@useobject{currentmarker}{}%
\end{pgfscope}%
\begin{pgfscope}%
\pgfsys@transformshift{2.254314in}{1.460425in}%
\pgfsys@useobject{currentmarker}{}%
\end{pgfscope}%
\begin{pgfscope}%
\pgfsys@transformshift{2.764608in}{1.691602in}%
\pgfsys@useobject{currentmarker}{}%
\end{pgfscope}%
\begin{pgfscope}%
\pgfsys@transformshift{2.996063in}{1.750052in}%
\pgfsys@useobject{currentmarker}{}%
\end{pgfscope}%
\begin{pgfscope}%
\pgfsys@transformshift{1.370370in}{1.051623in}%
\pgfsys@useobject{currentmarker}{}%
\end{pgfscope}%
\begin{pgfscope}%
\pgfsys@transformshift{1.892764in}{1.390797in}%
\pgfsys@useobject{currentmarker}{}%
\end{pgfscope}%
\begin{pgfscope}%
\pgfsys@transformshift{3.085708in}{1.786028in}%
\pgfsys@useobject{currentmarker}{}%
\end{pgfscope}%
\begin{pgfscope}%
\pgfsys@transformshift{3.118066in}{1.612567in}%
\pgfsys@useobject{currentmarker}{}%
\end{pgfscope}%
\begin{pgfscope}%
\pgfsys@transformshift{2.539127in}{1.531126in}%
\pgfsys@useobject{currentmarker}{}%
\end{pgfscope}%
\begin{pgfscope}%
\pgfsys@transformshift{1.342014in}{0.980703in}%
\pgfsys@useobject{currentmarker}{}%
\end{pgfscope}%
\begin{pgfscope}%
\pgfsys@transformshift{1.323542in}{0.857668in}%
\pgfsys@useobject{currentmarker}{}%
\end{pgfscope}%
\begin{pgfscope}%
\pgfsys@transformshift{0.830614in}{0.823950in}%
\pgfsys@useobject{currentmarker}{}%
\end{pgfscope}%
\begin{pgfscope}%
\pgfsys@transformshift{3.059195in}{1.471866in}%
\pgfsys@useobject{currentmarker}{}%
\end{pgfscope}%
\begin{pgfscope}%
\pgfsys@transformshift{2.192357in}{1.261543in}%
\pgfsys@useobject{currentmarker}{}%
\end{pgfscope}%
\begin{pgfscope}%
\pgfsys@transformshift{2.707652in}{1.618904in}%
\pgfsys@useobject{currentmarker}{}%
\end{pgfscope}%
\begin{pgfscope}%
\pgfsys@transformshift{2.949221in}{1.475507in}%
\pgfsys@useobject{currentmarker}{}%
\end{pgfscope}%
\begin{pgfscope}%
\pgfsys@transformshift{1.425928in}{1.091952in}%
\pgfsys@useobject{currentmarker}{}%
\end{pgfscope}%
\begin{pgfscope}%
\pgfsys@transformshift{0.994435in}{0.855209in}%
\pgfsys@useobject{currentmarker}{}%
\end{pgfscope}%
\begin{pgfscope}%
\pgfsys@transformshift{2.835025in}{1.655734in}%
\pgfsys@useobject{currentmarker}{}%
\end{pgfscope}%
\begin{pgfscope}%
\pgfsys@transformshift{3.111671in}{1.844110in}%
\pgfsys@useobject{currentmarker}{}%
\end{pgfscope}%
\begin{pgfscope}%
\pgfsys@transformshift{1.099748in}{0.960682in}%
\pgfsys@useobject{currentmarker}{}%
\end{pgfscope}%
\begin{pgfscope}%
\pgfsys@transformshift{1.720001in}{1.346481in}%
\pgfsys@useobject{currentmarker}{}%
\end{pgfscope}%
\begin{pgfscope}%
\pgfsys@transformshift{2.425112in}{1.563903in}%
\pgfsys@useobject{currentmarker}{}%
\end{pgfscope}%
\begin{pgfscope}%
\pgfsys@transformshift{0.892673in}{0.832178in}%
\pgfsys@useobject{currentmarker}{}%
\end{pgfscope}%
\begin{pgfscope}%
\pgfsys@transformshift{1.724678in}{1.245751in}%
\pgfsys@useobject{currentmarker}{}%
\end{pgfscope}%
\begin{pgfscope}%
\pgfsys@transformshift{2.059337in}{1.302500in}%
\pgfsys@useobject{currentmarker}{}%
\end{pgfscope}%
\begin{pgfscope}%
\pgfsys@transformshift{2.825914in}{1.690743in}%
\pgfsys@useobject{currentmarker}{}%
\end{pgfscope}%
\begin{pgfscope}%
\pgfsys@transformshift{2.647254in}{1.679565in}%
\pgfsys@useobject{currentmarker}{}%
\end{pgfscope}%
\begin{pgfscope}%
\pgfsys@transformshift{2.541234in}{1.561930in}%
\pgfsys@useobject{currentmarker}{}%
\end{pgfscope}%
\begin{pgfscope}%
\pgfsys@transformshift{1.378492in}{1.053928in}%
\pgfsys@useobject{currentmarker}{}%
\end{pgfscope}%
\begin{pgfscope}%
\pgfsys@transformshift{1.636712in}{1.026879in}%
\pgfsys@useobject{currentmarker}{}%
\end{pgfscope}%
\begin{pgfscope}%
\pgfsys@transformshift{1.971466in}{1.154450in}%
\pgfsys@useobject{currentmarker}{}%
\end{pgfscope}%
\begin{pgfscope}%
\pgfsys@transformshift{2.402251in}{1.554933in}%
\pgfsys@useobject{currentmarker}{}%
\end{pgfscope}%
\begin{pgfscope}%
\pgfsys@transformshift{2.829628in}{1.586640in}%
\pgfsys@useobject{currentmarker}{}%
\end{pgfscope}%
\begin{pgfscope}%
\pgfsys@transformshift{1.651813in}{0.964818in}%
\pgfsys@useobject{currentmarker}{}%
\end{pgfscope}%
\begin{pgfscope}%
\pgfsys@transformshift{2.218557in}{1.239380in}%
\pgfsys@useobject{currentmarker}{}%
\end{pgfscope}%
\begin{pgfscope}%
\pgfsys@transformshift{1.286771in}{1.055694in}%
\pgfsys@useobject{currentmarker}{}%
\end{pgfscope}%
\begin{pgfscope}%
\pgfsys@transformshift{2.406673in}{1.573822in}%
\pgfsys@useobject{currentmarker}{}%
\end{pgfscope}%
\begin{pgfscope}%
\pgfsys@transformshift{2.904211in}{1.571279in}%
\pgfsys@useobject{currentmarker}{}%
\end{pgfscope}%
\begin{pgfscope}%
\pgfsys@transformshift{2.831641in}{1.635316in}%
\pgfsys@useobject{currentmarker}{}%
\end{pgfscope}%
\begin{pgfscope}%
\pgfsys@transformshift{0.887476in}{0.795921in}%
\pgfsys@useobject{currentmarker}{}%
\end{pgfscope}%
\begin{pgfscope}%
\pgfsys@transformshift{2.211570in}{1.420077in}%
\pgfsys@useobject{currentmarker}{}%
\end{pgfscope}%
\begin{pgfscope}%
\pgfsys@transformshift{2.698044in}{1.591908in}%
\pgfsys@useobject{currentmarker}{}%
\end{pgfscope}%
\begin{pgfscope}%
\pgfsys@transformshift{1.845108in}{1.272700in}%
\pgfsys@useobject{currentmarker}{}%
\end{pgfscope}%
\begin{pgfscope}%
\pgfsys@transformshift{1.340977in}{1.029152in}%
\pgfsys@useobject{currentmarker}{}%
\end{pgfscope}%
\begin{pgfscope}%
\pgfsys@transformshift{1.134144in}{0.924014in}%
\pgfsys@useobject{currentmarker}{}%
\end{pgfscope}%
\begin{pgfscope}%
\pgfsys@transformshift{1.741924in}{1.198112in}%
\pgfsys@useobject{currentmarker}{}%
\end{pgfscope}%
\begin{pgfscope}%
\pgfsys@transformshift{2.972973in}{1.808074in}%
\pgfsys@useobject{currentmarker}{}%
\end{pgfscope}%
\begin{pgfscope}%
\pgfsys@transformshift{1.876284in}{1.222692in}%
\pgfsys@useobject{currentmarker}{}%
\end{pgfscope}%
\begin{pgfscope}%
\pgfsys@transformshift{1.087930in}{0.989459in}%
\pgfsys@useobject{currentmarker}{}%
\end{pgfscope}%
\begin{pgfscope}%
\pgfsys@transformshift{1.250321in}{0.969954in}%
\pgfsys@useobject{currentmarker}{}%
\end{pgfscope}%
\begin{pgfscope}%
\pgfsys@transformshift{1.562051in}{1.061863in}%
\pgfsys@useobject{currentmarker}{}%
\end{pgfscope}%
\begin{pgfscope}%
\pgfsys@transformshift{1.057569in}{0.945685in}%
\pgfsys@useobject{currentmarker}{}%
\end{pgfscope}%
\begin{pgfscope}%
\pgfsys@transformshift{3.014690in}{1.829683in}%
\pgfsys@useobject{currentmarker}{}%
\end{pgfscope}%
\begin{pgfscope}%
\pgfsys@transformshift{3.194578in}{1.934535in}%
\pgfsys@useobject{currentmarker}{}%
\end{pgfscope}%
\begin{pgfscope}%
\pgfsys@transformshift{2.160989in}{1.386701in}%
\pgfsys@useobject{currentmarker}{}%
\end{pgfscope}%
\begin{pgfscope}%
\pgfsys@transformshift{2.616314in}{1.644156in}%
\pgfsys@useobject{currentmarker}{}%
\end{pgfscope}%
\begin{pgfscope}%
\pgfsys@transformshift{2.643042in}{1.632753in}%
\pgfsys@useobject{currentmarker}{}%
\end{pgfscope}%
\begin{pgfscope}%
\pgfsys@transformshift{2.959634in}{1.743130in}%
\pgfsys@useobject{currentmarker}{}%
\end{pgfscope}%
\begin{pgfscope}%
\pgfsys@transformshift{3.174521in}{1.803715in}%
\pgfsys@useobject{currentmarker}{}%
\end{pgfscope}%
\begin{pgfscope}%
\pgfsys@transformshift{2.079036in}{1.357602in}%
\pgfsys@useobject{currentmarker}{}%
\end{pgfscope}%
\begin{pgfscope}%
\pgfsys@transformshift{2.821059in}{1.625376in}%
\pgfsys@useobject{currentmarker}{}%
\end{pgfscope}%
\begin{pgfscope}%
\pgfsys@transformshift{2.106322in}{1.340265in}%
\pgfsys@useobject{currentmarker}{}%
\end{pgfscope}%
\begin{pgfscope}%
\pgfsys@transformshift{2.415609in}{1.628564in}%
\pgfsys@useobject{currentmarker}{}%
\end{pgfscope}%
\begin{pgfscope}%
\pgfsys@transformshift{3.117021in}{1.755695in}%
\pgfsys@useobject{currentmarker}{}%
\end{pgfscope}%
\begin{pgfscope}%
\pgfsys@transformshift{1.016025in}{0.864440in}%
\pgfsys@useobject{currentmarker}{}%
\end{pgfscope}%
\begin{pgfscope}%
\pgfsys@transformshift{2.059386in}{1.373715in}%
\pgfsys@useobject{currentmarker}{}%
\end{pgfscope}%
\begin{pgfscope}%
\pgfsys@transformshift{2.265733in}{1.514248in}%
\pgfsys@useobject{currentmarker}{}%
\end{pgfscope}%
\begin{pgfscope}%
\pgfsys@transformshift{1.378394in}{1.050063in}%
\pgfsys@useobject{currentmarker}{}%
\end{pgfscope}%
\begin{pgfscope}%
\pgfsys@transformshift{0.981205in}{0.791310in}%
\pgfsys@useobject{currentmarker}{}%
\end{pgfscope}%
\begin{pgfscope}%
\pgfsys@transformshift{1.083839in}{0.866649in}%
\pgfsys@useobject{currentmarker}{}%
\end{pgfscope}%
\begin{pgfscope}%
\pgfsys@transformshift{3.049868in}{1.663490in}%
\pgfsys@useobject{currentmarker}{}%
\end{pgfscope}%
\begin{pgfscope}%
\pgfsys@transformshift{2.758112in}{1.617592in}%
\pgfsys@useobject{currentmarker}{}%
\end{pgfscope}%
\begin{pgfscope}%
\pgfsys@transformshift{1.327647in}{0.921693in}%
\pgfsys@useobject{currentmarker}{}%
\end{pgfscope}%
\begin{pgfscope}%
\pgfsys@transformshift{1.041134in}{0.816046in}%
\pgfsys@useobject{currentmarker}{}%
\end{pgfscope}%
\begin{pgfscope}%
\pgfsys@transformshift{1.091102in}{0.980271in}%
\pgfsys@useobject{currentmarker}{}%
\end{pgfscope}%
\begin{pgfscope}%
\pgfsys@transformshift{2.047327in}{1.394801in}%
\pgfsys@useobject{currentmarker}{}%
\end{pgfscope}%
\begin{pgfscope}%
\pgfsys@transformshift{1.981169in}{1.450934in}%
\pgfsys@useobject{currentmarker}{}%
\end{pgfscope}%
\begin{pgfscope}%
\pgfsys@transformshift{0.883000in}{0.834332in}%
\pgfsys@useobject{currentmarker}{}%
\end{pgfscope}%
\begin{pgfscope}%
\pgfsys@transformshift{1.588943in}{1.221924in}%
\pgfsys@useobject{currentmarker}{}%
\end{pgfscope}%
\begin{pgfscope}%
\pgfsys@transformshift{1.286695in}{1.018256in}%
\pgfsys@useobject{currentmarker}{}%
\end{pgfscope}%
\begin{pgfscope}%
\pgfsys@transformshift{2.333700in}{1.572724in}%
\pgfsys@useobject{currentmarker}{}%
\end{pgfscope}%
\begin{pgfscope}%
\pgfsys@transformshift{2.055476in}{1.254121in}%
\pgfsys@useobject{currentmarker}{}%
\end{pgfscope}%
\begin{pgfscope}%
\pgfsys@transformshift{1.449461in}{1.052307in}%
\pgfsys@useobject{currentmarker}{}%
\end{pgfscope}%
\begin{pgfscope}%
\pgfsys@transformshift{2.972763in}{1.602948in}%
\pgfsys@useobject{currentmarker}{}%
\end{pgfscope}%
\begin{pgfscope}%
\pgfsys@transformshift{2.629420in}{1.484090in}%
\pgfsys@useobject{currentmarker}{}%
\end{pgfscope}%
\begin{pgfscope}%
\pgfsys@transformshift{1.080851in}{0.723090in}%
\pgfsys@useobject{currentmarker}{}%
\end{pgfscope}%
\begin{pgfscope}%
\pgfsys@transformshift{2.494802in}{1.583111in}%
\pgfsys@useobject{currentmarker}{}%
\end{pgfscope}%
\begin{pgfscope}%
\pgfsys@transformshift{2.385108in}{1.486122in}%
\pgfsys@useobject{currentmarker}{}%
\end{pgfscope}%
\begin{pgfscope}%
\pgfsys@transformshift{1.599215in}{1.185561in}%
\pgfsys@useobject{currentmarker}{}%
\end{pgfscope}%
\begin{pgfscope}%
\pgfsys@transformshift{1.651195in}{1.064778in}%
\pgfsys@useobject{currentmarker}{}%
\end{pgfscope}%
\begin{pgfscope}%
\pgfsys@transformshift{1.159487in}{0.912299in}%
\pgfsys@useobject{currentmarker}{}%
\end{pgfscope}%
\begin{pgfscope}%
\pgfsys@transformshift{1.639835in}{1.231119in}%
\pgfsys@useobject{currentmarker}{}%
\end{pgfscope}%
\begin{pgfscope}%
\pgfsys@transformshift{1.975175in}{1.354448in}%
\pgfsys@useobject{currentmarker}{}%
\end{pgfscope}%
\begin{pgfscope}%
\pgfsys@transformshift{2.864197in}{1.755426in}%
\pgfsys@useobject{currentmarker}{}%
\end{pgfscope}%
\begin{pgfscope}%
\pgfsys@transformshift{3.112752in}{1.855348in}%
\pgfsys@useobject{currentmarker}{}%
\end{pgfscope}%
\begin{pgfscope}%
\pgfsys@transformshift{2.899758in}{1.704589in}%
\pgfsys@useobject{currentmarker}{}%
\end{pgfscope}%
\begin{pgfscope}%
\pgfsys@transformshift{2.483586in}{1.486180in}%
\pgfsys@useobject{currentmarker}{}%
\end{pgfscope}%
\begin{pgfscope}%
\pgfsys@transformshift{3.047586in}{1.822245in}%
\pgfsys@useobject{currentmarker}{}%
\end{pgfscope}%
\begin{pgfscope}%
\pgfsys@transformshift{1.037591in}{0.837405in}%
\pgfsys@useobject{currentmarker}{}%
\end{pgfscope}%
\begin{pgfscope}%
\pgfsys@transformshift{2.281388in}{1.407490in}%
\pgfsys@useobject{currentmarker}{}%
\end{pgfscope}%
\begin{pgfscope}%
\pgfsys@transformshift{1.554085in}{1.149011in}%
\pgfsys@useobject{currentmarker}{}%
\end{pgfscope}%
\begin{pgfscope}%
\pgfsys@transformshift{3.225527in}{1.888759in}%
\pgfsys@useobject{currentmarker}{}%
\end{pgfscope}%
\begin{pgfscope}%
\pgfsys@transformshift{1.839639in}{1.183995in}%
\pgfsys@useobject{currentmarker}{}%
\end{pgfscope}%
\begin{pgfscope}%
\pgfsys@transformshift{2.626210in}{1.481570in}%
\pgfsys@useobject{currentmarker}{}%
\end{pgfscope}%
\begin{pgfscope}%
\pgfsys@transformshift{1.598983in}{1.255893in}%
\pgfsys@useobject{currentmarker}{}%
\end{pgfscope}%
\begin{pgfscope}%
\pgfsys@transformshift{1.926732in}{1.354646in}%
\pgfsys@useobject{currentmarker}{}%
\end{pgfscope}%
\begin{pgfscope}%
\pgfsys@transformshift{1.481200in}{1.118367in}%
\pgfsys@useobject{currentmarker}{}%
\end{pgfscope}%
\begin{pgfscope}%
\pgfsys@transformshift{3.079558in}{1.862711in}%
\pgfsys@useobject{currentmarker}{}%
\end{pgfscope}%
\begin{pgfscope}%
\pgfsys@transformshift{1.096663in}{0.850094in}%
\pgfsys@useobject{currentmarker}{}%
\end{pgfscope}%
\begin{pgfscope}%
\pgfsys@transformshift{0.844833in}{0.872468in}%
\pgfsys@useobject{currentmarker}{}%
\end{pgfscope}%
\begin{pgfscope}%
\pgfsys@transformshift{1.170986in}{0.897889in}%
\pgfsys@useobject{currentmarker}{}%
\end{pgfscope}%
\begin{pgfscope}%
\pgfsys@transformshift{2.478542in}{1.560696in}%
\pgfsys@useobject{currentmarker}{}%
\end{pgfscope}%
\begin{pgfscope}%
\pgfsys@transformshift{2.275891in}{1.498840in}%
\pgfsys@useobject{currentmarker}{}%
\end{pgfscope}%
\begin{pgfscope}%
\pgfsys@transformshift{1.297523in}{1.089043in}%
\pgfsys@useobject{currentmarker}{}%
\end{pgfscope}%
\begin{pgfscope}%
\pgfsys@transformshift{2.672414in}{1.600151in}%
\pgfsys@useobject{currentmarker}{}%
\end{pgfscope}%
\begin{pgfscope}%
\pgfsys@transformshift{3.154956in}{1.751306in}%
\pgfsys@useobject{currentmarker}{}%
\end{pgfscope}%
\begin{pgfscope}%
\pgfsys@transformshift{3.203930in}{1.920322in}%
\pgfsys@useobject{currentmarker}{}%
\end{pgfscope}%
\begin{pgfscope}%
\pgfsys@transformshift{1.333126in}{1.023618in}%
\pgfsys@useobject{currentmarker}{}%
\end{pgfscope}%
\begin{pgfscope}%
\pgfsys@transformshift{2.469772in}{1.458692in}%
\pgfsys@useobject{currentmarker}{}%
\end{pgfscope}%
\begin{pgfscope}%
\pgfsys@transformshift{3.138468in}{1.544798in}%
\pgfsys@useobject{currentmarker}{}%
\end{pgfscope}%
\begin{pgfscope}%
\pgfsys@transformshift{2.006718in}{1.420894in}%
\pgfsys@useobject{currentmarker}{}%
\end{pgfscope}%
\begin{pgfscope}%
\pgfsys@transformshift{2.109220in}{1.410114in}%
\pgfsys@useobject{currentmarker}{}%
\end{pgfscope}%
\begin{pgfscope}%
\pgfsys@transformshift{2.808134in}{1.647053in}%
\pgfsys@useobject{currentmarker}{}%
\end{pgfscope}%
\begin{pgfscope}%
\pgfsys@transformshift{1.787837in}{1.135564in}%
\pgfsys@useobject{currentmarker}{}%
\end{pgfscope}%
\begin{pgfscope}%
\pgfsys@transformshift{0.937576in}{0.811406in}%
\pgfsys@useobject{currentmarker}{}%
\end{pgfscope}%
\begin{pgfscope}%
\pgfsys@transformshift{3.107668in}{1.764034in}%
\pgfsys@useobject{currentmarker}{}%
\end{pgfscope}%
\begin{pgfscope}%
\pgfsys@transformshift{1.050375in}{0.901482in}%
\pgfsys@useobject{currentmarker}{}%
\end{pgfscope}%
\begin{pgfscope}%
\pgfsys@transformshift{2.321632in}{1.330210in}%
\pgfsys@useobject{currentmarker}{}%
\end{pgfscope}%
\begin{pgfscope}%
\pgfsys@transformshift{2.837916in}{1.626268in}%
\pgfsys@useobject{currentmarker}{}%
\end{pgfscope}%
\begin{pgfscope}%
\pgfsys@transformshift{2.424887in}{1.613599in}%
\pgfsys@useobject{currentmarker}{}%
\end{pgfscope}%
\begin{pgfscope}%
\pgfsys@transformshift{3.102195in}{1.893624in}%
\pgfsys@useobject{currentmarker}{}%
\end{pgfscope}%
\begin{pgfscope}%
\pgfsys@transformshift{2.883884in}{1.764779in}%
\pgfsys@useobject{currentmarker}{}%
\end{pgfscope}%
\begin{pgfscope}%
\pgfsys@transformshift{1.646971in}{1.120555in}%
\pgfsys@useobject{currentmarker}{}%
\end{pgfscope}%
\begin{pgfscope}%
\pgfsys@transformshift{1.651390in}{1.090931in}%
\pgfsys@useobject{currentmarker}{}%
\end{pgfscope}%
\begin{pgfscope}%
\pgfsys@transformshift{3.044617in}{1.668865in}%
\pgfsys@useobject{currentmarker}{}%
\end{pgfscope}%
\begin{pgfscope}%
\pgfsys@transformshift{2.982428in}{1.673983in}%
\pgfsys@useobject{currentmarker}{}%
\end{pgfscope}%
\begin{pgfscope}%
\pgfsys@transformshift{2.577630in}{1.555370in}%
\pgfsys@useobject{currentmarker}{}%
\end{pgfscope}%
\begin{pgfscope}%
\pgfsys@transformshift{2.748542in}{1.617028in}%
\pgfsys@useobject{currentmarker}{}%
\end{pgfscope}%
\begin{pgfscope}%
\pgfsys@transformshift{2.055936in}{1.385001in}%
\pgfsys@useobject{currentmarker}{}%
\end{pgfscope}%
\begin{pgfscope}%
\pgfsys@transformshift{1.114275in}{0.859042in}%
\pgfsys@useobject{currentmarker}{}%
\end{pgfscope}%
\begin{pgfscope}%
\pgfsys@transformshift{2.134288in}{1.456266in}%
\pgfsys@useobject{currentmarker}{}%
\end{pgfscope}%
\begin{pgfscope}%
\pgfsys@transformshift{1.732161in}{1.333498in}%
\pgfsys@useobject{currentmarker}{}%
\end{pgfscope}%
\begin{pgfscope}%
\pgfsys@transformshift{2.247868in}{1.522867in}%
\pgfsys@useobject{currentmarker}{}%
\end{pgfscope}%
\begin{pgfscope}%
\pgfsys@transformshift{2.377195in}{1.487424in}%
\pgfsys@useobject{currentmarker}{}%
\end{pgfscope}%
\begin{pgfscope}%
\pgfsys@transformshift{2.334273in}{1.633051in}%
\pgfsys@useobject{currentmarker}{}%
\end{pgfscope}%
\begin{pgfscope}%
\pgfsys@transformshift{1.065825in}{0.967479in}%
\pgfsys@useobject{currentmarker}{}%
\end{pgfscope}%
\begin{pgfscope}%
\pgfsys@transformshift{1.731200in}{1.184342in}%
\pgfsys@useobject{currentmarker}{}%
\end{pgfscope}%
\begin{pgfscope}%
\pgfsys@transformshift{1.629604in}{1.195093in}%
\pgfsys@useobject{currentmarker}{}%
\end{pgfscope}%
\begin{pgfscope}%
\pgfsys@transformshift{1.554345in}{1.162217in}%
\pgfsys@useobject{currentmarker}{}%
\end{pgfscope}%
\begin{pgfscope}%
\pgfsys@transformshift{1.628124in}{1.137086in}%
\pgfsys@useobject{currentmarker}{}%
\end{pgfscope}%
\begin{pgfscope}%
\pgfsys@transformshift{0.954063in}{0.768735in}%
\pgfsys@useobject{currentmarker}{}%
\end{pgfscope}%
\begin{pgfscope}%
\pgfsys@transformshift{1.177483in}{0.937419in}%
\pgfsys@useobject{currentmarker}{}%
\end{pgfscope}%
\begin{pgfscope}%
\pgfsys@transformshift{2.633013in}{1.537293in}%
\pgfsys@useobject{currentmarker}{}%
\end{pgfscope}%
\begin{pgfscope}%
\pgfsys@transformshift{2.322003in}{1.522040in}%
\pgfsys@useobject{currentmarker}{}%
\end{pgfscope}%
\begin{pgfscope}%
\pgfsys@transformshift{3.221495in}{2.021399in}%
\pgfsys@useobject{currentmarker}{}%
\end{pgfscope}%
\begin{pgfscope}%
\pgfsys@transformshift{0.929945in}{0.897037in}%
\pgfsys@useobject{currentmarker}{}%
\end{pgfscope}%
\begin{pgfscope}%
\pgfsys@transformshift{1.122200in}{0.781343in}%
\pgfsys@useobject{currentmarker}{}%
\end{pgfscope}%
\begin{pgfscope}%
\pgfsys@transformshift{0.846585in}{0.646959in}%
\pgfsys@useobject{currentmarker}{}%
\end{pgfscope}%
\begin{pgfscope}%
\pgfsys@transformshift{2.005854in}{1.431337in}%
\pgfsys@useobject{currentmarker}{}%
\end{pgfscope}%
\begin{pgfscope}%
\pgfsys@transformshift{1.875086in}{1.295653in}%
\pgfsys@useobject{currentmarker}{}%
\end{pgfscope}%
\begin{pgfscope}%
\pgfsys@transformshift{1.740921in}{1.142207in}%
\pgfsys@useobject{currentmarker}{}%
\end{pgfscope}%
\begin{pgfscope}%
\pgfsys@transformshift{1.993063in}{1.361690in}%
\pgfsys@useobject{currentmarker}{}%
\end{pgfscope}%
\begin{pgfscope}%
\pgfsys@transformshift{3.210869in}{1.956822in}%
\pgfsys@useobject{currentmarker}{}%
\end{pgfscope}%
\begin{pgfscope}%
\pgfsys@transformshift{0.811126in}{0.719628in}%
\pgfsys@useobject{currentmarker}{}%
\end{pgfscope}%
\begin{pgfscope}%
\pgfsys@transformshift{2.484748in}{1.551791in}%
\pgfsys@useobject{currentmarker}{}%
\end{pgfscope}%
\begin{pgfscope}%
\pgfsys@transformshift{1.666458in}{1.134296in}%
\pgfsys@useobject{currentmarker}{}%
\end{pgfscope}%
\begin{pgfscope}%
\pgfsys@transformshift{0.869477in}{0.819480in}%
\pgfsys@useobject{currentmarker}{}%
\end{pgfscope}%
\begin{pgfscope}%
\pgfsys@transformshift{3.161992in}{2.013887in}%
\pgfsys@useobject{currentmarker}{}%
\end{pgfscope}%
\begin{pgfscope}%
\pgfsys@transformshift{2.280667in}{1.415372in}%
\pgfsys@useobject{currentmarker}{}%
\end{pgfscope}%
\begin{pgfscope}%
\pgfsys@transformshift{1.968501in}{1.357364in}%
\pgfsys@useobject{currentmarker}{}%
\end{pgfscope}%
\begin{pgfscope}%
\pgfsys@transformshift{2.051413in}{1.385976in}%
\pgfsys@useobject{currentmarker}{}%
\end{pgfscope}%
\begin{pgfscope}%
\pgfsys@transformshift{1.566841in}{1.142136in}%
\pgfsys@useobject{currentmarker}{}%
\end{pgfscope}%
\begin{pgfscope}%
\pgfsys@transformshift{3.152202in}{1.760919in}%
\pgfsys@useobject{currentmarker}{}%
\end{pgfscope}%
\begin{pgfscope}%
\pgfsys@transformshift{2.803047in}{1.638395in}%
\pgfsys@useobject{currentmarker}{}%
\end{pgfscope}%
\begin{pgfscope}%
\pgfsys@transformshift{2.942017in}{1.666618in}%
\pgfsys@useobject{currentmarker}{}%
\end{pgfscope}%
\begin{pgfscope}%
\pgfsys@transformshift{1.006902in}{0.836911in}%
\pgfsys@useobject{currentmarker}{}%
\end{pgfscope}%
\begin{pgfscope}%
\pgfsys@transformshift{1.215515in}{0.991302in}%
\pgfsys@useobject{currentmarker}{}%
\end{pgfscope}%
\begin{pgfscope}%
\pgfsys@transformshift{1.817243in}{1.241012in}%
\pgfsys@useobject{currentmarker}{}%
\end{pgfscope}%
\begin{pgfscope}%
\pgfsys@transformshift{1.703076in}{1.194846in}%
\pgfsys@useobject{currentmarker}{}%
\end{pgfscope}%
\begin{pgfscope}%
\pgfsys@transformshift{3.198753in}{2.084030in}%
\pgfsys@useobject{currentmarker}{}%
\end{pgfscope}%
\begin{pgfscope}%
\pgfsys@transformshift{1.000793in}{0.971063in}%
\pgfsys@useobject{currentmarker}{}%
\end{pgfscope}%
\begin{pgfscope}%
\pgfsys@transformshift{2.690189in}{1.354584in}%
\pgfsys@useobject{currentmarker}{}%
\end{pgfscope}%
\begin{pgfscope}%
\pgfsys@transformshift{2.333922in}{1.596163in}%
\pgfsys@useobject{currentmarker}{}%
\end{pgfscope}%
\begin{pgfscope}%
\pgfsys@transformshift{2.286767in}{1.490589in}%
\pgfsys@useobject{currentmarker}{}%
\end{pgfscope}%
\begin{pgfscope}%
\pgfsys@transformshift{2.166727in}{1.354828in}%
\pgfsys@useobject{currentmarker}{}%
\end{pgfscope}%
\begin{pgfscope}%
\pgfsys@transformshift{1.158066in}{0.889587in}%
\pgfsys@useobject{currentmarker}{}%
\end{pgfscope}%
\begin{pgfscope}%
\pgfsys@transformshift{1.020990in}{0.908916in}%
\pgfsys@useobject{currentmarker}{}%
\end{pgfscope}%
\begin{pgfscope}%
\pgfsys@transformshift{2.190143in}{1.490844in}%
\pgfsys@useobject{currentmarker}{}%
\end{pgfscope}%
\begin{pgfscope}%
\pgfsys@transformshift{2.000706in}{1.289555in}%
\pgfsys@useobject{currentmarker}{}%
\end{pgfscope}%
\begin{pgfscope}%
\pgfsys@transformshift{2.225144in}{1.443739in}%
\pgfsys@useobject{currentmarker}{}%
\end{pgfscope}%
\begin{pgfscope}%
\pgfsys@transformshift{1.794139in}{1.244802in}%
\pgfsys@useobject{currentmarker}{}%
\end{pgfscope}%
\begin{pgfscope}%
\pgfsys@transformshift{1.949429in}{1.376705in}%
\pgfsys@useobject{currentmarker}{}%
\end{pgfscope}%
\begin{pgfscope}%
\pgfsys@transformshift{1.893269in}{1.362553in}%
\pgfsys@useobject{currentmarker}{}%
\end{pgfscope}%
\begin{pgfscope}%
\pgfsys@transformshift{2.081267in}{1.404518in}%
\pgfsys@useobject{currentmarker}{}%
\end{pgfscope}%
\begin{pgfscope}%
\pgfsys@transformshift{1.754131in}{1.260828in}%
\pgfsys@useobject{currentmarker}{}%
\end{pgfscope}%
\begin{pgfscope}%
\pgfsys@transformshift{1.678204in}{0.984830in}%
\pgfsys@useobject{currentmarker}{}%
\end{pgfscope}%
\begin{pgfscope}%
\pgfsys@transformshift{3.086321in}{1.827249in}%
\pgfsys@useobject{currentmarker}{}%
\end{pgfscope}%
\begin{pgfscope}%
\pgfsys@transformshift{0.952979in}{0.877627in}%
\pgfsys@useobject{currentmarker}{}%
\end{pgfscope}%
\begin{pgfscope}%
\pgfsys@transformshift{1.987490in}{1.383727in}%
\pgfsys@useobject{currentmarker}{}%
\end{pgfscope}%
\begin{pgfscope}%
\pgfsys@transformshift{0.942894in}{0.868423in}%
\pgfsys@useobject{currentmarker}{}%
\end{pgfscope}%
\begin{pgfscope}%
\pgfsys@transformshift{1.270677in}{0.915549in}%
\pgfsys@useobject{currentmarker}{}%
\end{pgfscope}%
\begin{pgfscope}%
\pgfsys@transformshift{1.641770in}{1.208401in}%
\pgfsys@useobject{currentmarker}{}%
\end{pgfscope}%
\begin{pgfscope}%
\pgfsys@transformshift{2.642897in}{1.716710in}%
\pgfsys@useobject{currentmarker}{}%
\end{pgfscope}%
\begin{pgfscope}%
\pgfsys@transformshift{1.784145in}{1.176607in}%
\pgfsys@useobject{currentmarker}{}%
\end{pgfscope}%
\begin{pgfscope}%
\pgfsys@transformshift{1.652917in}{1.151453in}%
\pgfsys@useobject{currentmarker}{}%
\end{pgfscope}%
\begin{pgfscope}%
\pgfsys@transformshift{2.048930in}{1.382023in}%
\pgfsys@useobject{currentmarker}{}%
\end{pgfscope}%
\begin{pgfscope}%
\pgfsys@transformshift{2.794197in}{1.611388in}%
\pgfsys@useobject{currentmarker}{}%
\end{pgfscope}%
\begin{pgfscope}%
\pgfsys@transformshift{0.933652in}{0.849886in}%
\pgfsys@useobject{currentmarker}{}%
\end{pgfscope}%
\begin{pgfscope}%
\pgfsys@transformshift{3.164631in}{1.725383in}%
\pgfsys@useobject{currentmarker}{}%
\end{pgfscope}%
\begin{pgfscope}%
\pgfsys@transformshift{1.585266in}{1.150856in}%
\pgfsys@useobject{currentmarker}{}%
\end{pgfscope}%
\begin{pgfscope}%
\pgfsys@transformshift{1.870436in}{1.317390in}%
\pgfsys@useobject{currentmarker}{}%
\end{pgfscope}%
\begin{pgfscope}%
\pgfsys@transformshift{3.186615in}{1.880032in}%
\pgfsys@useobject{currentmarker}{}%
\end{pgfscope}%
\begin{pgfscope}%
\pgfsys@transformshift{2.247311in}{1.325574in}%
\pgfsys@useobject{currentmarker}{}%
\end{pgfscope}%
\begin{pgfscope}%
\pgfsys@transformshift{1.348826in}{1.077228in}%
\pgfsys@useobject{currentmarker}{}%
\end{pgfscope}%
\begin{pgfscope}%
\pgfsys@transformshift{0.868329in}{0.757539in}%
\pgfsys@useobject{currentmarker}{}%
\end{pgfscope}%
\begin{pgfscope}%
\pgfsys@transformshift{2.191645in}{1.411857in}%
\pgfsys@useobject{currentmarker}{}%
\end{pgfscope}%
\begin{pgfscope}%
\pgfsys@transformshift{1.463899in}{1.013181in}%
\pgfsys@useobject{currentmarker}{}%
\end{pgfscope}%
\begin{pgfscope}%
\pgfsys@transformshift{2.174645in}{1.446959in}%
\pgfsys@useobject{currentmarker}{}%
\end{pgfscope}%
\begin{pgfscope}%
\pgfsys@transformshift{2.828028in}{1.646063in}%
\pgfsys@useobject{currentmarker}{}%
\end{pgfscope}%
\begin{pgfscope}%
\pgfsys@transformshift{2.175345in}{1.463404in}%
\pgfsys@useobject{currentmarker}{}%
\end{pgfscope}%
\begin{pgfscope}%
\pgfsys@transformshift{2.495108in}{1.595275in}%
\pgfsys@useobject{currentmarker}{}%
\end{pgfscope}%
\begin{pgfscope}%
\pgfsys@transformshift{1.502668in}{1.082902in}%
\pgfsys@useobject{currentmarker}{}%
\end{pgfscope}%
\begin{pgfscope}%
\pgfsys@transformshift{1.150856in}{0.901071in}%
\pgfsys@useobject{currentmarker}{}%
\end{pgfscope}%
\begin{pgfscope}%
\pgfsys@transformshift{2.870411in}{1.667083in}%
\pgfsys@useobject{currentmarker}{}%
\end{pgfscope}%
\begin{pgfscope}%
\pgfsys@transformshift{2.234068in}{1.475929in}%
\pgfsys@useobject{currentmarker}{}%
\end{pgfscope}%
\begin{pgfscope}%
\pgfsys@transformshift{1.023715in}{0.962243in}%
\pgfsys@useobject{currentmarker}{}%
\end{pgfscope}%
\begin{pgfscope}%
\pgfsys@transformshift{1.352409in}{1.069319in}%
\pgfsys@useobject{currentmarker}{}%
\end{pgfscope}%
\begin{pgfscope}%
\pgfsys@transformshift{2.956278in}{1.800343in}%
\pgfsys@useobject{currentmarker}{}%
\end{pgfscope}%
\begin{pgfscope}%
\pgfsys@transformshift{2.832148in}{1.654812in}%
\pgfsys@useobject{currentmarker}{}%
\end{pgfscope}%
\begin{pgfscope}%
\pgfsys@transformshift{2.139278in}{1.341119in}%
\pgfsys@useobject{currentmarker}{}%
\end{pgfscope}%
\begin{pgfscope}%
\pgfsys@transformshift{0.815783in}{0.744355in}%
\pgfsys@useobject{currentmarker}{}%
\end{pgfscope}%
\begin{pgfscope}%
\pgfsys@transformshift{2.244031in}{1.418874in}%
\pgfsys@useobject{currentmarker}{}%
\end{pgfscope}%
\begin{pgfscope}%
\pgfsys@transformshift{2.686015in}{1.653729in}%
\pgfsys@useobject{currentmarker}{}%
\end{pgfscope}%
\begin{pgfscope}%
\pgfsys@transformshift{2.686736in}{1.719755in}%
\pgfsys@useobject{currentmarker}{}%
\end{pgfscope}%
\begin{pgfscope}%
\pgfsys@transformshift{3.165807in}{1.831660in}%
\pgfsys@useobject{currentmarker}{}%
\end{pgfscope}%
\begin{pgfscope}%
\pgfsys@transformshift{2.092186in}{1.574494in}%
\pgfsys@useobject{currentmarker}{}%
\end{pgfscope}%
\begin{pgfscope}%
\pgfsys@transformshift{0.851548in}{0.777877in}%
\pgfsys@useobject{currentmarker}{}%
\end{pgfscope}%
\begin{pgfscope}%
\pgfsys@transformshift{2.796348in}{1.397805in}%
\pgfsys@useobject{currentmarker}{}%
\end{pgfscope}%
\begin{pgfscope}%
\pgfsys@transformshift{2.838805in}{1.593072in}%
\pgfsys@useobject{currentmarker}{}%
\end{pgfscope}%
\begin{pgfscope}%
\pgfsys@transformshift{1.831201in}{1.360953in}%
\pgfsys@useobject{currentmarker}{}%
\end{pgfscope}%
\begin{pgfscope}%
\pgfsys@transformshift{2.689437in}{1.633762in}%
\pgfsys@useobject{currentmarker}{}%
\end{pgfscope}%
\begin{pgfscope}%
\pgfsys@transformshift{3.197197in}{1.833019in}%
\pgfsys@useobject{currentmarker}{}%
\end{pgfscope}%
\begin{pgfscope}%
\pgfsys@transformshift{1.684826in}{1.269674in}%
\pgfsys@useobject{currentmarker}{}%
\end{pgfscope}%
\begin{pgfscope}%
\pgfsys@transformshift{1.161255in}{0.979614in}%
\pgfsys@useobject{currentmarker}{}%
\end{pgfscope}%
\begin{pgfscope}%
\pgfsys@transformshift{1.553169in}{1.117543in}%
\pgfsys@useobject{currentmarker}{}%
\end{pgfscope}%
\begin{pgfscope}%
\pgfsys@transformshift{1.236630in}{0.913683in}%
\pgfsys@useobject{currentmarker}{}%
\end{pgfscope}%
\begin{pgfscope}%
\pgfsys@transformshift{2.981750in}{1.688584in}%
\pgfsys@useobject{currentmarker}{}%
\end{pgfscope}%
\begin{pgfscope}%
\pgfsys@transformshift{2.100723in}{1.430961in}%
\pgfsys@useobject{currentmarker}{}%
\end{pgfscope}%
\begin{pgfscope}%
\pgfsys@transformshift{3.146459in}{1.900880in}%
\pgfsys@useobject{currentmarker}{}%
\end{pgfscope}%
\begin{pgfscope}%
\pgfsys@transformshift{2.726704in}{1.689266in}%
\pgfsys@useobject{currentmarker}{}%
\end{pgfscope}%
\begin{pgfscope}%
\pgfsys@transformshift{2.143550in}{1.467144in}%
\pgfsys@useobject{currentmarker}{}%
\end{pgfscope}%
\begin{pgfscope}%
\pgfsys@transformshift{3.136114in}{1.810595in}%
\pgfsys@useobject{currentmarker}{}%
\end{pgfscope}%
\begin{pgfscope}%
\pgfsys@transformshift{2.465394in}{1.670282in}%
\pgfsys@useobject{currentmarker}{}%
\end{pgfscope}%
\begin{pgfscope}%
\pgfsys@transformshift{2.834310in}{1.624598in}%
\pgfsys@useobject{currentmarker}{}%
\end{pgfscope}%
\begin{pgfscope}%
\pgfsys@transformshift{2.776966in}{1.705739in}%
\pgfsys@useobject{currentmarker}{}%
\end{pgfscope}%
\begin{pgfscope}%
\pgfsys@transformshift{2.161602in}{1.427972in}%
\pgfsys@useobject{currentmarker}{}%
\end{pgfscope}%
\begin{pgfscope}%
\pgfsys@transformshift{2.798029in}{1.721010in}%
\pgfsys@useobject{currentmarker}{}%
\end{pgfscope}%
\begin{pgfscope}%
\pgfsys@transformshift{1.696523in}{1.179026in}%
\pgfsys@useobject{currentmarker}{}%
\end{pgfscope}%
\begin{pgfscope}%
\pgfsys@transformshift{1.333631in}{1.101932in}%
\pgfsys@useobject{currentmarker}{}%
\end{pgfscope}%
\begin{pgfscope}%
\pgfsys@transformshift{1.060641in}{0.880194in}%
\pgfsys@useobject{currentmarker}{}%
\end{pgfscope}%
\begin{pgfscope}%
\pgfsys@transformshift{0.882765in}{0.798469in}%
\pgfsys@useobject{currentmarker}{}%
\end{pgfscope}%
\begin{pgfscope}%
\pgfsys@transformshift{2.947883in}{1.763341in}%
\pgfsys@useobject{currentmarker}{}%
\end{pgfscope}%
\begin{pgfscope}%
\pgfsys@transformshift{1.706917in}{1.232561in}%
\pgfsys@useobject{currentmarker}{}%
\end{pgfscope}%
\begin{pgfscope}%
\pgfsys@transformshift{3.074332in}{1.674499in}%
\pgfsys@useobject{currentmarker}{}%
\end{pgfscope}%
\begin{pgfscope}%
\pgfsys@transformshift{2.114292in}{1.404568in}%
\pgfsys@useobject{currentmarker}{}%
\end{pgfscope}%
\begin{pgfscope}%
\pgfsys@transformshift{1.482221in}{1.184559in}%
\pgfsys@useobject{currentmarker}{}%
\end{pgfscope}%
\begin{pgfscope}%
\pgfsys@transformshift{0.851635in}{0.842408in}%
\pgfsys@useobject{currentmarker}{}%
\end{pgfscope}%
\begin{pgfscope}%
\pgfsys@transformshift{3.116840in}{1.847128in}%
\pgfsys@useobject{currentmarker}{}%
\end{pgfscope}%
\begin{pgfscope}%
\pgfsys@transformshift{3.212257in}{1.826890in}%
\pgfsys@useobject{currentmarker}{}%
\end{pgfscope}%
\begin{pgfscope}%
\pgfsys@transformshift{2.332445in}{1.391252in}%
\pgfsys@useobject{currentmarker}{}%
\end{pgfscope}%
\begin{pgfscope}%
\pgfsys@transformshift{1.999789in}{1.162596in}%
\pgfsys@useobject{currentmarker}{}%
\end{pgfscope}%
\begin{pgfscope}%
\pgfsys@transformshift{2.749023in}{1.782326in}%
\pgfsys@useobject{currentmarker}{}%
\end{pgfscope}%
\begin{pgfscope}%
\pgfsys@transformshift{1.702201in}{1.214409in}%
\pgfsys@useobject{currentmarker}{}%
\end{pgfscope}%
\begin{pgfscope}%
\pgfsys@transformshift{3.124683in}{1.727722in}%
\pgfsys@useobject{currentmarker}{}%
\end{pgfscope}%
\begin{pgfscope}%
\pgfsys@transformshift{1.079797in}{0.819224in}%
\pgfsys@useobject{currentmarker}{}%
\end{pgfscope}%
\begin{pgfscope}%
\pgfsys@transformshift{1.612587in}{0.990827in}%
\pgfsys@useobject{currentmarker}{}%
\end{pgfscope}%
\begin{pgfscope}%
\pgfsys@transformshift{2.804812in}{1.605553in}%
\pgfsys@useobject{currentmarker}{}%
\end{pgfscope}%
\begin{pgfscope}%
\pgfsys@transformshift{0.836821in}{0.799969in}%
\pgfsys@useobject{currentmarker}{}%
\end{pgfscope}%
\begin{pgfscope}%
\pgfsys@transformshift{1.064596in}{0.931892in}%
\pgfsys@useobject{currentmarker}{}%
\end{pgfscope}%
\begin{pgfscope}%
\pgfsys@transformshift{1.402259in}{1.007949in}%
\pgfsys@useobject{currentmarker}{}%
\end{pgfscope}%
\begin{pgfscope}%
\pgfsys@transformshift{2.041322in}{1.435926in}%
\pgfsys@useobject{currentmarker}{}%
\end{pgfscope}%
\begin{pgfscope}%
\pgfsys@transformshift{2.358826in}{1.300278in}%
\pgfsys@useobject{currentmarker}{}%
\end{pgfscope}%
\begin{pgfscope}%
\pgfsys@transformshift{2.988219in}{1.687025in}%
\pgfsys@useobject{currentmarker}{}%
\end{pgfscope}%
\begin{pgfscope}%
\pgfsys@transformshift{1.460448in}{1.024424in}%
\pgfsys@useobject{currentmarker}{}%
\end{pgfscope}%
\begin{pgfscope}%
\pgfsys@transformshift{1.047579in}{0.865953in}%
\pgfsys@useobject{currentmarker}{}%
\end{pgfscope}%
\begin{pgfscope}%
\pgfsys@transformshift{1.914964in}{1.329692in}%
\pgfsys@useobject{currentmarker}{}%
\end{pgfscope}%
\begin{pgfscope}%
\pgfsys@transformshift{1.479616in}{1.075630in}%
\pgfsys@useobject{currentmarker}{}%
\end{pgfscope}%
\begin{pgfscope}%
\pgfsys@transformshift{2.125532in}{1.447252in}%
\pgfsys@useobject{currentmarker}{}%
\end{pgfscope}%
\begin{pgfscope}%
\pgfsys@transformshift{1.074879in}{0.914166in}%
\pgfsys@useobject{currentmarker}{}%
\end{pgfscope}%
\begin{pgfscope}%
\pgfsys@transformshift{2.111063in}{1.408383in}%
\pgfsys@useobject{currentmarker}{}%
\end{pgfscope}%
\begin{pgfscope}%
\pgfsys@transformshift{2.836579in}{1.658728in}%
\pgfsys@useobject{currentmarker}{}%
\end{pgfscope}%
\begin{pgfscope}%
\pgfsys@transformshift{2.459702in}{1.647981in}%
\pgfsys@useobject{currentmarker}{}%
\end{pgfscope}%
\begin{pgfscope}%
\pgfsys@transformshift{2.663395in}{1.579041in}%
\pgfsys@useobject{currentmarker}{}%
\end{pgfscope}%
\begin{pgfscope}%
\pgfsys@transformshift{1.956950in}{1.300377in}%
\pgfsys@useobject{currentmarker}{}%
\end{pgfscope}%
\begin{pgfscope}%
\pgfsys@transformshift{0.899547in}{0.767687in}%
\pgfsys@useobject{currentmarker}{}%
\end{pgfscope}%
\begin{pgfscope}%
\pgfsys@transformshift{3.089243in}{1.867811in}%
\pgfsys@useobject{currentmarker}{}%
\end{pgfscope}%
\begin{pgfscope}%
\pgfsys@transformshift{2.855637in}{1.621951in}%
\pgfsys@useobject{currentmarker}{}%
\end{pgfscope}%
\begin{pgfscope}%
\pgfsys@transformshift{2.132209in}{1.426405in}%
\pgfsys@useobject{currentmarker}{}%
\end{pgfscope}%
\begin{pgfscope}%
\pgfsys@transformshift{1.421956in}{0.986427in}%
\pgfsys@useobject{currentmarker}{}%
\end{pgfscope}%
\begin{pgfscope}%
\pgfsys@transformshift{2.774962in}{1.558749in}%
\pgfsys@useobject{currentmarker}{}%
\end{pgfscope}%
\begin{pgfscope}%
\pgfsys@transformshift{2.802796in}{1.626915in}%
\pgfsys@useobject{currentmarker}{}%
\end{pgfscope}%
\begin{pgfscope}%
\pgfsys@transformshift{1.302511in}{0.897195in}%
\pgfsys@useobject{currentmarker}{}%
\end{pgfscope}%
\begin{pgfscope}%
\pgfsys@transformshift{1.045987in}{0.835570in}%
\pgfsys@useobject{currentmarker}{}%
\end{pgfscope}%
\begin{pgfscope}%
\pgfsys@transformshift{0.957994in}{0.762869in}%
\pgfsys@useobject{currentmarker}{}%
\end{pgfscope}%
\begin{pgfscope}%
\pgfsys@transformshift{2.220888in}{1.315358in}%
\pgfsys@useobject{currentmarker}{}%
\end{pgfscope}%
\begin{pgfscope}%
\pgfsys@transformshift{3.042097in}{1.693074in}%
\pgfsys@useobject{currentmarker}{}%
\end{pgfscope}%
\begin{pgfscope}%
\pgfsys@transformshift{2.772891in}{1.571652in}%
\pgfsys@useobject{currentmarker}{}%
\end{pgfscope}%
\begin{pgfscope}%
\pgfsys@transformshift{2.160495in}{1.504360in}%
\pgfsys@useobject{currentmarker}{}%
\end{pgfscope}%
\begin{pgfscope}%
\pgfsys@transformshift{2.547008in}{1.562975in}%
\pgfsys@useobject{currentmarker}{}%
\end{pgfscope}%
\begin{pgfscope}%
\pgfsys@transformshift{2.395783in}{1.486798in}%
\pgfsys@useobject{currentmarker}{}%
\end{pgfscope}%
\begin{pgfscope}%
\pgfsys@transformshift{1.726516in}{1.294071in}%
\pgfsys@useobject{currentmarker}{}%
\end{pgfscope}%
\begin{pgfscope}%
\pgfsys@transformshift{2.135383in}{1.372647in}%
\pgfsys@useobject{currentmarker}{}%
\end{pgfscope}%
\begin{pgfscope}%
\pgfsys@transformshift{2.070471in}{1.388978in}%
\pgfsys@useobject{currentmarker}{}%
\end{pgfscope}%
\begin{pgfscope}%
\pgfsys@transformshift{2.184692in}{1.386012in}%
\pgfsys@useobject{currentmarker}{}%
\end{pgfscope}%
\begin{pgfscope}%
\pgfsys@transformshift{1.508475in}{1.094657in}%
\pgfsys@useobject{currentmarker}{}%
\end{pgfscope}%
\begin{pgfscope}%
\pgfsys@transformshift{2.125331in}{1.425070in}%
\pgfsys@useobject{currentmarker}{}%
\end{pgfscope}%
\begin{pgfscope}%
\pgfsys@transformshift{2.212083in}{1.429054in}%
\pgfsys@useobject{currentmarker}{}%
\end{pgfscope}%
\begin{pgfscope}%
\pgfsys@transformshift{1.984180in}{1.405715in}%
\pgfsys@useobject{currentmarker}{}%
\end{pgfscope}%
\begin{pgfscope}%
\pgfsys@transformshift{1.659498in}{1.182393in}%
\pgfsys@useobject{currentmarker}{}%
\end{pgfscope}%
\begin{pgfscope}%
\pgfsys@transformshift{2.627341in}{1.596677in}%
\pgfsys@useobject{currentmarker}{}%
\end{pgfscope}%
\begin{pgfscope}%
\pgfsys@transformshift{3.224701in}{2.029976in}%
\pgfsys@useobject{currentmarker}{}%
\end{pgfscope}%
\begin{pgfscope}%
\pgfsys@transformshift{1.952313in}{1.114681in}%
\pgfsys@useobject{currentmarker}{}%
\end{pgfscope}%
\begin{pgfscope}%
\pgfsys@transformshift{2.089075in}{1.382253in}%
\pgfsys@useobject{currentmarker}{}%
\end{pgfscope}%
\begin{pgfscope}%
\pgfsys@transformshift{0.925593in}{0.914018in}%
\pgfsys@useobject{currentmarker}{}%
\end{pgfscope}%
\begin{pgfscope}%
\pgfsys@transformshift{1.101475in}{0.933163in}%
\pgfsys@useobject{currentmarker}{}%
\end{pgfscope}%
\begin{pgfscope}%
\pgfsys@transformshift{3.054937in}{1.713377in}%
\pgfsys@useobject{currentmarker}{}%
\end{pgfscope}%
\begin{pgfscope}%
\pgfsys@transformshift{1.987430in}{1.283634in}%
\pgfsys@useobject{currentmarker}{}%
\end{pgfscope}%
\begin{pgfscope}%
\pgfsys@transformshift{1.031396in}{0.930331in}%
\pgfsys@useobject{currentmarker}{}%
\end{pgfscope}%
\begin{pgfscope}%
\pgfsys@transformshift{1.223567in}{0.924875in}%
\pgfsys@useobject{currentmarker}{}%
\end{pgfscope}%
\begin{pgfscope}%
\pgfsys@transformshift{1.280042in}{1.010528in}%
\pgfsys@useobject{currentmarker}{}%
\end{pgfscope}%
\begin{pgfscope}%
\pgfsys@transformshift{1.920348in}{1.136951in}%
\pgfsys@useobject{currentmarker}{}%
\end{pgfscope}%
\begin{pgfscope}%
\pgfsys@transformshift{0.849279in}{0.861961in}%
\pgfsys@useobject{currentmarker}{}%
\end{pgfscope}%
\begin{pgfscope}%
\pgfsys@transformshift{2.085711in}{1.297163in}%
\pgfsys@useobject{currentmarker}{}%
\end{pgfscope}%
\begin{pgfscope}%
\pgfsys@transformshift{1.093438in}{0.794775in}%
\pgfsys@useobject{currentmarker}{}%
\end{pgfscope}%
\begin{pgfscope}%
\pgfsys@transformshift{2.986858in}{1.572101in}%
\pgfsys@useobject{currentmarker}{}%
\end{pgfscope}%
\begin{pgfscope}%
\pgfsys@transformshift{1.410382in}{1.048177in}%
\pgfsys@useobject{currentmarker}{}%
\end{pgfscope}%
\begin{pgfscope}%
\pgfsys@transformshift{2.612454in}{1.707946in}%
\pgfsys@useobject{currentmarker}{}%
\end{pgfscope}%
\begin{pgfscope}%
\pgfsys@transformshift{0.842801in}{0.733762in}%
\pgfsys@useobject{currentmarker}{}%
\end{pgfscope}%
\begin{pgfscope}%
\pgfsys@transformshift{2.970383in}{1.629843in}%
\pgfsys@useobject{currentmarker}{}%
\end{pgfscope}%
\begin{pgfscope}%
\pgfsys@transformshift{2.685488in}{1.528606in}%
\pgfsys@useobject{currentmarker}{}%
\end{pgfscope}%
\begin{pgfscope}%
\pgfsys@transformshift{0.879345in}{0.779829in}%
\pgfsys@useobject{currentmarker}{}%
\end{pgfscope}%
\begin{pgfscope}%
\pgfsys@transformshift{1.270801in}{0.960133in}%
\pgfsys@useobject{currentmarker}{}%
\end{pgfscope}%
\begin{pgfscope}%
\pgfsys@transformshift{2.351630in}{1.335321in}%
\pgfsys@useobject{currentmarker}{}%
\end{pgfscope}%
\begin{pgfscope}%
\pgfsys@transformshift{1.839648in}{1.242612in}%
\pgfsys@useobject{currentmarker}{}%
\end{pgfscope}%
\begin{pgfscope}%
\pgfsys@transformshift{2.989683in}{1.679717in}%
\pgfsys@useobject{currentmarker}{}%
\end{pgfscope}%
\begin{pgfscope}%
\pgfsys@transformshift{1.823454in}{1.348983in}%
\pgfsys@useobject{currentmarker}{}%
\end{pgfscope}%
\begin{pgfscope}%
\pgfsys@transformshift{1.015104in}{0.836684in}%
\pgfsys@useobject{currentmarker}{}%
\end{pgfscope}%
\begin{pgfscope}%
\pgfsys@transformshift{0.893169in}{0.850294in}%
\pgfsys@useobject{currentmarker}{}%
\end{pgfscope}%
\begin{pgfscope}%
\pgfsys@transformshift{1.566564in}{1.165774in}%
\pgfsys@useobject{currentmarker}{}%
\end{pgfscope}%
\begin{pgfscope}%
\pgfsys@transformshift{2.627072in}{1.687593in}%
\pgfsys@useobject{currentmarker}{}%
\end{pgfscope}%
\begin{pgfscope}%
\pgfsys@transformshift{1.318019in}{0.992220in}%
\pgfsys@useobject{currentmarker}{}%
\end{pgfscope}%
\begin{pgfscope}%
\pgfsys@transformshift{1.496460in}{1.110511in}%
\pgfsys@useobject{currentmarker}{}%
\end{pgfscope}%
\begin{pgfscope}%
\pgfsys@transformshift{3.068662in}{1.774858in}%
\pgfsys@useobject{currentmarker}{}%
\end{pgfscope}%
\begin{pgfscope}%
\pgfsys@transformshift{3.042928in}{1.771714in}%
\pgfsys@useobject{currentmarker}{}%
\end{pgfscope}%
\begin{pgfscope}%
\pgfsys@transformshift{1.428240in}{1.029419in}%
\pgfsys@useobject{currentmarker}{}%
\end{pgfscope}%
\begin{pgfscope}%
\pgfsys@transformshift{1.485322in}{1.134201in}%
\pgfsys@useobject{currentmarker}{}%
\end{pgfscope}%
\begin{pgfscope}%
\pgfsys@transformshift{1.578567in}{1.175852in}%
\pgfsys@useobject{currentmarker}{}%
\end{pgfscope}%
\begin{pgfscope}%
\pgfsys@transformshift{1.235674in}{0.934995in}%
\pgfsys@useobject{currentmarker}{}%
\end{pgfscope}%
\begin{pgfscope}%
\pgfsys@transformshift{2.470205in}{1.691435in}%
\pgfsys@useobject{currentmarker}{}%
\end{pgfscope}%
\begin{pgfscope}%
\pgfsys@transformshift{3.190158in}{2.154075in}%
\pgfsys@useobject{currentmarker}{}%
\end{pgfscope}%
\begin{pgfscope}%
\pgfsys@transformshift{2.445190in}{1.441485in}%
\pgfsys@useobject{currentmarker}{}%
\end{pgfscope}%
\begin{pgfscope}%
\pgfsys@transformshift{1.786562in}{1.150417in}%
\pgfsys@useobject{currentmarker}{}%
\end{pgfscope}%
\begin{pgfscope}%
\pgfsys@transformshift{1.069339in}{0.951593in}%
\pgfsys@useobject{currentmarker}{}%
\end{pgfscope}%
\begin{pgfscope}%
\pgfsys@transformshift{2.656064in}{1.684193in}%
\pgfsys@useobject{currentmarker}{}%
\end{pgfscope}%
\begin{pgfscope}%
\pgfsys@transformshift{1.275727in}{0.971630in}%
\pgfsys@useobject{currentmarker}{}%
\end{pgfscope}%
\begin{pgfscope}%
\pgfsys@transformshift{2.491015in}{1.565142in}%
\pgfsys@useobject{currentmarker}{}%
\end{pgfscope}%
\begin{pgfscope}%
\pgfsys@transformshift{2.887542in}{1.562502in}%
\pgfsys@useobject{currentmarker}{}%
\end{pgfscope}%
\begin{pgfscope}%
\pgfsys@transformshift{2.980448in}{1.850931in}%
\pgfsys@useobject{currentmarker}{}%
\end{pgfscope}%
\begin{pgfscope}%
\pgfsys@transformshift{2.655157in}{1.659569in}%
\pgfsys@useobject{currentmarker}{}%
\end{pgfscope}%
\begin{pgfscope}%
\pgfsys@transformshift{2.993350in}{1.640639in}%
\pgfsys@useobject{currentmarker}{}%
\end{pgfscope}%
\begin{pgfscope}%
\pgfsys@transformshift{1.496141in}{1.175712in}%
\pgfsys@useobject{currentmarker}{}%
\end{pgfscope}%
\begin{pgfscope}%
\pgfsys@transformshift{2.274800in}{1.412416in}%
\pgfsys@useobject{currentmarker}{}%
\end{pgfscope}%
\begin{pgfscope}%
\pgfsys@transformshift{2.313848in}{1.400989in}%
\pgfsys@useobject{currentmarker}{}%
\end{pgfscope}%
\begin{pgfscope}%
\pgfsys@transformshift{3.156264in}{1.999944in}%
\pgfsys@useobject{currentmarker}{}%
\end{pgfscope}%
\begin{pgfscope}%
\pgfsys@transformshift{2.221399in}{1.282723in}%
\pgfsys@useobject{currentmarker}{}%
\end{pgfscope}%
\begin{pgfscope}%
\pgfsys@transformshift{2.972710in}{1.648040in}%
\pgfsys@useobject{currentmarker}{}%
\end{pgfscope}%
\begin{pgfscope}%
\pgfsys@transformshift{2.770796in}{1.635777in}%
\pgfsys@useobject{currentmarker}{}%
\end{pgfscope}%
\begin{pgfscope}%
\pgfsys@transformshift{2.625247in}{1.473840in}%
\pgfsys@useobject{currentmarker}{}%
\end{pgfscope}%
\begin{pgfscope}%
\pgfsys@transformshift{2.640733in}{1.650113in}%
\pgfsys@useobject{currentmarker}{}%
\end{pgfscope}%
\begin{pgfscope}%
\pgfsys@transformshift{2.382807in}{1.452482in}%
\pgfsys@useobject{currentmarker}{}%
\end{pgfscope}%
\begin{pgfscope}%
\pgfsys@transformshift{1.387592in}{1.107848in}%
\pgfsys@useobject{currentmarker}{}%
\end{pgfscope}%
\begin{pgfscope}%
\pgfsys@transformshift{1.311906in}{0.999659in}%
\pgfsys@useobject{currentmarker}{}%
\end{pgfscope}%
\begin{pgfscope}%
\pgfsys@transformshift{1.886225in}{1.273228in}%
\pgfsys@useobject{currentmarker}{}%
\end{pgfscope}%
\begin{pgfscope}%
\pgfsys@transformshift{2.589336in}{1.501314in}%
\pgfsys@useobject{currentmarker}{}%
\end{pgfscope}%
\begin{pgfscope}%
\pgfsys@transformshift{1.136181in}{0.925126in}%
\pgfsys@useobject{currentmarker}{}%
\end{pgfscope}%
\begin{pgfscope}%
\pgfsys@transformshift{3.117219in}{1.740536in}%
\pgfsys@useobject{currentmarker}{}%
\end{pgfscope}%
\begin{pgfscope}%
\pgfsys@transformshift{3.159973in}{1.835583in}%
\pgfsys@useobject{currentmarker}{}%
\end{pgfscope}%
\begin{pgfscope}%
\pgfsys@transformshift{3.220643in}{1.723456in}%
\pgfsys@useobject{currentmarker}{}%
\end{pgfscope}%
\begin{pgfscope}%
\pgfsys@transformshift{2.391082in}{1.478173in}%
\pgfsys@useobject{currentmarker}{}%
\end{pgfscope}%
\begin{pgfscope}%
\pgfsys@transformshift{1.590678in}{1.018630in}%
\pgfsys@useobject{currentmarker}{}%
\end{pgfscope}%
\begin{pgfscope}%
\pgfsys@transformshift{1.750343in}{1.332502in}%
\pgfsys@useobject{currentmarker}{}%
\end{pgfscope}%
\begin{pgfscope}%
\pgfsys@transformshift{1.237853in}{0.948204in}%
\pgfsys@useobject{currentmarker}{}%
\end{pgfscope}%
\begin{pgfscope}%
\pgfsys@transformshift{0.922781in}{0.785893in}%
\pgfsys@useobject{currentmarker}{}%
\end{pgfscope}%
\begin{pgfscope}%
\pgfsys@transformshift{0.961016in}{0.923123in}%
\pgfsys@useobject{currentmarker}{}%
\end{pgfscope}%
\begin{pgfscope}%
\pgfsys@transformshift{2.446964in}{1.501322in}%
\pgfsys@useobject{currentmarker}{}%
\end{pgfscope}%
\begin{pgfscope}%
\pgfsys@transformshift{2.028199in}{1.267696in}%
\pgfsys@useobject{currentmarker}{}%
\end{pgfscope}%
\begin{pgfscope}%
\pgfsys@transformshift{1.851312in}{1.283012in}%
\pgfsys@useobject{currentmarker}{}%
\end{pgfscope}%
\begin{pgfscope}%
\pgfsys@transformshift{2.839453in}{1.661944in}%
\pgfsys@useobject{currentmarker}{}%
\end{pgfscope}%
\begin{pgfscope}%
\pgfsys@transformshift{2.220315in}{1.486886in}%
\pgfsys@useobject{currentmarker}{}%
\end{pgfscope}%
\begin{pgfscope}%
\pgfsys@transformshift{2.918932in}{1.604531in}%
\pgfsys@useobject{currentmarker}{}%
\end{pgfscope}%
\begin{pgfscope}%
\pgfsys@transformshift{3.162782in}{1.776224in}%
\pgfsys@useobject{currentmarker}{}%
\end{pgfscope}%
\begin{pgfscope}%
\pgfsys@transformshift{1.597207in}{1.047168in}%
\pgfsys@useobject{currentmarker}{}%
\end{pgfscope}%
\begin{pgfscope}%
\pgfsys@transformshift{1.436581in}{0.888929in}%
\pgfsys@useobject{currentmarker}{}%
\end{pgfscope}%
\begin{pgfscope}%
\pgfsys@transformshift{3.023743in}{1.782091in}%
\pgfsys@useobject{currentmarker}{}%
\end{pgfscope}%
\begin{pgfscope}%
\pgfsys@transformshift{1.752324in}{1.214696in}%
\pgfsys@useobject{currentmarker}{}%
\end{pgfscope}%
\begin{pgfscope}%
\pgfsys@transformshift{1.619074in}{1.272397in}%
\pgfsys@useobject{currentmarker}{}%
\end{pgfscope}%
\begin{pgfscope}%
\pgfsys@transformshift{0.842622in}{0.739709in}%
\pgfsys@useobject{currentmarker}{}%
\end{pgfscope}%
\begin{pgfscope}%
\pgfsys@transformshift{2.554149in}{1.649913in}%
\pgfsys@useobject{currentmarker}{}%
\end{pgfscope}%
\begin{pgfscope}%
\pgfsys@transformshift{1.764322in}{1.162167in}%
\pgfsys@useobject{currentmarker}{}%
\end{pgfscope}%
\begin{pgfscope}%
\pgfsys@transformshift{1.149571in}{0.880997in}%
\pgfsys@useobject{currentmarker}{}%
\end{pgfscope}%
\begin{pgfscope}%
\pgfsys@transformshift{2.566318in}{1.446088in}%
\pgfsys@useobject{currentmarker}{}%
\end{pgfscope}%
\begin{pgfscope}%
\pgfsys@transformshift{0.981609in}{0.747628in}%
\pgfsys@useobject{currentmarker}{}%
\end{pgfscope}%
\begin{pgfscope}%
\pgfsys@transformshift{1.055627in}{0.836696in}%
\pgfsys@useobject{currentmarker}{}%
\end{pgfscope}%
\begin{pgfscope}%
\pgfsys@transformshift{1.623780in}{1.028772in}%
\pgfsys@useobject{currentmarker}{}%
\end{pgfscope}%
\begin{pgfscope}%
\pgfsys@transformshift{3.137193in}{1.462057in}%
\pgfsys@useobject{currentmarker}{}%
\end{pgfscope}%
\begin{pgfscope}%
\pgfsys@transformshift{2.504706in}{1.414281in}%
\pgfsys@useobject{currentmarker}{}%
\end{pgfscope}%
\begin{pgfscope}%
\pgfsys@transformshift{2.083159in}{1.332581in}%
\pgfsys@useobject{currentmarker}{}%
\end{pgfscope}%
\begin{pgfscope}%
\pgfsys@transformshift{1.271104in}{1.026878in}%
\pgfsys@useobject{currentmarker}{}%
\end{pgfscope}%
\begin{pgfscope}%
\pgfsys@transformshift{1.952976in}{1.182991in}%
\pgfsys@useobject{currentmarker}{}%
\end{pgfscope}%
\begin{pgfscope}%
\pgfsys@transformshift{1.201871in}{0.996093in}%
\pgfsys@useobject{currentmarker}{}%
\end{pgfscope}%
\begin{pgfscope}%
\pgfsys@transformshift{1.993070in}{1.160787in}%
\pgfsys@useobject{currentmarker}{}%
\end{pgfscope}%
\begin{pgfscope}%
\pgfsys@transformshift{2.890922in}{1.798119in}%
\pgfsys@useobject{currentmarker}{}%
\end{pgfscope}%
\begin{pgfscope}%
\pgfsys@transformshift{1.929376in}{1.229984in}%
\pgfsys@useobject{currentmarker}{}%
\end{pgfscope}%
\begin{pgfscope}%
\pgfsys@transformshift{1.857414in}{1.215908in}%
\pgfsys@useobject{currentmarker}{}%
\end{pgfscope}%
\begin{pgfscope}%
\pgfsys@transformshift{1.381038in}{1.032218in}%
\pgfsys@useobject{currentmarker}{}%
\end{pgfscope}%
\begin{pgfscope}%
\pgfsys@transformshift{1.447012in}{1.057617in}%
\pgfsys@useobject{currentmarker}{}%
\end{pgfscope}%
\begin{pgfscope}%
\pgfsys@transformshift{2.356279in}{1.361648in}%
\pgfsys@useobject{currentmarker}{}%
\end{pgfscope}%
\begin{pgfscope}%
\pgfsys@transformshift{1.276332in}{0.863968in}%
\pgfsys@useobject{currentmarker}{}%
\end{pgfscope}%
\begin{pgfscope}%
\pgfsys@transformshift{1.047778in}{0.850912in}%
\pgfsys@useobject{currentmarker}{}%
\end{pgfscope}%
\begin{pgfscope}%
\pgfsys@transformshift{1.476896in}{1.144234in}%
\pgfsys@useobject{currentmarker}{}%
\end{pgfscope}%
\begin{pgfscope}%
\pgfsys@transformshift{1.180864in}{0.773805in}%
\pgfsys@useobject{currentmarker}{}%
\end{pgfscope}%
\begin{pgfscope}%
\pgfsys@transformshift{1.012626in}{0.804942in}%
\pgfsys@useobject{currentmarker}{}%
\end{pgfscope}%
\begin{pgfscope}%
\pgfsys@transformshift{2.435402in}{1.521768in}%
\pgfsys@useobject{currentmarker}{}%
\end{pgfscope}%
\begin{pgfscope}%
\pgfsys@transformshift{1.011242in}{0.859403in}%
\pgfsys@useobject{currentmarker}{}%
\end{pgfscope}%
\begin{pgfscope}%
\pgfsys@transformshift{2.968285in}{1.758205in}%
\pgfsys@useobject{currentmarker}{}%
\end{pgfscope}%
\begin{pgfscope}%
\pgfsys@transformshift{1.362493in}{1.089455in}%
\pgfsys@useobject{currentmarker}{}%
\end{pgfscope}%
\begin{pgfscope}%
\pgfsys@transformshift{1.025830in}{0.941852in}%
\pgfsys@useobject{currentmarker}{}%
\end{pgfscope}%
\begin{pgfscope}%
\pgfsys@transformshift{3.138656in}{1.829940in}%
\pgfsys@useobject{currentmarker}{}%
\end{pgfscope}%
\begin{pgfscope}%
\pgfsys@transformshift{1.358848in}{0.980912in}%
\pgfsys@useobject{currentmarker}{}%
\end{pgfscope}%
\begin{pgfscope}%
\pgfsys@transformshift{2.461322in}{1.627620in}%
\pgfsys@useobject{currentmarker}{}%
\end{pgfscope}%
\begin{pgfscope}%
\pgfsys@transformshift{2.417844in}{1.380725in}%
\pgfsys@useobject{currentmarker}{}%
\end{pgfscope}%
\begin{pgfscope}%
\pgfsys@transformshift{2.028689in}{1.377610in}%
\pgfsys@useobject{currentmarker}{}%
\end{pgfscope}%
\begin{pgfscope}%
\pgfsys@transformshift{2.078536in}{1.272946in}%
\pgfsys@useobject{currentmarker}{}%
\end{pgfscope}%
\begin{pgfscope}%
\pgfsys@transformshift{2.152180in}{1.352342in}%
\pgfsys@useobject{currentmarker}{}%
\end{pgfscope}%
\begin{pgfscope}%
\pgfsys@transformshift{2.406161in}{1.467721in}%
\pgfsys@useobject{currentmarker}{}%
\end{pgfscope}%
\begin{pgfscope}%
\pgfsys@transformshift{2.771325in}{1.510682in}%
\pgfsys@useobject{currentmarker}{}%
\end{pgfscope}%
\begin{pgfscope}%
\pgfsys@transformshift{3.102421in}{1.677060in}%
\pgfsys@useobject{currentmarker}{}%
\end{pgfscope}%
\begin{pgfscope}%
\pgfsys@transformshift{2.220621in}{1.401934in}%
\pgfsys@useobject{currentmarker}{}%
\end{pgfscope}%
\begin{pgfscope}%
\pgfsys@transformshift{1.286336in}{0.928378in}%
\pgfsys@useobject{currentmarker}{}%
\end{pgfscope}%
\begin{pgfscope}%
\pgfsys@transformshift{1.351715in}{1.037440in}%
\pgfsys@useobject{currentmarker}{}%
\end{pgfscope}%
\begin{pgfscope}%
\pgfsys@transformshift{3.228823in}{2.046580in}%
\pgfsys@useobject{currentmarker}{}%
\end{pgfscope}%
\begin{pgfscope}%
\pgfsys@transformshift{2.763925in}{1.760408in}%
\pgfsys@useobject{currentmarker}{}%
\end{pgfscope}%
\begin{pgfscope}%
\pgfsys@transformshift{3.100374in}{1.798657in}%
\pgfsys@useobject{currentmarker}{}%
\end{pgfscope}%
\begin{pgfscope}%
\pgfsys@transformshift{2.277751in}{1.561696in}%
\pgfsys@useobject{currentmarker}{}%
\end{pgfscope}%
\begin{pgfscope}%
\pgfsys@transformshift{2.121593in}{1.478684in}%
\pgfsys@useobject{currentmarker}{}%
\end{pgfscope}%
\begin{pgfscope}%
\pgfsys@transformshift{2.575187in}{1.593816in}%
\pgfsys@useobject{currentmarker}{}%
\end{pgfscope}%
\begin{pgfscope}%
\pgfsys@transformshift{2.327035in}{1.554331in}%
\pgfsys@useobject{currentmarker}{}%
\end{pgfscope}%
\begin{pgfscope}%
\pgfsys@transformshift{1.332124in}{0.965058in}%
\pgfsys@useobject{currentmarker}{}%
\end{pgfscope}%
\begin{pgfscope}%
\pgfsys@transformshift{2.469502in}{1.543156in}%
\pgfsys@useobject{currentmarker}{}%
\end{pgfscope}%
\begin{pgfscope}%
\pgfsys@transformshift{0.868232in}{0.781764in}%
\pgfsys@useobject{currentmarker}{}%
\end{pgfscope}%
\begin{pgfscope}%
\pgfsys@transformshift{1.340157in}{1.036419in}%
\pgfsys@useobject{currentmarker}{}%
\end{pgfscope}%
\begin{pgfscope}%
\pgfsys@transformshift{1.579695in}{1.160236in}%
\pgfsys@useobject{currentmarker}{}%
\end{pgfscope}%
\begin{pgfscope}%
\pgfsys@transformshift{3.159870in}{1.529623in}%
\pgfsys@useobject{currentmarker}{}%
\end{pgfscope}%
\begin{pgfscope}%
\pgfsys@transformshift{1.957434in}{1.311434in}%
\pgfsys@useobject{currentmarker}{}%
\end{pgfscope}%
\begin{pgfscope}%
\pgfsys@transformshift{1.137116in}{0.918026in}%
\pgfsys@useobject{currentmarker}{}%
\end{pgfscope}%
\begin{pgfscope}%
\pgfsys@transformshift{2.425059in}{1.554160in}%
\pgfsys@useobject{currentmarker}{}%
\end{pgfscope}%
\begin{pgfscope}%
\pgfsys@transformshift{2.579898in}{1.502466in}%
\pgfsys@useobject{currentmarker}{}%
\end{pgfscope}%
\begin{pgfscope}%
\pgfsys@transformshift{1.214463in}{0.872407in}%
\pgfsys@useobject{currentmarker}{}%
\end{pgfscope}%
\begin{pgfscope}%
\pgfsys@transformshift{3.214825in}{1.900484in}%
\pgfsys@useobject{currentmarker}{}%
\end{pgfscope}%
\begin{pgfscope}%
\pgfsys@transformshift{1.817363in}{1.286827in}%
\pgfsys@useobject{currentmarker}{}%
\end{pgfscope}%
\begin{pgfscope}%
\pgfsys@transformshift{2.963627in}{1.808923in}%
\pgfsys@useobject{currentmarker}{}%
\end{pgfscope}%
\begin{pgfscope}%
\pgfsys@transformshift{1.051066in}{0.777169in}%
\pgfsys@useobject{currentmarker}{}%
\end{pgfscope}%
\begin{pgfscope}%
\pgfsys@transformshift{2.193759in}{1.473504in}%
\pgfsys@useobject{currentmarker}{}%
\end{pgfscope}%
\begin{pgfscope}%
\pgfsys@transformshift{2.958953in}{1.537444in}%
\pgfsys@useobject{currentmarker}{}%
\end{pgfscope}%
\begin{pgfscope}%
\pgfsys@transformshift{2.177928in}{1.463779in}%
\pgfsys@useobject{currentmarker}{}%
\end{pgfscope}%
\begin{pgfscope}%
\pgfsys@transformshift{1.137111in}{0.889089in}%
\pgfsys@useobject{currentmarker}{}%
\end{pgfscope}%
\begin{pgfscope}%
\pgfsys@transformshift{1.365687in}{0.969587in}%
\pgfsys@useobject{currentmarker}{}%
\end{pgfscope}%
\begin{pgfscope}%
\pgfsys@transformshift{1.517625in}{1.035072in}%
\pgfsys@useobject{currentmarker}{}%
\end{pgfscope}%
\begin{pgfscope}%
\pgfsys@transformshift{2.086474in}{1.442376in}%
\pgfsys@useobject{currentmarker}{}%
\end{pgfscope}%
\begin{pgfscope}%
\pgfsys@transformshift{1.284304in}{0.941219in}%
\pgfsys@useobject{currentmarker}{}%
\end{pgfscope}%
\begin{pgfscope}%
\pgfsys@transformshift{1.449326in}{1.136548in}%
\pgfsys@useobject{currentmarker}{}%
\end{pgfscope}%
\begin{pgfscope}%
\pgfsys@transformshift{1.252420in}{0.975400in}%
\pgfsys@useobject{currentmarker}{}%
\end{pgfscope}%
\begin{pgfscope}%
\pgfsys@transformshift{1.206268in}{1.002882in}%
\pgfsys@useobject{currentmarker}{}%
\end{pgfscope}%
\begin{pgfscope}%
\pgfsys@transformshift{1.335175in}{1.027712in}%
\pgfsys@useobject{currentmarker}{}%
\end{pgfscope}%
\begin{pgfscope}%
\pgfsys@transformshift{1.737057in}{1.374993in}%
\pgfsys@useobject{currentmarker}{}%
\end{pgfscope}%
\begin{pgfscope}%
\pgfsys@transformshift{2.714055in}{1.633451in}%
\pgfsys@useobject{currentmarker}{}%
\end{pgfscope}%
\begin{pgfscope}%
\pgfsys@transformshift{3.218205in}{2.040531in}%
\pgfsys@useobject{currentmarker}{}%
\end{pgfscope}%
\begin{pgfscope}%
\pgfsys@transformshift{1.161644in}{0.988989in}%
\pgfsys@useobject{currentmarker}{}%
\end{pgfscope}%
\begin{pgfscope}%
\pgfsys@transformshift{2.504737in}{1.360524in}%
\pgfsys@useobject{currentmarker}{}%
\end{pgfscope}%
\begin{pgfscope}%
\pgfsys@transformshift{2.440336in}{1.445882in}%
\pgfsys@useobject{currentmarker}{}%
\end{pgfscope}%
\begin{pgfscope}%
\pgfsys@transformshift{1.509229in}{1.171908in}%
\pgfsys@useobject{currentmarker}{}%
\end{pgfscope}%
\begin{pgfscope}%
\pgfsys@transformshift{3.198700in}{1.692798in}%
\pgfsys@useobject{currentmarker}{}%
\end{pgfscope}%
\begin{pgfscope}%
\pgfsys@transformshift{2.173283in}{1.419527in}%
\pgfsys@useobject{currentmarker}{}%
\end{pgfscope}%
\begin{pgfscope}%
\pgfsys@transformshift{1.396165in}{1.068092in}%
\pgfsys@useobject{currentmarker}{}%
\end{pgfscope}%
\begin{pgfscope}%
\pgfsys@transformshift{3.125458in}{1.756420in}%
\pgfsys@useobject{currentmarker}{}%
\end{pgfscope}%
\begin{pgfscope}%
\pgfsys@transformshift{1.093106in}{0.878443in}%
\pgfsys@useobject{currentmarker}{}%
\end{pgfscope}%
\begin{pgfscope}%
\pgfsys@transformshift{3.098193in}{1.612695in}%
\pgfsys@useobject{currentmarker}{}%
\end{pgfscope}%
\begin{pgfscope}%
\pgfsys@transformshift{1.261737in}{0.893430in}%
\pgfsys@useobject{currentmarker}{}%
\end{pgfscope}%
\begin{pgfscope}%
\pgfsys@transformshift{3.001289in}{1.616390in}%
\pgfsys@useobject{currentmarker}{}%
\end{pgfscope}%
\begin{pgfscope}%
\pgfsys@transformshift{2.528806in}{1.679978in}%
\pgfsys@useobject{currentmarker}{}%
\end{pgfscope}%
\begin{pgfscope}%
\pgfsys@transformshift{2.761170in}{1.720201in}%
\pgfsys@useobject{currentmarker}{}%
\end{pgfscope}%
\begin{pgfscope}%
\pgfsys@transformshift{2.734579in}{1.423218in}%
\pgfsys@useobject{currentmarker}{}%
\end{pgfscope}%
\begin{pgfscope}%
\pgfsys@transformshift{2.824010in}{1.404835in}%
\pgfsys@useobject{currentmarker}{}%
\end{pgfscope}%
\begin{pgfscope}%
\pgfsys@transformshift{1.097869in}{1.005007in}%
\pgfsys@useobject{currentmarker}{}%
\end{pgfscope}%
\begin{pgfscope}%
\pgfsys@transformshift{2.416889in}{1.229718in}%
\pgfsys@useobject{currentmarker}{}%
\end{pgfscope}%
\begin{pgfscope}%
\pgfsys@transformshift{1.363649in}{1.039780in}%
\pgfsys@useobject{currentmarker}{}%
\end{pgfscope}%
\begin{pgfscope}%
\pgfsys@transformshift{2.920837in}{1.454663in}%
\pgfsys@useobject{currentmarker}{}%
\end{pgfscope}%
\begin{pgfscope}%
\pgfsys@transformshift{2.579399in}{1.574887in}%
\pgfsys@useobject{currentmarker}{}%
\end{pgfscope}%
\begin{pgfscope}%
\pgfsys@transformshift{2.736349in}{1.570864in}%
\pgfsys@useobject{currentmarker}{}%
\end{pgfscope}%
\begin{pgfscope}%
\pgfsys@transformshift{2.510705in}{1.595883in}%
\pgfsys@useobject{currentmarker}{}%
\end{pgfscope}%
\begin{pgfscope}%
\pgfsys@transformshift{2.883435in}{1.590662in}%
\pgfsys@useobject{currentmarker}{}%
\end{pgfscope}%
\begin{pgfscope}%
\pgfsys@transformshift{1.088349in}{0.959203in}%
\pgfsys@useobject{currentmarker}{}%
\end{pgfscope}%
\begin{pgfscope}%
\pgfsys@transformshift{1.326174in}{1.074464in}%
\pgfsys@useobject{currentmarker}{}%
\end{pgfscope}%
\begin{pgfscope}%
\pgfsys@transformshift{0.976771in}{0.810993in}%
\pgfsys@useobject{currentmarker}{}%
\end{pgfscope}%
\begin{pgfscope}%
\pgfsys@transformshift{1.373506in}{0.936235in}%
\pgfsys@useobject{currentmarker}{}%
\end{pgfscope}%
\begin{pgfscope}%
\pgfsys@transformshift{1.455788in}{1.146624in}%
\pgfsys@useobject{currentmarker}{}%
\end{pgfscope}%
\begin{pgfscope}%
\pgfsys@transformshift{0.819635in}{0.813970in}%
\pgfsys@useobject{currentmarker}{}%
\end{pgfscope}%
\begin{pgfscope}%
\pgfsys@transformshift{1.505311in}{1.131542in}%
\pgfsys@useobject{currentmarker}{}%
\end{pgfscope}%
\begin{pgfscope}%
\pgfsys@transformshift{1.937005in}{1.404568in}%
\pgfsys@useobject{currentmarker}{}%
\end{pgfscope}%
\begin{pgfscope}%
\pgfsys@transformshift{2.446352in}{1.549730in}%
\pgfsys@useobject{currentmarker}{}%
\end{pgfscope}%
\begin{pgfscope}%
\pgfsys@transformshift{3.068722in}{1.596768in}%
\pgfsys@useobject{currentmarker}{}%
\end{pgfscope}%
\begin{pgfscope}%
\pgfsys@transformshift{2.461918in}{1.679360in}%
\pgfsys@useobject{currentmarker}{}%
\end{pgfscope}%
\begin{pgfscope}%
\pgfsys@transformshift{3.009085in}{1.710294in}%
\pgfsys@useobject{currentmarker}{}%
\end{pgfscope}%
\begin{pgfscope}%
\pgfsys@transformshift{2.504160in}{1.593736in}%
\pgfsys@useobject{currentmarker}{}%
\end{pgfscope}%
\begin{pgfscope}%
\pgfsys@transformshift{2.530816in}{1.632996in}%
\pgfsys@useobject{currentmarker}{}%
\end{pgfscope}%
\begin{pgfscope}%
\pgfsys@transformshift{1.046940in}{0.941581in}%
\pgfsys@useobject{currentmarker}{}%
\end{pgfscope}%
\begin{pgfscope}%
\pgfsys@transformshift{1.164379in}{0.965996in}%
\pgfsys@useobject{currentmarker}{}%
\end{pgfscope}%
\begin{pgfscope}%
\pgfsys@transformshift{1.948967in}{1.328798in}%
\pgfsys@useobject{currentmarker}{}%
\end{pgfscope}%
\begin{pgfscope}%
\pgfsys@transformshift{3.179323in}{1.801909in}%
\pgfsys@useobject{currentmarker}{}%
\end{pgfscope}%
\begin{pgfscope}%
\pgfsys@transformshift{2.906505in}{1.602547in}%
\pgfsys@useobject{currentmarker}{}%
\end{pgfscope}%
\begin{pgfscope}%
\pgfsys@transformshift{1.368882in}{0.864099in}%
\pgfsys@useobject{currentmarker}{}%
\end{pgfscope}%
\begin{pgfscope}%
\pgfsys@transformshift{1.276141in}{0.979356in}%
\pgfsys@useobject{currentmarker}{}%
\end{pgfscope}%
\begin{pgfscope}%
\pgfsys@transformshift{1.707029in}{1.300395in}%
\pgfsys@useobject{currentmarker}{}%
\end{pgfscope}%
\begin{pgfscope}%
\pgfsys@transformshift{2.151329in}{1.359940in}%
\pgfsys@useobject{currentmarker}{}%
\end{pgfscope}%
\begin{pgfscope}%
\pgfsys@transformshift{3.078255in}{1.703643in}%
\pgfsys@useobject{currentmarker}{}%
\end{pgfscope}%
\begin{pgfscope}%
\pgfsys@transformshift{1.975341in}{1.179706in}%
\pgfsys@useobject{currentmarker}{}%
\end{pgfscope}%
\begin{pgfscope}%
\pgfsys@transformshift{3.035001in}{1.692875in}%
\pgfsys@useobject{currentmarker}{}%
\end{pgfscope}%
\begin{pgfscope}%
\pgfsys@transformshift{1.421873in}{1.095736in}%
\pgfsys@useobject{currentmarker}{}%
\end{pgfscope}%
\begin{pgfscope}%
\pgfsys@transformshift{2.383039in}{1.497072in}%
\pgfsys@useobject{currentmarker}{}%
\end{pgfscope}%
\begin{pgfscope}%
\pgfsys@transformshift{0.967142in}{0.911381in}%
\pgfsys@useobject{currentmarker}{}%
\end{pgfscope}%
\begin{pgfscope}%
\pgfsys@transformshift{2.992465in}{1.698002in}%
\pgfsys@useobject{currentmarker}{}%
\end{pgfscope}%
\begin{pgfscope}%
\pgfsys@transformshift{2.520764in}{1.433858in}%
\pgfsys@useobject{currentmarker}{}%
\end{pgfscope}%
\begin{pgfscope}%
\pgfsys@transformshift{2.980121in}{1.408336in}%
\pgfsys@useobject{currentmarker}{}%
\end{pgfscope}%
\begin{pgfscope}%
\pgfsys@transformshift{1.764862in}{1.209074in}%
\pgfsys@useobject{currentmarker}{}%
\end{pgfscope}%
\begin{pgfscope}%
\pgfsys@transformshift{1.705390in}{1.243925in}%
\pgfsys@useobject{currentmarker}{}%
\end{pgfscope}%
\begin{pgfscope}%
\pgfsys@transformshift{1.626615in}{1.196386in}%
\pgfsys@useobject{currentmarker}{}%
\end{pgfscope}%
\begin{pgfscope}%
\pgfsys@transformshift{2.301366in}{1.328911in}%
\pgfsys@useobject{currentmarker}{}%
\end{pgfscope}%
\begin{pgfscope}%
\pgfsys@transformshift{0.868742in}{0.722401in}%
\pgfsys@useobject{currentmarker}{}%
\end{pgfscope}%
\begin{pgfscope}%
\pgfsys@transformshift{1.393207in}{1.058857in}%
\pgfsys@useobject{currentmarker}{}%
\end{pgfscope}%
\begin{pgfscope}%
\pgfsys@transformshift{1.322576in}{0.998844in}%
\pgfsys@useobject{currentmarker}{}%
\end{pgfscope}%
\begin{pgfscope}%
\pgfsys@transformshift{2.610789in}{1.573750in}%
\pgfsys@useobject{currentmarker}{}%
\end{pgfscope}%
\begin{pgfscope}%
\pgfsys@transformshift{1.877121in}{1.340368in}%
\pgfsys@useobject{currentmarker}{}%
\end{pgfscope}%
\begin{pgfscope}%
\pgfsys@transformshift{0.879137in}{0.829204in}%
\pgfsys@useobject{currentmarker}{}%
\end{pgfscope}%
\begin{pgfscope}%
\pgfsys@transformshift{1.971499in}{1.425323in}%
\pgfsys@useobject{currentmarker}{}%
\end{pgfscope}%
\begin{pgfscope}%
\pgfsys@transformshift{2.467842in}{1.640310in}%
\pgfsys@useobject{currentmarker}{}%
\end{pgfscope}%
\begin{pgfscope}%
\pgfsys@transformshift{3.094993in}{1.704460in}%
\pgfsys@useobject{currentmarker}{}%
\end{pgfscope}%
\begin{pgfscope}%
\pgfsys@transformshift{2.425102in}{1.574514in}%
\pgfsys@useobject{currentmarker}{}%
\end{pgfscope}%
\begin{pgfscope}%
\pgfsys@transformshift{2.663132in}{1.575994in}%
\pgfsys@useobject{currentmarker}{}%
\end{pgfscope}%
\begin{pgfscope}%
\pgfsys@transformshift{2.526709in}{1.518319in}%
\pgfsys@useobject{currentmarker}{}%
\end{pgfscope}%
\begin{pgfscope}%
\pgfsys@transformshift{1.712058in}{1.298162in}%
\pgfsys@useobject{currentmarker}{}%
\end{pgfscope}%
\begin{pgfscope}%
\pgfsys@transformshift{1.194558in}{0.952572in}%
\pgfsys@useobject{currentmarker}{}%
\end{pgfscope}%
\begin{pgfscope}%
\pgfsys@transformshift{2.004081in}{1.377054in}%
\pgfsys@useobject{currentmarker}{}%
\end{pgfscope}%
\begin{pgfscope}%
\pgfsys@transformshift{2.823424in}{1.620129in}%
\pgfsys@useobject{currentmarker}{}%
\end{pgfscope}%
\begin{pgfscope}%
\pgfsys@transformshift{1.791989in}{1.199039in}%
\pgfsys@useobject{currentmarker}{}%
\end{pgfscope}%
\begin{pgfscope}%
\pgfsys@transformshift{2.798400in}{1.704676in}%
\pgfsys@useobject{currentmarker}{}%
\end{pgfscope}%
\begin{pgfscope}%
\pgfsys@transformshift{2.048804in}{1.348195in}%
\pgfsys@useobject{currentmarker}{}%
\end{pgfscope}%
\begin{pgfscope}%
\pgfsys@transformshift{2.919913in}{1.665465in}%
\pgfsys@useobject{currentmarker}{}%
\end{pgfscope}%
\begin{pgfscope}%
\pgfsys@transformshift{1.848244in}{1.216244in}%
\pgfsys@useobject{currentmarker}{}%
\end{pgfscope}%
\begin{pgfscope}%
\pgfsys@transformshift{1.604797in}{1.072509in}%
\pgfsys@useobject{currentmarker}{}%
\end{pgfscope}%
\begin{pgfscope}%
\pgfsys@transformshift{1.439290in}{0.978204in}%
\pgfsys@useobject{currentmarker}{}%
\end{pgfscope}%
\begin{pgfscope}%
\pgfsys@transformshift{3.224098in}{2.021361in}%
\pgfsys@useobject{currentmarker}{}%
\end{pgfscope}%
\begin{pgfscope}%
\pgfsys@transformshift{2.540901in}{1.446744in}%
\pgfsys@useobject{currentmarker}{}%
\end{pgfscope}%
\begin{pgfscope}%
\pgfsys@transformshift{2.379618in}{1.351384in}%
\pgfsys@useobject{currentmarker}{}%
\end{pgfscope}%
\begin{pgfscope}%
\pgfsys@transformshift{2.255482in}{1.355523in}%
\pgfsys@useobject{currentmarker}{}%
\end{pgfscope}%
\begin{pgfscope}%
\pgfsys@transformshift{1.960333in}{1.368142in}%
\pgfsys@useobject{currentmarker}{}%
\end{pgfscope}%
\begin{pgfscope}%
\pgfsys@transformshift{2.015299in}{1.437202in}%
\pgfsys@useobject{currentmarker}{}%
\end{pgfscope}%
\begin{pgfscope}%
\pgfsys@transformshift{1.992833in}{1.241242in}%
\pgfsys@useobject{currentmarker}{}%
\end{pgfscope}%
\begin{pgfscope}%
\pgfsys@transformshift{1.200992in}{1.043647in}%
\pgfsys@useobject{currentmarker}{}%
\end{pgfscope}%
\begin{pgfscope}%
\pgfsys@transformshift{2.302993in}{1.487427in}%
\pgfsys@useobject{currentmarker}{}%
\end{pgfscope}%
\begin{pgfscope}%
\pgfsys@transformshift{1.878333in}{1.310049in}%
\pgfsys@useobject{currentmarker}{}%
\end{pgfscope}%
\begin{pgfscope}%
\pgfsys@transformshift{3.084191in}{1.799942in}%
\pgfsys@useobject{currentmarker}{}%
\end{pgfscope}%
\begin{pgfscope}%
\pgfsys@transformshift{2.422335in}{1.493672in}%
\pgfsys@useobject{currentmarker}{}%
\end{pgfscope}%
\begin{pgfscope}%
\pgfsys@transformshift{2.249726in}{1.404161in}%
\pgfsys@useobject{currentmarker}{}%
\end{pgfscope}%
\begin{pgfscope}%
\pgfsys@transformshift{1.959312in}{1.358793in}%
\pgfsys@useobject{currentmarker}{}%
\end{pgfscope}%
\begin{pgfscope}%
\pgfsys@transformshift{3.167081in}{1.798584in}%
\pgfsys@useobject{currentmarker}{}%
\end{pgfscope}%
\begin{pgfscope}%
\pgfsys@transformshift{1.727544in}{1.301371in}%
\pgfsys@useobject{currentmarker}{}%
\end{pgfscope}%
\begin{pgfscope}%
\pgfsys@transformshift{2.782877in}{1.701364in}%
\pgfsys@useobject{currentmarker}{}%
\end{pgfscope}%
\begin{pgfscope}%
\pgfsys@transformshift{1.270026in}{0.945114in}%
\pgfsys@useobject{currentmarker}{}%
\end{pgfscope}%
\begin{pgfscope}%
\pgfsys@transformshift{1.475473in}{1.161156in}%
\pgfsys@useobject{currentmarker}{}%
\end{pgfscope}%
\begin{pgfscope}%
\pgfsys@transformshift{1.308864in}{0.990401in}%
\pgfsys@useobject{currentmarker}{}%
\end{pgfscope}%
\begin{pgfscope}%
\pgfsys@transformshift{2.665970in}{1.573822in}%
\pgfsys@useobject{currentmarker}{}%
\end{pgfscope}%
\begin{pgfscope}%
\pgfsys@transformshift{3.016851in}{1.768015in}%
\pgfsys@useobject{currentmarker}{}%
\end{pgfscope}%
\begin{pgfscope}%
\pgfsys@transformshift{2.843025in}{1.694263in}%
\pgfsys@useobject{currentmarker}{}%
\end{pgfscope}%
\begin{pgfscope}%
\pgfsys@transformshift{2.660089in}{1.531596in}%
\pgfsys@useobject{currentmarker}{}%
\end{pgfscope}%
\begin{pgfscope}%
\pgfsys@transformshift{1.110338in}{0.934138in}%
\pgfsys@useobject{currentmarker}{}%
\end{pgfscope}%
\begin{pgfscope}%
\pgfsys@transformshift{1.241930in}{0.996250in}%
\pgfsys@useobject{currentmarker}{}%
\end{pgfscope}%
\begin{pgfscope}%
\pgfsys@transformshift{1.085199in}{0.913597in}%
\pgfsys@useobject{currentmarker}{}%
\end{pgfscope}%
\begin{pgfscope}%
\pgfsys@transformshift{3.072472in}{1.890450in}%
\pgfsys@useobject{currentmarker}{}%
\end{pgfscope}%
\begin{pgfscope}%
\pgfsys@transformshift{1.650301in}{1.137980in}%
\pgfsys@useobject{currentmarker}{}%
\end{pgfscope}%
\begin{pgfscope}%
\pgfsys@transformshift{3.147954in}{1.971921in}%
\pgfsys@useobject{currentmarker}{}%
\end{pgfscope}%
\begin{pgfscope}%
\pgfsys@transformshift{1.229335in}{0.875265in}%
\pgfsys@useobject{currentmarker}{}%
\end{pgfscope}%
\begin{pgfscope}%
\pgfsys@transformshift{1.532744in}{1.208415in}%
\pgfsys@useobject{currentmarker}{}%
\end{pgfscope}%
\begin{pgfscope}%
\pgfsys@transformshift{3.088926in}{1.923042in}%
\pgfsys@useobject{currentmarker}{}%
\end{pgfscope}%
\begin{pgfscope}%
\pgfsys@transformshift{2.387426in}{1.595682in}%
\pgfsys@useobject{currentmarker}{}%
\end{pgfscope}%
\begin{pgfscope}%
\pgfsys@transformshift{1.176096in}{0.937897in}%
\pgfsys@useobject{currentmarker}{}%
\end{pgfscope}%
\begin{pgfscope}%
\pgfsys@transformshift{1.776677in}{1.062098in}%
\pgfsys@useobject{currentmarker}{}%
\end{pgfscope}%
\begin{pgfscope}%
\pgfsys@transformshift{2.956393in}{1.833385in}%
\pgfsys@useobject{currentmarker}{}%
\end{pgfscope}%
\begin{pgfscope}%
\pgfsys@transformshift{1.822787in}{1.234689in}%
\pgfsys@useobject{currentmarker}{}%
\end{pgfscope}%
\begin{pgfscope}%
\pgfsys@transformshift{1.207386in}{0.913677in}%
\pgfsys@useobject{currentmarker}{}%
\end{pgfscope}%
\begin{pgfscope}%
\pgfsys@transformshift{1.930049in}{1.335976in}%
\pgfsys@useobject{currentmarker}{}%
\end{pgfscope}%
\begin{pgfscope}%
\pgfsys@transformshift{0.829991in}{0.815222in}%
\pgfsys@useobject{currentmarker}{}%
\end{pgfscope}%
\begin{pgfscope}%
\pgfsys@transformshift{1.032287in}{0.865080in}%
\pgfsys@useobject{currentmarker}{}%
\end{pgfscope}%
\begin{pgfscope}%
\pgfsys@transformshift{2.733308in}{1.665247in}%
\pgfsys@useobject{currentmarker}{}%
\end{pgfscope}%
\begin{pgfscope}%
\pgfsys@transformshift{0.855999in}{0.892142in}%
\pgfsys@useobject{currentmarker}{}%
\end{pgfscope}%
\begin{pgfscope}%
\pgfsys@transformshift{1.955583in}{1.424176in}%
\pgfsys@useobject{currentmarker}{}%
\end{pgfscope}%
\begin{pgfscope}%
\pgfsys@transformshift{2.531408in}{1.379699in}%
\pgfsys@useobject{currentmarker}{}%
\end{pgfscope}%
\begin{pgfscope}%
\pgfsys@transformshift{2.361470in}{1.384869in}%
\pgfsys@useobject{currentmarker}{}%
\end{pgfscope}%
\begin{pgfscope}%
\pgfsys@transformshift{2.561499in}{1.343329in}%
\pgfsys@useobject{currentmarker}{}%
\end{pgfscope}%
\begin{pgfscope}%
\pgfsys@transformshift{1.757001in}{1.227612in}%
\pgfsys@useobject{currentmarker}{}%
\end{pgfscope}%
\begin{pgfscope}%
\pgfsys@transformshift{1.364324in}{0.987138in}%
\pgfsys@useobject{currentmarker}{}%
\end{pgfscope}%
\begin{pgfscope}%
\pgfsys@transformshift{2.328030in}{1.417843in}%
\pgfsys@useobject{currentmarker}{}%
\end{pgfscope}%
\begin{pgfscope}%
\pgfsys@transformshift{2.788146in}{1.582659in}%
\pgfsys@useobject{currentmarker}{}%
\end{pgfscope}%
\begin{pgfscope}%
\pgfsys@transformshift{1.214990in}{0.931825in}%
\pgfsys@useobject{currentmarker}{}%
\end{pgfscope}%
\begin{pgfscope}%
\pgfsys@transformshift{1.398036in}{1.070188in}%
\pgfsys@useobject{currentmarker}{}%
\end{pgfscope}%
\begin{pgfscope}%
\pgfsys@transformshift{1.574201in}{1.083580in}%
\pgfsys@useobject{currentmarker}{}%
\end{pgfscope}%
\begin{pgfscope}%
\pgfsys@transformshift{1.888845in}{1.137876in}%
\pgfsys@useobject{currentmarker}{}%
\end{pgfscope}%
\begin{pgfscope}%
\pgfsys@transformshift{2.381751in}{1.413292in}%
\pgfsys@useobject{currentmarker}{}%
\end{pgfscope}%
\begin{pgfscope}%
\pgfsys@transformshift{1.625422in}{1.191902in}%
\pgfsys@useobject{currentmarker}{}%
\end{pgfscope}%
\begin{pgfscope}%
\pgfsys@transformshift{2.960325in}{1.632125in}%
\pgfsys@useobject{currentmarker}{}%
\end{pgfscope}%
\begin{pgfscope}%
\pgfsys@transformshift{2.751804in}{1.782974in}%
\pgfsys@useobject{currentmarker}{}%
\end{pgfscope}%
\begin{pgfscope}%
\pgfsys@transformshift{1.752098in}{1.355594in}%
\pgfsys@useobject{currentmarker}{}%
\end{pgfscope}%
\begin{pgfscope}%
\pgfsys@transformshift{1.052203in}{0.900045in}%
\pgfsys@useobject{currentmarker}{}%
\end{pgfscope}%
\begin{pgfscope}%
\pgfsys@transformshift{2.427907in}{1.513709in}%
\pgfsys@useobject{currentmarker}{}%
\end{pgfscope}%
\begin{pgfscope}%
\pgfsys@transformshift{2.837203in}{1.407581in}%
\pgfsys@useobject{currentmarker}{}%
\end{pgfscope}%
\begin{pgfscope}%
\pgfsys@transformshift{0.861809in}{0.889022in}%
\pgfsys@useobject{currentmarker}{}%
\end{pgfscope}%
\begin{pgfscope}%
\pgfsys@transformshift{1.541827in}{1.067080in}%
\pgfsys@useobject{currentmarker}{}%
\end{pgfscope}%
\begin{pgfscope}%
\pgfsys@transformshift{1.586619in}{1.222952in}%
\pgfsys@useobject{currentmarker}{}%
\end{pgfscope}%
\begin{pgfscope}%
\pgfsys@transformshift{1.825460in}{1.298012in}%
\pgfsys@useobject{currentmarker}{}%
\end{pgfscope}%
\begin{pgfscope}%
\pgfsys@transformshift{1.366210in}{1.023732in}%
\pgfsys@useobject{currentmarker}{}%
\end{pgfscope}%
\begin{pgfscope}%
\pgfsys@transformshift{2.821918in}{1.733676in}%
\pgfsys@useobject{currentmarker}{}%
\end{pgfscope}%
\begin{pgfscope}%
\pgfsys@transformshift{2.188204in}{1.550648in}%
\pgfsys@useobject{currentmarker}{}%
\end{pgfscope}%
\begin{pgfscope}%
\pgfsys@transformshift{1.233540in}{0.867786in}%
\pgfsys@useobject{currentmarker}{}%
\end{pgfscope}%
\begin{pgfscope}%
\pgfsys@transformshift{2.998823in}{1.639824in}%
\pgfsys@useobject{currentmarker}{}%
\end{pgfscope}%
\begin{pgfscope}%
\pgfsys@transformshift{2.460445in}{1.613620in}%
\pgfsys@useobject{currentmarker}{}%
\end{pgfscope}%
\begin{pgfscope}%
\pgfsys@transformshift{1.771066in}{1.322521in}%
\pgfsys@useobject{currentmarker}{}%
\end{pgfscope}%
\begin{pgfscope}%
\pgfsys@transformshift{1.272689in}{0.898597in}%
\pgfsys@useobject{currentmarker}{}%
\end{pgfscope}%
\begin{pgfscope}%
\pgfsys@transformshift{2.474866in}{1.596264in}%
\pgfsys@useobject{currentmarker}{}%
\end{pgfscope}%
\begin{pgfscope}%
\pgfsys@transformshift{1.501644in}{1.091260in}%
\pgfsys@useobject{currentmarker}{}%
\end{pgfscope}%
\begin{pgfscope}%
\pgfsys@transformshift{1.608804in}{1.245657in}%
\pgfsys@useobject{currentmarker}{}%
\end{pgfscope}%
\begin{pgfscope}%
\pgfsys@transformshift{2.078163in}{1.486111in}%
\pgfsys@useobject{currentmarker}{}%
\end{pgfscope}%
\begin{pgfscope}%
\pgfsys@transformshift{1.416081in}{1.091930in}%
\pgfsys@useobject{currentmarker}{}%
\end{pgfscope}%
\begin{pgfscope}%
\pgfsys@transformshift{1.876491in}{1.191794in}%
\pgfsys@useobject{currentmarker}{}%
\end{pgfscope}%
\begin{pgfscope}%
\pgfsys@transformshift{1.887274in}{1.287700in}%
\pgfsys@useobject{currentmarker}{}%
\end{pgfscope}%
\begin{pgfscope}%
\pgfsys@transformshift{2.197126in}{1.385138in}%
\pgfsys@useobject{currentmarker}{}%
\end{pgfscope}%
\begin{pgfscope}%
\pgfsys@transformshift{0.950538in}{0.908040in}%
\pgfsys@useobject{currentmarker}{}%
\end{pgfscope}%
\begin{pgfscope}%
\pgfsys@transformshift{1.447304in}{1.007819in}%
\pgfsys@useobject{currentmarker}{}%
\end{pgfscope}%
\begin{pgfscope}%
\pgfsys@transformshift{1.755399in}{1.188781in}%
\pgfsys@useobject{currentmarker}{}%
\end{pgfscope}%
\begin{pgfscope}%
\pgfsys@transformshift{1.910709in}{1.318938in}%
\pgfsys@useobject{currentmarker}{}%
\end{pgfscope}%
\begin{pgfscope}%
\pgfsys@transformshift{1.093697in}{0.948651in}%
\pgfsys@useobject{currentmarker}{}%
\end{pgfscope}%
\begin{pgfscope}%
\pgfsys@transformshift{2.477692in}{1.635542in}%
\pgfsys@useobject{currentmarker}{}%
\end{pgfscope}%
\begin{pgfscope}%
\pgfsys@transformshift{2.381046in}{1.493069in}%
\pgfsys@useobject{currentmarker}{}%
\end{pgfscope}%
\begin{pgfscope}%
\pgfsys@transformshift{2.341106in}{1.558874in}%
\pgfsys@useobject{currentmarker}{}%
\end{pgfscope}%
\begin{pgfscope}%
\pgfsys@transformshift{1.600858in}{1.088113in}%
\pgfsys@useobject{currentmarker}{}%
\end{pgfscope}%
\begin{pgfscope}%
\pgfsys@transformshift{2.176618in}{1.428921in}%
\pgfsys@useobject{currentmarker}{}%
\end{pgfscope}%
\begin{pgfscope}%
\pgfsys@transformshift{3.136923in}{1.715320in}%
\pgfsys@useobject{currentmarker}{}%
\end{pgfscope}%
\begin{pgfscope}%
\pgfsys@transformshift{1.334723in}{1.119231in}%
\pgfsys@useobject{currentmarker}{}%
\end{pgfscope}%
\begin{pgfscope}%
\pgfsys@transformshift{1.335757in}{1.014014in}%
\pgfsys@useobject{currentmarker}{}%
\end{pgfscope}%
\begin{pgfscope}%
\pgfsys@transformshift{2.745211in}{1.764357in}%
\pgfsys@useobject{currentmarker}{}%
\end{pgfscope}%
\begin{pgfscope}%
\pgfsys@transformshift{2.260392in}{1.421025in}%
\pgfsys@useobject{currentmarker}{}%
\end{pgfscope}%
\begin{pgfscope}%
\pgfsys@transformshift{1.616425in}{1.170463in}%
\pgfsys@useobject{currentmarker}{}%
\end{pgfscope}%
\begin{pgfscope}%
\pgfsys@transformshift{3.011419in}{1.671998in}%
\pgfsys@useobject{currentmarker}{}%
\end{pgfscope}%
\begin{pgfscope}%
\pgfsys@transformshift{3.053851in}{1.690236in}%
\pgfsys@useobject{currentmarker}{}%
\end{pgfscope}%
\begin{pgfscope}%
\pgfsys@transformshift{2.125266in}{1.245843in}%
\pgfsys@useobject{currentmarker}{}%
\end{pgfscope}%
\begin{pgfscope}%
\pgfsys@transformshift{2.435438in}{1.629810in}%
\pgfsys@useobject{currentmarker}{}%
\end{pgfscope}%
\begin{pgfscope}%
\pgfsys@transformshift{1.867031in}{1.166087in}%
\pgfsys@useobject{currentmarker}{}%
\end{pgfscope}%
\begin{pgfscope}%
\pgfsys@transformshift{1.849402in}{1.350394in}%
\pgfsys@useobject{currentmarker}{}%
\end{pgfscope}%
\begin{pgfscope}%
\pgfsys@transformshift{0.970569in}{0.916333in}%
\pgfsys@useobject{currentmarker}{}%
\end{pgfscope}%
\begin{pgfscope}%
\pgfsys@transformshift{2.423315in}{1.645726in}%
\pgfsys@useobject{currentmarker}{}%
\end{pgfscope}%
\begin{pgfscope}%
\pgfsys@transformshift{2.730898in}{1.708447in}%
\pgfsys@useobject{currentmarker}{}%
\end{pgfscope}%
\begin{pgfscope}%
\pgfsys@transformshift{1.944067in}{1.362802in}%
\pgfsys@useobject{currentmarker}{}%
\end{pgfscope}%
\begin{pgfscope}%
\pgfsys@transformshift{0.995500in}{0.703234in}%
\pgfsys@useobject{currentmarker}{}%
\end{pgfscope}%
\begin{pgfscope}%
\pgfsys@transformshift{3.223730in}{2.146467in}%
\pgfsys@useobject{currentmarker}{}%
\end{pgfscope}%
\begin{pgfscope}%
\pgfsys@transformshift{2.453635in}{1.656821in}%
\pgfsys@useobject{currentmarker}{}%
\end{pgfscope}%
\begin{pgfscope}%
\pgfsys@transformshift{0.920077in}{0.755373in}%
\pgfsys@useobject{currentmarker}{}%
\end{pgfscope}%
\begin{pgfscope}%
\pgfsys@transformshift{1.790653in}{1.249853in}%
\pgfsys@useobject{currentmarker}{}%
\end{pgfscope}%
\begin{pgfscope}%
\pgfsys@transformshift{1.501159in}{1.112313in}%
\pgfsys@useobject{currentmarker}{}%
\end{pgfscope}%
\begin{pgfscope}%
\pgfsys@transformshift{2.437795in}{1.530500in}%
\pgfsys@useobject{currentmarker}{}%
\end{pgfscope}%
\begin{pgfscope}%
\pgfsys@transformshift{1.229174in}{0.991786in}%
\pgfsys@useobject{currentmarker}{}%
\end{pgfscope}%
\begin{pgfscope}%
\pgfsys@transformshift{1.153523in}{0.984179in}%
\pgfsys@useobject{currentmarker}{}%
\end{pgfscope}%
\begin{pgfscope}%
\pgfsys@transformshift{1.983666in}{1.261079in}%
\pgfsys@useobject{currentmarker}{}%
\end{pgfscope}%
\begin{pgfscope}%
\pgfsys@transformshift{3.035918in}{1.663180in}%
\pgfsys@useobject{currentmarker}{}%
\end{pgfscope}%
\begin{pgfscope}%
\pgfsys@transformshift{2.821677in}{1.633643in}%
\pgfsys@useobject{currentmarker}{}%
\end{pgfscope}%
\begin{pgfscope}%
\pgfsys@transformshift{3.056269in}{1.825108in}%
\pgfsys@useobject{currentmarker}{}%
\end{pgfscope}%
\begin{pgfscope}%
\pgfsys@transformshift{2.356277in}{1.574052in}%
\pgfsys@useobject{currentmarker}{}%
\end{pgfscope}%
\begin{pgfscope}%
\pgfsys@transformshift{1.013453in}{0.821539in}%
\pgfsys@useobject{currentmarker}{}%
\end{pgfscope}%
\begin{pgfscope}%
\pgfsys@transformshift{2.603951in}{1.737391in}%
\pgfsys@useobject{currentmarker}{}%
\end{pgfscope}%
\begin{pgfscope}%
\pgfsys@transformshift{2.759788in}{1.316471in}%
\pgfsys@useobject{currentmarker}{}%
\end{pgfscope}%
\begin{pgfscope}%
\pgfsys@transformshift{3.099950in}{1.889084in}%
\pgfsys@useobject{currentmarker}{}%
\end{pgfscope}%
\begin{pgfscope}%
\pgfsys@transformshift{2.429090in}{1.347948in}%
\pgfsys@useobject{currentmarker}{}%
\end{pgfscope}%
\begin{pgfscope}%
\pgfsys@transformshift{2.947096in}{1.637719in}%
\pgfsys@useobject{currentmarker}{}%
\end{pgfscope}%
\begin{pgfscope}%
\pgfsys@transformshift{2.393553in}{1.284508in}%
\pgfsys@useobject{currentmarker}{}%
\end{pgfscope}%
\begin{pgfscope}%
\pgfsys@transformshift{1.653729in}{1.311769in}%
\pgfsys@useobject{currentmarker}{}%
\end{pgfscope}%
\begin{pgfscope}%
\pgfsys@transformshift{2.116406in}{1.428023in}%
\pgfsys@useobject{currentmarker}{}%
\end{pgfscope}%
\begin{pgfscope}%
\pgfsys@transformshift{2.493791in}{1.687563in}%
\pgfsys@useobject{currentmarker}{}%
\end{pgfscope}%
\begin{pgfscope}%
\pgfsys@transformshift{2.944677in}{1.603356in}%
\pgfsys@useobject{currentmarker}{}%
\end{pgfscope}%
\begin{pgfscope}%
\pgfsys@transformshift{3.150033in}{1.700360in}%
\pgfsys@useobject{currentmarker}{}%
\end{pgfscope}%
\begin{pgfscope}%
\pgfsys@transformshift{2.182106in}{1.370303in}%
\pgfsys@useobject{currentmarker}{}%
\end{pgfscope}%
\begin{pgfscope}%
\pgfsys@transformshift{2.739641in}{1.715917in}%
\pgfsys@useobject{currentmarker}{}%
\end{pgfscope}%
\begin{pgfscope}%
\pgfsys@transformshift{2.090337in}{1.537922in}%
\pgfsys@useobject{currentmarker}{}%
\end{pgfscope}%
\begin{pgfscope}%
\pgfsys@transformshift{0.950265in}{0.752938in}%
\pgfsys@useobject{currentmarker}{}%
\end{pgfscope}%
\begin{pgfscope}%
\pgfsys@transformshift{2.317601in}{1.339943in}%
\pgfsys@useobject{currentmarker}{}%
\end{pgfscope}%
\begin{pgfscope}%
\pgfsys@transformshift{3.165941in}{1.808210in}%
\pgfsys@useobject{currentmarker}{}%
\end{pgfscope}%
\begin{pgfscope}%
\pgfsys@transformshift{1.535549in}{1.127496in}%
\pgfsys@useobject{currentmarker}{}%
\end{pgfscope}%
\begin{pgfscope}%
\pgfsys@transformshift{1.430600in}{1.042752in}%
\pgfsys@useobject{currentmarker}{}%
\end{pgfscope}%
\begin{pgfscope}%
\pgfsys@transformshift{2.015891in}{1.419312in}%
\pgfsys@useobject{currentmarker}{}%
\end{pgfscope}%
\begin{pgfscope}%
\pgfsys@transformshift{2.082069in}{1.247588in}%
\pgfsys@useobject{currentmarker}{}%
\end{pgfscope}%
\begin{pgfscope}%
\pgfsys@transformshift{2.605481in}{1.524412in}%
\pgfsys@useobject{currentmarker}{}%
\end{pgfscope}%
\begin{pgfscope}%
\pgfsys@transformshift{3.179656in}{1.981719in}%
\pgfsys@useobject{currentmarker}{}%
\end{pgfscope}%
\begin{pgfscope}%
\pgfsys@transformshift{3.101946in}{1.937694in}%
\pgfsys@useobject{currentmarker}{}%
\end{pgfscope}%
\begin{pgfscope}%
\pgfsys@transformshift{2.539557in}{1.589159in}%
\pgfsys@useobject{currentmarker}{}%
\end{pgfscope}%
\begin{pgfscope}%
\pgfsys@transformshift{1.085178in}{0.856656in}%
\pgfsys@useobject{currentmarker}{}%
\end{pgfscope}%
\begin{pgfscope}%
\pgfsys@transformshift{2.482191in}{1.615719in}%
\pgfsys@useobject{currentmarker}{}%
\end{pgfscope}%
\begin{pgfscope}%
\pgfsys@transformshift{2.490946in}{1.521839in}%
\pgfsys@useobject{currentmarker}{}%
\end{pgfscope}%
\begin{pgfscope}%
\pgfsys@transformshift{1.009463in}{0.810241in}%
\pgfsys@useobject{currentmarker}{}%
\end{pgfscope}%
\begin{pgfscope}%
\pgfsys@transformshift{2.088177in}{1.217364in}%
\pgfsys@useobject{currentmarker}{}%
\end{pgfscope}%
\begin{pgfscope}%
\pgfsys@transformshift{3.066861in}{1.818710in}%
\pgfsys@useobject{currentmarker}{}%
\end{pgfscope}%
\begin{pgfscope}%
\pgfsys@transformshift{2.536799in}{1.653235in}%
\pgfsys@useobject{currentmarker}{}%
\end{pgfscope}%
\begin{pgfscope}%
\pgfsys@transformshift{2.200996in}{1.418019in}%
\pgfsys@useobject{currentmarker}{}%
\end{pgfscope}%
\begin{pgfscope}%
\pgfsys@transformshift{2.594447in}{1.600766in}%
\pgfsys@useobject{currentmarker}{}%
\end{pgfscope}%
\begin{pgfscope}%
\pgfsys@transformshift{1.093144in}{0.912215in}%
\pgfsys@useobject{currentmarker}{}%
\end{pgfscope}%
\begin{pgfscope}%
\pgfsys@transformshift{1.126541in}{0.905613in}%
\pgfsys@useobject{currentmarker}{}%
\end{pgfscope}%
\begin{pgfscope}%
\pgfsys@transformshift{3.225803in}{1.957127in}%
\pgfsys@useobject{currentmarker}{}%
\end{pgfscope}%
\begin{pgfscope}%
\pgfsys@transformshift{1.147563in}{0.928625in}%
\pgfsys@useobject{currentmarker}{}%
\end{pgfscope}%
\begin{pgfscope}%
\pgfsys@transformshift{2.481619in}{1.545714in}%
\pgfsys@useobject{currentmarker}{}%
\end{pgfscope}%
\begin{pgfscope}%
\pgfsys@transformshift{1.824022in}{1.290177in}%
\pgfsys@useobject{currentmarker}{}%
\end{pgfscope}%
\begin{pgfscope}%
\pgfsys@transformshift{3.197543in}{1.658268in}%
\pgfsys@useobject{currentmarker}{}%
\end{pgfscope}%
\begin{pgfscope}%
\pgfsys@transformshift{1.447229in}{1.059739in}%
\pgfsys@useobject{currentmarker}{}%
\end{pgfscope}%
\begin{pgfscope}%
\pgfsys@transformshift{1.937058in}{1.398734in}%
\pgfsys@useobject{currentmarker}{}%
\end{pgfscope}%
\begin{pgfscope}%
\pgfsys@transformshift{1.553745in}{1.184521in}%
\pgfsys@useobject{currentmarker}{}%
\end{pgfscope}%
\begin{pgfscope}%
\pgfsys@transformshift{2.153837in}{1.394899in}%
\pgfsys@useobject{currentmarker}{}%
\end{pgfscope}%
\begin{pgfscope}%
\pgfsys@transformshift{2.425916in}{1.520475in}%
\pgfsys@useobject{currentmarker}{}%
\end{pgfscope}%
\begin{pgfscope}%
\pgfsys@transformshift{2.289694in}{1.345916in}%
\pgfsys@useobject{currentmarker}{}%
\end{pgfscope}%
\begin{pgfscope}%
\pgfsys@transformshift{1.131336in}{0.976406in}%
\pgfsys@useobject{currentmarker}{}%
\end{pgfscope}%
\begin{pgfscope}%
\pgfsys@transformshift{3.025262in}{1.741939in}%
\pgfsys@useobject{currentmarker}{}%
\end{pgfscope}%
\begin{pgfscope}%
\pgfsys@transformshift{3.098923in}{1.797634in}%
\pgfsys@useobject{currentmarker}{}%
\end{pgfscope}%
\begin{pgfscope}%
\pgfsys@transformshift{3.153504in}{1.739699in}%
\pgfsys@useobject{currentmarker}{}%
\end{pgfscope}%
\begin{pgfscope}%
\pgfsys@transformshift{1.675047in}{1.127895in}%
\pgfsys@useobject{currentmarker}{}%
\end{pgfscope}%
\begin{pgfscope}%
\pgfsys@transformshift{2.182712in}{1.471394in}%
\pgfsys@useobject{currentmarker}{}%
\end{pgfscope}%
\begin{pgfscope}%
\pgfsys@transformshift{1.437384in}{1.014822in}%
\pgfsys@useobject{currentmarker}{}%
\end{pgfscope}%
\begin{pgfscope}%
\pgfsys@transformshift{2.192427in}{1.471280in}%
\pgfsys@useobject{currentmarker}{}%
\end{pgfscope}%
\begin{pgfscope}%
\pgfsys@transformshift{1.356106in}{1.043110in}%
\pgfsys@useobject{currentmarker}{}%
\end{pgfscope}%
\begin{pgfscope}%
\pgfsys@transformshift{1.994532in}{1.336626in}%
\pgfsys@useobject{currentmarker}{}%
\end{pgfscope}%
\begin{pgfscope}%
\pgfsys@transformshift{1.641247in}{1.168009in}%
\pgfsys@useobject{currentmarker}{}%
\end{pgfscope}%
\begin{pgfscope}%
\pgfsys@transformshift{2.353144in}{1.231594in}%
\pgfsys@useobject{currentmarker}{}%
\end{pgfscope}%
\begin{pgfscope}%
\pgfsys@transformshift{2.339480in}{1.477093in}%
\pgfsys@useobject{currentmarker}{}%
\end{pgfscope}%
\begin{pgfscope}%
\pgfsys@transformshift{3.127738in}{1.891921in}%
\pgfsys@useobject{currentmarker}{}%
\end{pgfscope}%
\begin{pgfscope}%
\pgfsys@transformshift{2.607142in}{1.645762in}%
\pgfsys@useobject{currentmarker}{}%
\end{pgfscope}%
\begin{pgfscope}%
\pgfsys@transformshift{0.942816in}{0.851007in}%
\pgfsys@useobject{currentmarker}{}%
\end{pgfscope}%
\begin{pgfscope}%
\pgfsys@transformshift{3.010064in}{1.612674in}%
\pgfsys@useobject{currentmarker}{}%
\end{pgfscope}%
\begin{pgfscope}%
\pgfsys@transformshift{3.080692in}{1.788523in}%
\pgfsys@useobject{currentmarker}{}%
\end{pgfscope}%
\begin{pgfscope}%
\pgfsys@transformshift{2.509101in}{1.368148in}%
\pgfsys@useobject{currentmarker}{}%
\end{pgfscope}%
\begin{pgfscope}%
\pgfsys@transformshift{1.121185in}{0.931277in}%
\pgfsys@useobject{currentmarker}{}%
\end{pgfscope}%
\begin{pgfscope}%
\pgfsys@transformshift{3.096648in}{1.826943in}%
\pgfsys@useobject{currentmarker}{}%
\end{pgfscope}%
\begin{pgfscope}%
\pgfsys@transformshift{1.170635in}{0.958232in}%
\pgfsys@useobject{currentmarker}{}%
\end{pgfscope}%
\begin{pgfscope}%
\pgfsys@transformshift{3.088546in}{1.521608in}%
\pgfsys@useobject{currentmarker}{}%
\end{pgfscope}%
\begin{pgfscope}%
\pgfsys@transformshift{1.642217in}{1.190002in}%
\pgfsys@useobject{currentmarker}{}%
\end{pgfscope}%
\begin{pgfscope}%
\pgfsys@transformshift{2.269702in}{1.531666in}%
\pgfsys@useobject{currentmarker}{}%
\end{pgfscope}%
\begin{pgfscope}%
\pgfsys@transformshift{2.656496in}{1.512450in}%
\pgfsys@useobject{currentmarker}{}%
\end{pgfscope}%
\begin{pgfscope}%
\pgfsys@transformshift{1.238856in}{0.865377in}%
\pgfsys@useobject{currentmarker}{}%
\end{pgfscope}%
\begin{pgfscope}%
\pgfsys@transformshift{0.910641in}{0.897355in}%
\pgfsys@useobject{currentmarker}{}%
\end{pgfscope}%
\begin{pgfscope}%
\pgfsys@transformshift{2.138308in}{1.295035in}%
\pgfsys@useobject{currentmarker}{}%
\end{pgfscope}%
\begin{pgfscope}%
\pgfsys@transformshift{1.540312in}{1.225184in}%
\pgfsys@useobject{currentmarker}{}%
\end{pgfscope}%
\begin{pgfscope}%
\pgfsys@transformshift{2.330614in}{1.433370in}%
\pgfsys@useobject{currentmarker}{}%
\end{pgfscope}%
\begin{pgfscope}%
\pgfsys@transformshift{0.927231in}{0.828488in}%
\pgfsys@useobject{currentmarker}{}%
\end{pgfscope}%
\begin{pgfscope}%
\pgfsys@transformshift{2.842495in}{1.327321in}%
\pgfsys@useobject{currentmarker}{}%
\end{pgfscope}%
\begin{pgfscope}%
\pgfsys@transformshift{2.995980in}{1.648846in}%
\pgfsys@useobject{currentmarker}{}%
\end{pgfscope}%
\begin{pgfscope}%
\pgfsys@transformshift{1.838266in}{1.352330in}%
\pgfsys@useobject{currentmarker}{}%
\end{pgfscope}%
\begin{pgfscope}%
\pgfsys@transformshift{2.765650in}{1.572932in}%
\pgfsys@useobject{currentmarker}{}%
\end{pgfscope}%
\begin{pgfscope}%
\pgfsys@transformshift{1.362180in}{1.124855in}%
\pgfsys@useobject{currentmarker}{}%
\end{pgfscope}%
\begin{pgfscope}%
\pgfsys@transformshift{1.774482in}{1.338975in}%
\pgfsys@useobject{currentmarker}{}%
\end{pgfscope}%
\begin{pgfscope}%
\pgfsys@transformshift{2.213963in}{1.294239in}%
\pgfsys@useobject{currentmarker}{}%
\end{pgfscope}%
\begin{pgfscope}%
\pgfsys@transformshift{1.113643in}{0.944074in}%
\pgfsys@useobject{currentmarker}{}%
\end{pgfscope}%
\begin{pgfscope}%
\pgfsys@transformshift{1.996816in}{1.462817in}%
\pgfsys@useobject{currentmarker}{}%
\end{pgfscope}%
\begin{pgfscope}%
\pgfsys@transformshift{1.895513in}{1.330299in}%
\pgfsys@useobject{currentmarker}{}%
\end{pgfscope}%
\end{pgfscope}%
\begin{pgfscope}%
\pgfpathrectangle{\pgfqpoint{0.684105in}{0.571603in}}{\pgfqpoint{2.665895in}{1.657828in}}%
\pgfusepath{clip}%
\pgfsetrectcap%
\pgfsetroundjoin%
\pgfsetlinewidth{0.803000pt}%
\definecolor{currentstroke}{rgb}{0.690196,0.690196,0.690196}%
\pgfsetstrokecolor{currentstroke}%
\pgfsetdash{}{0pt}%
\pgfpathmoveto{\pgfqpoint{0.778160in}{0.571603in}}%
\pgfpathlineto{\pgfqpoint{0.778160in}{2.229431in}}%
\pgfusepath{stroke}%
\end{pgfscope}%
\begin{pgfscope}%
\pgfsetbuttcap%
\pgfsetroundjoin%
\definecolor{currentfill}{rgb}{0.000000,0.000000,0.000000}%
\pgfsetfillcolor{currentfill}%
\pgfsetlinewidth{0.803000pt}%
\definecolor{currentstroke}{rgb}{0.000000,0.000000,0.000000}%
\pgfsetstrokecolor{currentstroke}%
\pgfsetdash{}{0pt}%
\pgfsys@defobject{currentmarker}{\pgfqpoint{0.000000in}{-0.048611in}}{\pgfqpoint{0.000000in}{0.000000in}}{%
\pgfpathmoveto{\pgfqpoint{0.000000in}{0.000000in}}%
\pgfpathlineto{\pgfqpoint{0.000000in}{-0.048611in}}%
\pgfusepath{stroke,fill}%
}%
\begin{pgfscope}%
\pgfsys@transformshift{0.778160in}{0.571603in}%
\pgfsys@useobject{currentmarker}{}%
\end{pgfscope}%
\end{pgfscope}%
\begin{pgfscope}%
\definecolor{textcolor}{rgb}{0.000000,0.000000,0.000000}%
\pgfsetstrokecolor{textcolor}%
\pgfsetfillcolor{textcolor}%
\pgftext[x=0.778160in,y=0.474381in,,top]{\color{textcolor}{\rmfamily\fontsize{10.000000}{12.000000}\selectfont\catcode`\^=\active\def^{\ifmmode\sp\else\^{}\fi}\catcode`\%=\active\def%{\%}$\mathdefault{1.0}$}}%
\end{pgfscope}%
\begin{pgfscope}%
\pgfpathrectangle{\pgfqpoint{0.684105in}{0.571603in}}{\pgfqpoint{2.665895in}{1.657828in}}%
\pgfusepath{clip}%
\pgfsetrectcap%
\pgfsetroundjoin%
\pgfsetlinewidth{0.803000pt}%
\definecolor{currentstroke}{rgb}{0.690196,0.690196,0.690196}%
\pgfsetstrokecolor{currentstroke}%
\pgfsetdash{}{0pt}%
\pgfpathmoveto{\pgfqpoint{1.268374in}{0.571603in}}%
\pgfpathlineto{\pgfqpoint{1.268374in}{2.229431in}}%
\pgfusepath{stroke}%
\end{pgfscope}%
\begin{pgfscope}%
\pgfsetbuttcap%
\pgfsetroundjoin%
\definecolor{currentfill}{rgb}{0.000000,0.000000,0.000000}%
\pgfsetfillcolor{currentfill}%
\pgfsetlinewidth{0.803000pt}%
\definecolor{currentstroke}{rgb}{0.000000,0.000000,0.000000}%
\pgfsetstrokecolor{currentstroke}%
\pgfsetdash{}{0pt}%
\pgfsys@defobject{currentmarker}{\pgfqpoint{0.000000in}{-0.048611in}}{\pgfqpoint{0.000000in}{0.000000in}}{%
\pgfpathmoveto{\pgfqpoint{0.000000in}{0.000000in}}%
\pgfpathlineto{\pgfqpoint{0.000000in}{-0.048611in}}%
\pgfusepath{stroke,fill}%
}%
\begin{pgfscope}%
\pgfsys@transformshift{1.268374in}{0.571603in}%
\pgfsys@useobject{currentmarker}{}%
\end{pgfscope}%
\end{pgfscope}%
\begin{pgfscope}%
\definecolor{textcolor}{rgb}{0.000000,0.000000,0.000000}%
\pgfsetstrokecolor{textcolor}%
\pgfsetfillcolor{textcolor}%
\pgftext[x=1.268374in,y=0.474381in,,top]{\color{textcolor}{\rmfamily\fontsize{10.000000}{12.000000}\selectfont\catcode`\^=\active\def^{\ifmmode\sp\else\^{}\fi}\catcode`\%=\active\def%{\%}$\mathdefault{1.2}$}}%
\end{pgfscope}%
\begin{pgfscope}%
\pgfpathrectangle{\pgfqpoint{0.684105in}{0.571603in}}{\pgfqpoint{2.665895in}{1.657828in}}%
\pgfusepath{clip}%
\pgfsetrectcap%
\pgfsetroundjoin%
\pgfsetlinewidth{0.803000pt}%
\definecolor{currentstroke}{rgb}{0.690196,0.690196,0.690196}%
\pgfsetstrokecolor{currentstroke}%
\pgfsetdash{}{0pt}%
\pgfpathmoveto{\pgfqpoint{1.758589in}{0.571603in}}%
\pgfpathlineto{\pgfqpoint{1.758589in}{2.229431in}}%
\pgfusepath{stroke}%
\end{pgfscope}%
\begin{pgfscope}%
\pgfsetbuttcap%
\pgfsetroundjoin%
\definecolor{currentfill}{rgb}{0.000000,0.000000,0.000000}%
\pgfsetfillcolor{currentfill}%
\pgfsetlinewidth{0.803000pt}%
\definecolor{currentstroke}{rgb}{0.000000,0.000000,0.000000}%
\pgfsetstrokecolor{currentstroke}%
\pgfsetdash{}{0pt}%
\pgfsys@defobject{currentmarker}{\pgfqpoint{0.000000in}{-0.048611in}}{\pgfqpoint{0.000000in}{0.000000in}}{%
\pgfpathmoveto{\pgfqpoint{0.000000in}{0.000000in}}%
\pgfpathlineto{\pgfqpoint{0.000000in}{-0.048611in}}%
\pgfusepath{stroke,fill}%
}%
\begin{pgfscope}%
\pgfsys@transformshift{1.758589in}{0.571603in}%
\pgfsys@useobject{currentmarker}{}%
\end{pgfscope}%
\end{pgfscope}%
\begin{pgfscope}%
\definecolor{textcolor}{rgb}{0.000000,0.000000,0.000000}%
\pgfsetstrokecolor{textcolor}%
\pgfsetfillcolor{textcolor}%
\pgftext[x=1.758589in,y=0.474381in,,top]{\color{textcolor}{\rmfamily\fontsize{10.000000}{12.000000}\selectfont\catcode`\^=\active\def^{\ifmmode\sp\else\^{}\fi}\catcode`\%=\active\def%{\%}$\mathdefault{1.4}$}}%
\end{pgfscope}%
\begin{pgfscope}%
\pgfpathrectangle{\pgfqpoint{0.684105in}{0.571603in}}{\pgfqpoint{2.665895in}{1.657828in}}%
\pgfusepath{clip}%
\pgfsetrectcap%
\pgfsetroundjoin%
\pgfsetlinewidth{0.803000pt}%
\definecolor{currentstroke}{rgb}{0.690196,0.690196,0.690196}%
\pgfsetstrokecolor{currentstroke}%
\pgfsetdash{}{0pt}%
\pgfpathmoveto{\pgfqpoint{2.248803in}{0.571603in}}%
\pgfpathlineto{\pgfqpoint{2.248803in}{2.229431in}}%
\pgfusepath{stroke}%
\end{pgfscope}%
\begin{pgfscope}%
\pgfsetbuttcap%
\pgfsetroundjoin%
\definecolor{currentfill}{rgb}{0.000000,0.000000,0.000000}%
\pgfsetfillcolor{currentfill}%
\pgfsetlinewidth{0.803000pt}%
\definecolor{currentstroke}{rgb}{0.000000,0.000000,0.000000}%
\pgfsetstrokecolor{currentstroke}%
\pgfsetdash{}{0pt}%
\pgfsys@defobject{currentmarker}{\pgfqpoint{0.000000in}{-0.048611in}}{\pgfqpoint{0.000000in}{0.000000in}}{%
\pgfpathmoveto{\pgfqpoint{0.000000in}{0.000000in}}%
\pgfpathlineto{\pgfqpoint{0.000000in}{-0.048611in}}%
\pgfusepath{stroke,fill}%
}%
\begin{pgfscope}%
\pgfsys@transformshift{2.248803in}{0.571603in}%
\pgfsys@useobject{currentmarker}{}%
\end{pgfscope}%
\end{pgfscope}%
\begin{pgfscope}%
\definecolor{textcolor}{rgb}{0.000000,0.000000,0.000000}%
\pgfsetstrokecolor{textcolor}%
\pgfsetfillcolor{textcolor}%
\pgftext[x=2.248803in,y=0.474381in,,top]{\color{textcolor}{\rmfamily\fontsize{10.000000}{12.000000}\selectfont\catcode`\^=\active\def^{\ifmmode\sp\else\^{}\fi}\catcode`\%=\active\def%{\%}$\mathdefault{1.6}$}}%
\end{pgfscope}%
\begin{pgfscope}%
\pgfpathrectangle{\pgfqpoint{0.684105in}{0.571603in}}{\pgfqpoint{2.665895in}{1.657828in}}%
\pgfusepath{clip}%
\pgfsetrectcap%
\pgfsetroundjoin%
\pgfsetlinewidth{0.803000pt}%
\definecolor{currentstroke}{rgb}{0.690196,0.690196,0.690196}%
\pgfsetstrokecolor{currentstroke}%
\pgfsetdash{}{0pt}%
\pgfpathmoveto{\pgfqpoint{2.739017in}{0.571603in}}%
\pgfpathlineto{\pgfqpoint{2.739017in}{2.229431in}}%
\pgfusepath{stroke}%
\end{pgfscope}%
\begin{pgfscope}%
\pgfsetbuttcap%
\pgfsetroundjoin%
\definecolor{currentfill}{rgb}{0.000000,0.000000,0.000000}%
\pgfsetfillcolor{currentfill}%
\pgfsetlinewidth{0.803000pt}%
\definecolor{currentstroke}{rgb}{0.000000,0.000000,0.000000}%
\pgfsetstrokecolor{currentstroke}%
\pgfsetdash{}{0pt}%
\pgfsys@defobject{currentmarker}{\pgfqpoint{0.000000in}{-0.048611in}}{\pgfqpoint{0.000000in}{0.000000in}}{%
\pgfpathmoveto{\pgfqpoint{0.000000in}{0.000000in}}%
\pgfpathlineto{\pgfqpoint{0.000000in}{-0.048611in}}%
\pgfusepath{stroke,fill}%
}%
\begin{pgfscope}%
\pgfsys@transformshift{2.739017in}{0.571603in}%
\pgfsys@useobject{currentmarker}{}%
\end{pgfscope}%
\end{pgfscope}%
\begin{pgfscope}%
\definecolor{textcolor}{rgb}{0.000000,0.000000,0.000000}%
\pgfsetstrokecolor{textcolor}%
\pgfsetfillcolor{textcolor}%
\pgftext[x=2.739017in,y=0.474381in,,top]{\color{textcolor}{\rmfamily\fontsize{10.000000}{12.000000}\selectfont\catcode`\^=\active\def^{\ifmmode\sp\else\^{}\fi}\catcode`\%=\active\def%{\%}$\mathdefault{1.8}$}}%
\end{pgfscope}%
\begin{pgfscope}%
\pgfpathrectangle{\pgfqpoint{0.684105in}{0.571603in}}{\pgfqpoint{2.665895in}{1.657828in}}%
\pgfusepath{clip}%
\pgfsetrectcap%
\pgfsetroundjoin%
\pgfsetlinewidth{0.803000pt}%
\definecolor{currentstroke}{rgb}{0.690196,0.690196,0.690196}%
\pgfsetstrokecolor{currentstroke}%
\pgfsetdash{}{0pt}%
\pgfpathmoveto{\pgfqpoint{3.229231in}{0.571603in}}%
\pgfpathlineto{\pgfqpoint{3.229231in}{2.229431in}}%
\pgfusepath{stroke}%
\end{pgfscope}%
\begin{pgfscope}%
\pgfsetbuttcap%
\pgfsetroundjoin%
\definecolor{currentfill}{rgb}{0.000000,0.000000,0.000000}%
\pgfsetfillcolor{currentfill}%
\pgfsetlinewidth{0.803000pt}%
\definecolor{currentstroke}{rgb}{0.000000,0.000000,0.000000}%
\pgfsetstrokecolor{currentstroke}%
\pgfsetdash{}{0pt}%
\pgfsys@defobject{currentmarker}{\pgfqpoint{0.000000in}{-0.048611in}}{\pgfqpoint{0.000000in}{0.000000in}}{%
\pgfpathmoveto{\pgfqpoint{0.000000in}{0.000000in}}%
\pgfpathlineto{\pgfqpoint{0.000000in}{-0.048611in}}%
\pgfusepath{stroke,fill}%
}%
\begin{pgfscope}%
\pgfsys@transformshift{3.229231in}{0.571603in}%
\pgfsys@useobject{currentmarker}{}%
\end{pgfscope}%
\end{pgfscope}%
\begin{pgfscope}%
\definecolor{textcolor}{rgb}{0.000000,0.000000,0.000000}%
\pgfsetstrokecolor{textcolor}%
\pgfsetfillcolor{textcolor}%
\pgftext[x=3.229231in,y=0.474381in,,top]{\color{textcolor}{\rmfamily\fontsize{10.000000}{12.000000}\selectfont\catcode`\^=\active\def^{\ifmmode\sp\else\^{}\fi}\catcode`\%=\active\def%{\%}$\mathdefault{2.0}$}}%
\end{pgfscope}%
\begin{pgfscope}%
\definecolor{textcolor}{rgb}{0.000000,0.000000,0.000000}%
\pgfsetstrokecolor{textcolor}%
\pgfsetfillcolor{textcolor}%
\pgftext[x=2.017052in,y=0.284413in,,top]{\color{textcolor}{\rmfamily\fontsize{10.000000}{12.000000}\selectfont\catcode`\^=\active\def^{\ifmmode\sp\else\^{}\fi}\catcode`\%=\active\def%{\%}Input order, $\alpha$}}%
\end{pgfscope}%
\begin{pgfscope}%
\pgfpathrectangle{\pgfqpoint{0.684105in}{0.571603in}}{\pgfqpoint{2.665895in}{1.657828in}}%
\pgfusepath{clip}%
\pgfsetrectcap%
\pgfsetroundjoin%
\pgfsetlinewidth{0.803000pt}%
\definecolor{currentstroke}{rgb}{0.690196,0.690196,0.690196}%
\pgfsetstrokecolor{currentstroke}%
\pgfsetdash{}{0pt}%
\pgfpathmoveto{\pgfqpoint{0.684105in}{0.763781in}}%
\pgfpathlineto{\pgfqpoint{3.350000in}{0.763781in}}%
\pgfusepath{stroke}%
\end{pgfscope}%
\begin{pgfscope}%
\pgfsetbuttcap%
\pgfsetroundjoin%
\definecolor{currentfill}{rgb}{0.000000,0.000000,0.000000}%
\pgfsetfillcolor{currentfill}%
\pgfsetlinewidth{0.803000pt}%
\definecolor{currentstroke}{rgb}{0.000000,0.000000,0.000000}%
\pgfsetstrokecolor{currentstroke}%
\pgfsetdash{}{0pt}%
\pgfsys@defobject{currentmarker}{\pgfqpoint{-0.048611in}{0.000000in}}{\pgfqpoint{-0.000000in}{0.000000in}}{%
\pgfpathmoveto{\pgfqpoint{-0.000000in}{0.000000in}}%
\pgfpathlineto{\pgfqpoint{-0.048611in}{0.000000in}}%
\pgfusepath{stroke,fill}%
}%
\begin{pgfscope}%
\pgfsys@transformshift{0.684105in}{0.763781in}%
\pgfsys@useobject{currentmarker}{}%
\end{pgfscope}%
\end{pgfscope}%
\begin{pgfscope}%
\definecolor{textcolor}{rgb}{0.000000,0.000000,0.000000}%
\pgfsetstrokecolor{textcolor}%
\pgfsetfillcolor{textcolor}%
\pgftext[x=0.409413in, y=0.711020in, left, base]{\color{textcolor}{\rmfamily\fontsize{10.000000}{12.000000}\selectfont\catcode`\^=\active\def^{\ifmmode\sp\else\^{}\fi}\catcode`\%=\active\def%{\%}$\mathdefault{1.0}$}}%
\end{pgfscope}%
\begin{pgfscope}%
\pgfpathrectangle{\pgfqpoint{0.684105in}{0.571603in}}{\pgfqpoint{2.665895in}{1.657828in}}%
\pgfusepath{clip}%
\pgfsetrectcap%
\pgfsetroundjoin%
\pgfsetlinewidth{0.803000pt}%
\definecolor{currentstroke}{rgb}{0.690196,0.690196,0.690196}%
\pgfsetstrokecolor{currentstroke}%
\pgfsetdash{}{0pt}%
\pgfpathmoveto{\pgfqpoint{0.684105in}{1.368109in}}%
\pgfpathlineto{\pgfqpoint{3.350000in}{1.368109in}}%
\pgfusepath{stroke}%
\end{pgfscope}%
\begin{pgfscope}%
\pgfsetbuttcap%
\pgfsetroundjoin%
\definecolor{currentfill}{rgb}{0.000000,0.000000,0.000000}%
\pgfsetfillcolor{currentfill}%
\pgfsetlinewidth{0.803000pt}%
\definecolor{currentstroke}{rgb}{0.000000,0.000000,0.000000}%
\pgfsetstrokecolor{currentstroke}%
\pgfsetdash{}{0pt}%
\pgfsys@defobject{currentmarker}{\pgfqpoint{-0.048611in}{0.000000in}}{\pgfqpoint{-0.000000in}{0.000000in}}{%
\pgfpathmoveto{\pgfqpoint{-0.000000in}{0.000000in}}%
\pgfpathlineto{\pgfqpoint{-0.048611in}{0.000000in}}%
\pgfusepath{stroke,fill}%
}%
\begin{pgfscope}%
\pgfsys@transformshift{0.684105in}{1.368109in}%
\pgfsys@useobject{currentmarker}{}%
\end{pgfscope}%
\end{pgfscope}%
\begin{pgfscope}%
\definecolor{textcolor}{rgb}{0.000000,0.000000,0.000000}%
\pgfsetstrokecolor{textcolor}%
\pgfsetfillcolor{textcolor}%
\pgftext[x=0.409413in, y=1.315347in, left, base]{\color{textcolor}{\rmfamily\fontsize{10.000000}{12.000000}\selectfont\catcode`\^=\active\def^{\ifmmode\sp\else\^{}\fi}\catcode`\%=\active\def%{\%}$\mathdefault{1.5}$}}%
\end{pgfscope}%
\begin{pgfscope}%
\pgfpathrectangle{\pgfqpoint{0.684105in}{0.571603in}}{\pgfqpoint{2.665895in}{1.657828in}}%
\pgfusepath{clip}%
\pgfsetrectcap%
\pgfsetroundjoin%
\pgfsetlinewidth{0.803000pt}%
\definecolor{currentstroke}{rgb}{0.690196,0.690196,0.690196}%
\pgfsetstrokecolor{currentstroke}%
\pgfsetdash{}{0pt}%
\pgfpathmoveto{\pgfqpoint{0.684105in}{1.972436in}}%
\pgfpathlineto{\pgfqpoint{3.350000in}{1.972436in}}%
\pgfusepath{stroke}%
\end{pgfscope}%
\begin{pgfscope}%
\pgfsetbuttcap%
\pgfsetroundjoin%
\definecolor{currentfill}{rgb}{0.000000,0.000000,0.000000}%
\pgfsetfillcolor{currentfill}%
\pgfsetlinewidth{0.803000pt}%
\definecolor{currentstroke}{rgb}{0.000000,0.000000,0.000000}%
\pgfsetstrokecolor{currentstroke}%
\pgfsetdash{}{0pt}%
\pgfsys@defobject{currentmarker}{\pgfqpoint{-0.048611in}{0.000000in}}{\pgfqpoint{-0.000000in}{0.000000in}}{%
\pgfpathmoveto{\pgfqpoint{-0.000000in}{0.000000in}}%
\pgfpathlineto{\pgfqpoint{-0.048611in}{0.000000in}}%
\pgfusepath{stroke,fill}%
}%
\begin{pgfscope}%
\pgfsys@transformshift{0.684105in}{1.972436in}%
\pgfsys@useobject{currentmarker}{}%
\end{pgfscope}%
\end{pgfscope}%
\begin{pgfscope}%
\definecolor{textcolor}{rgb}{0.000000,0.000000,0.000000}%
\pgfsetstrokecolor{textcolor}%
\pgfsetfillcolor{textcolor}%
\pgftext[x=0.409413in, y=1.919675in, left, base]{\color{textcolor}{\rmfamily\fontsize{10.000000}{12.000000}\selectfont\catcode`\^=\active\def^{\ifmmode\sp\else\^{}\fi}\catcode`\%=\active\def%{\%}$\mathdefault{2.0}$}}%
\end{pgfscope}%
\begin{pgfscope}%
\definecolor{textcolor}{rgb}{0.000000,0.000000,0.000000}%
\pgfsetstrokecolor{textcolor}%
\pgfsetfillcolor{textcolor}%
\pgftext[x=0.353857in,y=1.400517in,,bottom,rotate=90.000000]{\color{textcolor}{\rmfamily\fontsize{10.000000}{12.000000}\selectfont\catcode`\^=\active\def^{\ifmmode\sp\else\^{}\fi}\catcode`\%=\active\def%{\%}Predicted order}}%
\end{pgfscope}%
\begin{pgfscope}%
\pgfpathrectangle{\pgfqpoint{0.684105in}{0.571603in}}{\pgfqpoint{2.665895in}{1.657828in}}%
\pgfusepath{clip}%
\pgfsetrectcap%
\pgfsetroundjoin%
\pgfsetlinewidth{1.505625pt}%
\definecolor{currentstroke}{rgb}{0.000000,0.000000,0.000000}%
\pgfsetstrokecolor{currentstroke}%
\pgfsetdash{}{0pt}%
\pgfpathmoveto{\pgfqpoint{1.567365in}{1.123805in}}%
\pgfpathlineto{\pgfqpoint{3.228823in}{1.837621in}}%
\pgfpathlineto{\pgfqpoint{0.805282in}{0.796390in}}%
\pgfpathlineto{\pgfqpoint{1.895513in}{1.264788in}}%
\pgfusepath{stroke}%
\end{pgfscope}%
\begin{pgfscope}%
\pgfsetrectcap%
\pgfsetmiterjoin%
\pgfsetlinewidth{0.803000pt}%
\definecolor{currentstroke}{rgb}{0.000000,0.000000,0.000000}%
\pgfsetstrokecolor{currentstroke}%
\pgfsetdash{}{0pt}%
\pgfpathmoveto{\pgfqpoint{0.684105in}{0.571603in}}%
\pgfpathlineto{\pgfqpoint{0.684105in}{2.229431in}}%
\pgfusepath{stroke}%
\end{pgfscope}%
\begin{pgfscope}%
\pgfsetrectcap%
\pgfsetmiterjoin%
\pgfsetlinewidth{0.803000pt}%
\definecolor{currentstroke}{rgb}{0.000000,0.000000,0.000000}%
\pgfsetstrokecolor{currentstroke}%
\pgfsetdash{}{0pt}%
\pgfpathmoveto{\pgfqpoint{3.350000in}{0.571603in}}%
\pgfpathlineto{\pgfqpoint{3.350000in}{2.229431in}}%
\pgfusepath{stroke}%
\end{pgfscope}%
\begin{pgfscope}%
\pgfsetrectcap%
\pgfsetmiterjoin%
\pgfsetlinewidth{0.803000pt}%
\definecolor{currentstroke}{rgb}{0.000000,0.000000,0.000000}%
\pgfsetstrokecolor{currentstroke}%
\pgfsetdash{}{0pt}%
\pgfpathmoveto{\pgfqpoint{0.684105in}{0.571603in}}%
\pgfpathlineto{\pgfqpoint{3.350000in}{0.571603in}}%
\pgfusepath{stroke}%
\end{pgfscope}%
\begin{pgfscope}%
\pgfsetrectcap%
\pgfsetmiterjoin%
\pgfsetlinewidth{0.803000pt}%
\definecolor{currentstroke}{rgb}{0.000000,0.000000,0.000000}%
\pgfsetstrokecolor{currentstroke}%
\pgfsetdash{}{0pt}%
\pgfpathmoveto{\pgfqpoint{0.684105in}{2.229431in}}%
\pgfpathlineto{\pgfqpoint{3.350000in}{2.229431in}}%
\pgfusepath{stroke}%
\end{pgfscope}%
\end{pgfpicture}%
\makeatother%
\endgroup%

\vspace*{-5pt}
\caption{Comparison of actual fractional orders and predicted orders.}
\label{fig:accuracy}
\end{figure}

Each of the 1000 fractional order step responses was interpolated or sampled
into time steps of $\Delta t = 0.1$, and those points were applied to the input
layer of the neural network trained on the integer order data. The results are
illustrated in Figure~\ref{fig:accuracy}, illustrating an excellent match. The
mean square error for the data is 0.00121 with $R^2 = 0.9897$. 

