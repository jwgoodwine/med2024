\section{Fractional Calculus}

  This section gives a very brief overview of only the fractional calclulus
  topics necessary to implement the methods in this paper and is therefore
  incomplete. The interested reader is referred to the references above for a
  complete exposition.

  The obvious task for fractional calculus to to define derivatives ``in
  between'' the usual integer order derivatives, \textit{e.g.}, the one half
  derivative of $x(t)$. Consider the first and second order backwards finite
  difference equations for $\Delta t \ll 1$
 \[
 \frac{\d x}{\d t}\left(t\right) \approx \frac{x(t) - x \left(t - \Delta
 t\right)}{\Delta t}
 \]
 and
 \[
    \frac{\d^2 x}{\d t^2}\left(t\right) \approx \frac{x(t) - 2 x \left(t - \Delta
    t\right) + x(t - 2 \Delta t)}{\left( \Delta t \right)^2}.
 \]
 As is well known, from these it is clear that the formula for an arbitrary
 positive integer order derivative is
 \begin{equation}
  \frac{\d^n x}{\d t^n}(t) \approx \frac{\sum_{i=0}^n \left( -1 \right)^i
  \left( \frac{n!}{i! \left( n - i \right)!} \right) x ( t - i \Delta t )}{\left( \Delta t
  \right)^n},
  \label{eq:finitediff}
 \end{equation}
 where the fraction with the factorials in the numerator and denominator is the
 binomial coefficient. Note that the number of terms in the sum is one more than
 the order of the derivative.  The only terms in Equation~\ref{eq:finitediff}
 that must be integers are the factorials. Because the gamma function can be
 considered a generalization of the factorial because $n! = \Gamma(n+1)$ for
 integer values of $n$, a generalized equation results with
 \begin{equation}
    \frac{\d^\alpha x}{d t^\alpha}(t) \approx \frac{\sum_{i=0}^{\lceil
    \frac{t}{\Delta t} \rceil} \left( -1 \right)^i \left(
    \frac{\Gamma(\alpha+1)}{i! \Gamma\left( \alpha - i + 1\right)} \right) x ( t
    - i \Delta t )}{\left( \Delta t \right)^\alpha}.
    \label{eq:gw}
 \end{equation}
  
  The generalized binomial coefficient in Equation~\ref{eq:gw} is nonzero for
  all values of $i$ (except when $\alpha$ is an integer), and hence the sum will
  include times spanning all of history. In the controls context assuming zero
  initial conditions and zero values for all time prior to zero, the sum will
  include terms at all time steps between $0$ and the current $t$. This
  highlights an important distinction between integer order derivatives and
  fractional order derivatives, which is that the latter are \emph{nonlocal}.  

  The method to compute the fractional step responses that the neural network
  will identify was computed by a different method than what is expressed in
  Equation~\ref{eq:gw}, but is similar in spirit. Specifically, we use the
  \texttt{numfracpy} python library to numerically compute fractional step
  responses, and from its documentation\footnote{http://tinyurl.com/yyytpv8d} the step response is computed using a
  fractional order Adams predictor corrector method given by
  \begin{align*}
  x^P_{n+1} &= \sum_{j=0}^{m-1} \frac{t^j_{n+1}}{j!}x_0^j + \left( \Delta t
  \right)^\alpha \sum_{j=0}^n b_{j,n+1} f\left( t_j, x_j \right) \\
  x_{n+1} &=  \sum_{j=0}^{m-1} \frac{t^j_{n+1}}{j!}x_0^j + \sum_{j=0}^n \big[ a_{j,n+1}
  f \left( t_j, x_j \right) + \\ &\  \hspace*{1in} a_{n+1,n+1} f\left( t_{n+1},
x^P_{n+1} \right) \big]
  \end{align*}
  where
  \[
  b_{j,n+1} = \frac{1}{\Gamma (\alpha + 1 )} \left[(n-j+1)^\alpha -
  (n-j)^\alpha \right]
  \] and
  \[\alpha_{0,n+1} = \frac{\left(\Delta t \right)^\alpha \left[n^{\alpha + 1} -
  (n-\alpha)(n+1)^\alpha\right]}{\Gamma \left( \alpha + 2 \right)},
  \]
  \begin{multline*}
  a_{j,n+1} = \frac{\left( \Delta t \right)^\alpha}{\Gamma \left( \alpha + 2
  \right)} \big[ (n-j+2)^{\alpha + 1} \\
  -2 (n - j + 1)^{\alpha + 1} + (n-j)^{\alpha + 1} \big]
  \end{multline*} for $1 \leq j \leq n$
  and
  \[
  \alpha_{n+1,n+1} = \frac{\left( \Delta t \right)^\alpha}{\Gamma\left(\alpha +
  2\right)}.
  \]
  The summations in the expressions for $x_{n+1}$ are the parts of the
  expressions that include the nonlocal terms.
