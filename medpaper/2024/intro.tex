\section{INTRODUCTION}

  Fractional calculus and fractional order dynamics are increasingly important
  in modern engineered systems. Unlike integer order derivatives, fractional
  order derivatives, and hence the dynamics that depend on them, are
  \emph{nonlocal}. As such, many modern, large scale engineered systems may
  exhibit fractional order dynamics and responses because interactions among
  various components in the system may be significantly displaced in time or
  space. In instances where significant fractional order dynamics are present,
  control algorithms which directly address the fractional nature of the system
  may be superior.  Therefore, tools to readily identify if significant
  fractional order dymancs are present are needed.

  There is a vast literature on fractional calculus. Some textbooks include
  \cite{fracbook,fracbook2,oustaloup}.  Fractional-order control is also a
  topical area such as in \cite{fraccontrol,YQChenAcc}. An excellent review
  article illustrating the very broad range of applications of fractional
  calculus and control in science and engineering is \cite{SUN2018213}.


  Our main interets are identifying cases where fractional order models may
  provide useful ``reduced order'' models for large scale systems
  \cite{goodwinemed2023,goodwinemmar2023} and for exact models for many large
  scale systems
  \cite{Goodwine2014Modeling,Leyden2016Using,Leyden2019Large,bg:xnids2022,bg:xninonlinear2020}.
  While this paper does not build upon it, our closest publication to this would
  be \cite{bg:chenSII2022} where we created a symmetric neural network with a
  sequential set of identical layers. When it was trained on first derivatives
  of functions, the middle layer could represent the half derivative. 
 
 There are many different definitions of the fractional derivative. As will be
 outlined in the next section, a common feature of these is replacing factorial
 functions appearing in many integer-order representations of the derivative
 with gamma functions. The Riemann-Liouville, Caputo and the Gr\"unwald-Letnikov
 definitions are perhaps the most common examples of fractional derivative
 definitions, and the reader is referred to the references
 \cite{Machado20111140,4609961,series/lnee/Ortigueira11,das2011functional} for
 descriptions and definitions of each.


